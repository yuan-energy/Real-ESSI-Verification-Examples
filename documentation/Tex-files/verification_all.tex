\documentclass[fleqn,11pt]{article}
\setlength{\textwidth}{17.5cm}
\setlength{\textheight}{23.2cm}
\setlength{\hoffset}{-2.0cm}
\setlength{\voffset}{-2.5cm}

\usepackage{graphicx}
\usepackage{lastpage}
% \usepackage{datetime}
\usepackage{fancyhdr}
\usepackage{float}
\usepackage{caption}
\usepackage{subcaption}
\usepackage{color}
\usepackage[utf8]{inputenc}
\usepackage[T1]{fontenc}
\usepackage{textcomp}
\usepackage{gensymb}
\usepackage{amsmath}

\usepackage{multirow}
\usepackage{hyperref}



% list and color set.......
%%%%%%%%%%%%%%%%%%%%%%%%%%%%%%%%%%
\usepackage{color}
\usepackage{xcolor}
\usepackage{listings}
\lstloadlanguages{Matlab,Python}
\lstdefinelanguage{fei}
{morekeywords={
20NodeBrick, 27NodeBrick, 8NodeBrick, a0, a1, a2, a3, a4, acceleration, acceleration_file, acceleration_scale_unit, add, algorithm, algorithm
, all
, alpha, and, armstrong_frederick_cr, armstrong_frederick_ha, at, ax, ay, az, bending_Iy, bending_Iz, beta, beta_min, binary
, check, compressive_strength, constant, constitutive, constrain, constraint, contact_plane_vector, control, convergence, cR1, cR2, critical_stress_ratio_M, cross_section, crushing_strength, damping, databasename, define, direction, displacement, displacement_control, displacement_file, displacement_scale_unit, dof, dofs, dofs
, domain, domain
, drained, druckerprager_angle, dynamic, e0, each, eigen, Elastic_Membrane_Plate, elastic_modulus, elastic_modulus_horizontal, elastic_modulus_vertical, element, element_file, else, equal, equaldof, factor, field, filename
, fix, force_file, forces, free, friction_ratio, gamma, Gauss, h_in, help
, host, if, imposed, Imx, Imy, Imz, in, increment, initial_confining_strain, initial_confining_stress, initial_mean_pressure, integration_points, IntegrationRule, integrator, isotropic_hardening_rate, joint_1_offset, joint_2_offset, K_f, K_s, k_x, k_y, k_z, kappa, kd_in, kinematic_hardening_rate, lambda, length, linear
, load, loading, M_in, magnitude, magnitudes, mass, mass_density, master, material, maximum_gap, maximum_number_of_iterations, maximum_strain, maximum_time_step, mean, Membrane_Plate_Fiber, mesh, method, minimum_bulk_modulus, minimum_time_step, model, motion, mx, my, mz, name, new, node, nodes, nodes_file, normal_stiffness, number_of_drm_elements, number_of_drm_nodes, number_of_iterations, number_of_modes, number_of_steps, number_of_subincrements, number_of_times_reaching_maximum_strain, of, output, password, path_series, path_time_series, penalty, penalty_for_applying_generalized_displacement, point, points, poisson_ratio, poisson_ratio_h_h, poisson_ratio_h_v, porosity, port, pressure, pressure_reference_p0, print, R0, reduction, reference_void_ratio, remove, response, restore, rho_f, rho_s, rx, ry, rz
, sanisand2004_A0, sanisand2004_c, sanisand2004_ch, sanisand2004_cz, sanisand2004_ec_ref, sanisand2004_G0, sanisand2004_h0, sanisand2004_lambda_c, sanisand2004_m, sanisand2004_Mc, sanisand2004_nb, sanisand2004_nd, sanisand2004_p_cut, sanisand2004_Pat, sanisand2004_xi, sanisand2004_z_max, sanisand2008_A0, sanisand2008_alpha_cc, sanisand2008_c, sanisand2008_ch, sanisand2008_cz, sanisand2008_ec_ref, sanisand2008_G0, sanisand2008_h0, sanisand2008_K0, sanisand2008_k_c, sanisand2008_lambda, sanisand2008_m, sanisand2008_nb, sanisand2008_nd, sanisand2008_p0, sanisand2008_p_in, sanisand2008_p_r, sanisand2008_Pat, sanisand2008_rho_c, sanisand2008_theta_c, sanisand2008_X, sanisand2008_xi, sanisand2008_z_max, save, scale_factor, section, self_weight, series_file, shear, shear_modulus, shear_modulus_h_v, simple, simulate, single, slave, socket, solver, stage, state, static, steps, stiffness, stiffness_to_use, strain, strain_at_compressive_strength, strain_at_crushing_strength, strain_hardening_ratio, strain_increment_size, stress, surface, tangential_stiffness, tensile_strength, tension_softening_stiffness, test, testing, thickness, time_step, to, tolerance_1, tolerance_2, torsion_Jx, triaxial, type, undrained, unit, use, username, using, ux, uy, uz, value, variable_node_brick_8_to_27, velocity_file, velocity_scale_unit, viscoelastic_modulus, von_mises_radius, while, whos
, with, x_max, x_min, xi_in, xz_plane_vector, y_max, y_min, yield_strength, z_max, z_min},
morekeywords={[2] 20NodeBrick, 20NodeBrick_elastic, 20NodeBrick_upU, 27NodeBrick, 27NodeBrick_elastic, 27NodeBrickLT, 3NodeShell_ANDES, 4NodeShell_ANDES, 4NodeShell_MITC4, 4NodeShell_NewMITC4, 8NodeBrick, 8NodeBrick_elastic, 8NodeBrick_up, 8NodeBrick_upU, 8NodeBrickLT, beam_9dof_elastic, beam_displacement_based, beam_elastic, beam_elastic_lumped_mass, camclay, camclay_accelerated, Caughey3rd, Caughey4th, contact, DOFTYPE, druckerprager_isotropic_hardening, druckerprager_isotropic_hardening_accelerated, druckerprager_kinematic_hardening, druckerprager_kinematic_hardening_accelerated, druckerprager_perfectly_plastic, druckerprager_perfectly_plastic_accelerated, Energy_Increment, F_fluid_x, F_fluid_y, F_fluid_z, FORCETYPE, from_reactions, Fx, Fy, Fz, Hilber_Hughes_Taylor, linear, linear_elastic_crossanisotropic, linear_elastic_isotropic_3d, Modified_Newton, Mx, My, mysql, Mz, New_PisanoLT, Newmark, Newton, Norm_Displacement_Increment, Norm_Unbalance, path_series, path_time_series, pisano, pisanoLT, ProfileSPD, Rayleigh, sanisand2004, sanisand2008, static, transient, truss, UMFPack, uniaxial_concrete02, uniaxial_elastic, uniaxial_steel01, uniaxial_steel02, variable transient, vonmises_isotropic_hardening, vonmises_isotropic_hardening_accelerated, vonmises_kinematic_hardening, vonmises_kinematic_hardening_accelerated, vonmises_linear_kinematic_hardening, vonmises_linear_kinematic_hardening_accelerated, vonmises_perfectly_plastic, vonmises_perfectly_plastic_accelerated, vonmises_perfectly_plastic_LT, With_no_convergence_check,
elastic_modulus_1atm,m_in,n_in,a_in,eplcum_cr_in},
sensitive=true,
morecomment=[l]{//},
morestring=[b]",
}

\definecolor{dkgreen}{rgb}{0,0.6,0}
\definecolor{gray}{rgb}{0.5,0.5,0.5}
\definecolor{mauve}{rgb}{0.58,0,0.82}
\definecolor{ltgray}{rgb}{0.98,0.98,0.98}
\definecolor{grayish}{rgb}{0.85, 0.85, 0.85}
\lstset{language=fei, 
  frame=single,
  backgroundcolor=\color{ltgray},
  aboveskip=3mm,
  belowskip=3mm,
  showstringspaces=false,
  columns=flexible,
  basicstyle={\small\ttfamily},
  numberstyle=\tiny\color{black},
  keywordstyle=\color{blue},
  commentstyle=\color{gray},
  stringstyle=\color{mauve},
  breaklines=true,
  numbers=left,
  breakatwhitespace=true,
  tabsize=3,
  %rulecolor=\color{mauve},
}


%%%%%%%%%%%%%%%%%%%%%%%%%%%%%%%%%

\pagestyle{fancy}

\def\today
{\number\day.\space \ifcase\month\or
January\or
February\or
March\or
April\or
May\or
June\or
July\or
August\or
September\or
October\or
November\or
December\fi,\space
\number\year}

%%%%
\newcount\hh
\newcount\mm
\mm=\time
\hh=\time
\divide\hh by 60
\divide\mm by 60
\multiply\mm by 60
\mm=-\mm
\advance\mm by \time
\def\hhmm{\number\hh:\ifnum\mm<10{}0\fi\number\mm}

%%%%%
\lhead{\small \it Verification for Real ESSI Simulator}
\chead{\small \it }
\rhead{\small \it \thepage{} of \pageref{LastPage} }
%
\lfoot{\small \it UC Davis}
\rfoot{\small \it \today, \hhmm}
\cfoot{\small \it Draft}
\addtolength{\headheight}{14pt}


\newcommand{\tabincell}[2]{\begin{tabular}{@{}#1@{}}#2\end{tabular}}





\begin{document}

%%%%%%%%%%%%%%%%%%%%Start Here%%%%%%%%%%%%%%%%%%%%%%%
%%%%%%%%%%%%%%%%%%%%Start Here%%%%%%%%%%%%%%%%%%%%%%%
%%%%%%%%%%%%%%%%%%%%Start Here%%%%%%%%%%%%%%%%%%%%%%%
%-------------------------------------------------------------------------------------------------------------%
%-------------------------------------------------------------------------------------------------------------%

\thispagestyle{fancy}


\tableofcontents{}





\newpage


\newpage
 \section{Elastic Beam Element Under Static Loading} 
 
 
 %%%%%%%%%%%%%%%%%%%%%%%%%%%%%%%%%%%%%%%%%%%%%%%%%%%%%%%%%%%%%%%%%%%%%%%%%%%%%%%%
 \subsection{Simulate the cantilever by 1 elastic beam element} 
 
 
 %%%%%%%%%%%%%%%%%%%%%%%%%%%%%%%%%%%%%%%%%%%%%%%%%%%%%%%%%%%%%%%%%%%%%%%%%%%%%%%%
 % \paragraph{Problem description:} 
 % 
 % \begin{itemize}
 %   \item Structure dimensions
 % 
 %     Structure Length=6m, width=1m, height=1m.
 % 
 %   \item Element dimensions
 % 
 %     Element length=1m, width=1m, height=1m.
 % \end{itemize}
 
This is a simple beam example under static loading in three directions. The diagram below shows the loading in one bending direction.

 \begin{figure}[!htb]
   \centering
   \includegraphics[width=8cm]{../Figure-files/_Chapter_Appendix_Illustrative_Examples/cantilever.pdf}
   \caption{The cantilever model}
   \label{fig-canti_1beam-static}
 \end{figure}
 
 
 \paragraph{ESSI model fei file: }
 
 \begin{lstlisting}
model name "beam_1element" ;
// define the node coordinates
add node #  1 at (   0.0*m ,    0.0*m,     0.0*m)  with 6 dofs;
add node #  2 at (   1.0*m ,    0.0*m,     0.0*m)  with 6 dofs;
// Geometry: width and height. Help the beam definition.
b=0.2*m;
h=0.2*m;
I=b*h^3/12.0;
// define the beam element
add element # 1 type beam_elastic with nodes (1,2) 
  cross_section =   b*h 
  elastic_modulus =  1e9*N/m^2
  shear_modulus =  5e8*N/m^2
  torsion_Jx =  0.33*b*h^3
  bending_Iy = I
  bending_Iz = I
  mass_density = 0*kg/m^3
  xz_plane_vector = ( 1, 0, 1) 
  joint_1_offset = (0*m, 0*m, 0*m) 
  joint_2_offset = (0*m, 0*m, 0*m);
// add boundary condition
fix node #      1 dofs all;
// axial loading
new loading stage "axial";
add load # 1 to node # 2 type linear Fx = 1*N;
define load factor increment 1;
define algorithm With_no_convergence_check ;
define solver ProfileSPD;
simulate 1 steps using static algorithm;
// bending in one direction
new loading stage "bending1";
remove load # 1;
add load # 2 to node # 2 type linear Fy = 1*N;
define load factor increment 1;
define algorithm With_no_convergence_check ;
define solver ProfileSPD;
simulate 1 steps using static algorithm;
// bending in the other direction
new loading stage "bending2";
remove load # 2;
add load # 3 to node # 2 type linear Fz = 1*N;
define load factor increment 1;
define algorithm With_no_convergence_check ;
define solver ProfileSPD;
simulate 1 steps using static algorithm;
  
bye;
\end{lstlisting}
 
 
The    ESSI   model   fei   files   for   this   example   can   be   downloaded
 \href{https://github.com/BorisJeremic/Real-ESSI-Examples/blob/master/model_fei_file/Simple_beam_static_cantilever/Simple_beam_static_cantilever.tgz?raw=true}{here}.
 
 
 
 
 
 
 
%  
%  
%  %%%%%%%%%%%%%%%%%%%%%%%%%%%%%%%%%%%%%%%%%%%%%%%%%%%%%%%%%%%%%%%%%%%%%%%%%%%%%%%%
%%%%%%%%%%%%%%%%%%%%%%%%%%%%%%%%%%%%%%%%%%%%%%%%%%%%%%%%%%%%%%%%%%%%%%%%%%%%%%%%
%%%%%%%%%%%%%%%%%%%%%%%%%%%%%%%%%%%%%%%%%%%%%%%%%%%%%%%%%%%%%%%%%%%%%%%%%%%%%%%%
%%%%%%%%%%%%%%%%%%%%%%%%%%%%%%%%%%%%%%%%%%%%%%%%%%%%%%%%%%%%%%%%%%%%%%%%%%%%%%%%
%%%%%%%%%%%%%%%%%%%%%%%%%%%%%%%%%%%%%%%%%%%%%%%%%%%%%%%%%%%%%%%%%%%%%%%%%%%%%%%%
%%%%%%%%%%%%%%%%%%%%%%%%%%%%%%%%%%%%%%%%%%%%%%%%%%%%%%%%%%%%%%%%%%%%
%%%%%%%%%%%%%%%%%%%%%%%%%%%%%%%%%%%%%%%%%%%%%%%%%%%%%%%%%%%%%%%%%%%%

\newpage
\section{Elastic Beam Element under Dynamic Loading} ~ 


%%%%%%%%%%%%%%%%%%%%%%%%%%%%%%%%%%%%%%%%%%%%%%%%%%%%%%%%%%%%%%%%%%%%%%%%%%%%%%%%
\subsection{Cantilever, One Elastic Beam Element} 


%%%%%%%%%%%%%%%%%%%%%%%%%%%%%%%%%%%%%%%%%%%%%%%%%%%%%%%%%%%%%%%%%%%%%%%%%%%%%%%%
\paragraph{Problem description:}

% \begin{itemize}
%   \item Structure dimensions% 
% 
%     Structure Length=1m, width=0.2m, height=0.2m.
% 
%   \item Element dimensions
% 
%     Element length=1m, width=0.2m, height=0.2m.
% \end{itemize}

\begin{figure}[!htb]
  \centering
  \includegraphics[width=8cm]{../Figure-files/_Chapter_Appendix_Illustrative_Examples/cantilever.pdf}
  \caption{The cantilever model.}
  \label{fig-canti_1beam}
\end{figure}


\paragraph{ESSI model fei file: } ~

\begin{lstlisting}
model name "beam_1element" ;

// add node
add node #  1 at (   0.0*m ,    0.0*m,     0.0*m)  with 6 dofs;
add node #  2 at (   1.0*m ,    0.0*m,     0.0*m)  with 6 dofs;
  // Geometry: width and height
b=0.2*m;
h=0.2*m;
// Materials: properties
natural_period    = 1*s;        
natural_frequency  = 2*pi/natural_period;
elastic_constant  = 1e9*N/m^2; 
I=b*h^3/12.0;
A=b*h;
L=1*m;
rho   = (1.8751)^4*elastic_constant*I/(natural_frequency^2*L^4*A);
possion_ratio=0.3;
// add elements
add element # 1 type beam_elastic with nodes (1,2) 
  cross_section =   b*h 
  elastic_modulus =  elastic_constant
  shear_modulus =  elastic_constant/2/(1+possion_ratio)
  torsion_Jx =  0.33*b*h^3
  bending_Iy =  b*h^3/12
  bending_Iz =  b*h^3/12
  mass_density = rho
  xz_plane_vector = ( 1, 0, 1) 
  joint_1_offset = (0*m, 0*m, 0*m) 
  joint_2_offset = (0*m, 0*m, 0*m);

// add boundary condition
fix node #      1 dofs all;

// // ----------------------------------------------------------------------------
// // --slowLoading---------------------------------------------------------------
// // add load in 180 seconds. (Slow)
// // ----------------------------------------------------------------------------
// new loading stage "slowLoading";
// add load # 1 to node # 2 type path_time_series 
//  Fz =  1.*N
//  series_file = "slowLoading.txt" ;
// define dynamic integrator Newmark with gamma = 0.5 beta = 0.25;
// define algorithm With_no_convergence_check ;
// define solver ProfileSPD;
// simulate 2000 steps using transient algorithm 
//  time_step = 0.1*s;

// // ----------------------------------------------------------------------------
// // --fastLoading---------------------------------------------------------------
// // add load in 0.6 seconds (Fast)
// // ----------------------------------------------------------------------------
// remove load # 1;
// new loading stage "fastLoading";
// add load # 2 to node # 2 type path_time_series 
//  Fz =  1.*N
//  series_file = "fastLoading.txt" ;
// define dynamic integrator Newmark with gamma = 0.5 beta = 0.25;
// define algorithm With_no_convergence_check ;
// define solver ProfileSPD;
// simulate 1000 steps using transient algorithm 
//  time_step = 0.01*s;

// // ----------------------------------------------------------------------------
// // --freeVibration-------------------------------------------------------------
// // add a load and then release to free vibration
// // ----------------------------------------------------------------------------
// remove load # 2;
new loading stage "freeVibration";
add load # 3 to node # 2 type path_time_series 
  Fz =  1.*N
  series_file = "freeVibration.txt" ;
define dynamic integrator Newmark with gamma = 0.5 beta = 0.25;
define algorithm With_no_convergence_check ;
define solver ProfileSPD;
simulate 2000 steps using transient algorithm 
  time_step = 0.01*s;

bye;
\end{lstlisting}


%%%%%%%%%%%%%%%%%%%%%%%%%%%%%%%%%%%%%%%%%%%%%%%%%%%%%%%%%%%%%%%%%%%%%%%%%%%%%%%%
\paragraph{Displacement Results} 

\begin{figure}[!htb]
  \centering
  \includegraphics[width=12cm]{../Figure-files/_Chapter_Appendix_Illustrative_Examples/beam-1element-slowLoading.pdf}
  \caption{Slow loading condition, vertical displacements or the cantilever tip.}
  \label{fig_beam1_slow}
\end{figure}


\begin{figure}[!htb]
  \centering
  \includegraphics[width=12cm]{../Figure-files/_Chapter_Appendix_Illustrative_Examples/beam-1element-fastLoading.pdf}
  \caption{Fast loading condition, vertical displacements of the cantilever tip.}
  \label{fig_beam1_fast}
\end{figure}

\begin{figure}[!htb]
  \centering
  \includegraphics[width=12cm]{../Figure-files/_Chapter_Appendix_Illustrative_Examples/beam-1element-freeVibration.pdf}
  \caption{Free vibration, vertical displacements of the cantilever tip.}
  \label{fig_beam1_freevib}
\end{figure}


The    ESSI   model   fei   files   for   this   example   can   be   downloaded
\href{https://github.com/BorisJeremic/Real-ESSI-Examples/blob/master/model_fei_file/beam_elastic_1element_dynamic/beam_elastic_1element_dynamic.tgz?raw=true}{here}.






















%%%%%%%%%%%%%%%%%%%%%%%%%%%%%%%%%%%%%%%%%%%%%%%%%%%%%%%%%%%%%%%%%%%%%%%%%%%%%%%%
%%%%%%%%%%%%%%%%%%%%%%%%%%%%%%%%%%%%%%%%%%%%%%%%%%%%%%%%%%%%%%%%%%%%%%%%%%%%%%%%
%%%%%%%%%%%%%%%%%%%%%%%%%%%%%%%%%%%%%%%%%%%%%%%%%%%%%%%%%%%%%%%%%%%%%%%%%%%%%%%%
%%%%%%%%%%%%%%%%%%%%%%%%%%%%%%%%%%%%%%%%%%%%%%%%%%%%%%%%%%%%%%%%%%%%%%%%%%%%%%%%
%%%%%%%%%%%%%%%%%%%%%%%%%%%%%%%%%%%%%%%%%%%%%%%%%%%%%%%%%%%%%%%%%%%%%%%%%%%%%%%%
%%%%%%%%%%%%%%%%%%%%%%%%%%%%%%%%%%%%%%%%%%%%%%%%%%%%%%%%%%%%%%%%%%%%
%%%%%%%%%%%%%%%%%%%%%%%%%%%%%%%%%%%%%%%%%%%%%%%%%%%%%%%%%%%%%%%%%%%%

\newpage
\subsection{Cantilever, 5 Elastic Beam Elements} 

\paragraph{Problem description:}

% \begin{itemize}
%   \item Structure dimensions
% 
%     Structure Length=1m, width=0.2m, height=0.2m.
% 
%   \item Element dimensions
% 
%     Element length=0.2m, width=0.2m, height=0.2m.
% \end{itemize}

\begin{figure}[!htb]
  \centering
  \includegraphics[width=8cm]{../Figure-files/_Chapter_Appendix_Illustrative_Examples/cantilever.pdf}
  \caption{The cantilever model.}
  \label{fig_cantilever_m5}
\end{figure}


\paragraph{ESSI model fei file: } ~

\begin{lstlisting}
model name "beam_5element" ;

// add node
add node #  1 at (   0.0*m ,    0.0*m,     0.0*m)  with 6 dofs;
add node #  2 at (   0.2*m ,    0.0*m,     0.0*m)  with 6 dofs;
add node #  3 at (   0.4*m ,    0.0*m,     0.0*m)  with 6 dofs;
add node #  4 at (   0.6*m ,    0.0*m,     0.0*m)  with 6 dofs;
add node #  5 at (   0.8*m ,    0.0*m,     0.0*m)  with 6 dofs;
add node #  6 at (   1.0*m ,    0.0*m,     0.0*m)  with 6 dofs;
  
// Geometry: width and height
b=0.2*m;
h=0.2*m;

// Materials: properties
natural_period    = 1*s;        
natural_frequency  = 2*pi/natural_period;
elastic_constant  = 1e9*N/m^2; 
I=b*h^3/12.0;
A=b*h;
L=1*m;
rho   = (1.8751)^4*elastic_constant*I/(natural_frequency^2*L^4*A);
possion_ratio=0.3;

// Cross section geometry: width and height
b=0.2*m;
h=0.2*m;

// add elements
ii=1;
while (ii<6) {
  add element # ii type beam_elastic with nodes (ii,ii+1) 
    cross_section =   b*h 
    elastic_modulus =  elastic_constant
    shear_modulus =  elastic_constant/2/(1+possion_ratio)
    torsion_Jx =  0.33*b*h^3
    bending_Iy =  b*h^3/12
    bending_Iz =  b*h^3/12
    mass_density = rho
    xz_plane_vector = ( 1, 0, 1) 
    joint_1_offset = (0*m, 0*m, 0*m) 
    joint_2_offset = (0*m, 0*m, 0*m);
  ii+=1;
}

// add boundary condition
fix node #      1 dofs all;

// // ----------------------------------------------------------------------------
// // --slowLoading---------------------------------------------------------------
// // add load in 180 seconds. 
// // ----------------------------------------------------------------------------
// new loading stage "slowLoading";
// add load # 1 to node # 6 type path_time_series 
//  Fz =  1.*N
//  series_file = "slowLoading.txt" ;
// define dynamic integrator Newmark with gamma = 0.5 beta = 0.25;
// define algorithm With_no_convergence_check ;
// define solver ProfileSPD;
// simulate 2000 steps using transient algorithm 
//  time_step = 0.1*s;

// // ----------------------------------------------------------------------------
// // --fastLoading---------------------------------------------------------------
// // add load in 0.6 seconds.
// // ----------------------------------------------------------------------------
// remove load # 1;
// new loading stage "fastLoading";
// add load # 2 to node # 6 type path_time_series 
//  Fz =  1.*N
//  series_file = "fastLoading.txt" ;
// define dynamic integrator Newmark with gamma = 0.5 beta = 0.25;
// define algorithm With_no_convergence_check ;
// define solver ProfileSPD;
// simulate 1000 steps using transient algorithm 
//  time_step = 0.01*s;

// // ----------------------------------------------------------------------------
// // --freeVibration-------------------------------------------------------------
// // add a load and then release for free vibration
// // ----------------------------------------------------------------------------
// remove load # 2;
new loading stage "freeVibration";
add load # 3 to node # 6 type path_time_series 
  Fz =  1.*N
  series_file = "freeVibration.txt" ;
define dynamic integrator Newmark with gamma = 0.5 beta = 0.25;
define algorithm With_no_convergence_check ;
define solver ProfileSPD;
simulate 100 steps using transient algorithm 
  time_step = 0.1*s;

bye;
\end{lstlisting}

%%%%%%%%%%%%%%%%%%%%%%%%%%%%%%%%%%%%%%%%%%%%%%%%%%%%%%%%%%%%%%%%%%%%%%%%%%%%%%%%
\paragraph{Displacement results} 

\begin{figure}[!htb]
  \centering
  \includegraphics[width=12cm]{../Figure-files/_Chapter_Appendix_Illustrative_Examples/beam-5element-slowLoading.pdf}
  \caption{Slow loading condition, vertical displacements of the cantilever tip.}
  \label{fig_beam5_slow}
\end{figure}


\begin{figure}[!htb]
  \centering
  \includegraphics[width=12cm]{../Figure-files/_Chapter_Appendix_Illustrative_Examples/beam-5element-fastLoading.pdf}
  \caption{Fast loading condition, vertical displacements of the cantilever tip.}
  \label{fig_beam5_fast}
\end{figure}

\begin{figure}[!htb]
  \centering
  \includegraphics[width=12cm]{../Figure-files/_Chapter_Appendix_Illustrative_Examples/beam-5element-freeVibration.pdf}
  \caption{Free vibration condition, vertical displacements of the cantilever tip.}
  \label{fig_beam5_freevibration}
\end{figure}



The    ESSI   model   fei   files   for   this   example   can   be   downloaded
\href{https://github.com/BorisJeremic/Real-ESSI-Examples/blob/master/model_fei_file/beam_elastic_5element_dynamic/beam_elastic_5element_dynamic.tgz?raw=true}{here}.






















%%%%%%%%%%%%%%%%%%%%%%%%%%%%%%%%%%%%%%%%%%%%%%%%%%%%%%%%%%%%%%%%%%%%%%%%%%%%%%%%
%%%%%%%%%%%%%%%%%%%%%%%%%%%%%%%%%%%%%%%%%%%%%%%%%%%%%%%%%%%%%%%%%%%%%%%%%%%%%%%%
%%%%%%%%%%%%%%%%%%%%%%%%%%%%%%%%%%%%%%%%%%%%%%%%%%%%%%%%%%%%%%%%%%%%%%%%%%%%%%%%
%%%%%%%%%%%%%%%%%%%%%%%%%%%%%%%%%%%%%%%%%%%%%%%%%%%%%%%%%%%%%%%%%%%%%%%%%%%%%%%%
%%%%%%%%%%%%%%%%%%%%%%%%%%%%%%%%%%%%%%%%%%%%%%%%%%%%%%%%%%%%%%%%%%%%%%%%%%%%%%%%
%%%%%%%%%%%%%%%%%%%%%%%%%%%%%%%%%%%%%%%%%%%%%%%%%%%%%%%%%%%%%%%%%%%%
%%%%%%%%%%%%%%%%%%%%%%%%%%%%%%%%%%%%%%%%%%%%%%%%%%%%%%%%%%%%%%%%%%%%

\newpage
\section{Cantilever, one 27 Node Brick Element, Dynamic Loading} 



%%%%%%%%%%%%%%%%%%%%%%%%%%%%%%%%%%%%%%%%%%%%%%%%%%%%%%%%%%%%%%%%%%%%%%%%%%%%%%%%
\subsection{Simulate the cantilever using one 27NodeBrick element} 





%%%%%%%%%%%%%%%%%%%%%%%%%%%%%%%%%%%%%%%%%%%%%%%%%%%%%%%%%%%%%%%%%%%%%%%%%%%%%%%%
\paragraph{Problem description:} 

% \begin{itemize}
%   \item Structure dimensions
% 
%     Structure Length=1m, width=0.2m, height=0.2m.
% 
%   \item Element dimensions
% 
%     Element length=1m, width=0.2m, height=0.2m.
% \end{itemize}
% 
\begin{figure}[!htb]
  \centering
  \includegraphics[width=8cm]{../Figure-files/_Chapter_Appendix_Illustrative_Examples/cantilever.pdf}
  \caption{The cantilever model.}
  \label{fig_cantilev_1}
\end{figure}


\paragraph{ESSI model fei file: } ~

\begin{lstlisting}
model name "brick_1element" ;

// Geometry: width and height
b=0.2*m;
h=0.2*m;

// Materials: properties
natural_period    = 1*s;        
natural_frequency  = 2*pi/natural_period;
elastic_constant  = 1e9*N/m^2; 
I=b*h^3/12.0;
A=b*h;
L=1*m;
rho   = (1.8751)^4*elastic_constant*I/(natural_frequency^2*L^4*A);
possion_ratio=0.3;


add material # 1 type linear_elastic_isotropic_3d_LT
  mass_density = rho
  elastic_modulus = elastic_constant
  poisson_ratio = possion_ratio;

add node #        1 at (   0.0000 *m,   0.2000 *m,  0.0000 *m) with 3 dofs;
add node #        2 at (   0.0000 *m,   0.0000 *m,  0.0000 *m) with 3 dofs;
add node #        3 at (   1.0000 *m,   0.2000 *m,  0.0000 *m) with 3 dofs;
add node #        4 at (   1.0000 *m,   0.0000 *m,  0.0000 *m) with 3 dofs;
add node #        5 at (   0.0000 *m,   0.0000 *m,  0.2000 *m) with 3 dofs;
add node #        6 at (   1.0000 *m,   0.0000 *m,  0.2000 *m) with 3 dofs;
add node #        7 at (   1.0000 *m,   0.2000 *m,  0.2000 *m) with 3 dofs;
add node #        8 at (   0.0000 *m,   0.2000 *m,  0.2000 *m) with 3 dofs;
add node #        9 at (   0.0000 *m,   0.1000 *m,  0.0000 *m) with 3 dofs;
add node #       10 at (   0.5000 *m,   0.2000 *m,  0.0000 *m) with 3 dofs;
add node #       11 at (   1.0000 *m,   0.1000 *m,  0.0000 *m) with 3 dofs;
add node #       12 at (   0.5000 *m,   0.0000 *m,  0.0000 *m) with 3 dofs;
add node #       13 at (   0.0000 *m,   0.1000 *m,  0.2000 *m) with 3 dofs;
add node #       14 at (   0.5000 *m,   0.2000 *m,  0.2000 *m) with 3 dofs;
add node #       15 at (   1.0000 *m,   0.1000 *m,  0.2000 *m) with 3 dofs;
add node #       16 at (   0.5000 *m,   0.0000 *m,  0.2000 *m) with 3 dofs;
add node #       17 at (   0.0000 *m,   0.0000 *m,  0.1000 *m) with 3 dofs;
add node #       18 at (   0.0000 *m,   0.2000 *m,  0.1000 *m) with 3 dofs;
add node #       19 at (   1.0000 *m,   0.2000 *m,  0.1000 *m) with 3 dofs;
add node #       20 at (   1.0000 *m,   0.0000 *m,  0.1000 *m) with 3 dofs;
add node #       21 at (   0.5000 *m,   0.1000 *m,  0.1000 *m) with 3 dofs;
add node #       22 at (   0.0000 *m,   0.1000 *m,  0.1000 *m) with 3 dofs;
add node #       23 at (   0.5000 *m,   0.2000 *m,  0.1000 *m) with 3 dofs;
add node #       24 at (   1.0000 *m,   0.1000 *m,  0.1000 *m) with 3 dofs;
add node #       25 at (   0.5000 *m,   0.0000 *m,  0.1000 *m) with 3 dofs;
add node #       26 at (   0.5000 *m,   0.1000 *m,  0.0000 *m) with 3 dofs;
add node #       27 at (   0.5000 *m,   0.1000 *m,  0.2000 *m) with 3 dofs;

add element #         1 type 27NodeBrickLT with nodes(       2,       1,       3,       4,       5,       8,       7,       6,       9,      10,      11,      12,      13,      14,      15,      16,      17,      18,      19,      20,      21,      22,      23,      24,      25,      26,      27) use material #        1; 

fix node # 1 dofs all;
fix node # 2 dofs all;
fix node # 5 dofs all;
fix node # 8 dofs all;
fix node # 9 dofs all;
fix node # 13 dofs all;
fix node # 17 dofs all;
fix node # 18 dofs all;
fix node # 22 dofs all;

  
// // ----------------------------------------------------------------------------
// // --slowLoading---------------------------------------------------------------
// // ----------------------------------------------------------------------------
// new loading stage "slowLoading";
// add load # 1 to node # 4 type path_time_series Fz=1/36.0*N series_file = "slowLoading.txt" ; 
// add load # 2 to node # 6 type path_time_series Fz=1/36.0*N series_file = "slowLoading.txt" ; 
// add load # 3 to node # 3 type path_time_series Fz=1/36.0*N series_file = "slowLoading.txt" ; 
// add load # 4 to node # 7 type path_time_series Fz=1/36.0*N series_file = "slowLoading.txt" ; 
// add load # 5 to node # 20 type path_time_series Fz=1/9.0*N series_file = "slowLoading.txt" ; 
// add load # 6 to node # 11 type path_time_series Fz=1/9.0*N series_file = "slowLoading.txt" ; 
// add load # 7 to node # 15 type path_time_series Fz=1/9.0*N series_file = "slowLoading.txt" ; 
// add load # 8 to node # 19 type path_time_series Fz=1/9.0*N series_file = "slowLoading.txt" ; 
// add load # 9 to node # 24 type path_time_series Fz=4/9.0*N series_file = "slowLoading.txt" ; 
// // add algorithm and solver
// define dynamic integrator Newmark with gamma = 0.5 beta = 0.25;
// define algorithm With_no_convergence_check ;
// define solver ProfileSPD;
// simulate 2000 steps using transient algorithm 
//  time_step = 0.1*s;

// // ----------------------------------------------------------------------------
// // --fastLoading---------------------------------------------------------------
// // ----------------------------------------------------------------------------
// new loading stage "fastLoading";
// add load # 101 to node # 4 type path_time_series Fz=1/36.0*N series_file = "fastLoading.txt" ; 
// add load # 102 to node # 6 type path_time_series Fz=1/36.0*N series_file = "fastLoading.txt" ; 
// add load # 103 to node # 3 type path_time_series Fz=1/36.0*N series_file = "fastLoading.txt" ; 
// add load # 104 to node # 7 type path_time_series Fz=1/36.0*N series_file = "fastLoading.txt" ; 
// add load # 105 to node # 20 type path_time_series Fz=1/9.0*N series_file = "fastLoading.txt" ; 
// add load # 106 to node # 11 type path_time_series Fz=1/9.0*N series_file = "fastLoading.txt" ; 
// add load # 107 to node # 15 type path_time_series Fz=1/9.0*N series_file = "fastLoading.txt" ; 
// add load # 108 to node # 19 type path_time_series Fz=1/9.0*N series_file = "fastLoading.txt" ; 
// add load # 109 to node # 24 type path_time_series Fz=4/9.0*N series_file = "fastLoading.txt" ; 
// // add algorithm and solver
// define dynamic integrator Newmark with gamma = 0.5 beta = 0.25;
// define algorithm With_no_convergence_check ;
// define solver ProfileSPD;
// simulate 1000 steps using transient algorithm 
//  time_step = 0.01*s;

// // ----------------------------------------------------------------------------
// // --freeVibration---------------------------------------------------------------
// // ----------------------------------------------------------------------------
new loading stage "freeVibration";
add load # 201 to node # 4 type path_time_series Fz=1/36.0*N series_file = "freeVibration.txt" ; 
add load # 202 to node # 6 type path_time_series Fz=1/36.0*N series_file = "freeVibration.txt" ; 
add load # 203 to node # 3 type path_time_series Fz=1/36.0*N series_file = "freeVibration.txt" ; 
add load # 204 to node # 7 type path_time_series Fz=1/36.0*N series_file = "freeVibration.txt" ; 
add load # 205 to node # 20 type path_time_series Fz=1/9.0*N series_file = "freeVibration.txt" ; 
add load # 206 to node # 11 type path_time_series Fz=1/9.0*N series_file = "freeVibration.txt" ; 
add load # 207 to node # 15 type path_time_series Fz=1/9.0*N series_file = "freeVibration.txt" ; 
add load # 208 to node # 19 type path_time_series Fz=1/9.0*N series_file = "freeVibration.txt" ; 
add load # 209 to node # 24 type path_time_series Fz=4/9.0*N series_file = "freeVibration.txt" ; 
// add algorithm and solver
define dynamic integrator Newmark with gamma = 0.5 beta = 0.25;
define algorithm With_no_convergence_check ;
define solver ProfileSPD;
simulate 10000 steps using transient algorithm 
  time_step = 0.001*s;

// end
bye;
\end{lstlisting}

\paragraph{Displacement results against time series} ~

\begin{figure}[!htb]
  \centering
  \includegraphics[width=12cm]{../Figure-files/_Chapter_Appendix_Illustrative_Examples/brick-1element-slowLoading.pdf}
  \caption{Slow loading condition, vertical displacements of the cantilever tip.}
  \label{fig_brick1-slow}
\end{figure}


\begin{figure}[!htb]
  \centering
  \includegraphics[width=12cm]{../Figure-files/_Chapter_Appendix_Illustrative_Examples/brick-1element-fastLoading.pdf}
  \caption{Fast loading condition, vertical displacements of the cantilever tip.}
  \label{fig_brick1-fast}
\end{figure}

\begin{figure}[!htb]
  \centering
  \includegraphics[width=12cm]{../Figure-files/_Chapter_Appendix_Illustrative_Examples/brick-1element-freeVibration.pdf}
  \caption{Free vibration condition, vertical displacements of the cantilever tip.}
  \label{fig_brick1-freevib}
\end{figure}


The    ESSI   model   fei   files   for   this   example   can   be   downloaded
\href{https://github.com/BorisJeremic/Real-ESSI-Examples/blob/master/model_fei_file/27NodeBrick_1element_dynamic/27NodeBrick_1element_dynamic.tgz?raw=true}{here}.










%%%%%%%%%%%%%%%%%%%%%%%%%%%%%%%%%%%%%%%%%%%%%%%%%%%%%%%%%%%%%%%%%%%%%%%%%%%%%%%%
%%%%%%%%%%%%%%%%%%%%%%%%%%%%%%%%%%%%%%%%%%%%%%%%%%%%%%%%%%%%%%%%%%%%%%%%%%%%%%%%
%%%%%%%%%%%%%%%%%%%%%%%%%%%%%%%%%%%%%%%%%%%%%%%%%%%%%%%%%%%%%%%%%%%%%%%%%%%%%%%%
%%%%%%%%%%%%%%%%%%%%%%%%%%%%%%%%%%%%%%%%%%%%%%%%%%%%%%%%%%%%%%%%%%%%%%%%%%%%%%%%
%%%%%%%%%%%%%%%%%%%%%%%%%%%%%%%%%%%%%%%%%%%%%%%%%%%%%%%%%%%%%%%%%%%%%%%%%%%%%%%%
%%%%%%%%%%%%%%%%%%%%%%%%%%%%%%%%%%%%%%%%%%%%%%%%%%%%%%%%%%%%%%%%%%%%
%%%%%%%%%%%%%%%%%%%%%%%%%%%%%%%%%%%%%%%%%%%%%%%%%%%%%%%%%%%%%%%%%%%%

\newpage
\subsection{Simulate Cantilever Using five 27 Node Brick Elements} 


%%%%%%%%%%%%%%%%%%%%%%%%%%%%%%%%%%%%%%%%%%%%%%%%%%%%%%%%%%%%%%%%%%%%%%%%%%%%%%%%
\paragraph{Problem description:} ~

% \begin{itemize}
%   \item Structure dimensions
% 
%     Structure Length=1m, width=0.2m, height=0.2m.
% 
%   \item Element dimensions
% 
%     Element length=0.2m, width=0.2m, height=0.2m.
% \end{itemize}

\begin{figure}[!htb]
  \centering
  \includegraphics[width=8cm]{../Figure-files/_Chapter_Appendix_Illustrative_Examples/cantilever.pdf}
  \caption{The cantilever model.}
  % \label{}
\end{figure}


%%%%%%%%%%%%%%%%%%%%%%%%%%%%%%%%%%%%%%%%%%%%%%%%%%%%%%%%%%%%%%%%%%%%%%%%%%%%%%%%
\paragraph{ESSI model fei file: }  ~

\begin{lstlisting}
model name "brick_5element" ;

// Geometry: width and height
b=0.2*m;
h=0.2*m;

// Materials: properties
natural_period    = 1*s;        
natural_frequency  = 2*pi/natural_period;
elastic_constant  = 1e9*N/m^2; 
I=b*h^3/12.0;
A=b*h;
L=1*m;
rho   = (1.8751)^4*elastic_constant*I/(natural_frequency^2*L^4*A);
possion_ratio=0.3;


add material # 1 type linear_elastic_isotropic_3d_LT
  mass_density = rho
  elastic_modulus = elastic_constant
  poisson_ratio = possion_ratio;

add node # 1 at (0.0*m, 0.0*m , 0.0*m) with 3 dofs;
add node # 2 at (0.1*m, 0.0*m , 0.0*m) with 3 dofs;
add node # 3 at (0.2*m, 0.0*m , 0.0*m) with 3 dofs;
add node # 4 at (0.0*m, 0.1*m , 0.0*m) with 3 dofs;
add node # 5 at (0.1*m, 0.1*m , 0.0*m) with 3 dofs;
add node # 6 at (0.2*m, 0.1*m , 0.0*m) with 3 dofs;
add node # 7 at (0.0*m, 0.2*m , 0.0*m) with 3 dofs;
add node # 8 at (0.1*m, 0.2*m , 0.0*m) with 3 dofs;
add node # 9 at (0.2*m, 0.2*m , 0.0*m) with 3 dofs;

fix node No 1 dofs ux uy uz;
fix node No 2 dofs ux uy uz;
fix node No 3 dofs ux uy uz;
fix node No 4 dofs ux uy uz;
fix node No 5 dofs ux uy uz;
fix node No 6 dofs ux uy uz;
fix node No 7 dofs ux uy uz;
fix node No 8 dofs ux uy uz;
fix node No 9 dofs ux uy uz;
e = 0;
hh = 0*m;
NBricks=5;
dz = 0.2*m;
while ( e < NBricks)
{
  hh += dz;
  add node # 10+18*e at (0.0*m, 0.0*m , hh - 0.5*dz) with 3 dofs;
  add node # 11+18*e at (0.1*m, 0.0*m , hh - 0.5*dz) with 3 dofs;
  add node # 12+18*e at (0.2*m, 0.0*m , hh - 0.5*dz) with 3 dofs;
  add node # 13+18*e at (0.0*m, 0.1*m , hh - 0.5*dz) with 3 dofs;
  add node # 14+18*e at (0.1*m, 0.1*m , hh - 0.5*dz) with 3 dofs;
  add node # 15+18*e at (0.2*m, 0.1*m , hh - 0.5*dz) with 3 dofs;
  add node # 16+18*e at (0.0*m, 0.2*m , hh - 0.5*dz) with 3 dofs;
  add node # 17+18*e at (0.1*m, 0.2*m , hh - 0.5*dz) with 3 dofs;
  add node # 18+18*e at (0.2*m, 0.2*m , hh - 0.5*dz) with 3 dofs;
  
  add node # 19+18*e at (0.0*m, 0.0*m , hh) with 3 dofs;
  add node # 20+18*e at (0.1*m, 0.0*m , hh) with 3 dofs;
  add node # 21+18*e at (0.2*m, 0.0*m , hh) with 3 dofs;
  add node # 22+18*e at (0.0*m, 0.1*m , hh) with 3 dofs;
  add node # 23+18*e at (0.1*m, 0.1*m , hh) with 3 dofs;
  add node # 24+18*e at (0.2*m, 0.1*m , hh) with 3 dofs;
  add node # 25+18*e at (0.0*m, 0.2*m , hh) with 3 dofs;
  add node # 26+18*e at (0.1*m, 0.2*m , hh) with 3 dofs;
  add node # 27+18*e at (0.2*m, 0.2*m , hh) with 3 dofs;

  add element # e+1 type 27NodeBrickLT with nodes 
    (
      21+18*e,
      27+18*e,
      25+18*e,
      19+18*e,

       3+18*e,
       9+18*e,
       7+18*e,
       1+18*e,

      24+18*e,
      26+18*e,
      22+18*e,
      20+18*e,

       6+18*e,
       8+18*e,
       4+18*e,
       2+18*e,

      12+18*e,
      18+18*e,
      16+18*e,
      10+18*e,

      14+18*e,
      15+18*e,
      17+18*e,
      13+18*e,
      11+18*e,
      23+18*e,
       5+18*e
    ) 
    use material # 1;

  e += 1;
};


e = e -1;

  
// // ----------------------------------------------------------------------------
// // --slowLoading---------------------------------------------------------------
// // add the 1 Newton load in 180 seconds. 
// // ----------------------------------------------------------------------------
// new loading stage "slowLoading";
// add load # 1 to node # (19+18*e) type path_time_series Fx=1/36.0*N series_file = "slowLoading.txt";
// add load # 2 to node # (20+18*e) type path_time_series Fx=1/9.0*N series_file = "slowLoading.txt";
// add load # 3 to node # (21+18*e) type path_time_series Fx=1/36.0*N series_file = "slowLoading.txt";
// add load # 4 to node # (22+18*e) type path_time_series Fx=1/9.0*N series_file = "slowLoading.txt";
// add load # 5 to node # (23+18*e) type path_time_series Fx=4/9.0*N series_file = "slowLoading.txt";
// add load # 6 to node # (24+18*e) type path_time_series Fx=1/9.0*N series_file = "slowLoading.txt";
// add load # 7 to node # (25+18*e) type path_time_series Fx=1/36.0*N series_file = "slowLoading.txt";
// add load # 8 to node # (26+18*e) type path_time_series Fx=1/9.0*N series_file = "slowLoading.txt";
// add load # 9 to node # (27+18*e) type path_time_series Fx=1/36.0*N series_file = "slowLoading.txt";
// // add algorithm and solver
// define dynamic integrator Newmark with gamma = 0.5 beta = 0.25;
// define algorithm With_no_convergence_check ;
// define solver ProfileSPD;
// simulate 2000 steps using transient algorithm 
//  time_step = 0.1*s;

// // ----------------------------------------------------------------------------
// // --fastLoading---------------------------------------------------------------
// // add the 1 Newton load in 0.6 seconds.
// // ----------------------------------------------------------------------------
// new loading stage "fastLoading";
// add load # 101 to node # (19+18*e) type path_time_series Fx=1/36.0*N series_file = "fastLoading.txt" ; 
// add load # 102 to node # (20+18*e) type path_time_series Fx=1/9.0*N series_file = "fastLoading.txt" ; 
// add load # 103 to node # (21+18*e) type path_time_series Fx=1/36.0*N series_file = "fastLoading.txt" ; 
// add load # 104 to node # (22+18*e) type path_time_series Fx=1/9.0*N series_file = "fastLoading.txt" ; 
// add load # 105 to node # (23+18*e) type path_time_series Fx=4/9.0*N series_file = "fastLoading.txt" ; 
// add load # 106 to node # (24+18*e) type path_time_series Fx=1/9.0*N series_file = "fastLoading.txt" ; 
// add load # 107 to node # (25+18*e) type path_time_series Fx=1/36.0*N series_file = "fastLoading.txt" ; 
// add load # 108 to node # (26+18*e) type path_time_series Fx=1/9.0*N series_file = "fastLoading.txt" ; 
// add load # 109 to node # (27+18*e) type path_time_series Fx=1/36.0*N series_file = "fastLoading.txt" ; 
// // add algorithm and solver
// define dynamic integrator Newmark with gamma = 0.5 beta = 0.25;
// define algorithm With_no_convergence_check ;
// define solver ProfileSPD;
// simulate 1000 steps using transient algorithm 
//  time_step = 0.01*s;

// // ----------------------------------------------------------------------------
// // --freeVibration---------------------------------------------------------------
// // add a load and then release to free vibration
// // ----------------------------------------------------------------------------
new loading stage "freeVibration";
add load # 201 to node # (19+18*e) type path_time_series Fx=1/36.0*N series_file = "freeVibration.txt" ; 
add load # 202 to node # (20+18*e) type path_time_series Fx=1/9.0*N series_file = "freeVibration.txt" ; 
add load # 203 to node # (21+18*e) type path_time_series Fx=1/36.0*N series_file = "freeVibration.txt" ; 
add load # 204 to node # (22+18*e) type path_time_series Fx=1/9.0*N series_file = "freeVibration.txt" ; 
add load # 205 to node # (23+18*e) type path_time_series Fx=4/9.0*N series_file = "freeVibration.txt" ; 
add load # 206 to node # (24+18*e) type path_time_series Fx=1/9.0*N series_file = "freeVibration.txt" ; 
add load # 207 to node # (25+18*e) type path_time_series Fx=1/36.0*N series_file = "freeVibration.txt" ; 
add load # 208 to node # (26+18*e) type path_time_series Fx=1/9.0*N series_file = "freeVibration.txt" ; 
add load # 209 to node # (27+18*e) type path_time_series Fx=1/36.0*N series_file = "freeVibration.txt" ; 
// add algorithm and solver
define dynamic integrator Newmark with gamma = 0.5 beta = 0.25;
define algorithm With_no_convergence_check ;
define solver ProfileSPD;
simulate 100 steps using transient algorithm 
  time_step = 0.1*s;

// end
bye;
\end{lstlisting}

%%%%%%%%%%%%%%%%%%%%%%%%%%%%%%%%%%%%%%%%%%%%%%%%%%%%%%%%%%%%%%%%%%%%%%%%%%%%%%%%
\paragraph{Displacement Results.}

\begin{figure}[!htb]
  \centering
  \includegraphics[width=12cm]{../Figure-files/_Chapter_Appendix_Illustrative_Examples/brick-5element-slowLoading.pdf}
  \caption{Slow loading condition, vertical displacements of the cantilever tip.}
  \label{fig_brick5-slow}
\end{figure}


\begin{figure}[!htb]
  \centering
  \includegraphics[width=12cm]{../Figure-files/_Chapter_Appendix_Illustrative_Examples/brick-5element-fastLoading.pdf}
  \caption{Fast loading condition, vertical displacements of the cantilever tip.}
  \label{fig_brick5-fast}
\end{figure}

\begin{figure}[!htb]
  \centering
  \includegraphics[width=12cm]{../Figure-files/_Chapter_Appendix_Illustrative_Examples/brick-5element-freeVibration.pdf}
  \caption{Free vibration condition, vertical displacements of the cantilever tip.}
  \label{fig_brick5-freevib}
\end{figure}




The    ESSI   model   fei   files   for   this   example   can   be   downloaded
\href{https://github.com/BorisJeremic/Real-ESSI-Examples/blob/master/model_fei_file/27NodeBrick_5element_dynamic/27NodeBrick_5element_dynamic.tgz?raw=true}{here}.

















%%%%%%%%%%%%%%%%%%%%%%%%%%%%%%%%%%%%%%%%%%%%%%%%%%%%%%%%%%%%%%%%%%%%%%%%%%%%%%%%
%%%%%%%%%%%%%%%%%%%%%%%%%%%%%%%%%%%%%%%%%%%%%%%%%%%%%%%%%%%%%%%%%%%%%%%%%%%%%%%%
%%%%%%%%%%%%%%%%%%%%%%%%%%%%%%%%%%%%%%%%%%%%%%%%%%%%%%%%%%%%%%%%%%%%%%%%%%%%%%%%
%%%%%%%%%%%%%%%%%%%%%%%%%%%%%%%%%%%%%%%%%%%%%%%%%%%%%%%%%%%%%%%%%%%%%%%%%%%%%%%%
%%%%%%%%%%%%%%%%%%%%%%%%%%%%%%%%%%%%%%%%%%%%%%%%%%%%%%%%%%%%%%%%%%%%%%%%%%%%%%%%
%%%%%%%%%%%%%%%%%%%%%%%%%%%%%%%%%%%%%%%%%%%%%%%%%%%%%%%%%%%%%%%%%%%%
%%%%%%%%%%%%%%%%%%%%%%%%%%%%%%%%%%%%%%%%%%%%%%%%%%%%%%%%%%%%%%%%%%%%

\newpage
\section{Elastic Beam Element under Dynamic Loading with concentrated mass} ~ 
%%%%%%%%%%%%%%%%%%%%%%%%%%%%%%%%%%%%%%%%%%%%%%%%%%%%%%%%%%%%%%%%%%%%%%%%%%%%%%%%
\subsection{Simulate Cantilever Using 1 Elastic Beam Element} 
%%%%%%%%%%%%%%%%%%%%%%%%%%%%%%%%%%%%%%%%%%%%%%%%%%%%%%%%%%%%%%%%%%%%%%%%%%%%%%%%
\paragraph{Problem description:} ~
% 
% \begin{itemize}
%   \item Structure dimensions
% 
%     Structure Length=1m, width=0.2m, height=0.2m.
% 
%   \item Element dimensions
% 
%     Element length=1m, width=0.2m, height=0.2m.
% \end{itemize}

\begin{figure}[!htb]
  \centering
  \includegraphics[width=8cm]{../Figure-files/_Chapter_Appendix_Illustrative_Examples/cantilever-mass.pdf}
  \caption{The cantilever-mass model.}
  \label{fig_cantilever_mass_1}
\end{figure}


%%%%%%%%%%%%%%%%%%%%%%%%%%%%%%%%%%%%%%%%%%%%%%%%%%%%%%%%%%%%%%%%%%%%%%%%%%%%%%%%
\paragraph{ESSI model fei file: } ~

\begin{lstlisting}
model name "beam-mass_1element" ;

// add node
add node #  1 at (   0.0*m ,    0.0*m,     0.0*m)  with 6 dofs;
add node #  2 at (   1.0*m ,    0.0*m,     0.0*m)  with 6 dofs;
  
// Geometry: width and height
b=0.2*m;
h=0.2*m;

// Materials: properties
natural_period    = 1*s;        
natural_frequency  = 2*pi/natural_period;
elastic_constant  = 1e9*N/m^2; 
I=b*h^3/12.0;
A=b*h;
L=1*m;
rho   = (1.8751)^4*elastic_constant*I/(natural_frequency^2*L^4*A);
possion_ratio=0.3;

// add elements
add element # 1 type beam_elastic with nodes (1,2) 
  cross_section =   b*h 
  elastic_modulus =  elastic_constant
  shear_modulus =  elastic_constant/2/(1+possion_ratio)
  torsion_Jx =  0.33*b*h^3
  bending_Iy =  b*h^3/12
  bending_Iz =  b*h^3/12
  mass_density = rho
  xz_plane_vector = ( 1, 0, 1) 
  joint_1_offset = (0*m, 0*m, 0*m) 
  joint_2_offset = (0*m, 0*m, 0*m);

// add boundary condition
fix node #      1 dofs all;

// add mass
beamMass=rho*A*L;
add mass to node # 2 
  mx = beamMass
  my = beamMass
  mz = beamMass
  Imx = 0*beamMass*L^2
  Imy = 0*beamMass*L^2
  Imz = 0*beamMass*L^2;

// // ----------------------------------------------------------------------------
// // --slowLoading---------------------------------------------------------------
// // ----------------------------------------------------------------------------
// new loading stage "slowLoading";
// add load # 1 to node # 2 type path_time_series 
//  Fz =  1.*N
//  series_file = "slowLoading.txt" ;
// define dynamic integrator Newmark with gamma = 0.5 beta = 0.25;
// define algorithm With_no_convergence_check ;
// define solver ProfileSPD;
// simulate 2000 steps using transient algorithm 
//  time_step = 0.1*s;

// // ----------------------------------------------------------------------------
// // --fastLoading---------------------------------------------------------------
// // ----------------------------------------------------------------------------
// remove load # 1;
// new loading stage "fastLoading";
// add load # 2 to node # 2 type path_time_series 
//  Fz =  1.*N
//  series_file = "fastLoading.txt" ;
// define dynamic integrator Newmark with gamma = 0.5 beta = 0.25;
// define algorithm With_no_convergence_check ;
// define solver ProfileSPD;
// simulate 1000 steps using transient algorithm 
//  time_step = 0.01*s;

// // ----------------------------------------------------------------------------
// // --freeVibration-------------------------------------------------------------
// // ----------------------------------------------------------------------------
// remove load # 2;
new loading stage "freeVibration";
add load # 3 to node # 2 type path_time_series 
  Fz =  1.*N
  series_file = "freeVibration.txt" ;
define dynamic integrator Newmark with gamma = 0.5 beta = 0.25;
define algorithm With_no_convergence_check ;
define solver ProfileSPD;
simulate 1000 steps using transient algorithm 
  time_step = 0.01*s;

bye;
\end{lstlisting}

\paragraph{Displacement results against time series} ~

\begin{figure}[!htb]
  \centering
  \includegraphics[width=12cm]{../Figure-files/_Chapter_Appendix_Illustrative_Examples/beam-mass-1element-slowLoading.pdf}
  \caption{Slow loading condition, vertical displacements of the cantilever tip.}
  \label{fig_beam-mass-slow}
\end{figure}


\begin{figure}[!htb]
  \centering
  \includegraphics[width=12cm]{../Figure-files/_Chapter_Appendix_Illustrative_Examples/beam-mass-1element-fastLoading.pdf}
  \caption{Fast loading condition, vertical displacements of the cantilever tip.}
  \label{fig_beam-mass-fast}
\end{figure}

\begin{figure}[!htb]
  \centering
  \includegraphics[width=12cm]{../Figure-files/_Chapter_Appendix_Illustrative_Examples/beam-mass-1element-freeVibration.pdf}
  \caption{Free vibration condition, vertical displacements of the cantilever tip.}
  \label{fig_beam-mass-freevibr}
\end{figure}


The    ESSI   model   fei   files   for   this   example   can   be   downloaded
\href{https://github.com/BorisJeremic/Real-ESSI-Examples/blob/master/model_fei_file/beam_elastic_mass_dynamic/beam_elastic_mass_dynamic.tgz?raw=true}{here}.









%%%%%%%%%%%%%%%%%%%%%%%%%%%%%%%%%%%%%%%%%%%%%%%%%%%%%%%%%%%%%%%%%%%%%%%%%%%%%%%%
%%%%%%%%%%%%%%%%%%%%%%%%%%%%%%%%%%%%%%%%%%%%%%%%%%%%%%%%%%%%%%%%%%%%%%%%%%%%%%%%
%%%%%%%%%%%%%%%%%%%%%%%%%%%%%%%%%%%%%%%%%%%%%%%%%%%%%%%%%%%%%%%%%%%%%%%%%%%%%%%%
%%%%%%%%%%%%%%%%%%%%%%%%%%%%%%%%%%%%%%%%%%%%%%%%%%%%%%%%%%%%%%%%%%%%%%%%%%%%%%%%
%%%%%%%%%%%%%%%%%%%%%%%%%%%%%%%%%%%%%%%%%%%%%%%%%%%%%%%%%%%%%%%%%%%%%%%%%%%%%%%%
%%%%%%%%%%%%%%%%%%%%%%%%%%%%%%%%%%%%%%%%%%%%%%%%%%%%%%%%%%%%%%%%%%%%
%%%%%%%%%%%%%%%%%%%%%%%%%%%%%%%%%%%%%%%%%%%%%%%%%%%%%%%%%%%%%%%%%%%%

\newpage
\section{Elastic Beam, 27 Node Brick Model With Concentrated Mass} 
%%%%%%%%%%%%%%%%%%%%%%%%%%%%%%%%%%%%%%%%%%%%%%%%%%%%%%%%%%%%%%%%%%%%%%%%%%%%%%%%
\subsection{Simulate the cantilever by one 27NodeBrick element} 
%%%%%%%%%%%%%%%%%%%%%%%%%%%%%%%%%%%%%%%%%%%%%%%%%%%%%%%%%%%%%%%%%%%%%%%%%%%%%%%%
\paragraph{Problem description:} ~

% \begin{itemize}
%   \item Structure dimensions
% 
%     Structure Length=1m, width=0.2m, height=0.2m.
% 
%   \item Element dimensions
% 
%     Element length=1m, width=0.2m, height=0.2m.
% \end{itemize}

\begin{figure}[!htb]
  \centering
  \includegraphics[width=8cm]{../Figure-files/_Chapter_Appendix_Illustrative_Examples/cantilever-mass.pdf}
  \caption{The cantilever-mass model.}
  % \label{}
\end{figure}


%%%%%%%%%%%%%%%%%%%%%%%%%%%%%%%%%%%%%%%%%%%%%%%%%%%%%%%%%%%%%%%%%%%%%%%%%%%%%%%%
\paragraph{ESSI model fei file: } ~

\begin{lstlisting}
model name "brick-mass_1element" ;

// Geometry: width and height
b=0.2*m;
h=0.2*m;

// Materials: properties
natural_period    = 1*s;        
natural_frequency  = 2*pi/natural_period;
elastic_constant  = 1e9*N/m^2; 
I=b*h^3/12.0;
A=b*h;
L=1*m;
rho   = (1.8751)^4*elastic_constant*I/(natural_frequency^2*L^4*A);
possion_ratio=0.3;

add material # 1 type linear_elastic_isotropic_3d_LT
  mass_density = rho
  elastic_modulus = elastic_constant
  poisson_ratio = possion_ratio;

add node #        1 at (   0.0000 *m,   0.2000 *m,  0.0000 *m) with 3 dofs;
add node #        2 at (   0.0000 *m,   0.0000 *m,  0.0000 *m) with 3 dofs;
add node #        3 at (   1.0000 *m,   0.2000 *m,  0.0000 *m) with 3 dofs;
add node #        4 at (   1.0000 *m,   0.0000 *m,  0.0000 *m) with 3 dofs;
add node #        5 at (   0.0000 *m,   0.0000 *m,  0.2000 *m) with 3 dofs;
add node #        6 at (   1.0000 *m,   0.0000 *m,  0.2000 *m) with 3 dofs;
add node #        7 at (   1.0000 *m,   0.2000 *m,  0.2000 *m) with 3 dofs;
add node #        8 at (   0.0000 *m,   0.2000 *m,  0.2000 *m) with 3 dofs;
add node #        9 at (   0.0000 *m,   0.1000 *m,  0.0000 *m) with 3 dofs;
add node #       10 at (   0.5000 *m,   0.2000 *m,  0.0000 *m) with 3 dofs;
add node #       11 at (   1.0000 *m,   0.1000 *m,  0.0000 *m) with 3 dofs;
add node #       12 at (   0.5000 *m,   0.0000 *m,  0.0000 *m) with 3 dofs;
add node #       13 at (   0.0000 *m,   0.1000 *m,  0.2000 *m) with 3 dofs;
add node #       14 at (   0.5000 *m,   0.2000 *m,  0.2000 *m) with 3 dofs;
add node #       15 at (   1.0000 *m,   0.1000 *m,  0.2000 *m) with 3 dofs;
add node #       16 at (   0.5000 *m,   0.0000 *m,  0.2000 *m) with 3 dofs;
add node #       17 at (   0.0000 *m,   0.0000 *m,  0.1000 *m) with 3 dofs;
add node #       18 at (   0.0000 *m,   0.2000 *m,  0.1000 *m) with 3 dofs;
add node #       19 at (   1.0000 *m,   0.2000 *m,  0.1000 *m) with 3 dofs;
add node #       20 at (   1.0000 *m,   0.0000 *m,  0.1000 *m) with 3 dofs;
add node #       21 at (   0.5000 *m,   0.1000 *m,  0.1000 *m) with 3 dofs;
add node #       22 at (   0.0000 *m,   0.1000 *m,  0.1000 *m) with 3 dofs;
add node #       23 at (   0.5000 *m,   0.2000 *m,  0.1000 *m) with 3 dofs;
add node #       24 at (   1.0000 *m,   0.1000 *m,  0.1000 *m) with 3 dofs;
add node #       25 at (   0.5000 *m,   0.0000 *m,  0.1000 *m) with 3 dofs;
add node #       26 at (   0.5000 *m,   0.1000 *m,  0.0000 *m) with 3 dofs;
add node #       27 at (   0.5000 *m,   0.1000 *m,  0.2000 *m) with 3 dofs;

add element #         1 type 27NodeBrickLT with nodes(       2,       1,       3,       4,       5,       8,       7,       6,       9,      10,      11,      12,      13,      14,      15,      16,      17,      18,      19,      20,      21,      22,      23,      24,      25,      26,      27) use material #        1; 

fix node # 1 dofs all;
fix node # 2 dofs all;
fix node # 5 dofs all;
fix node # 8 dofs all;
fix node # 9 dofs all;
fix node # 13 dofs all;
fix node # 17 dofs all;
fix node # 18 dofs all;
fix node # 22 dofs all;


// Mapping from 3 dofs to 6 dofs. 
add node #      1003 at (   1.0000 *m,   0.2000 *m,  0.0000 *m) with 6 dofs;
add node #      1004 at (   1.0000 *m,   0.0000 *m,  0.0000 *m) with 6 dofs;
add node #      1006 at (   1.0000 *m,   0.0000 *m,  0.2000 *m) with 6 dofs;
add node #      1007 at (   1.0000 *m,   0.2000 *m,  0.2000 *m) with 6 dofs;
// And connect the nodes at the same location.
add constraint equal dof with master node # 3 and slave node #  1003 dof to constrain ux uy uz;
add constraint equal dof with master node # 4 and slave node #  1004 dof to constrain ux uy uz;
add constraint equal dof with master node # 6 and slave node #  1006 dof to constrain ux uy uz;
add constraint equal dof with master node # 7 and slave node #  1007 dof to constrain ux uy uz;

add mass to node # 24   mx =  rho*A*L my =  rho*A*L mz = rho*A*L;

// add 6 beams to connect the mass 
smallb=0.01*m;
smallh=0.01*m;
smallE  = 1e9*N/m^2; 
smallnu=0.3;
smallrho=0*kg/m^3;
smallI=smallb*smallh^3/12.0;
add element # 11 type beam_elastic with nodes (1003,1004) 
  cross_section =   smallb*smallh 
  elastic_modulus =  smallE
  shear_modulus =  smallE/2/(1+smallnu)
  torsion_Jx =  0.33*smallb*smallh^3
  bending_Iy =  smallI
  bending_Iz =  smallI
  mass_density = smallrho
  xz_plane_vector = ( 1, 0, 1) 
  joint_1_offset = (0*m, 0*m, 0*m) 
  joint_2_offset = (0*m, 0*m, 0*m);
add element # 12 type beam_elastic with nodes (1003,1006) 
  cross_section =   smallb*smallh 
  elastic_modulus =  smallE
  shear_modulus =  smallE/2/(1+smallnu)
  torsion_Jx =  0.33*smallb*smallh^3
  bending_Iy =  smallI
  bending_Iz =  smallI
  mass_density = smallrho
  xz_plane_vector = ( 1, 0, 1) 
  joint_1_offset = (0*m, 0*m, 0*m) 
  joint_2_offset = (0*m, 0*m, 0*m);
add element # 13 type beam_elastic with nodes (1003,1007) 
  cross_section =   smallb*smallh 
  elastic_modulus =  smallE
  shear_modulus =  smallE/2/(1+smallnu)
  torsion_Jx =  0.33*smallb*smallh^3
  bending_Iy =  smallI
  bending_Iz =  smallI
  mass_density = smallrho
  xz_plane_vector = ( 1, 0, 1) 
  joint_1_offset = (0*m, 0*m, 0*m) 
  joint_2_offset = (0*m, 0*m, 0*m);
add element # 14 type beam_elastic with nodes (1004,1006) 
  cross_section =   smallb*smallh 
  elastic_modulus =  smallE
  shear_modulus =  smallE/2/(1+smallnu)
  torsion_Jx =  0.33*smallb*smallh^3
  bending_Iy =  smallI
  bending_Iz =  smallI
  mass_density = smallrho
  xz_plane_vector = ( 1, 0, 1) 
  joint_1_offset = (0*m, 0*m, 0*m) 
  joint_2_offset = (0*m, 0*m, 0*m);
add element # 15 type beam_elastic with nodes (1004,1007) 
  cross_section =   smallb*smallh 
  elastic_modulus =  smallE
  shear_modulus =  smallE/2/(1+smallnu)
  torsion_Jx =  0.33*smallb*smallh^3
  bending_Iy =  smallI
  bending_Iz =  smallI
  mass_density = smallrho
  xz_plane_vector = ( 1, 0, 1) 
  joint_1_offset = (0*m, 0*m, 0*m) 
  joint_2_offset = (0*m, 0*m, 0*m);
add element # 16 type beam_elastic with nodes (1006,1007) 
  cross_section =   smallb*smallh 
  elastic_modulus =  smallE
  shear_modulus =  smallE/2/(1+smallnu)
  torsion_Jx =  0.33*smallb*smallh^3
  bending_Iy =  smallI
  bending_Iz =  smallI
  mass_density = smallrho
  xz_plane_vector = ( 1, 0, 1) 
  joint_1_offset = (0*m, 0*m, 0*m) 
  joint_2_offset = (0*m, 0*m, 0*m);


// // ----------------------------------------------------------------------------
// // --slowLoading---------------------------------------------------------------
// // add the 1 Newton load in 180 seconds.
// // ----------------------------------------------------------------------------
// new loading stage "slowLoading";
// add load # 1 to node # 4 type path_time_series Fz=1/36.0*N series_file = "slowLoading.txt" ; 
// add load # 2 to node # 6 type path_time_series Fz=1/36.0*N series_file = "slowLoading.txt" ; 
// add load # 3 to node # 3 type path_time_series Fz=1/36.0*N series_file = "slowLoading.txt" ; 
// add load # 4 to node # 7 type path_time_series Fz=1/36.0*N series_file = "slowLoading.txt" ; 
// add load # 5 to node # 20 type path_time_series Fz=1/9.0*N series_file = "slowLoading.txt" ; 
// add load # 6 to node # 11 type path_time_series Fz=1/9.0*N series_file = "slowLoading.txt" ; 
// add load # 7 to node # 15 type path_time_series Fz=1/9.0*N series_file = "slowLoading.txt" ; 
// add load # 8 to node # 19 type path_time_series Fz=1/9.0*N series_file = "slowLoading.txt" ; 
// add load # 9 to node # 24 type path_time_series Fz=4/9.0*N series_file = "slowLoading.txt" ; 
// // add algorithm and solver
// define dynamic integrator Newmark with gamma = 0.5 beta = 0.25;
// define algorithm With_no_convergence_check ;
// define solver ProfileSPD;
// simulate 2000 steps using transient algorithm 
//  time_step = 0.1*s;

// // ----------------------------------------------------------------------------
// // --fastLoading---------------------------------------------------------------
// // add the 1 Newton load in 0.6 seconds.
// // ----------------------------------------------------------------------------
// new loading stage "fastLoading";
// add load # 101 to node # 4 type path_time_series Fz=1/36.0*N series_file = "fastLoading.txt" ; 
// add load # 102 to node # 6 type path_time_series Fz=1/36.0*N series_file = "fastLoading.txt" ; 
// add load # 103 to node # 3 type path_time_series Fz=1/36.0*N series_file = "fastLoading.txt" ; 
// add load # 104 to node # 7 type path_time_series Fz=1/36.0*N series_file = "fastLoading.txt" ; 
// add load # 105 to node # 20 type path_time_series Fz=1/9.0*N series_file = "fastLoading.txt" ; 
// add load # 106 to node # 11 type path_time_series Fz=1/9.0*N series_file = "fastLoading.txt" ; 
// add load # 107 to node # 15 type path_time_series Fz=1/9.0*N series_file = "fastLoading.txt" ; 
// add load # 108 to node # 19 type path_time_series Fz=1/9.0*N series_file = "fastLoading.txt" ; 
// add load # 109 to node # 24 type path_time_series Fz=4/9.0*N series_file = "fastLoading.txt" ; 
// // add algorithm and solver
// define dynamic integrator Newmark with gamma = 0.5 beta = 0.25;
// define algorithm With_no_convergence_check ;
// define solver ProfileSPD;
// simulate 1000 steps using transient algorithm 
//  time_step = 0.01*s;

// // ----------------------------------------------------------------------------
// // --freeVibration---------------------------------------------------------------
// // ----------------------------------------------------------------------------
new loading stage "freeVibration";
add load # 201 to node # 4 type path_time_series Fz=1/36.0*N series_file = "freeVibration.txt" ; 
add load # 202 to node # 6 type path_time_series Fz=1/36.0*N series_file = "freeVibration.txt" ; 
add load # 203 to node # 3 type path_time_series Fz=1/36.0*N series_file = "freeVibration.txt" ; 
add load # 204 to node # 7 type path_time_series Fz=1/36.0*N series_file = "freeVibration.txt" ; 
add load # 205 to node # 20 type path_time_series Fz=1/9.0*N series_file = "freeVibration.txt" ; 
add load # 206 to node # 11 type path_time_series Fz=1/9.0*N series_file = "freeVibration.txt" ; 
add load # 207 to node # 15 type path_time_series Fz=1/9.0*N series_file = "freeVibration.txt" ; 
add load # 208 to node # 19 type path_time_series Fz=1/9.0*N series_file = "freeVibration.txt" ; 
add load # 209 to node # 24 type path_time_series Fz=4/9.0*N series_file = "freeVibration.txt" ; 
// add algorithm and solver
define dynamic integrator Newmark with gamma = 0.5 beta = 0.25;
define algorithm With_no_convergence_check ;
define solver ProfileSPD;
simulate 100 steps using transient algorithm 
  time_step = 0.1*s;

// end
bye;
\end{lstlisting}

%%%%%%%%%%%%%%%%%%%%%%%%%%%%%%%%%%%%%%%%%%%%%%%%%%%%%%%%%%%%%%%%%%%%%%%%%%%%%%%%
\paragraph{Displacement Results.} 

\begin{figure}[!htb]
  \centering
  \includegraphics[width=12cm]{../Figure-files/_Chapter_Appendix_Illustrative_Examples/brick-mass-1element-slowLoading.pdf}
  \caption{Slow loading condition, vertical displacements of the cantilever tip.}
  \label{fig_27brick-mass-slow}
\end{figure}


\begin{figure}[!htb]
  \centering
  \includegraphics[width=12cm]{../Figure-files/_Chapter_Appendix_Illustrative_Examples/brick-mass-1element-fastLoading.pdf}
  \caption{Fast loading condition, vertical displacements of the cantilever tip.}
  \label{fig_27brick-mass-fast}
\end{figure}

\begin{figure}[!htb]
  \centering
  \includegraphics[width=12cm]{../Figure-files/_Chapter_Appendix_Illustrative_Examples/brick-mass-1element-freeVibration.pdf}
  \caption{Free vibration condition, vertical displacements of the cantilever tip.}
  \label{fig_27brick-mass-freeVib}
\end{figure}


The    ESSI   model   fei   files   for   this   example   can   be   downloaded
\href{https://github.com/BorisJeremic/Real-ESSI-Examples/blob/master/model_fei_file/27NodeBrick_1element_dynamic/27NodeBrick_1element_dynamic.tgz?raw=true}{here}.





















%%%%%%%%%%%%%%%%%%%%%%%%%%%%%%%%%%%%%%%%%%%%%%%%%%%%%%%%%%%%%%%%%%%%%%%%%%%%%%%%
%%%%%%%%%%%%%%%%%%%%%%%%%%%%%%%%%%%%%%%%%%%%%%%%%%%%%%%%%%%%%%%%%%%%%%%%%%%%%%%%
%%%%%%%%%%%%%%%%%%%%%%%%%%%%%%%%%%%%%%%%%%%%%%%%%%%%%%%%%%%%%%%%%%%%%%%%%%%%%%%%
%%%%%%%%%%%%%%%%%%%%%%%%%%%%%%%%%%%%%%%%%%%%%%%%%%%%%%%%%%%%%%%%%%%%%%%%%%%%%%%%
%%%%%%%%%%%%%%%%%%%%%%%%%%%%%%%%%%%%%%%%%%%%%%%%%%%%%%%%%%%%%%%%%%%%%%%%%%%%%%%%
%%%%%%%%%%%%%%%%%%%%%%%%%%%%%%%%%%%%%%%%%%%%%%%%%%%%%%%%%%%%%%%%%%%%
%%%%%%%%%%%%%%%%%%%%%%%%%%%%%%%%%%%%%%%%%%%%%%%%%%%%%%%%%%%%%%%%%%%%

\newpage
\section{Elastic   Beam  Element,  Dynamic  Loading,  Viscous    (Rayleigh/Caughey)  and
Numerical (Newmark/HHT) Damping}
%%%%%%%%%%%%%%%%%%%%%%%%%%%%%%%%%%%%%%%%%%%%%%%%%%%%%%%%%%%%%%%%%%%%%%%%%%%%%%%%
\subsection{Single Beam Element} 
%%%%%%%%%%%%%%%%%%%%%%%%%%%%%%%%%%%%%%%%%%%%%%%%%%%%%%%%%%%%%%%%%%%%%%%%%%%%%%%%
\paragraph{Problem description:} 

% \begin{itemize}
%   \item Structure dimensions
% 
%     Structure Length=1m, width=0.2m, height=0.2m.
% 
%   \item Element dimensions
% 
%     Element length=1m, width=0.2m, height=0.2m.
% \end{itemize}
% 
\begin{figure}[!htb]
  \centering
  \includegraphics[width=8cm]{../Figure-files/_Chapter_Appendix_Illustrative_Examples/cantilever.pdf}
  \caption{The cantilever-mass model.}
  \label{fig_cantilev_1beam_damping}
\end{figure}


%%%%%%%%%%%%%%%%%%%%%%%%%%%%%%%%%%%%%%%%%%%%%%%%%%%%%%%%%%%%%%%%%%%%%%%%%%%%%%%%
\paragraph{ESSI model fei file: } ~


\begin{lstlisting}
model name "beam_1element" ;

// add node
add node #  1 at (   0.0*m ,    0.0*m,     0.0*m)  with 6 dofs;
add node #  2 at (   1.0*m ,    0.0*m,     0.0*m)  with 6 dofs;
  
// Geometry: width and height
b=0.2*m;
h=0.2*m;

// Materials: properties
natural_period    = 1*s;        
natural_frequency  = 2*pi/natural_period;
elastic_constant  = 1e9*N/m^2; 
I=b*h^3/12.0;
A=b*h;
L=1*m;
rho   = (1.8751)^4*elastic_constant*I/(natural_frequency^2*L^4*A);
possion_ratio=0.3;

// add elements
add element # 1 type beam_elastic with nodes (1,2) 
  cross_section =   b*h 
  elastic_modulus =  elastic_constant
  shear_modulus =  elastic_constant/2/(1+possion_ratio)
  torsion_Jx =  0.33*b*h^3
  bending_Iy =  b*h^3/12
  bending_Iz =  b*h^3/12
  mass_density = rho
  xz_plane_vector = ( 1, 0, 1) 
  joint_1_offset = (0*m, 0*m, 0*m) 
  joint_2_offset = (0*m, 0*m, 0*m);

// add boundary condition
fix node #      1 dofs all;

// // ----------------------------------------------------------------------------
// // --no-damping-------------------------------------------------------------
// // ----------------------------------------------------------------------------
// new loading stage "no-damping";
// add load # 1 to node # 2 type path_time_series 
//  Fz =  1.*N
//  series_file = "freeVibration.txt" ;
// define dynamic integrator Newmark with gamma = 0.5 beta = 0.25;
// define algorithm With_no_convergence_check ;
// define solver ProfileSPD;
// simulate 100 steps using transient algorithm 
//  time_step = 0.1*s;

// // ----------------------------------------------------------------------------
// // --Newmark-damping-------------------------------------------------------------
// // ----------------------------------------------------------------------------
// remove load # 2;
// new loading stage "Newmark-damping";
// add load # 3 to node # 2 type path_time_series 
//  Fz =  1.*N
//  series_file = "freeVibration.txt" ;
// define dynamic integrator Newmark with gamma = 0.6 beta = 0.3025;
// define algorithm With_no_convergence_check ;
// define solver ProfileSPD;
// simulate 100 steps using transient algorithm 
//  time_step = 0.1*s;
// // ----------------------------------------------------------------------------
// // --HHT-damping-------------------------------------------------------------
// // ----------------------------------------------------------------------------
// remove load # 3;
// new loading stage "HHT-damping";
// add load # 4 to node # 6 type path_time_series 
//  Fz =  1.*kN
//  series_file = "freeVibration.txt" ;
// define dynamic integrator Hilber_Hughes_Taylor with alpha = -0.20;
// define algorithm With_no_convergence_check ;
// define solver ProfileSPD;
// simulate 300 steps using transient algorithm 
//  time_step = 0.1*s;
// // ----------------------------------------------------------------------------
// // --Rayleigh-damping-------------------------------------------------------------
// // ----------------------------------------------------------------------------
// remove load # 4;
// simulate using eigen algorithm number_of_modes = 2;
f1=0.996807/s;
f2=0.996807/s;
w1 = 2*pi*f1;
w2 = 2*pi*f2;
xi=0.05;
rayl_a1 = 2*xi/(w1 + w2);
rayl_a0 = rayl_a1*w1*w2;

add damping # 1 type Rayleigh with 
  a0 =  rayl_a0
  a1 =  rayl_a1
  stiffness_to_use = Initial_Stiffness;
add damping # 1 to element # 1;

new loading stage "Rayleigh-damping";
add load # 5 to node # 2 type path_time_series 
  Fz =  1.*N
  series_file = "freeVibration.txt" ;
define dynamic integrator Newmark with gamma = 0.5 beta = 0.25;
define algorithm With_no_convergence_check ;
define solver ProfileSPD;
simulate 100 steps using transient algorithm 
  time_step = 0.1*s;

// // ----------------------------------------------------------------------------
// // --Caughey3rd-damping--------------------------------------------------------
// // ----------------------------------------------------------------------------
// add damping # 2 type Caughey3rd with 
//  a0 =  0.560523/s
//  a1 =  0.0730746*s
//  a2 =  0.000361559*s^3
//  stiffness_to_use = Last_Committed_Stiffness;
// kk=1;
// while (kk<6) {
//  add damping # 2 to element # kk;
//  kk+=1;
// }
// new loading stage "Caughey3rd-damping";
// add load # 6 to node # 6 type path_time_series 
//  Fz =  10.*kN
//  series_file = "freeVibration.txt" ;
// define dynamic integrator Newmark with gamma = 0.6 beta = 0.3025;
// define algorithm With_no_convergence_check ;
// define solver ProfileSPD;
// simulate 100 steps using transient algorithm 
//  time_step = 0.2*s;


// // ----------------------------------------------------------------------------
// // --Caughey4th-damping--------------------------------------------------------
// // ----------------------------------------------------------------------------
// add damping # 2 type Caughey4th with 
//  a0 =  0.560523/s
//  a1 =  0.0756472*s
//  a2 =  0.000517195*s^3
//  a3 =  1.20005*10^(-6)*s^5
//  stiffness_to_use = Last_Committed_Stiffness;
// kk=1;
// while (kk<6) {
//  add damping # 2 to element # kk;
//  kk+=1;
// }
// new loading stage "Caughey4th-damping";
// add load # 6 to node # 6 type path_time_series 
//  Fz =  10.*kN
//  series_file = "freeVibration.txt" ;
// define dynamic integrator Newmark with gamma = 0.6 beta = 0.3025;
// define algorithm With_no_convergence_check ;
// define solver ProfileSPD;
// simulate 100 steps using transient algorithm 
//  time_step = 0.2*s;

bye;
\end{lstlisting}

\paragraph{Displacement results against time series} ~

\begin{figure}[!htb]
  \centering
  \includegraphics[width=12cm]{../Figure-files/_Chapter_Appendix_Illustrative_Examples/beam-1element-no-damping.pdf}
  \caption{Free vibration condition,  no damping, vertical displacements of the cantilever tip.}
  \label{fig_1beam_nodamping}
\end{figure}



\begin{figure}[!htb]
  \centering
  \includegraphics[width=12cm]{../Figure-files/_Chapter_Appendix_Illustrative_Examples/beam-1element-Rayleigh-damping.pdf}
  \caption{Free vibration condition, viscous (Rayleigh) damping, vertical displacements of the cantilever tip.}
  \label{fig_1beam_rayleigh}
\end{figure}

\begin{figure}[!htb]
  \centering
  \includegraphics[width=12cm]{../Figure-files/_Chapter_Appendix_Illustrative_Examples/beam-5element-Caughey3rd-damping.pdf}
  \caption{Free vibration condition, viscous (Caughey3rd) damping, vertical displacements of the cantilever tip.}
  \label{fig_1beam_Caughey3rd}
\end{figure}

\begin{figure}[!htb]
  \centering
  \includegraphics[width=12cm]{../Figure-files/_Chapter_Appendix_Illustrative_Examples/beam-5element-Caughey4th-damping.pdf}
  \caption{Free vibration condition, viscous (Caughey4th) damping, vertical displacements of the cantilever tip.}
  \label{fig_1beam_Caughey4th}
\end{figure}

\begin{figure}[!htb]
  \centering
  \includegraphics[width=12cm]{../Figure-files/_Chapter_Appendix_Illustrative_Examples/beam-1element-Newmark-damping.pdf}
  \caption{Free vibration condition, numerical (Newmark) damping, vertical displacements of the cantilever tip.}
  \label{fig_1beam_newmark}
\end{figure}

\begin{figure}[!htb]
  \centering
  \includegraphics[width=12cm]{../Figure-files/_Chapter_Appendix_Illustrative_Examples/beam-1element-HHT-damping.pdf}
  \caption{Free vibration condition, numerical (HHT) damping, vertical displacements of the cantilever tip.}
  \label{fig_1beam_HHT}
\end{figure}




The    ESSI   model   fei   files   for   this   example   can   be   downloaded
\href{https://github.com/BorisJeremic/Real-ESSI-Examples/blob/master/model_fei_file/beam_elastic_damping_dynamic/beam_elastic_damping_dynamic.tgz?raw=true}{here}.









%%%%%%%%%%%%%%%%%%%%%%%%%%%%%%%%%%%%%%%%%%%%%%%%%%%%%%%%%%%%%%%%%%%%%%%%%%%%%%%%
%%%%%%%%%%%%%%%%%%%%%%%%%%%%%%%%%%%%%%%%%%%%%%%%%%%%%%%%%%%%%%%%%%%%%%%%%%%%%%%%
%%%%%%%%%%%%%%%%%%%%%%%%%%%%%%%%%%%%%%%%%%%%%%%%%%%%%%%%%%%%%%%%%%%%%%%%%%%%%%%%
%%%%%%%%%%%%%%%%%%%%%%%%%%%%%%%%%%%%%%%%%%%%%%%%%%%%%%%%%%%%%%%%%%%%%%%%%%%%%%%%
%%%%%%%%%%%%%%%%%%%%%%%%%%%%%%%%%%%%%%%%%%%%%%%%%%%%%%%%%%%%%%%%%%%%%%%%%%%%%%%%
%%%%%%%%%%%%%%%%%%%%%%%%%%%%%%%%%%%%%%%%%%%%%%%%%%%%%%%%%%%%%%%%%%%%
%%%%%%%%%%%%%%%%%%%%%%%%%%%%%%%%%%%%%%%%%%%%%%%%%%%%%%%%%%%%%%%%%%%%
\section{Elastic Beam Element for a Simple Frame Structure}



%%%%%%%%%%%%%%%%%%%%%%%%%%%%%%%%%%%%%%%%%%%%%%%%%%%%%%%%%%%%%%%%%%%%
\paragraph{Problem Description} ~ 


\begin{itemize} 

\item Dimensions: hidth=$6m$, height=$6m$, force=$100N$

\item Element dimensions: length=$6$m,   
                    cross section width=$1$m,   
                    cross section height=$1$m,  
                    mass density $\rho=0.0 {\rm kN/m^3}$,  
                    Young's modulus $E=1E8~{\rm Pa}$,  
                    Poisson's ratio $\nu=0.0$.
\end{itemize}


\begin{figure}[!htb]
  \centering
  \includegraphics[width=10cm]{../Figure-files/_Chapter_Appendix_Illustrative_Examples/beam_elastic_frame_descrip.pdf}
  \caption{Elastic frame with \emph{beam\_elastic} elements.}
  \label{fig_frame}
\end{figure}





%%%%%%%%%%%%%%%%%%%%%%%%%%%%%%%%%%%%%%%%%%%%%%%%%%%%%%%%%%%%%%%%%%%%
\paragraph{ESSI model fei file:} ~

%\lstinputlisting[frame=single]{../Figure-files/1beam_elastic_presentation.fei}
\begin{lstlisting}
model name "beam_element_presentation" ;

add node # 1 at ( 0.00*m, 0.00*m, 0.00*m) with 6 dofs;
add node # 2 at ( 0.00*m, 0.00*m, 6.00*m) with 6 dofs;
add node # 3 at ( 6.00*m, 0.00*m, 6.00*m) with 6 dofs;
add node # 4 at ( 6.00*m, 0.00*m, 0.00*m) with 6 dofs;

elastic_constant  = 1e8*N/m^2; 
b=1*m;
h=1*m;
rho   = 0*kg/m^3;     // Mass density

add element # 1 type beam_elastic with nodes (1, 2) 
 cross_section =  b*h   elastic_modulus =  elastic_constant
 shear_modulus =  elastic_constant/2
 torsion_Jx =  0.33*b*h^3  bending_Iy =  b*h^3/12  bending_Iz =  h*b^3/12
 mass_density =   rho  xz_plane_vector = (1, 0, 1 ) 
   joint_1_offset = (0*m, 0*m, 0*m )  joint_2_offset = (0*m, 0*m, 0*m );

add element # 2 type beam_elastic with nodes (2,3) 
 cross_section =  b*h  elastic_modulus =  elastic_constant
 shear_modulus =  elastic_constant/2
 torsion_Jx =  0.33*b*h^3  bending_Iy =  b*h^3/12  bending_Iz =  h*b^3/12
 mass_density =   rho  xz_plane_vector = (1, 0, 1 ) 
   joint_1_offset = (0*m, 0*m, 0*m ) joint_2_offset = (0*m, 0*m, 0*m );

add element # 3 type beam_elastic with nodes (3,4) 
 cross_section =  b*h  elastic_modulus =  elastic_constant
 shear_modulus =  elastic_constant/2
 torsion_Jx =  0.33*b*h^3 bending_Iy =  b*h^3/12  bending_Iz =  h*b^3/12
 mass_density =   rho  xz_plane_vector = (1, 0, 1 ) 
   joint_1_offset = (0*m, 0*m, 0*m ) joint_2_offset = (0*m, 0*m, 0*m );

fix node #1 dofs all;
fix node #4 dofs all;

new loading stage "Fz";

add load # 1 to node # 2 type linear Fz=50*N;

define algorithm With_no_convergence_check;
define solver ProfileSPD;
define load factor increment 1;
simulate 1 steps using static algorithm;

bye;
\end{lstlisting}

The    ESSI   model   fei   files   for   this   example   can   be   downloaded
\href{https://github.com/BorisJeremic/Real-ESSI-Examples/blob/master/model_fei_file/beam_elastic_presentation_example/beam_elastic_presentation_example.tgz?raw=true}{here}.











%%%%%%%%%%%%%%%%%%%%%%%%%%%%%%%%%%%%%%%%%%%%%%%%%%%%%%%%%%%%%%%%%%%%%%%%%%%%%%%%
%%%%%%%%%%%%%%%%%%%%%%%%%%%%%%%%%%%%%%%%%%%%%%%%%%%%%%%%%%%%%%%%%%%%%%%%%%%%%%%%
%%%%%%%%%%%%%%%%%%%%%%%%%%%%%%%%%%%%%%%%%%%%%%%%%%%%%%%%%%%%%%%%%%%%%%%%%%%%%%%%
%%%%%%%%%%%%%%%%%%%%%%%%%%%%%%%%%%%%%%%%%%%%%%%%%%%%%%%%%%%%%%%%%%%%%%%%%%%%%%%%
%%%%%%%%%%%%%%%%%%%%%%%%%%%%%%%%%%%%%%%%%%%%%%%%%%%%%%%%%%%%%%%%%%%%%%%%%%%%%%%%
%%%%%%%%%%%%%%%%%%%%%%%%%%%%%%%%%%%%%%%%%%%%%%%%%%%%%%%%%%%%%%%%%%%%
\section{27NodeBrick Cantilever Beam for the static load}

%%%%%%%%%%%%%%%%%%%%%%%%%%%%%%%%%%%%%%%%%%%%%%%%%%%%%%%%%%%%%%%%%%%%
\paragraph{Problem description: } ~

Length=6m, Width=1m, Height=1m, Force=100N, E=1E8Pa, $\nu=0.0$.
The force direction is shown in Figure (\ref{fig Problem description for cantilever beams of different Poisson's 27}). 

\begin{figure}[!htb]
  \centering
  \includegraphics[width=7cm]{../Figure-files/_Chapter_Appendix_Illustrative_Examples/cantilever_6.pdf}
  \caption{Problem description for cantilever beam.}
  \label{fig Problem description for cantilever beams of different Poisson's 27}
\end{figure}

%%%%%%%%%%%%%%%%%%%%%%%%%%%%%%%%%%%%%%%%%%%%%%%%%%%%%%%%%%%%%%%%%%%%
\paragraph{Numerical model:} ~

The  27NodeBrick  elements  for  cantilever  beams  is shown in Figure (\ref{fig 27NodeBrick elements for cantilever beams of different Poisson's ratios}):

\begin{figure}[!htb]
  \centering
  \includegraphics[width=9cm]{../Figure-files/_Chapter_Appendix_Illustrative_Examples/beam_27brick_6div.pdf}
  \caption{27NodeBrick elements for cantilever beams made of solid elements.}
  \label{fig 27NodeBrick elements for cantilever beams of different Poisson's ratios}
\end{figure}

%%%%%%%%%%%%%%%%%%%%%%%%%%%%%%%%%%%%%%%%%%%%%%%%%%%%%%%%%%%%%%%%%%%%
\paragraph{ESSI model fei file: } ~


%\lstinputlisting[frame=single]{../Figure-files/5_27NodeBrick.fei}
\begin{lstlisting}
model name "6meter_cantilever_27brick" ;

add material # 1 type linear_elastic_isotropic_3d
  mass_density = 0*kg/m^3
  elastic_modulus = 1e8*N/m^2
  poisson_ratio = 0.0;

add node #  1 at (   0.00 *m,   1.00 *m,  0.00 *m) with 3 dofs;
add node #  2 at (   0.00 *m,   0.00 *m,  0.00 *m) with 3 dofs;
add node #  3 at (   6.00 *m,   1.00 *m,  0.00 *m) with 3 dofs;
add node #  4 at (   5.00 *m,   1.00 *m,  0.00 *m) with 3 dofs;
add node #  5 at (   4.00 *m,   1.00 *m,  0.00 *m) with 3 dofs;
add node #  6 at (   3.00 *m,   1.00 *m,  0.00 *m) with 3 dofs;
...
...
add node #117 at (   5.50 *m,   0.50 *m,  1.00 *m) with 3 dofs;

add element #  1 type 27NodeBrickLT with nodes(   2,  10,  8,  1,  15,  17,  28,  23,  29,  30,  31,  32,  33,  34,  35,  36,  37,  38,  39,  40,  41,  42,  43,  44,  45,  46,  47) use material #  1; 
add element #  2 type 27NodeBrickLT with nodes(  10,  11,  7,  8,  17,  18,  27,  28,  48,  49,  50,  30,  51,  52,  53,  34,  38,  54,  55,  39,  56,  57,  58,  59,  43,  60,  61) use material #  1; 
add element #  3 type 27NodeBrickLT with nodes(  11,  12,  6,  7,  18,  19,  26,  27,  62,  63,  64,  49,  65,  66,  67,  52,  54,  68,  69,  55,  70,  71,  72,  73,  58,  74,  75) use material #  1; 
add element #  4 type 27NodeBrickLT with nodes(  12,  13,  5,  6,  19,  20,  25,  26,  76,  77,  78,  63,  79,  80,  81,  66,  68,  82,  83,  69,  84,  85,  86,  87,  72,  88,  89) use material #  1; 
add element #  5 type 27NodeBrickLT with nodes(  13,  14,  4,  5,  20,  21,  24,  25,  90,  91,  92,  77,  93,  94,  95,  80,  82,  96,  97,  83,  98,  99, 100, 101,  86, 102, 103) use material #  1; 
add element #  6 type 27NodeBrickLT with nodes(  14,  9,   3,  4,  21,  16,  22,  24, 104, 105, 106,  91, 107, 108, 109,  94,  96, 110, 111,  97, 112, 113, 114, 115, 100, 116, 117) use material #  1; 

fix node # 1 dofs all;
fix node # 2 dofs all;
fix node # 15 dofs all;
fix node # 23 dofs all;
fix node # 32 dofs all;
fix node # 36 dofs all;
fix node # 37 dofs all;
fix node # 40 dofs all;
fix node # 45 dofs all;

new loading stage "Fz";
add load # 1 to node # 13 type linear Fz=2.777778*N; 
add load # 2 to node # 24 type linear Fz=2.777778*N; 
add load # 3 to node # 3 type linear Fz=2.777778*N; 
add load # 4 to node # 34 type linear Fz=2.777778*N; 
add load # 5 to node # 182 type linear Fz=11.111111*N; 
add load # 6 to node # 177 type linear Fz=11.111111*N; 
add load # 7 to node # 180 type linear Fz=11.111111*N; 
add load # 8 to node # 183 type linear Fz=11.111111*N; 
add load # 9 to node # 186 type linear Fz=44.444444*N; 

define algorithm With_no_convergence_check ;
define solver UMFPack;
define load factor increment 1;
simulate 1 steps using static algorithm;

bye;
\end{lstlisting}

The ESSI model fei files for this example can be downloaded 
\href{https://github.com/BorisJeremic/Real-ESSI-Examples/blob/master/model_fei_file/27NodeBrick_static/27NodeBrick_static.tgz?raw=true}{here}.









%%%%%%%%%%%%%%%%%%%%%%%%%%%%%%%%%%%%%%%%%%%%%%%%%%%%%%%%%%%%%%%%%%%%















%%%%%%%%%%%%%%%%%%%%%%%%%%%%%%%%%%%%%%%%%%%%%%%%%%%%%%%%%%%%%%%%%%%%%%%%%%%%%%%%
%%%%%%%%%%%%%%%%%%%%%%%%%%%%%%%%%%%%%%%%%%%%%%%%%%%%%%%%%%%%%%%%%%%%%%%%%%%%%%%%
%%%%%%%%%%%%%%%%%%%%%%%%%%%%%%%%%%%%%%%%%%%%%%%%%%%%%%%%%%%%%%%%%%%%%%%%%%%%%%%%
%%%%%%%%%%%%%%%%%%%%%%%%%%%%%%%%%%%%%%%%%%%%%%%%%%%%%%%%%%%%%%%%%%%%%%%%%%%%%%%%
%%%%%%%%%%%%%%%%%%%%%%%%%%%%%%%%%%%%%%%%%%%%%%%%%%%%%%%%%%%%%%%%%%%%%%%%%%%%%%%%
%%%%%%%%%%%%%%%%%%%%%%%%%%%%%%%%%%%%%%%%%%%%%%%%%%%%%%%%%%%%%%%%%%%%
%%%%%%%%%%%%%%%%%%%%%%%%%%%%%%%%%%%%%%%%%%%%%%%%%%%%%%%%%%%%%%%%%%%%
\section{4NodeANDES Cantilever Beams Under the Force Perpendicular to Plane}

%%%%%%%%%%%%%%%%%%%%%%%%%%%%%%%%%%%%%%%%%%%%%%%%%%%%%%%%%%%%%%%%%%%%
\paragraph{Problem description:} ~

Length=6m, Width=1m, Height=1m, Force=100N, E=1E8Pa, $\nu=0.0$. 

\begin{figure}[!htb]
  \centering
  \includegraphics[width=7cm]{../Figure-files/_Chapter_Appendix_Illustrative_Examples/cantilever_6.pdf}
  \caption{Cantilever beams}
  \label{fig Problem description for cantilever 4}
\end{figure}


%%%%%%%%%%%%%%%%%%%%%%%%%%%%%%%%%%%%%%%%%%%%%%%%%%%%%%%%%%%%%%%%%%%%
\paragraph{Numerical model:} ~

\vskip 12pt

For a force  direction  perpendicular  to  the  plane, only the bending
deformation is present.


The  model is shown  in Figure (\ref{fig 4NodeANDES elements for
cantilever beams under force perpendicular to plane}).

\begin{figure}[!htb]
  \centering
%  \begin{subfigure}{0.5\textwidth}
    \centering
    \includegraphics[width=10cm]{../Figure-files/_Chapter_Appendix_Illustrative_Examples/beam_ANDES_xy_bending_6div.pdf}
%    % \caption{Six 4NodeANDES elements}
%  \end{subfigure}
%  \captionsetup{justification=centering,margin=3cm}
  \caption{4NodeANDES elements for cantilever beams under force perpendicular to
  plane.}
  \label{fig 4NodeANDES elements for cantilever beams under force perpendicular to plane}
\end{figure}


%%%%%%%%%%%%%%%%%%%%%%%%%%%%%%%%%%%%%%%%%%%%%%%%%%%%%%%%%%%%%%%%%%%%
\paragraph{ESSI model fei file: } ~

%\lstinputlisting[frame=single]{../Figure-files/3_perpend_ANDES.fei}
\begin{lstlisting}
 model name "6meter_cantilever_4NodeANDES" ;
      
add material # 1 type linear_elastic_isotropic_3d
  mass_density = 0*kg/m^3
  elastic_modulus = 1e8*N/m^2
  poisson_ratio = 0.0;

add node #  1 at ( 0.0*m, 0.0*m, 0.0*m) with 6 dofs;
add node #  2 at ( 6.0*m, 0.0*m, 0.0*m) with 6 dofs;
add node #  3 at ( 1.0*m, 0.0*m, 0.0*m) with 6 dofs;
add node #  4 at ( 2.0*m, 0.0*m, 0.0*m) with 6 dofs;
add node #  5 at ( 3.0*m, 0.0*m, 0.0*m) with 6 dofs;
add node #  6 at ( 4.0*m, 0.0*m, 0.0*m) with 6 dofs;
add node #  7 at ( 5.0*m, 0.0*m, 0.0*m) with 6 dofs;
add node #  8 at ( 6.0*m, 1.0*m, 0.0*m) with 6 dofs;
add node #  9 at ( 0.0*m, 1.0*m, 0.0*m) with 6 dofs;
add node # 10 at ( 5.0*m, 1.0*m, 0.0*m) with 6 dofs;
add node # 11 at ( 4.0*m, 1.0*m, 0.0*m) with 6 dofs;
add node # 12 at ( 3.0*m, 1.0*m, 0.0*m) with 6 dofs;
add node # 13 at ( 2.0*m, 1.0*m, 0.0*m) with 6 dofs;
add node # 14 at ( 1.0*m, 1.0*m, 0.0*m) with 6 dofs;

h = 1*m; 
add element # 1 type 4NodeShell_ANDES with nodes (1,3,14,9) use material # 1 thickness = h ; 
add element # 2 type 4NodeShell_ANDES with nodes (3,4,13,14) use material # 1 thickness = h ; 
add element # 3 type 4NodeShell_ANDES with nodes (4,5,12,13) use material # 1 thickness = h ; 
add element # 4 type 4NodeShell_ANDES with nodes (5,6,11,12) use material # 1 thickness = h ; 
add element # 5 type 4NodeShell_ANDES with nodes (6,7,10,11) use material # 1 thickness = h ; 
add element # 6 type 4NodeShell_ANDES with nodes (7,2,8,10) use material # 1 thickness = h ; 

fix node #  1 dofs all    ;
fix node #  9 dofs all    ;

new loading stage "Fz";
add load # 1 to node # 8 type linear Fz=50*N;
add load # 2 to node # 2 type linear Fz=50*N;

define algorithm With_no_convergence_check ;
define solver ProfileSPD;
define load factor increment 1;
simulate 1 steps using static algorithm;

bye;
\end{lstlisting}

The ESSI model fei files for this example can be downloaded 
\href{https://github.com/BorisJeremic/Real-ESSI-Examples/blob/master/model_fei_file/ANDESshell_cantilever_perpendicular_to_plane/ANDESshell_cantilever_perpendicular_to_plane.tgz?raw=true}{here}.








%%%%%%%%%%%%%%%%%%%%%%%%%%%%%%%%%%%%%%%%%%%%%%%%%%%%%%%%%%%%%%%%%%%%













%%%%%%%%%%%%%%%%%%%%%%%%%%%%%%%%%%%%%%%%%%%%%%%%%%%%%%%%%%%%%%%%%%%%%%%%%%%%%%%%
%%%%%%%%%%%%%%%%%%%%%%%%%%%%%%%%%%%%%%%%%%%%%%%%%%%%%%%%%%%%%%%%%%%%%%%%%%%%%%%%
%%%%%%%%%%%%%%%%%%%%%%%%%%%%%%%%%%%%%%%%%%%%%%%%%%%%%%%%%%%%%%%%%%%%%%%%%%%%%%%%
%%%%%%%%%%%%%%%%%%%%%%%%%%%%%%%%%%%%%%%%%%%%%%%%%%%%%%%%%%%%%%%%%%%%%%%%%%%%%%%%
%%%%%%%%%%%%%%%%%%%%%%%%%%%%%%%%%%%%%%%%%%%%%%%%%%%%%%%%%%%%%%%%%%%%%%%%%%%%%%%%
%%%%%%%%%%%%%%%%%%%%%%%%%%%%%%%%%%%%%%%%%%%%%%%%%%%%%%%%%%%%%%%%%%%
\section{4NodeANDES Cantilever Beams under the In-Plane Force}

%%%%%%%%%%%%%%%%%%%%%%%%%%%%%%%%%%%%%%%%%%%%%%%%%%%%%%%%%%%%%%%%%%%%
\paragraph{Problem description:} ~

Length=6m, Width=1m, Height=1m, Force=100N, E=1E8Pa, $\nu=0.0$. 

\begin{figure}[!htb]
  \centering
  \includegraphics[width=7cm]{../Figure-files/_Chapter_Appendix_Illustrative_Examples/cantilever_6.pdf}
  \caption{Problem description for cantilever beams with in plane force}
  \label{fig Problem description for cantilever 4 2}
\end{figure}

%%%%%%%%%%%%%%%%%%%%%%%%%%%%%%%%%%%%%%%%%%%%%%%%%%%%%%%%%%%%%%%%%%%%
\paragraph{Numerical model:} ~

%When the force direction is inplane, both the bending and shear deformation are calculated in 4NodeANDES elements. 
\begin{lstlisting}

\end{lstlisting}

The 4NodeANDES elements under in-plane force is shown in Figure (\ref{fig 4NodeANDES elements for cantilever beams under inplane force}).

\begin{figure}[!htb]
  \centering
  \vskip 8pt
%  \begin{subfigure}{0.5\textwidth}
    \centering
    \includegraphics[width=10cm]{../Figure-files/_Chapter_Appendix_Illustrative_Examples/beam_ANDES_yz_inPlane_6div.pdf}
%  \end{subfigure}
%  \captionsetup{justification=centering,margin=3cm}
  \caption{4NodeANDES elements for cantilever beams under in-plane force}
  \label{fig 4NodeANDES elements for cantilever beams under inplane force}
\end{figure}


%%%%%%%%%%%%%%%%%%%%%%%%%%%%%%%%%%%%%%%%%%%%%%%%%%%%%%%%%%%%%%%%%%%%
\paragraph{ESSI model fei file: } ~

%\lstinputlisting[frame=single]{../Figure-files/4_inplane_ANDES.fei}
\begin{lstlisting}
model name "6meter_cantilever_4NodeANDES" ;

add material # 1 type linear_elastic_isotropic_3d
  mass_density = 0*kg/m^3
  elastic_modulus = 1e8*N/m^2
  poisson_ratio = 0.0;

add node #   1 at ( 0.00*m, 0.00*m, 0.00*m) with 6 dofs;
add node #   2 at ( 6.00*m, 0.00*m, 0.00*m) with 6 dofs;
add node #   3 at ( 1.00*m, 0.00*m, 0.00*m) with 6 dofs;
add node #   4 at ( 2.00*m, 0.00*m, 0.00*m) with 6 dofs;
add node #   5 at ( 3.00*m, 0.00*m, 0.00*m) with 6 dofs;
add node #   6 at ( 4.00*m, 0.00*m, 0.00*m) with 6 dofs;
add node #   7 at ( 5.00*m, 0.00*m, 0.00*m) with 6 dofs;
add node #   8 at ( 6.00*m, 1.00*m, 0.00*m) with 6 dofs;
add node #   9 at ( 0.00*m, 1.00*m, 0.00*m) with 6 dofs;
add node #  10 at ( 5.00*m, 1.00*m, 0.00*m) with 6 dofs;
add node #  11 at ( 4.00*m, 1.00*m, 0.00*m) with 6 dofs;
add node #  12 at ( 3.00*m, 1.00*m, 0.00*m) with 6 dofs;
add node #  13 at ( 2.00*m, 1.00*m, 0.00*m) with 6 dofs;
add node #  14 at ( 1.00*m, 1.00*m, 0.00*m) with 6 dofs;

h     = 1*m;  
add element # 1 type 4NodeShell_ANDES with nodes (1,3,14,9) use material # 1 thickness = h ; 
add element # 2 type 4NodeShell_ANDES with nodes (3,4,13,14) use material # 1 thickness = h ; 
add element # 3 type 4NodeShell_ANDES with nodes (4,5,12,13) use material # 1 thickness = h ; 
add element # 4 type 4NodeShell_ANDES with nodes (5,6,11,12) use material # 1 thickness = h ; 
add element # 5 type 4NodeShell_ANDES with nodes (6,7,10,11) use material # 1 thickness = h ; 
add element # 6 type 4NodeShell_ANDES with nodes (7,2,8,10) use material # 1 thickness = h ; 

fix node #  1 dofs all;
fix node #  9 dofs all;

new loading stage "Fy";
add load # 1 to node # 8 type linear Fy=50*N;
add load # 2 to node # 2 type linear Fy=50*N;

define algorithm With_no_convergence_check ;
define solver ProfileSPD;
define load factor increment 1;
simulate 1 steps using static algorithm;

bye;
\end{lstlisting}


The ESSI model fei files for this example can be downloaded 
\href{https://github.com/BorisJeremic/Real-ESSI-Examples/blob/master/model_fei_file/ANDESshell_cantilever_inplane/ANDESshell_cantilever_inplane.tgz?raw=true}{here}.








%%%%%%%%%%%%%%%%%%%%%%%%%%%%%%%%%%%%%%%%%%%%%%%%%%%%%%%%%%%%%%%%%%%%
























%%%%%%%%%%%%%%%%%%%%%%%%%%%%%%%%%%%%%%%%%%%%%%%%%%%%%%%%%%%%%%%%%%%%%%%%%%%%%%%%
%%%%%%%%%%%%%%%%%%%%%%%%%%%%%%%%%%%%%%%%%%%%%%%%%%%%%%%%%%%%%%%%%%%%%%%%%%%%%%%%
%%%%%%%%%%%%%%%%%%%%%%%%%%%%%%%%%%%%%%%%%%%%%%%%%%%%%%%%%%%%%%%%%%%%%%%%%%%%%%%%
%%%%%%%%%%%%%%%%%%%%%%%%%%%%%%%%%%%%%%%%%%%%%%%%%%%%%%%%%%%%%%%%%%%%%%%%%%%%%%%%
%%%%%%%%%%%%%%%%%%%%%%%%%%%%%%%%%%%%%%%%%%%%%%%%%%%%%%%%%%%%%%%%%%%%%%%%%%%%%%%%
%%%%%%%%%%%%%%%%%%%%%%%%%%%%%%%%%%%%%%%%%%%%%%%%%%%%%%%%%%%%%%%%%%%%
\section{27NodeBrick Cantilever Beams for Dynamic Input}



%%%%%%%%%%%%%%%%%%%%%%%%%%%%%%%%%%%%%%%%%%%%%%%%%%%%%%%%%%%%%%%%%%%%
\paragraph{Problem description:} ~



Length=20m, Width=1m, Height=1m, E=504MPa, $\nu=0.4$. 

All degree of freedoms at the bottom nodes are fixed. 

The load is a self weight with a dynamic displacement of supports.


\begin{figure}[!htb]
  \centering
  \includegraphics[width=8cm]{../Figure-files/_Chapter_Appendix_Illustrative_Examples/dynamic_example_diagram.pdf}
  \caption{Problem description for one simple dynamic example}
  \label{fig Problem description for one simple dynamic example}
\end{figure}



%%%%%%%%%%%%%%%%%%%%%%%%%%%%%%%%%%%%%%%%%%%%%%%%%%%%%%%%%%%%%%%%%%%%
\paragraph{Numerical model:} ~

The numerical model applied 27NodeBrick to simulate the 1D motion. 

\begin{figure}[!htb]
  \centering
  \includegraphics[width=16cm]{../Figure-files/_Chapter_Appendix_Illustrative_Examples/dynamic_example_numerical.pdf}
  \caption{Numerical model for one simple dynamic example}
  \label{fig Numerical model for one simple dynamic example}
\end{figure}

%%%%%%%%%%%%%%%%%%%%%%%%%%%%%%%%%%%%%%%%%%%%%%%%%%%%%%%%%%%%%%%%%%%%
\paragraph{ESSI model fei file: } ~

%\lstinputlisting[frame=single]{../Figure-files/6dynamic_example.fei}
\begin{lstlisting}
model name "dynamic_example";

add material # 1 type linear_elastic_isotropic_3d_LT
 mass_density    = 2000*kg/m^3
 elastic_modulus = 504000000.00*Pa
 poisson_ratio   = 0.4;

add node No 1 at (0*m, 0*m, 0*m) with 3 dofs;
add node No 2 at (0*m, 0.5*m, 0*m) with 3 dofs;
add node No 3 at (0*m, 1*m, 0*m) with 3 dofs;
add node No 4 at (0.5*m, 0*m, 0*m) with 3 dofs;
add node No 5 at (0.5*m, 0.5*m, 0*m) with 3 dofs;
add node No 6 at (0.5*m, 1*m, 0*m) with 3 dofs;
...
...
add node No 369 at (1*m, 1*m, 20*m) with 3 dofs;

add element # 1 type 27NodeBrickLT with nodes (27,21,19,25,9,3,1,7,24,20,22,26,6,2,4,8,18,12,10,16,14,15,11,13,17,23,5) use material # 1 ;
add element # 2 type 27NodeBrickLT with nodes (45,39,37,43,27,21,19,25,42,38,40,44,24,20,22,26,36,30,28,34,32,33,29,31,35,41,23) use material # 1 ;
add element # 3 type 27NodeBrickLT with nodes (63,57,55,61,45,39,37,43,60,56,58,62,42,38,40,44,54,48,46,52,50,51,47,49,53,59,41) use material # 1 ;
add element # 4 type 27NodeBrickLT with nodes (81,75,73,79,63,57,55,61,78,74,76,80,60,56,58,62,72,66,64,70,68,69,65,67,71,77,59) use material # 1 ;
add element # 5 type 27NodeBrickLT with nodes (99,93,91,97,81,75,73,79,96,92,94,98,78,74,76,80,90,84,82,88,86,87,83,85,89,95,77) use material # 1 ;
...
...
add element # 20 type 27NodeBrickLT with nodes (369,363,361,367,351,345,343,349,366,362,364,368,348,
 344,346,350,360,354,352,358,356,357,353,355,359,365,347) use material # 1 ;

add acceleration field # 1 ax = 0*g ay = 0*g az = -1*g ;
add load # 1 to element # 1 type self_weight use acceleration field # 1;
add load # 2 to element # 2 type self_weight use acceleration field # 1;
add load # 3 to element # 3 type self_weight use acceleration field # 1;
add load # 4 to element # 4 type self_weight use acceleration field # 1;
add load # 5 to element # 5 type self_weight use acceleration field # 1;
add load # 6 to element # 6 type self_weight use acceleration field # 1;
...
...
add load # 20 to element # 20 type self_weight use acceleration field # 1;

fix node No 1 dofs    uy uz;
fix node No 2 dofs    uy uz;
fix node No 3 dofs    uy uz;
fix node No 4 dofs    uy uz;
fix node No 5 dofs    uy uz;
fix node No 6 dofs    uy uz;
...
...
fix node No 369 dofs    uy uz;

zeta = 0.0166667;
fq1  = 3.75;
fq2  = 11.25;
omega1 = 2*pi*fq1;
omega2 = 2*pi*fq2;
zeta1 = zeta;
zeta2 = zeta;
alpha1 = 2*omega1*omega2*(zeta1*omega2-zeta2*omega1)/(omega2*omega2-omega1*omega1);
beta1  = 2*              (zeta2*omega2-zeta1*omega1)/(omega2*omega2-omega1*omega1);
add damping # 1 
   type Rayleigh 
   with 
   a0 = alpha1/s 
   a1 = beta1*s 
   stiffness_to_use = Initial_Stiffness;

add damping # 1 to element # 1;
add damping # 1 to element # 2;
add damping # 1 to element # 3;
add damping # 1 to element # 4;
add damping # 1 to element # 5;
add damping # 1 to element # 6;
...
...
add damping # 1 to element # 20;

new loading stage "impose_motion";

add imposed motion # 1001 to node # 1 dof ux 
 displacement_scale_unit = 1*m       displacement_file    = "dis.txt" 
 velocity_scale_unit     = 1*m/s     velocity_file        = "vel.txt" 
 acceleration_scale_unit = 1*m/s^2   acceleration_file    = "acc.txt";

add imposed motion # 1002 to node # 2 dof ux 
 displacement_scale_unit = 1*m         displacement_file    = "dis.txt" 
 velocity_scale_unit     = 1*m/s       velocity_file        = "vel.txt" 
 acceleration_scale_unit = 1*m/s^2     acceleration_file    = "acc.txt";

add imposed motion # 1003 to node # 3 dof ux 
 displacement_scale_unit = 1*m         displacement_file   = "dis.txt" 
 velocity_scale_unit     = 1*m/s       velocity_file       = "vel.txt" 
 acceleration_scale_unit = 1*m/s^2     acceleration_file   = "acc.txt";
...
...
add imposed motion # 1009 to node # 9 dof ux 
 displacement_scale_unit = 1*m        displacement_file   = "dis.txt" 
 velocity_scale_unit     = 1*m/s      velocity_file       = "vel.txt" 
 acceleration_scale_unit = 1*m/s^2    acceleration_file   = "acc.txt";

define dynamic integrator Newmark with gamma = 0.5 beta = 0.25;
define algorithm With_no_convergence_check;
define solver ProfileSPD;
simulate 50 steps using transient algorithm time_step = 0.005*s;
         
bye;
\end{lstlisting}


The ESSI model fei files for this example can be downloaded 
\href{https://github.com/BorisJeremic/Real-ESSI-Examples/blob/master/model_fei_file/27NodeBrick_dynamic_impose_motion/27NodeBrick_dynamic_impose_motion.tgz?raw=true}{here}.











%%%%%%%%%%%%%%%%%%%%%%%%%%%%%%%%%%%%%%%%%%%%%%%%%%%%%%%%%%%%%%%%%%%%













%%%%%%%%%%%%%%%%%%%%%%%%%%%%%%%%%%%%%%%%%%%%%%%%%%%%%%%%%%%%%%%%%%%%%%%%%%%%%%%%
%%%%%%%%%%%%%%%%%%%%%%%%%%%%%%%%%%%%%%%%%%%%%%%%%%%%%%%%%%%%%%%%%%%%%%%%%%%%%%%%
%%%%%%%%%%%%%%%%%%%%%%%%%%%%%%%%%%%%%%%%%%%%%%%%%%%%%%%%%%%%%%%%%%%%%%%%%%%%%%%%
%%%%%%%%%%%%%%%%%%%%%%%%%%%%%%%%%%%%%%%%%%%%%%%%%%%%%%%%%%%%%%%%%%%%%%%%%%%%%%%%
%%%%%%%%%%%%%%%%%%%%%%%%%%%%%%%%%%%%%%%%%%%%%%%%%%%%%%%%%%%%%%%%%%%%%%%%%%%%%%%%
%%%%%%%%%%%%%%%%%%%%%%%%%%%%%%%%%%%%%%%%%%%%%%%%%%%%%%%%%%%%%%%%%%%%
\section{4NodeANDES Square Plate with Four Edges Clamped}



%%%%%%%%%%%%%%%%%%%%%%%%%%%%%%%%%%%%%%%%%%%%%%%%%%%%%%%%%%%%%%%%%%%%
\paragraph{Problem description:} ~



Length=20m, Width=20m, Height=1m, Force=100N, E=1E8Pa, $\nu=0.3$. 

The four edges are \textbf{clamped}. 

The load is a self weight.


\begin{figure}[!htb]
  \centering
  \includegraphics[width=10cm]{../Figure-files/_Chapter_Appendix_Illustrative_Examples/square_plate_descrp.pdf}
  \caption{Square plate with four edges clamped }
  \label{fig 4NodeANDES edges clamped square plate with element side length for program description }
\end{figure}


\clearpage
%%%%%%%%%%%%%%%%%%%%%%%%%%%%%%%%%%%%%%%%%%%%%%%%%%%%%%%%%%%%%%%%%%%%
\paragraph{Numerical model:} ~

The element side length is 1 meter. 


\begin{figure}[!htb]
  \centering
  \includegraphics[width=10cm]{../Figure-files/_Chapter_Appendix_Illustrative_Examples/square_plate4_2.pdf}
  \caption{4NodeANDES edge clamped square plate with element side length 1m }
  \label{fig 4NodeANDES edges clamped square plate with element side length 1m }
\end{figure}


%%%%%%%%%%%%%%%%%%%%%%%%%%%%%%%%%%%%%%%%%%%%%%%%%%%%%%%%%%%%%%%%%%%%
\paragraph{ESSI model fei file: } ~


%\lstinputlisting[frame=single]{../Figure-files/7shell.fei}
\begin{lstlisting}
model name "square_plate" ;

add material # 1 type linear_elastic_isotropic_3d
  mass_density = 1e2*kg/m^3   elastic_modulus = 1e8*N/m^2   poisson_ratio = 0.3;

add node #  1 at (   0.00*m, 0.00*m, 0.00*m) with 6 dofs;
add node #  2 at (  20.00*m, 0.00*m, 0.00*m) with 6 dofs;
add node #  3 at (   1.00*m, 0.00*m, 0.00*m) with 6 dofs;
add node #  4 at (   2.00*m, 0.00*m, 0.00*m) with 6 dofs;
add node #  5 at (   3.00*m, 0.00*m, 0.00*m) with 6 dofs;
add node #  6 at (   4.00*m, 0.00*m, 0.00*m) with 6 dofs;
...
...
add node #  441 at ( 19.00*m, 19.00*m, 0.00*m) with 6 dofs;

h     = 1*m;       
add element #       1 type 4NodeShell_ANDES with nodes(  1,  3,   81,  80) use material #   1 thickness=h;
add element #       2 type 4NodeShell_ANDES with nodes(  3,  4,  100,  81) use material #   1 thickness=h;
add element #       3 type 4NodeShell_ANDES with nodes(  4,  5,  119, 100) use material #   1 thickness=h;
add element #       4 type 4NodeShell_ANDES with nodes(  5,  6,  138, 119) use material #   1 thickness=h;
add element #       5 type 4NodeShell_ANDES with nodes(  6,  7,  157, 138) use material #   1 thickness=h;
add element #       6 type 4NodeShell_ANDES with nodes(  7,  8,  176, 157) use material #   1 thickness=h;
...
...
add element #     400 type 4NodeShell_ANDES with nodes(  441, 41, 22,  43) use material #   1 thickness=h;


fix node #      1 dofs all    ;
fix node #      2 dofs all    ;
fix node #      3 dofs all    ;
fix node #      4 dofs all    ;
fix node #      5 dofs all    ;
fix node #      6 dofs all    ;
...
...
fix node #     80 dofs all    ;


new loading stage "self_weight";
add acceleration field # 1   ax =  0*g   ay =  0*g   az =  1*m/s^2;
add load # 1 to element # 1 type self_weight use acceleration field # 1;
add load # 2 to element # 2 type self_weight use acceleration field # 1;
add load # 3 to element # 3 type self_weight use acceleration field # 1;
add load # 4 to element # 4 type self_weight use acceleration field # 1;
add load # 5 to element # 5 type self_weight use acceleration field # 1;
add load # 6 to element # 6 type self_weight use acceleration field # 1;
...
...
add load # 400 to element # 400 type self_weight use acceleration field # 1;


define algorithm With_no_convergence_check ;
define solver ProfileSPD;
define load factor increment 1;
simulate 1 steps using static algorithm;

bye;
\end{lstlisting}


The ESSI model fei files for this example can be downloaded 
\href{https://github.com/BorisJeremic/Real-ESSI-Examples/blob/master/model_fei_file/ANDESshell_square_plate/ANDESshell_square_plate.tgz?raw=true}{here}.




















%%%%%%%%%%%%%%%%%%%%%%%%%%%%%%%%%%%%%%%%%%%%%%%%%%%%%%%%%%%%%%%%%%%%













%%%%%%%%%%%%%%%%%%%%%%%%%%%%%%%%%%%%%%%%%%%%%%%%%%%%%%%%%%%%%%%%%%%%%%%%%%%%%%%%
%%%%%%%%%%%%%%%%%%%%%%%%%%%%%%%%%%%%%%%%%%%%%%%%%%%%%%%%%%%%%%%%%%%%%%%%%%%%%%%%
%%%%%%%%%%%%%%%%%%%%%%%%%%%%%%%%%%%%%%%%%%%%%%%%%%%%%%%%%%%%%%%%%%%%%%%%%%%%%%%%
%%%%%%%%%%%%%%%%%%%%%%%%%%%%%%%%%%%%%%%%%%%%%%%%%%%%%%%%%%%%%%%%%%%%%%%%%%%%%%%%
%%%%%%%%%%%%%%%%%%%%%%%%%%%%%%%%%%%%%%%%%%%%%%%%%%%%%%%%%%%%%%%%%%%%%%%%%%%%%%%%
%%%%%%%%%%%%%%%%%%%%%%%%%%%%%%%%%%%%%%%%%%%%%%%%%%%%%%%%%%%%%%%%%%%%
\section{One Dimensional DRM Model}



%Domain  Reduction  Method  (DRM)  is  a  finite element methodology for modeling
%earthquake  ground  motion.  Please  look  at the reference\footnote{Bielak, J.,
%Loukakis,  K., Hisada, Y., and Yoshimura, C. (2003). Domain reduction method for
%three-dimensional  earthquake  modeling  in  localized  regions, Part I: Theory.
%Bulletin  of  the  Seismological  Society  of America, 93(2), 817-824.} for more
%information.


%%%%%%%%%%%%%%%%%%%%%%%%%%%%%%%%%%%%%%%%%%%%%%%%%%%%%%%%%%%%%%%%%%%%
\paragraph{Problem description:} ~


A simple 1D DRM model is shown in Fig.(\ref{fig_Program_description_for_the_1D_DRM_model}). 
The "DRM element", "Exterior node" and "Boundary node" are required
to be designated in the DRM HDF5 input. The format and script for the HDF5 input
is available in DSL/input manual.

\begin{figure}[!htb]
  \centering
  \includegraphics[width=18cm]{../Figure-files/_Chapter_Appendix_Illustrative_Examples/DRM_1D_descrp.pdf}
  \caption{1D DRM model.}
  \label{fig_Program_description_for_the_1D_DRM_model}
\end{figure}




%%%%%%%%%%%%%%%%%%%%%%%%%%%%%%%%%%%%%%%%%%%%%%%%%%%%%%%%%%%%%%%%%%%%
\paragraph{Numerical model:} ~


\begin{figure}[!htb]
  \centering
  \includegraphics[width=15cm]{../Figure-files/_Chapter_Appendix_Illustrative_Examples/DRM_1D_result41.png}
  \caption{1D DRM model.}
  \label{fig Diagram for the 1D DRM model}
\end{figure}




%%%%%%%%%%%%%%%%%%%%%%%%%%%%%%%%%%%%%%%%%%%%%%%%%%%%%%%%%%%%%%%%%%%%
\paragraph{ESSI model fei file: } ~

%\lstinputlisting[frame=single]{../Figure-files/DRM_1D_script.fei}
\begin{lstlisting}
model name "DRM" ;

//Material for soil
add material # 1 type linear_elastic_isotropic_3d_LT
  mass_density = 2000*kg/m^3
  elastic_modulus = 1300*MPa
  poisson_ratio = 0.3;

//Material for DRM layer
add material # 2 type linear_elastic_isotropic_3d_LT
  mass_density = 2000*kg/m^3
  elastic_modulus = 1300*MPa
  poisson_ratio = 0.3;

//Material for exterior layer
add material # 3 type linear_elastic_isotropic_3d_LT
  mass_density = 2000*kg/m^3
  elastic_modulus = 1300*MPa
  poisson_ratio = 0.3;
//
add node # 1 at (  0.00*m,  0.00*m,  0.00*m) with 3 dofs;
add node # 2 at (  5.00*m,  0.00*m,  0.00*m) with 3 dofs;
add node # 3 at (  5.00*m,  5.00*m,  0.00*m) with 3 dofs;
add node # 4 at (  0.00*m,  5.00*m,  0.00*m) with 3 dofs;
add node # 5 at (  5.00*m,  0.00*m, 50.00*m) with 3 dofs;
add node # 6 at (  5.00*m,  0.00*m,  5.00*m) with 3 dofs;
...
...
add node # 52 at (  0.00*m, 5.00*m,  -5.00*m) with 3 dofs;

//
add element #       1 type 8NodeBrickLT with nodes( 1, 4, 3, 2, 24, 44, 34, 6) use material # 1;
add element #       2 type 8NodeBrickLT with nodes( 24, 44, 34, 6, 23, 43, 33, 7) use material # 1;
...
add element #      12 type 8NodeBrickLT with nodes( 48, 47, 45, 46, 52, 51, 49, 50) use material # 3;

//
fix node # 1 dofs uy ;
fix node # 1 dofs uz ;
fix node # 2 dofs uy ;
fix node # 2 dofs uz ;
fix node # 3 dofs uy ;
fix node # 3 dofs uz ;
fix node # 4 dofs uy ;
fix node # 4 dofs uz ;
...
fix node # 51 dofs ux ;


new loading stage "1D";
add domain reduction method loading #  1
  hdf5_file = "input.hdf5";

define algorithm With_no_convergence_check ;
define solver ProfileSPD;
define dynamic integrator Newmark with  gamma = 0.5  beta = 0.25;
simulate 999 steps using transient algorithm time_step = 0.01*s;

bye;
\end{lstlisting}

The ESSI model fei files for this example can be downloaded 
\href{https://github.com/BorisJeremic/Real-ESSI-Examples/blob/master/model_fei_file/8NodeBrick_DRM_1D/8NodeBrick_DRM_1D.tgz?raw=true}{here}.

The same model for this example with 27NodeBrickLT can be downloaded 
\href{https://github.com/BorisJeremic/Real-ESSI-Examples/blob/master/model_fei_file/27NodeBrick_DRM_1D/27NodeBrick_DRM_1D.tgz?raw=true}{here}.








%%%%%%%%%%%%%%%%%%%%%%%%%%%%%%%%%%%%%%%%%%%%%%%%%%%%%%%%%%%%%%%%%%%%
\paragraph{Long 1D DRM model 1000:1 } ~

To show the wave propagation explicitly, a long 1D model (1000:1) similar to the
1D DRM model above was made in this section.

The           model           description           is          same          to
Fig.(\ref{fig_Program_description_for_the_1D_DRM_model})  except  this model use
far more soil elements.

The general view is shown in Fig.(\ref{fig_Long_1D_DRM_model}) below.

\begin{figure}[!htb]
  \centering
  \includegraphics[width=10cm]{../Figure-files/_Chapter_Appendix_Illustrative_Examples/long_DRM_full.png}
  \caption{Long 1D DRM model}
  \label{fig_Long_1D_DRM_model}
\end{figure}

There is still now outgoing waves at the exterior layers, which is shown in Fig(\ref{fig_Long_1D_DRM_model_exterior_layer}).
\begin{figure}[!htb]
  \centering
  \includegraphics[width=5cm]{../Figure-files/_Chapter_Appendix_Illustrative_Examples/long_DRM_part.png}
  \caption{Long 1D DRM model: exterior layer}
  \label{fig_Long_1D_DRM_model_exterior_layer}
\end{figure}


The ESSI model fei files for this example can be downloaded 
\href{https://github.com/BorisJeremic/Real-ESSI-Examples/blob/master/model_fei_file/8NodeBrick_DRM_1D_long/8NodeBrick_DRM_1D_long.tgz?raw=true}{here}.

The results can also be seen from this 
\href{http://sokocalo.engr.ucdavis.edu/~jeremic/lecture_notes_online_material/_Chapter_Applications_Earthquake_Soil_Structure_Interaction_General_Aspects/Animation_DRM_1D.mp4}{video}.














%%%%%%%%%%%%%%%%%%%%%%%%%%%%%%%%%%%%%%%%%%%%%%%%%%%%%%%%%%%%%%%%%%%%
\clearpage  
% add new page to avoid the figure confusion with previous section
%%%%%%%%%%%%%%%%%%%%%%%%%%%%%%%%%%%%%%%%%%%%%%%%%%%%%%%%%%%%%%%%%%%%












%%%%%%%%%%%%%%%%%%%%%%%%%%%%%%%%%%%%%%%%%%%%%%%%%%%%%%%%%%%%%%%%%%%%%%%%%%%%%%%%
%%%%%%%%%%%%%%%%%%%%%%%%%%%%%%%%%%%%%%%%%%%%%%%%%%%%%%%%%%%%%%%%%%%%%%%%%%%%%%%%
%%%%%%%%%%%%%%%%%%%%%%%%%%%%%%%%%%%%%%%%%%%%%%%%%%%%%%%%%%%%%%%%%%%%%%%%%%%%%%%%
%%%%%%%%%%%%%%%%%%%%%%%%%%%%%%%%%%%%%%%%%%%%%%%%%%%%%%%%%%%%%%%%%%%%%%%%%%%%%%%%
%%%%%%%%%%%%%%%%%%%%%%%%%%%%%%%%%%%%%%%%%%%%%%%%%%%%%%%%%%%%%%%%%%%%%%%%%%%%%%%%
\section{Three Dimensional DRM Model}

%%%%%%%%%%%%%%%%%%%%%%%%%%%%%%%%%%%%%%%%%%%%%%%%%%%%%%%%%%%%%%%%%%%%
\paragraph{Problem description:} ~

As shown in Fig.(\ref{fig The diagram for Domain Reduction Method DRM }), the DRM layer is used to add the earthquake motion. 


\begin{figure}[!htb]
  \centering
  \includegraphics[width=10cm]{../Figure-files/_Chapter_Appendix_Illustrative_Examples/DRM_3D_descp_3.pdf}
  \caption{The diagram for 3D Domain Reduction Method example.}
  \label{fig The diagram for Domain Reduction Method DRM }
\end{figure}


% In 2D, the DRM model is shown in Fig.(\ref{fig The diagram for Domain Reduction Method DRM 2d }). 
% \begin{figure}[!htb]
%   \centering
%   \includegraphics[width=10cm]{../Figure-files/DRM_diagram_2d.png}
%   \caption{The diagram for 2D Domain Reduction Method (DRM) }
%   \label{fig The diagram for Domain Reduction Method DRM 2d }
% \end{figure}




%%%%%%%%%%%%%%%%%%%%%%%%%%%%%%%%%%%%%%%%%%%%%%%%%%%%%%%%%%%%%%%%%%%%
\paragraph{Numerical result:} ~

\begin{figure}[!htb]
  \centering
  \includegraphics[width=15cm]{../Figure-files/_Chapter_Appendix_Illustrative_Examples/3d_drm_result429.png}
  \caption{Diagram for the 3D DRM model.}
  \label{fig Diagram for the 3D DRM model}
\end{figure}


%%%%%%%%%%%%%%%%%%%%%%%%%%%%%%%%%%%%%%%%%%%%%%%%%%%%%%%%%%%%%%%%%%%%
\paragraph{ESSI model fei file: } ~

%\lstinputlisting[frame=single]{../Figure-files/DRM_3D_script.fei}
\begin{lstlisting}
model name "DRM" ;

//Material for soil
add material # 1 type linear_elastic_isotropic_3d_LT
  mass_density = 2000*kg/m^3
  elastic_modulus = 1300*MPa
  poisson_ratio = 0.3;

//Material for DRM layer
add material # 2 type linear_elastic_isotropic_3d_LT
  mass_density = 2000*kg/m^3
  elastic_modulus = 1300*MPa
  poisson_ratio = 0.3;

//Material for exterior layer
add material # 3 type linear_elastic_isotropic_3d_LT
  mass_density = 2000*kg/m^3
  elastic_modulus = 1300*MPa
  poisson_ratio = 0.3;

//
add node #  1 at (   0.00*m,   0.00*m,   0.00*m) with 3 dofs;
add node #  2 at (   50.00*m,   0.00*m,   0.00*m) with 3 dofs;
add node #  3 at (   5.00*m,   0.00*m,   0.00*m) with 3 dofs;
add node #  4 at (   10.00*m,   0.00*m,   0.00*m) with 3 dofs;
add node #  5 at (   15.00*m,   0.00*m,   0.00*m) with 3 dofs;
add node #  6 at (   20.00*m,   0.00*m,   0.00*m) with 3 dofs;
add node #  7 at (   25.00*m,   0.00*m,   0.00*m) with 3 dofs;
...
...
add node # 2925 at (   55.00*m,   55.00*m,   -5.00*m) with 3 dofs;

//
add element #  1 type 8NodeBrickLT with nodes(  1,  40,  41,  3, 150, 441, 603, 151) use material # 1;
add element #  2 type 8NodeBrickLT with nodes(  3,  41,  50,  4, 151, 603, 684, 160) use material # 1;
...
add element # 2352 type 8NodeBrickLT with nodes( 2925, 2924, 2922, 2923, 2921, 2920, 2918, 2919) use material # 3;

//
fix node #  1332 dofs all  ;
fix node #  1334 dofs all  ;
...
...
fix node #  2924 dofs all  ;

new loading stage "3D";
add domain reduction method loading #  1
  hdf5_file = "input.hdf5";

define algorithm With_no_convergence_check ;
define solver ProfileSPD;
define dynamic integrator Newmark with  gamma = 0.5  beta = 0.25;

simulate 999 steps using transient algorithm  time_step = 0.01*s;

bye;
\end{lstlisting}

The    ESSI   model   fei   files   for   this   example   can   be   downloaded
\href{https://github.com/BorisJeremic/Real-ESSI-Examples/blob/master/model_fei_file/8NodeBrick_DRM_3D/8NodeBrick_DRM_3D.tgz?raw=true}{here}.

The   same   model  for  this  example  with  27NodeBrickLT  can  be  downloaded
\href{https://github.com/BorisJeremic/Real-ESSI-Examples/blob/master/model_fei_file/27NodeBrick_DRM_3D/27NodeBrick_DRM_3D.tgz?raw=true}{here}.






%%%%%%%%%%%%%%%%%%%%%%%%%%%%%%%%%%%%%%%%%%%%%%%%%%%%%%%%%%%%%%%%%%%%%%%%%%%%%%%%
%%%%%%%%%%%%%%%%%%%%%%%%%%%%%%%%%%%%%%%%%%%%%%%%%%%%%%%%%%%%%%%%%%%%%%%%%%%%%%%%
%%%%%%%%%%%%%%%%%%%%%%%%%%%%%%%%%%%%%%%%%%%%%%%%%%%%%%%%%%%%%%%%%%%%%%%%%%%%%%%%
%%%%%%%%%%%%%%%%%%%%%%%%%%%%%%%%%%%%%%%%%%%%%%%%%%%%%%%%%%%%%%%%%%%%%%%%%%%%%%%%
%%%%%%%%%%%%%%%%%%%%%%%%%%%%%%%%%%%%%%%%%%%%%%%%%%%%%%%%%%%%%%%%%%%%%%%%%%%%%%%%
%%%%%%%%%%%%%%%%%%%%%%%%%%%%%%%%%%%%%%%%%%%%%%%%%%%%%%%%%%%%%%%%%%%%
%%%%%%%%%%%%%%%%%%%%%%%%%%%%%%%%%%%%%%%%%%%%%%%%%%%%%%%%%%%%%%%%%%%%
\section{ShearBeam Element for Pisano Materials}

%%%%%%%%%%%%%%%%%%%%%%%%%%%%%%%%%%%%%%%%%%%%%%%%%%%%%%%%%%%%%%%%%%%%
\paragraph{Problem description:} ~

In  the  element  type  "ShearBeamLT",  only  one Gauss point exists. 
ShearBeamLT  element  was used here to test the Pisan{\'o}
material model.

Vertical    force    $F_z$   was used to apply confinement to the element. Then,
cyclic force  $F_x$ is  used to load. 
point.

\begin{figure}[!htb]
  \centering
  \includegraphics[width=10cm]{../Figure-files/_Chapter_Appendix_Illustrative_Examples/pisano_descrip.pdf}
  \caption{ShearBeam element.}
  \label{fig_ShearBeam}
\end{figure}



%%%%%%%%%%%%%%%%%%%%%%%%%%%%%%%%%%%%%%%%%%%%%%%%%%%%%%%%%%%%%%%%%%%%
\paragraph{Results} ~

Resulting stress-strain relationship is shown in Fig.(\ref{fig_ShearBeam_result}). 

\begin{figure}[!htb]
  \centering
  \includegraphics[width=12cm]{../Figure-files/_Chapter_Appendix_Illustrative_Examples/pisanoLT_test01.png}
  \caption{Shear stress-strain response.}
  \label{fig_ShearBeam_result}
\end{figure}

%%%%%%%%%%%%%%%%%%%%%%%%%%%%%%%%%%%%%%%%%%%%%%%%%%%%%%%%%%%%%%%%%%%%
\paragraph{ESSI model fei file: } ~

%\lstinputlisting[frame=single]{../Figure-files/2test_pisano_01.fei}
\begin{lstlisting}
model name "pisanoLT";

add node # 1 at (0*m,0*m,0*m)  with 3 dofs;
add node # 2 at (0*m,0*m,1*m)  with 3 dofs;

fix node # 1 dofs all;
fix node # 2 dofs uy;

add material # 1 type New_PisanoLT
 mass_density = 2000*kg/m^3
 elastic_modulus_1atm =  325*MPa  poisson_ratio =  0.3
 M_in =  1.4   kd_in =  0.0  xi_in =  0.0  h_in =  700  m_in =  0.7
 initial_confining_stress =  0*kPa n_in = 0  a_in = 0.0  eplcum_cr_in = 1e-6;

add element # 1 type ShearBeamLT with nodes (1, 2) \
     cross_section = 1*m^2 use material # 1;

new loading stage "confinement";

add load # 1 to node # 2 type linear Fz = -200*kN;
define load factor increment 0.01;
define algorithm With_no_convergence_check ;
define solver  UMFPack;
simulate 100 steps using static algorithm;

new loading stage "test01";
gamma_max = 3e-3;
add imposed motion # 2 to node # 2 dof ux
 displacement_scale_unit = gamma_max*m displacement_file = "input_sine.txt"
 velocity_scale_unit = gamma_max*m/s velocity_file = "input_sine.txt"
 acceleration_scale_unit = gamma_max*m/s^2 acceleration_file = "input_sine.txt";

define load factor increment 0.0005;
define algorithm With_no_convergence_check;
define solver  UMFPack;
simulate 2000 steps using static algorithm;

bye;
\end{lstlisting}

The    ESSI   model   fei   files   for   this   example   can   be   downloaded
\href{https://github.com/BorisJeremic/Real-ESSI-Examples/blob/master/model_fei_file/shearbeam_pisano_plastic/shearbeam_pisano_plastic.tgz?raw=true}{here}.


























%%%%%%%%%%%%%%%%%%%%%%%%%%%%%%%%%%%%%%%%%%%%%%%%%%%%%%%%%%%%%%%%%%%%%%%%%%%%%%%%
%%%%%%%%%%%%%%%%%%%%%%%%%%%%%%%%%%%%%%%%%%%%%%%%%%%%%%%%%%%%%%%%%%%%%%%%%%%%%%%%
%%%%%%%%%%%%%%%%%%%%%%%%%%%%%%%%%%%%%%%%%%%%%%%%%%%%%%%%%%%%%%%%%%%%%%%%%%%%%%%%
%%%%%%%%%%%%%%%%%%%%%%%%%%%%%%%%%%%%%%%%%%%%%%%%%%%%%%%%%%%%%%%%%%%%%%%%%%%%%%%%
%%%%%%%%%%%%%%%%%%%%%%%%%%%%%%%%%%%%%%%%%%%%%%%%%%%%%%%%%%%%%%%%%%%%%%%%%%%%%%%%
%%%%%%%%%%%%%%%%%%%%%%%%%%%%%%%%%%%%%%%%%%%%%%%%%%%%%%%%%%%%%%%%%%%%
%%%%%%%%%%%%%%%%%%%%%%%%%%%%%%%%%%%%%%%%%%%%%%%%%%%%%%%%%%%%%%%%%%%%
\section{8NodeBrickLT Element for Drucker Prager Armstrong Federick Material}

%%%%%%%%%%%%%%%%%%%%%%%%%%%%%%%%%%%%%%%%%%%%%%%%%%%%%%%%%%%%%%%%%%%%
\paragraph{Problem description:} ~

This example is used to test the materials properties, such as G/Gmax against strains. The element type is 8NodeBrickLT. And there are two stages of loading. The first loading stage is confinement and the second loading stage is shearing. 

The boundary condition is specially designed such that each Gauss point has the same stress state. 





%%%%%%%%%%%%%%%%%%%%%%%%%%%%%%%%%%%%%%%%%%%%%%%%%%%%%%%%%%%%%%%%%%%%
\paragraph{Results} ~

Resulting stress-strain relationship is shown in Fig.(\ref{fig_8nodebrick_result}). 

\begin{figure}[!htb]
  \centering
  \includegraphics[width=12cm]{../Figure-files/_Chapter_Appendix_Illustrative_Examples/drucker_prager_armstrong_federick.pdf}
  \caption{Shear stress-strain response.}
  \label{fig_8nodebrick_result}
\end{figure}

%%%%%%%%%%%%%%%%%%%%%%%%%%%%%%%%%%%%%%%%%%%%%%%%%%%%%%%%%%%%%%%%%%%%
\paragraph{ESSI model fei file: } ~

%\lstinputlisting[frame=single]{../Figure-files/2test_pisano_01.fei}

\begin{lstlisting}
// Drucker Prager Armstrong Frederick
// This model is created by Jose.
model name "druckeraf";

// Parameters:
phi   = 5;
ha      = 1000;
cr      = 973;
gam        = 0.01;
Ncyc       = 5;
Nsteps     = 1000;
H=1;
vp=1000*m/s;
vs=500*m/s; 
rho=2000*kg/m^3;
p0 = 250*kPa;
G = rho*vs^2;
M = rho*vp^2;

//From wiki (https://en.wikipedia.org/wiki/Elastic_modulus)
E = G*(3*M-4*G)/(M-G);
nu = (M-2*G)/(2*M-2*G);

K0 = 1.0;
phirad = pi*phi/180;
M = 6*sin(phirad)/(3-sin(phirad));

// Define the material:
add material # 1 type DruckerPragerArmstrongFrederickLT
    mass_density = 0*kg/m^3 
    elastic_modulus =  E
    poisson_ratio =  nu
    druckerprager_k = M
    armstrong_frederick_ha = ha*Pa 
    armstrong_frederick_cr = cr*Pa
    isotropic_hardening_rate =  0*E
    initial_confining_stress = 1*Pa;

// define the node:
add node # 1 at (0*m,0*m,1*m)  with 3 dofs;
add node # 2 at (1*m,0*m,1*m)  with 3 dofs;
add node # 3 at (1*m,1*m,1*m)  with 3 dofs;
add node # 4 at (0*m,1*m,1*m)  with 3 dofs;

add node # 5 at (0*m,0*m,0*m)  with 3 dofs;
add node # 6 at (1*m,0*m,0*m)  with 3 dofs;
add node # 7 at (1*m,1*m,0*m)  with 3 dofs;
add node # 8 at (0*m,1*m,0*m)  with 3 dofs;

// add equal degree of freedom in three directions
add constraint equal dof with master node # 2 and slave node # 3 dof to constrain ux;
add constraint equal dof with master node # 2 and slave node # 6 dof to constrain ux;
add constraint equal dof with master node # 2 and slave node # 7 dof to constrain ux;

add constraint equal dof with master node # 3 and slave node # 4 dof to constrain uy;
add constraint equal dof with master node # 3 and slave node # 8 dof to constrain uy;
add constraint equal dof with master node # 3 and slave node # 7 dof to constrain uy;

add constraint equal dof with master node # 1 and slave node # 2 dof to constrain uz;
add constraint equal dof with master node # 1 and slave node # 3 dof to constrain uz;
add constraint equal dof with master node # 1 and slave node # 4 dof to constrain uz;

// Define the element.
add element # 1 type 8NodeBrickLT with nodes (1, 2,3 , 4, 5, 6,7, 8) use material # 1;

new loading stage "confinement";
fix node # 1 dofs ux uy;
fix node # 2 dofs uy;
fix node # 4 dofs ux;

fix node # 5 dofs ux uy uz;
fix node # 6 dofs uy uz;
fix node # 7 dofs uz;
fix node # 8 dofs ux uz;

sigma_z = -3*p0/(1+2*K0);
sigma_x = K0*sigma_z;
sigma_y = K0*sigma_z;

//Z-face
add load # 1 to node # 1 type linear  Fz = sigma_z*m^2/4;
add load # 2 to node # 2 type linear  Fz = sigma_z*m^2/4;
add load # 3 to node # 3 type linear  Fz = sigma_z*m^2/4;
add load # 4 to node # 4 type linear  Fz = sigma_z*m^2/4;

//X-face
add load # 5 to node # 2 type linear  Fx = sigma_x*m^2/4;
add load # 6 to node # 6 type linear  Fx = sigma_x*m^2/4;
add load # 7 to node # 7 type linear  Fx = sigma_x*m^2/4;
add load # 8 to node # 3 type linear  Fx = sigma_x*m^2/4;

add load # 9 to node # 3 type linear   Fy = sigma_y*m^2/4;
add load # 10 to node # 7 type linear  Fy = sigma_y*m^2/4;
add load # 11 to node # 8 type linear  Fy = sigma_y*m^2/4;
add load # 12 to node # 4 type linear  Fy = sigma_y*m^2/4;

Nsteps_static=100;
define load factor increment 1/Nsteps_static;

define solver  UMFPack;
define convergence test Norm_Displacement_Increment  
    tolerance =  1e-6
    maximum_iterations =  100
    verbose_level = 4;
define algorithm Newton ;

define NDMaterialLT constitutive integration algorithm Euler_One_Step
    yield_function_relative_tolerance =  0.002
    stress_relative_tolerance =  0.002
    maximum_iterations = 1000;

simulate Nsteps_static steps using static algorithm;


new loading stage "shearing";
compute reaction forces;
add load # 13 to node # 1 type from_reactions;
add load # 14 to node # 4 type from_reactions;

free node # 1 dofs ux;
free node # 4 dofs ux;
fix node # 3 dofs uy;
fix node # 6 dofs ux;
fix node # 7 dofs ux uy;
fix node # 8 dofs uy;

add constraint equal dof with master node # 1 and slave node # 3 dof to constrain ux;
add constraint equal dof with master node # 1 and slave node # 4 dof to constrain ux;
add constraint equal dof with master node # 1 and slave node # 2 dof to constrain ux;
remove constraint equaldof node # 6;
remove constraint equaldof node # 7;
remove constraint equaldof node # 8;

n = 1;
while(n<=1)
{
    add load # 14+n to node # n type path_time_series 
     Fx = 170.*kN 
     series_file = "path.txt";        
    n+=1;
}

define load factor increment 1/Nsteps;

define solver  UMFPack;
define convergence test Norm_Displacement_Increment  
    tolerance =  1e-5
    maximum_iterations =  100
    verbose_level = 4;
define algorithm Newton ;

define NDMaterialLT constitutive integration algorithm Euler_One_Step
    yield_function_relative_tolerance =  0.0002
    stress_relative_tolerance =  0.002
    maximum_iterations = 1000;

simulate Ncyc*Nsteps steps using static algorithm;

bye;    
\end{lstlisting}

The    ESSI   model   fei   files   for   this   example   can   be   downloaded
\href{https://github.com/BorisJeremic/Real-ESSI-Examples/blob/master/model_fei_file/shearbeam_pisano_plastic8NodeBrickLT_DruckerPragerArmstrongFrederick/8NodeBrickLT_DruckerPragerArmstrongFrederick.tgz?raw=true}{here}.









\section{Beam theory}

Problem description: Length=6m, Width=1m, Height=1m, F=100N, E=1E8Pa, $\nu=0.0$. The force direction was shown in Figure (\ref{fig Problem description for cantilever beam theory}). 

\begin{figure}[H]
  \centering
  \includegraphics[width=7cm]{../Figure-files/cantilever_6.pdf}
  \caption{Problem description for cantilever beams}
  \label{fig Problem description for cantilever beam theory}
\end{figure}


The basic idea to calculate the shear deformation of a beam is 

\begin{equation}
  \delta=\frac{FL}{GA_v}
\end{equation}

where $A_v$ is the not the gross cross sectional area of the beam. $A_v$ should be the shear area. Thus,

\begin{equation}
  \kappa = \frac{A}{A_v}
\end{equation}

where $\kappa$ is the form factor, shear correction factor or shear deformation coefficient, $A$ is the gross sectional area and $A_v$ is the shear area of the section. 

The history of $\kappa$ value is long. 
\begin{enumerate}
  \item Timoshenko (1940) \footnote{Strength of materials, Timoshenko, Krieger Pub Co, 1940} define the form factor for rectangular section is 1.5. 
  \item Cowper (1970) \footnote{Cowper, G. R. "The shear coefficient in Timoshenko’s beam theory." Journal of applied mechanics 33.2 (1966): 335-340.} gave the formula for the form factor:
    \begin{equation}
      \kappa=\frac{12+11\nu}{10(1+\nu)}
    \end{equation}
  \item Renton (1991) \footnote{Renton, J. D. "Generalized beam theory applied to shear stiffness." International Journal of Solids and Structures 27.15 (1991): 1955-1967.}  provided a closed form solution for shear area of rectangular sections. For a rectangular section of depth $2a$ and breadth $2b$.
    \begin{equation}
      \kappa=\frac{6}{5}+ (\frac{\nu}{1+\nu})^2 \sum_{m=0}^{\infty}\sum_{n=1}^{\infty} \frac{144(b/a)^4}{\pi^6 (2m+1)^2 n^2 [(2m+1)^2(b/2a)^2+n^2]}
    \end{equation}
\end{enumerate}





For square cross section, $b=a$, therefore, 
\begin{equation}
  \kappa= \frac{6}{5}+ (\frac{\nu}{1+\nu})^2 \sum_{m=0}^{\infty}\sum_{n=1}^{\infty} \frac{144}{\pi^6 (2m+1)^2 n^2 [(2m+1)^2(1/2)^2+n^2]}
\end{equation}

The summation of the series are very hard. $Matlab$ and $Mathematica$ cannot solve it directly. According to the Renton (1991), the intermediate values are given by
\begin{equation}
  \kappa=\frac{6}{5}+ C_1 (\frac{\nu}{1+\nu})^2 (\frac{b}{a})^4
\end{equation}

When $b=a$, the equation becomes
\begin{equation}
  \kappa=\frac{6}{5}+ 0.1392 (\frac{\nu}{1+\nu})^2 
\end{equation}






\newpage



\section{Verification of 8NodeBrick elements}
\vskip 24pt

\subsection{Verification of 8NodeBrick cantilever beams}




Problem description: Length=6m, Width=1m, Height=1m, Force=100N, E=1E8Pa, $\nu=0.0$. Use the shear deformation coefficient $\kappa=1.2$. The force direction was shown in Figure (\ref{fig Problem description for cantilever beams}). 

\begin{figure}[H]
  \centering
  \includegraphics[width=7cm]{../Figure-files/cantilever_6.pdf}
  \caption{Problem description for cantilever beams}
  \label{fig Problem description for cantilever beams}
\end{figure}


Theoretical displacement (bending and shear deformation):
\begin{equation}
  \begin{aligned}
  d &=\frac{FL^3}{3EI}+\frac{FL}{GA_v} \\
  &= \frac{FL^3}{3E\frac{bh^3}{12}}+\frac{FL}{\frac{E}{2(1+\nu)} \frac{bh}{\kappa}} \\ 
    &= \frac{100 N \times 6^3 m^3}{3\times 10^8 N/m^2 \times \frac{1}{12} m^4}+ 
    \frac{100 N\times 6 m}{\frac{10}{2} \times 10^7 N/m^2\times 1 m^2 \times \frac{5}{6}} \\ 
    &=8.64\times 10^{-4} m + 0.144 \times 10^{-4} m   \\
   & =8.784\times 10^{-4} \ m
   \end{aligned}
\end{equation}



Numerical model:



The 8NodeBrick elements were shown in Figure (\ref{fig 8NodeBrick elements for cantilever beams}).

\begin{figure}[H]
  \centering
  \begin{subfigure}{0.5\textwidth}
    \centering
    \includegraphics[width=9cm]{../Figure-files/beam_8brick_1div.pdf}
    \caption{One 8NodeBrick element}
  \end{subfigure}
  \vskip 8pt
  \begin{subfigure}{0.5\textwidth}
    \centering
    \includegraphics[width=9cm]{../Figure-files/beam_8brick_2div.pdf}
    \caption{Two 8NodeBrick elements}
  \end{subfigure}
  \vskip 8pt
  \begin{subfigure}{0.5\textwidth}
    \centering
    \includegraphics[width=9cm]{../Figure-files/beam_8brick_6div.pdf}
    \caption{Six 8NodeBrick elements}
  \end{subfigure}
  \captionsetup{justification=centering,margin=3cm}
  \caption{8NodeBrick elements for cantilever beams}
  \label{fig 8NodeBrick elements for cantilever beams}
\end{figure}


% \begin{figure}[H]
%   \centering
%   \includegraphics[width=9cm]{../Figure-files/beam_8brick_2div.pdf}
%   % \caption{}
%   % \label{}
% \end{figure}

% \begin{figure}[H]
%   \centering
%   \includegraphics[width=9cm]{../Figure-files/beam_8brick_6div.pdf}
%   % \caption{}
%   % \label{}
% \end{figure}



An example ESSI script is shown below.




All the ESSI results were listed in Table (\ref{table 8NodeBrick cantilever beams results for different element number}). 
The theoretical solution is 8.784E-04 $m$.
\begin{table}[H]
  \centering
    \caption{Results for 8NodeBrick cantilever beams of different element numbers}
    \begin{tabular}{|c|c|c|c|}
      \hline
      Element number & 1        & 2        & 6         \\  \hline
      8NodeBrick     & 4.61E-05 $m$ & 1.59E-04 $m$ & 5.84E-04 $m$     \\ \hline
      Error           & 94.75\%  & 81.87\%  & 33.52\%           \\ 
      \hline 
    \end{tabular}
    \label{table 8NodeBrick cantilever beams results for different element number}
\end{table}

The errors were plotted in Figure (\ref{fig error 8NodeBrick cantilever beam for different element number}).
\begin{figure}[H]
  % \centering
  % \begin{subfigure}{0.5\textwidth}
    \centering
    \includegraphics[width=6cm]{../Figure-files/error8brick_beam_different_element_number.jpeg}
  % \end{subfigure}
  % \begin{subfigure}{0.5\textwidth}
    % \centering
    % \includegraphics[width=7cm]{../Figure-files/error8brick_beam_different_element_number100.jpeg}
  % \end{subfigure}
  \captionsetup{justification=centering,margin=3cm}
  \caption{8NodeBrick cantilever beam for different element number\\
    Displacement error   versus   Number of elements}
  \label{fig error 8NodeBrick cantilever beam for different element number}
\end{figure}







The ESSI model fei files for the table above are \href{https://github.com/yuan-energy/ESSI_Verification/blob/master/8NodeBrick/cantilever_different_element_number/cantilever_different_element_number.tar.gz?raw=true}{here}






% \newpage
% \begin{itemize}
%   \item \textbf{\emph{Cantilever: different geometry}}
% \end{itemize}

% In the figures above, only the model with geometry $6m\times 1m \times 1m$ was drawed. In the ESSI models, the geometry $10m\times 1m \times 1m$ and the geometry $20m\times 1m \times 1m$ were also calculated. In three different geometry models, all the element sizes were $1m\times 1m \times 1m$. Therefore, the number of elements used in each model were $6,\ 10\ and\ 20$ respectively.

% The ESSI results for different geometry were listed in Table (\ref{table Results for 8NodeBrick cantilever beams of different geometry}). 

% \begin{table}[H]
%   \centering
%   \caption{Results for 8NodeBrick cantilever beams of different geometry}
%   \label{table Results for 8NodeBrick cantilever beams of different geometry}
%   \begin{tabular}{|c|c|c|c|c|c|}
%   \hline
%   Geometry & 8NodeBrick & Theoretical(bending) & Theoretical(shear) & Theoretical(all) & Error   \\ \hline
%   1:6      & 5.84E-04 $m$ & 8.64E-04      $m$       & 1.20E-05    $m$       & 8.76E-04  $m$       & 33.33\% \\ \hline
%   1:10     & 2.68E-03 $m$ & 4.00E-03      $m$       & 2.00E-05    $m$       & 4.02E-03  $m$       & 33.33\% \\ \hline
%   1:20     & 2.14E-02 $m$ & 3.20E-02      $m$       & 4.00E-05    $m$       & 3.20E-02  $m$       & 33.33\% \\
%   \hline
%   \end{tabular}
% \end{table}

% The ESSI model fei files for the table above are \href{https://github.com/yuan-energy/ESSI_Verification/blob/master/8NodeBrick/cantilever_different_geometry/cantilever_different_geometry.tar.gz?raw=true}{here}






\newpage
\subsection{Verification of 8NodeBrick cantilever beam for different Poisson's ratio}




Problem description: Length=6m, Width=1m, Height=1m, Force=100N, E=1E8Pa, $\nu=0.0-0.49$. The force direction was shown in Figure (\ref{fig Problem description for cantilever beams of different Poisson's ratios}). 

\begin{figure}[H]
  \centering
  \includegraphics[width=7cm]{../Figure-files/cantilever_6.pdf}
  \caption{Problem description for cantilever beams of different Poisson's ratios}
  \label{fig Problem description for cantilever beams of different Poisson's ratios}
\end{figure}

The theoretical solution for $\nu=0.0$ was calculated below, while the solution for other Poisson's ratio were calculated by the similar process. 

Theoretical displacement (bending and shear deformation):
\begin{equation}
  \begin{aligned}
  d &=\frac{FL^3}{3EI}+\frac{FL}{GA_v} \\
  &= \frac{FL^3}{3E\frac{bh^3}{12}}+\frac{FL}{\frac{E}{2(1+\nu)} \frac{bh}{\kappa}} \\ 
    &= \frac{100 N \times 6^3 m^3}{3\times 10^8 N/m^2 \times \frac{1}{12} m^4}+ 
    \frac{100 N\times 6 m}{\frac{10}{2} \times 10^7 N/m^2\times 1 m^2 \times \frac{5}{6}} \\ 
    &=8.64\times 10^{-4} m + 0.144 \times 10^{-4} m   \\
   & =8.784\times 10^{-4} \ m
   \end{aligned}
\end{equation}

The rotation angle at the end:
\begin{equation}
  \theta =\frac{FL^2}{2EI} 
   =\frac{100 N \times 6^2 m^2} {2\times 10^8 N/m^2 \times \frac{1}{12} m^4} 
 =2.16 \times 10^{-4} \ rad = 0.0124 \degree 
\end{equation}

The 8NodeBrick elements for cantilever beams of different Poisson's ratios were shown in Figure (\ref{fig 8NodeBrick elements for cantilever beams of different Poisson's ratios}):
\begin{figure}[H]
  \centering
  \includegraphics[width=9cm]{../Figure-files/beam_8brick_6div.pdf}
  \caption{8NodeBrick elements for cantilever beams of different Poisson's ratios}
  \label{fig 8NodeBrick elements for cantilever beams of different Poisson's ratios}
\end{figure}


All the displacement results were listed in Table (\ref{table Displacement results for 8NodeBrick cantilever beams of different Poisson ratios}) - (\ref{table Displacement results for 8NodeBrick cantilever beams of different Poisson ratios div4}). 

\begin{table}[H]
  \centering
  \captionsetup{justification=centering,margin=3cm}
  \caption{\emph{\textbf{Displacement}} results for 8NodeBrick cantilever beams with \textcolor{red}{element side length 1 m}}
  \label{table Displacement results for 8NodeBrick cantilever beams of different Poisson ratios}
  \begin{tabular}{|c|c|c|c|c|c|}
    \hline 
\tabincell{c}{Poisson's \\ ratio}    & \tabincell{c}{8NodeBrick\\displacement} & \tabincell{c}{Theory displacement\\(bending)} & \tabincell{c}{Theory displacement\\(shear)} & \tabincell{c}{Theory\\displacement(all)}   & Error \\ \hline
0.00 & 5.840E-04 $m$ & 8.640E-04 $m$ & 1.440E-05 $m$  & 8.784E-04 $m$  & 33.52\%   \\ \hline
0.05 & 5.924E-04 $m$ & 8.640E-04 $m$ & 1.512E-05 $m$  & 8.791E-04 $m$  & 32.62\%   \\ \hline
0.10 & 5.969E-04 $m$ & 8.640E-04 $m$ & 1.586E-05 $m$  & 8.799E-04 $m$  & 32.16\%   \\ \hline
0.15 & 5.971E-04 $m$ & 8.640E-04 $m$ & 1.659E-05 $m$  & 8.806E-04 $m$  & 32.20\%   \\ \hline
0.20 & 5.922E-04 $m$ & 8.640E-04 $m$ & 1.734E-05 $m$  & 8.813E-04 $m$  & 32.81\%   \\ \hline
0.25 & 5.814E-04 $m$ & 8.640E-04 $m$ & 1.808E-05 $m$  & 8.821E-04 $m$  & 34.09\%   \\ \hline
0.30 & 5.634E-04 $m$ & 8.640E-04 $m$ & 1.884E-05 $m$  & 8.828E-04 $m$  & 36.19\%   \\ \hline
0.35 & 5.364E-04 $m$ & 8.640E-04 $m$ & 1.959E-05 $m$  & 8.836E-04 $m$  & 39.29\%   \\ \hline
0.40 & 4.970E-04 $m$ & 8.640E-04 $m$ & 2.035E-05 $m$  & 8.844E-04 $m$  & 43.80\%   \\ \hline
0.45 & 4.353E-04 $m$ & 8.640E-04 $m$ & 2.111E-05 $m$  & 8.851E-04 $m$  & 50.82\%   \\ \hline
0.49 & 3.142E-04 $m$ & 8.640E-04 $m$ & 2.173E-05 $m$  & 8.857E-04 $m$  & 64.52\%   \\ \hline
  \end{tabular}
  % \caption{}
\end{table}



Then, in the same geometry, the element side length was cut into 0.5m. 

\begin{table}[H]
  \centering
  \captionsetup{justification=centering,margin=3cm}
  \caption{\emph{\textbf{Displacement}} results for 8NodeBrick cantilever beams with \textcolor{red}{element side length 0.5 m}}
  \label{table Displacement results for 8NodeBrick cantilever beams of different Poisson ratios div2}
  \begin{tabular}{|c|c|c|c|c|c|}
    \hline 
\tabincell{c}{Poisson's \\ ratio}    & \tabincell{c}{8NodeBrick\\displacement} & \tabincell{c}{Theory displacement\\(bending)} & \tabincell{c}{Theory displacement\\(shear)} & \tabincell{c}{Theory\\displacement(all)}   & Error \\ \hline
0.00 & 7.787E-04 $m$ & 8.640E-04 $m$ & 1.440E-05 $m$  & 8.784E-04 $m$ & 11.35\%    \\ \hline
0.05 & 7.824E-04 $m$ & 8.640E-04 $m$ & 1.512E-05 $m$  & 8.791E-04 $m$ & 11.00\%    \\ \hline
0.10 & 7.839E-04 $m$ & 8.640E-04 $m$ & 1.586E-05 $m$  & 8.799E-04 $m$ & 10.91\%    \\ \hline
0.15 & 7.829E-04 $m$ & 8.640E-04 $m$ & 1.659E-05 $m$  & 8.806E-04 $m$ & 11.09\%    \\ \hline
0.20 & 7.790E-04 $m$ & 8.640E-04 $m$ & 1.734E-05 $m$  & 8.813E-04 $m$ & 11.61\%    \\ \hline
0.25 & 7.717E-04 $m$ & 8.640E-04 $m$ & 1.808E-05 $m$  & 8.821E-04 $m$ & 12.51\%    \\ \hline
0.30 & 7.597E-04 $m$ & 8.640E-04 $m$ & 1.884E-05 $m$  & 8.828E-04 $m$ & 13.95\%    \\ \hline
0.35 & 7.406E-04 $m$ & 8.640E-04 $m$ & 1.959E-05 $m$  & 8.836E-04 $m$ & 16.18\%    \\ \hline
0.40 & 7.089E-04 $m$ & 8.640E-04 $m$ & 2.035E-05 $m$  & 8.844E-04 $m$ & 19.84\%    \\ \hline
0.45 & 6.466E-04 $m$ & 8.640E-04 $m$ & 2.111E-05 $m$  & 8.851E-04 $m$ & 26.95\%    \\ \hline
0.49 & 4.990E-04 $m$ & 8.640E-04 $m$ & 2.173E-05 $m$  & 8.857E-04 $m$ & 43.66\%    \\ \hline
  \end{tabular}
  % \caption{}
\end{table}

Finally, in the same geometry, the element side length was cut into 0.25m. 

\begin{table}[H]
  \centering
  \captionsetup{justification=centering,margin=3cm}
  \caption{\emph{\textbf{Displacement}} results for 8NodeBrick cantilever beams with \textcolor{red}{element side length 0.25 m}}
  \label{table Displacement results for 8NodeBrick cantilever beams of different Poisson ratios div4}
  \begin{tabular}{|c|c|c|c|c|c|}
    \hline 
\tabincell{c}{Poisson's \\ ratio}    & \tabincell{c}{8NodeBrick\\displacement} & \tabincell{c}{Theory displacement\\(bending)} & \tabincell{c}{Theory displacement\\(shear)} & \tabincell{c}{Theory\\displacement(all)}   & Error \\ \hline
0.00 & 8.511E-04 $m$ & 8.640E-04 $m$ & 1.440E-05 $m$  & 8.784E-04 $m$  & 3.11\%     \\ \hline
0.05 & 8.525E-04 $m$ & 8.640E-04 $m$ & 1.512E-05 $m$  & 8.791E-04 $m$  & 3.03\%     \\ \hline
0.10 & 8.527E-04 $m$ & 8.640E-04 $m$ & 1.586E-05 $m$  & 8.799E-04 $m$  & 3.09\%     \\ \hline
0.15 & 8.518E-04 $m$ & 8.640E-04 $m$ & 1.659E-05 $m$  & 8.806E-04 $m$  & 3.27\%     \\ \hline
0.20 & 8.494E-04 $m$ & 8.640E-04 $m$ & 1.734E-05 $m$  & 8.813E-04 $m$  & 3.62\%     \\ \hline
0.25 & 8.455E-04 $m$ & 8.640E-04 $m$ & 1.808E-05 $m$  & 8.821E-04 $m$  & 4.15\%     \\ \hline
0.30 & 8.393E-04 $m$ & 8.640E-04 $m$ & 1.884E-05 $m$  & 8.828E-04 $m$  & 4.93\%     \\ \hline
0.35 & 8.299E-04 $m$ & 8.640E-04 $m$ & 1.959E-05 $m$  & 8.836E-04 $m$  & 6.08\%     \\ \hline
0.40 & 8.141E-04 $m$ & 8.640E-04 $m$ & 2.035E-05 $m$  & 8.844E-04 $m$  & 7.94\%     \\ \hline
0.45 & 7.801E-04 $m$ & 8.640E-04 $m$ & 2.111E-05 $m$  & 8.851E-04 $m$  & 11.86\%    \\ \hline
0.49 & 6.603E-04 $m$ & 8.640E-04 $m$ & 2.173E-05 $m$  & 8.857E-04 $m$  & 25.45\%    \\ \hline
  \end{tabular}
  % \caption{}
\end{table}





% \begin{table}[H]
%   \centering
%   \caption{\emph{\textbf{Displacement}} results for 8NodeBrick cantilever beams of different Poisson's ratios}
%   \label{table Displacement results for 8NodeBrick cantilever beams of different Poisson ratios}
%   \begin{tabular}{|c|c|c|c|c|c|}
%     \hline 
% \tabincell{c}{Poisson's \\ ratio}    & \tabincell{c}{8NodeBrick\\displacement} & \tabincell{c}{Theory displacement\\(bending)} & \tabincell{c}{Theory displacement\\(shear)} & \tabincell{c}{Theory\\displacement(all)}   & Error \\ \hline
% 0.00 & 5.840E-04 $m$ & 8.640E-04 $m$ & 1.200E-05  $m$  & 8.740E-04 $m$  & 33.33\%    \\ \hline
% 0.05 & 5.924E-04 $m$ & 8.640E-04 $m$ & 1.260E-05  $m$  & 8.745E-04 $m$  & 32.42\%    \\ \hline
% 0.10 & 5.969E-04 $m$ & 8.640E-04 $m$ & 1.320E-05  $m$  & 8.750E-04 $m$  & 31.95\%    \\ \hline
% 0.15 & 5.971E-04 $m$ & 8.640E-04 $m$ & 1.380E-05  $m$  & 8.755E-04 $m$  & 31.98\%    \\ \hline
% 0.20 & 5.922E-04 $m$ & 8.640E-04 $m$ & 1.440E-05  $m$  & 8.760E-04 $m$  & 32.58\%    \\ \hline
% 0.25 & 5.814E-04 $m$ & 8.640E-04 $m$ & 1.500E-05  $m$  & 8.765E-04 $m$  & 33.86\%    \\ \hline
% 0.30 & 5.634E-04 $m$ & 8.640E-04 $m$ & 1.560E-05  $m$  & 8.770E-04 $m$  & 35.95\%    \\ \hline
% 0.35 & 5.364E-04 $m$ & 8.640E-04 $m$ & 1.620E-05  $m$  & 8.775E-04 $m$  & 39.06\%    \\ \hline
% 0.40 & 4.970E-04 $m$ & 8.640E-04 $m$ & 1.680E-05  $m$  & 8.780E-04 $m$  & 43.57\%    \\ \hline
% 0.45 & 4.353E-04 $m$ & 8.640E-04 $m$ & 1.740E-05  $m$  & 8.785E-04 $m$  & 50.61\%    \\ \hline
% 0.49 & 3.142E-04 $m$ & 8.640E-04 $m$ & 1.788E-05  $m$  & 8.789E-04 $m$  & 64.37\%    \\ \hline
%   \end{tabular}
%   % \caption{}
% \end{table}



The errors were plotted in Figure (\ref{fig error 8NodeBrick cantilever beam for different Poisson's ratio}).
\begin{figure}[H]
  % \centering
  \begin{subfigure}{0.5\textwidth}
    \centering
    \includegraphics[width=6cm]{../Figure-files/error8brick_beam_different_poisson_ratio_disp_div.jpeg}
    \caption{Error scale 0\% - 80\%}
  \end{subfigure}
  \begin{subfigure}{0.5\textwidth}
    \centering
    \includegraphics[width=6cm]{../Figure-files/error8brick_beam_different_poisson_ratio_disp_div100.jpeg}
    \caption{Error scale 0\% - 100\%}
  \end{subfigure}
  \captionsetup{justification=centering,margin=3cm}
  \caption{8NodeBrick cantilever beam for different Poisson's ratio\\
      \emph{\textbf{Displacement error}}   versus   Poisson's ratio}
  \label{fig error 8NodeBrick cantilever beam for different Poisson's ratio}
\end{figure}


The angle results were listed in Table (\ref{table angle results for 8NodeBrick cantilever beams of different Poissons ratios}).
\begin{table}[H]
  \centering
  \captionsetup{justification=centering,margin=3cm}
  \caption{\emph{\textbf{Rotation angle}} results for 8NodeBrick cantilever beams with \textcolor{red}{element side length 1 m}}
  \label{table angle results for 8NodeBrick cantilever beams of different Poissons ratios}
\begin{tabular}{|c|c|c|c|}
\hline
\tabincell{c}{Poisson's \\ ratio} & \tabincell{c}{8NodeBrick \\ angle (unit:\degree)}  & \tabincell{c}{Theory angle\\(unit:\degree)}  & Error   \\ \hline
0.00            & 8.25E-03   & 1.24E-02     & 33.46\% \\ \hline
0.05            & 8.36E-03   & 1.24E-02     & 32.55\% \\ \hline
0.10            & 8.42E-03   & 1.24E-02     & 32.08\% \\ \hline
0.15            & 8.42E-03   & 1.24E-02     & 32.10\% \\ \hline
0.20            & 8.35E-03   & 1.24E-02     & 32.67\% \\ \hline
0.25            & 8.20E-03   & 1.24E-02     & 33.90\% \\ \hline
0.30            & 7.95E-03   & 1.24E-02     & 35.89\% \\ \hline
0.35            & 7.59E-03   & 1.24E-02     & 38.83\% \\ \hline
0.40            & 7.07E-03   & 1.24E-02     & 43.00\% \\ \hline
0.45            & 6.30E-03   & 1.24E-02     & 49.21\% \\ \hline
0.49            & 4.93E-03   & 1.24E-02     & 60.20\% \\
\hline
\end{tabular}
  % \caption{}
\end{table}



Then, in the same geometry, element side length was cut into 0.5m. The angle results were listed in Table (\ref{table angle results for 8NodeBrick cantilever beams of different Poissons ratios div2}).
\begin{table}[H]
  \centering
  \captionsetup{justification=centering,margin=3cm}
  \caption{\emph{\textbf{Rotation angle}} results for 8NodeBrick cantilever beams with with \textcolor{red}{element side length 0.5 m}}
  \label{table angle results for 8NodeBrick cantilever beams of different Poissons ratios div2}
\begin{tabular}{|c|c|c|c|}
\hline
\tabincell{c}{Poisson's \\ ratio} & \tabincell{c}{8NodeBrick \\ angle (unit:\degree)}  & \tabincell{c}{Theory angle\\(unit:\degree)}  & Error   \\ \hline
0.00            & 1.10E-02 & 1.24E-02 & 11.28\% \\ \hline
0.05            & 1.10E-02 & 1.24E-02 & 10.91\% \\ \hline
0.10            & 1.11E-02 & 1.24E-02 & 10.78\% \\ \hline
0.15            & 1.10E-02 & 1.24E-02 & 10.90\% \\ \hline
0.20            & 1.10E-02 & 1.24E-02 & 11.32\% \\ \hline
0.25            & 1.09E-02 & 1.24E-02 & 12.09\% \\ \hline
0.30            & 1.07E-02 & 1.24E-02 & 13.33\% \\ \hline
0.35            & 1.05E-02 & 1.24E-02 & 15.29\% \\ \hline
0.40            & 1.01E-02 & 1.24E-02 & 18.53\% \\ \hline
0.45            & 9.32E-03 & 1.24E-02 & 24.87\% \\ \hline
0.49            & 7.52E-03 & 1.24E-02 & 39.35\% \\
\hline
\end{tabular}
  % \caption{}
\end{table}





Finally, in the same geometry, element side length was cut into 0.25m. The angle results were listed in Table (\ref{table angle results for 8NodeBrick cantilever beams of different Poissons ratios div4}).
\begin{table}[H]
  \centering
  \captionsetup{justification=centering,margin=3cm}
  \caption{\emph{\textbf{Rotation angle}} results for 8NodeBrick cantilever beams with with \textcolor{red}{element side length 0.25 m}}
  \label{table angle results for 8NodeBrick cantilever beams of different Poissons ratios div4}
\begin{tabular}{|c|c|c|c|}
\hline
\tabincell{c}{Poisson's \\ ratio} & \tabincell{c}{8NodeBrick \\ angle (unit:\degree)}  & \tabincell{c}{Theory angle\\(unit:\degree)}  & Error   \\ \hline
0.00            & 1.20E-02 & 1.24E-02 & 3.06\%  \\ \hline
0.05            & 1.20E-02 & 1.24E-02 & 2.97\%  \\ \hline
0.10            & 1.20E-02 & 1.24E-02 & 2.99\%  \\ \hline
0.15            & 1.20E-02 & 1.24E-02 & 3.12\%  \\ \hline
0.20            & 1.20E-02 & 1.24E-02 & 3.38\%  \\ \hline
0.25            & 1.19E-02 & 1.24E-02 & 3.79\%  \\ \hline
0.30            & 1.19E-02 & 1.24E-02 & 4.40\%  \\ \hline
0.35            & 1.17E-02 & 1.24E-02 & 5.33\%  \\ \hline
0.40            & 1.15E-02 & 1.24E-02 & 6.87\%  \\ \hline
0.45            & 1.11E-02 & 1.24E-02 & 10.22\% \\ \hline
0.49            & 9.64E-03 & 1.24E-02 & 22.23\% \\
\hline
\end{tabular}
  % \caption{}
\end{table}












The errors were plotted in Figure (\ref{table angle error 8NodeBrick cantilever beam for different Poisson ratio}).

\begin{figure}[H]
  % \centering
  \begin{subfigure}{0.5\textwidth}
    \centering
    \includegraphics[width=6cm]{../Figure-files/error8brick_beam_different_poisson_ratio_angle_div.jpeg}
    \caption{Error scale 30\% - 70\%}
  \end{subfigure}
  \begin{subfigure}{0.5\textwidth}
    \centering
    \includegraphics[width=6cm]{../Figure-files/error8brick_beam_different_poisson_ratio_angle_div100.jpeg}
    \caption{Error scale 0\% - 100\%}
  \end{subfigure}
  \captionsetup{justification=centering,margin=3cm}
  \caption{8NodeBrick cantilever beam for different Poisson's ratio\\
      \emph{\textbf{Rotation angle error}}   versus   Poisson's ratio}
  \label{table angle error 8NodeBrick cantilever beam for different Poisson ratio}
\end{figure}



The ESSI model fei files for the table above are \href{https://github.com/yuan-energy/ESSI_Verification/blob/master/8NodeBrick/cantilever_different_Poisson/cantilever_different_Poisson.tar.gz?raw=true}{here}




\newpage
\subsection{Test of irregular shaped 8NodeBrick cantilever beams}

Cantilever model was used as an example. 
Three different shapes were tested. 


In the first test, the upper two nodes of each element were moved one half element size along the $y-axis$, while the lower two nodes were kept at the same location. The element shape was shown in Figure (\ref{fig irregular shape 1 8NodeBrick cantilever beams }).

\begin{figure}[H]
  \centering
  \includegraphics[width=9cm]{../Figure-files/beam_brick_shape1.pdf}
  \caption{8NodeBrick cantilever beams for irregular \textbf{\emph{Shape 1}} }
  \label{fig irregular shape 1 8NodeBrick cantilever beams }
\end{figure}


In the second test, the upper two nodes of each element were moved 90\% element size along the $y-axis$, while the lower two nodes were moved 90\% element size in the other direction along the $y-axis$. The element shape was shown in Figure (\ref{fig irregular shape 2 8NodeBrick cantilever beams }).

\begin{figure}[H]
  \centering
  \includegraphics[width=9cm]{../Figure-files/beam_brick_shape2.pdf}
  \caption{8NodeBrick cantilever beams for irregular \textbf{\emph{Shape 2}} }
  \label{fig irregular shape 2 8NodeBrick cantilever beams }
\end{figure}



In the third test, the upper two nodes of each element were moved one half element size with different directions along the $y-axis$, while the lower two nodes were kept at the same location. The element shape was shown in Figure (\ref{fig irregular shape 3 8NodeBrick cantilever beams }).

\begin{figure}[H]
  \centering
  \includegraphics[width=9cm]{../Figure-files/beam_brick_shape3.pdf}
  \caption{8NodeBrick cantilever beams for irregular \textbf{\emph{Shape 3}} }
  \label{fig irregular shape 3 8NodeBrick cantilever beams }
\end{figure}

The boundary conditions were shown in Figure (\ref{fig 8NodeBrick cantilever beam boundary conditions shape 1}), (\ref{fig 8NodeBrick cantilever beam boundary conditions shape 2}) and (\ref{fig 8NodeBrick cantilever beam boundary conditions shape 3}) .

\begin{figure}[H]
  \centering
    \begin{subfigure}{0.5\textwidth}
      \centering
      \includegraphics[width=9cm]{../Figure-files/beam_brick_shape1_vertical.pdf}
      \caption{Veritical force}
    \end{subfigure}
    \begin{subfigure}{0.5\textwidth}
      \centering
      \includegraphics[width=9cm]{../Figure-files/beam_brick_shape1_horizontal.pdf}
      \caption{Horizontal force}
    \end{subfigure}
  \caption{8NodeBrick cantilever beam boundary conditions for irregular \textbf{\emph{Shape 1}} }
  \label{fig 8NodeBrick cantilever beam boundary conditions shape 1}
\end{figure}


\begin{figure}[H]
  \centering
    \begin{subfigure}{0.5\textwidth}
      \centering
      \includegraphics[width=9cm]{../Figure-files/beam_brick_shape2_vertical.pdf}
      \caption{Veritical force}
    \end{subfigure}
    \begin{subfigure}{0.5\textwidth}
      \centering
      \includegraphics[width=9cm]{../Figure-files/beam_brick_shape2_horizontal.pdf}
      \caption{Horizontal force}
    \end{subfigure}
  \caption{8NodeBrick cantilever beam boundary conditions for irregular \textbf{\emph{Shape 2}} }
  \label{fig 8NodeBrick cantilever beam boundary conditions shape 2}
\end{figure}


\begin{figure}[H]
  \centering
    \begin{subfigure}{0.5\textwidth}
      \centering
      \includegraphics[width=9cm]{../Figure-files/beam_brick_shape3_vertical.pdf}
      \caption{Veritical force}
    \end{subfigure}
    \begin{subfigure}{0.5\textwidth}
      \centering
      \includegraphics[width=9cm]{../Figure-files/beam_brick_shape3_horizontal.pdf}
      \caption{Horizontal force}
    \end{subfigure}
  \caption{8NodeBrick cantilever beam boundary conditions for irregular \textbf{\emph{Shape 3}} }
  \label{fig 8NodeBrick cantilever beam boundary conditions shape 3}
\end{figure}


The ESSI results were listed in Table (\ref{table Results for 8NodeBrick cantilever beams of irregular shapes}). 
\begin{table}[H]
  \centering
  \caption{Results for 8NodeBrick cantilever beams of irregular shapes}
  \label{table Results for 8NodeBrick cantilever beams of irregular shapes}
  \begin{tabular}{|c|c|c|c|c|c|}
    \hline 
    Element Type   & Force direction & Normal shape & Shape 1 & Shape 2 & Shape 3  \\ \hline 
    8NodeBrick     & Vertical ($z$)     & 5.840E-04 $m$  & 5.751E-04 $m$ & 2.959E-04 $m$ & 3.883E-04 $m$   \\ \hline
    8NodeBrick     & Transverse ($y$)      & 5.840E-04 $m$  & 4.529E-04 $m$ & 1.390E-04 $m$ & 4.744E-04 $m$   \\ \hline
    Theoretical    &      -              & 8.784E-04 $m$  & 8.784E-04 $m$ & 8.784E-04 $m$ & 8.784E-04 $m$ \\ \hline
  \end{tabular}
\end{table}

The errors were listed in Table (\ref{table Errors for irregular shaped 8NodeBrick compared to theoretical solution}) and (\ref{talbe Errors for irregular shaped 8NodeBrick compared to normal shape}).


\begin{table}[H]
  \centering
  \caption{Errors for irregular shaped 8NodeBrick compared to theoretical solution}
  \label{table Errors for irregular shaped 8NodeBrick compared to theoretical solution}
  \begin{tabular}{|c|c|c|c|c|c|}
    \hline 
    Element Type   & Force direction & Normal shape & Shape 1 & Shape 2 & Shape 3  \\ \hline 
    8NodeBrick     & Vertical ($z$)     & 33.52\% & 34.53\% & 66.31\% & 55.79\%  \\ \hline
    8NodeBrick     & Transverse ($y$)   & 33.52\% & 48.44\% & 84.18\% & 45.99\%  \\ \hline
  \end{tabular}
  % \caption{}
\end{table}

\begin{table}[H]
  \centering
  \caption{Errors for irregular shaped 8NodeBrick compared to normal shape}
  \label{talbe Errors for irregular shaped 8NodeBrick compared to normal shape}
  \begin{tabular}{|c|c|c|c|c|c|}
    \hline 
    Element Type   & Force direction & Normal shape & Shape 1 & Shape 2 & Shape 3  \\ \hline 
    8NodeBrick     & Vertical ($z$)     & 0.00\% & 1.52\%  & 49.33\% & 33.51\%       \\ \hline
    8NodeBrick     & Transverse ($y$)   & 0.00\% & 22.45\% & 76.20\% & 18.77\%       \\ \hline
  \end{tabular}
  % \caption{}
\end{table}

The ESSI model fei files for the table above are \href{https://github.com/yuan-energy/ESSI_Verification/blob/master/8NodeBrick/cantilever_irregular_element/cantilever_irregular_element.tar.gz?raw=true}{here}



% The errors were listed below, compared to the theoretical solution.
% \begin{table}[H]
%   \centering
%   \begin{tabular}{|c|c|c|c|c|}
%     \hline 
%     \multicolumn{5}{|c|}{Test for brick shape displacement errors}   \\ \hline
%     Element Type  & Normal shape & Shape 1 & Shape 2 & Shape 3  \\ \hline 
%     8NodeBrick     &     &    &   & \\ \hline
%     27NodeBrick    &     &    &   &  \\ \hline
%   \end{tabular}
%   % \caption{}
% \end{table}

\newpage
Then, the irregular beam was divided into small elements. 


Problem description: Length=12m, Width=2m, Height=2m, q=400N/m, E=1E8Pa, $\nu=0.0$. Use the shear deformation coefficient $\kappa=1.2$. The force direction was shown in Figure (\ref{fig Problem description for cantilever beams under uniform pressure}). 

\begin{figure}[H]
  \centering
  \includegraphics[width=7cm]{../Figure-files/cantilever_12_uniform_load.pdf}
  \caption{Problem description for cantilever beams under uniform load  }
  \label{fig Problem description for cantilever beams under uniform pressure}
\end{figure}


Theoretical displacement (bending and shear deformation):
\begin{equation}
  \begin{aligned}
  d &=\frac{qL^4}{8EI} + \frac{q \frac{L^2}{2}}{GA_v} \\ 
    &=\frac{qL^4}{8E\frac{bh^3}{12} }+\frac{q \frac{L^2}{2}}{\frac{E}{2(1+\nu)}\frac{bh}{\kappa}} \\
    &= \frac{400 N/m \times 12^4 m^4}{8\times 10^8 N/m^2 \times \frac{2^4}{12} m^4} 
       + \frac{400 N/m \times \frac{12^2}{2} m^2} {\frac{10^8}{2} N/m^2 \times 2m\times 2m\times \frac{5}{6}} \\ 
    &=7.776\times 10^{-3} m  +1.728\times 10^{-4} m \\
    &=7.9488\times 10^{-3} m
   \end{aligned}
\end{equation}







The ESSI displacement results were listed in Table (\ref{table Results for 8NodeBrick cantilever beams of irregular shapes with more elements}).
\begin{table}[H]
  \centering
  \caption{Results for 8NodeBrick cantilever beams of irregular shapes with more elements}
  \label{table Results for 8NodeBrick cantilever beams of irregular shapes with more elements}
\begin{tabular}{|c|c|c|c|c|c|}
\hline
\multirow{2}{*}{Element Type} & \multirow{2}{*}{Shape}  & \multirow{2}{*}{Force direction}  & \multicolumn{3}{|c|}{Number of division} \\  \cline{4-6}
                        &        &                  &  1 &  2 &  4  \\ \hline
8NodeBrick              & shape1 & Vertical ($z$)   & 5.37E-03 $m$  & 7.08E-03 $m$  & 7.71E-03  $m$ \\ \hline
8NodeBrick              & shape1 & Transverse ($y$) & 4.60E-03 $m$  & 6.66E-03 $m$  & 7.58E-03  $m$ \\ \hline
8NodeBrick              & shape2 & Vertical ($z$)   & 2.74E-03 $m$& 4.75E-03 $m$& 6.43E-03  $m$ \\ \hline
8NodeBrick              & shape2 & Transverse ($y$) & 1.46E-03 $m$& 2.72E-03 $m$& 4.63E-03  $m$ \\ \hline
8NodeBrick              & shape3 & Vertical ($z$)   & 9.21E-04 $m$  & 6.60E-03 $m$  & 7.56E-03  $m$ \\ \hline
8NodeBrick              & shape3 & Transverse ($y$) & 1.09E-03 $m$  & 6.09E-03 $m$  & 7.37E-03  $m$ \\ \hline
 \multicolumn{3}{|c|}{Theoretical solution}      & 7.95E-03 $m$  & 7.95E-03 $m$  & 7.95E-03  $m$ \\
\hline
\end{tabular}
\end{table}

The error were listed in Table (\ref{table Errors for 8NodeBrick cantilever beams of irregular shapes with more elements}). 

\begin{table}[H]
  \centering
  \caption{Errors for 8NodeBrick cantilever beams of irregular shapes with more elements}
  \label{table Errors for 8NodeBrick cantilever beams of irregular shapes with more elements}
\begin{tabular}{|c|c|c|c|c|c|}
\hline
\multirow{2}{*}{Element Type} & \multirow{2}{*}{Shape}  & \multirow{2}{*}{Force direction}  & \multicolumn{3}{|c|}{Number of division} \\  \cline{4-6}
                        &        &                  &  1 &  2 &  4  \\ \hline
8NodeBrick   & shape1      & Vertical ($z$)   & 32.42\% & 10.95\% & 3.01\%     \\ \hline
8NodeBrick   & shape1      & Transverse ($y$) & 42.16\% & 16.17\% & 4.69\%     \\ \hline
8NodeBrick   & shape2      & Vertical ($z$)   & 65.59\% & 40.22\% & 19.05\%    \\ \hline
8NodeBrick   & shape2      & Transverse ($y$) & 81.57\% & 65.76\% & 41.81\%    \\ \hline
8NodeBrick   & shape3      & Vertical ($z$)   & 88.42\% & 16.97\% & 4.89\%     \\ \hline
8NodeBrick   & shape3      & Transverse ($y$) & 86.24\% & 23.36\% & 7.28\%     \\
\hline
\end{tabular}
\end{table}



% \begin{figure}[H]
%   % \centering
%   \begin{subfigure}{0.5\textwidth}
%     \centering
%     \includegraphics[width=6cm]{../Figure-files/error8brick_beam_irregular_shape1.jpeg}
%     \caption{Error scale 0\% - 50\%}
%   \end{subfigure}
%   \begin{subfigure}{0.5\textwidth}
%     \centering
%     \includegraphics[width=6cm]{../Figure-files/error8brick_beam_irregular_shape1100.jpeg}
%     \caption{Error scale 0\% - 100\%}
%   \end{subfigure}
%   \captionsetup{justification=centering,margin=3cm}
%   \caption{8NodeBrick cantilever beam for irregular Shape 1\\
%       Displacement error   versus   Number of division}
%   % \caption{}
%   % \label{}
% \end{figure}

The errors were shown in Figure (\ref{fig shape 1 8NodeBrick cantilever beam for irregular more elements}), (\ref{fig shape 2 8NodeBrick cantilever beam for irregular more elements}) and (\ref{fig shape 3 8NodeBrick cantilever beam for irregular more elements}). 
\begin{figure}[H]
  % \centering
    \centering
    \includegraphics[width=6cm]{../Figure-files/error8brick_beam_irregular_shape1100.jpeg}
  \captionsetup{justification=centering,margin=3cm}
  \caption{8NodeBrick cantilever beam for irregular \emph{\textbf{Shape 1}}\\
      Displacement error   versus   Number of division}
  % \caption{}
  \label{fig shape 1 8NodeBrick cantilever beam for irregular more elements}
\end{figure}


% \begin{figure}[H]
%   % \centering
%   \begin{subfigure}{0.5\textwidth}
%     \centering
%     \includegraphics[width=6cm]{../Figure-files/error8brick_beam_irregular_shape2.jpeg}
%     \caption{Error scale 0\% - 80\%}
%   \end{subfigure}
%   \begin{subfigure}{0.5\textwidth}
%     \centering
%     \includegraphics[width=6cm]{../Figure-files/error8brick_beam_irregular_shape2100.jpeg}
%     \caption{Error scale 0\% - 100\%}
%   \end{subfigure}
%   \captionsetup{justification=centering,margin=3cm}
%   \caption{Two sub}
%   % \caption{}
%   % \label{}
% \end{figure}

\begin{figure}[H]
  % \centering
    \centering
    \includegraphics[width=6cm]{../Figure-files/error8brick_beam_irregular_shape2.jpeg}
  \captionsetup{justification=centering,margin=3cm}
  \caption{8NodeBrick cantilever beam for irregular \emph{\textbf{Shape 2}} \\
      Displacement error   versus   Number of division}
  % \caption{}
  \label{fig shape 2 8NodeBrick cantilever beam for irregular more elements}
\end{figure}



% \begin{figure}[H]
%   % \centering
%   \begin{subfigure}{0.5\textwidth}
%     \centering
%     \includegraphics[width=6cm]{../Figure-files/error8brick_beam_irregular_shape3.jpeg}
%     \caption{Error scale 0\% - 80\%}
%   \end{subfigure}
%   \begin{subfigure}{0.5\textwidth}
%     \centering
%     \includegraphics[width=6cm]{../Figure-files/error8brick_beam_irregular_shape3100.jpeg}
%     \caption{Error scale 0\% - 100\%}
%   \end{subfigure}
%   \captionsetup{justification=centering,margin=3cm}
%   \caption{Two sub}
%   % \caption{}
%   % \label{}
% \end{figure}

\begin{figure}[H]
  % \centering
    \centering
    \includegraphics[width=6cm]{../Figure-files/error8brick_beam_irregular_shape3.jpeg}
  \captionsetup{justification=centering,margin=3cm}
  \caption{8NodeBrick cantilever beam for irregular \emph{\textbf{Shape 3}}\\
      Displacement error   versus   Number of division}
  % \caption{}
  \label{fig shape 3 8NodeBrick cantilever beam for irregular more elements}
\end{figure}



The ESSI model fei files for the table above are \href{https://github.com/yuan-energy/ESSI_Verification/blob/master/8NodeBrick/cantilever_irregular_element_cut/cantilever_irregular_element_cut.tar.gz?raw=true}{here}











% \newpage
% \subsection{Verification of 8NodeBrick edge clamped beams }

% Problem description: Length=6m, Width=1m, Height=1m, Force=100N, E=1E8Pa, $\nu=0.0$. Use the shear deformation coefficient $\kappa=1.2$. The force direction was shown in Figure (\ref{fig Problem description for clamped beams}). 

% \begin{figure}[H]
%   \centering
%   \includegraphics[width=7cm]{../Figure-files/clamped_beam.pdf}
%   \caption{Problem description for clamped beams}
%   \label{fig Problem description for clamped beams}
% \end{figure}

% % \subsection{Verification of edge clamped beams - one line elements}



% The elment types and element sizes were same to the cantilever model. Only the boundary conditions and external force locations were changed. 

% Numerical model:

% The 8NodeBrick elements were shown in Figure (\ref{fig 8NodeBrick elements for clamped beams}).

% \begin{figure}[H]
%   \centering
%   \includegraphics[width=9cm]{../Figure-files/beam_8brick.pdf}
%   \caption{8NodeBrick elements for clamped beams}
%   \label{fig 8NodeBrick elements for clamped beams}
% \end{figure}

% Theoretical displacement (bending and shear deformation):
% \begin{equation}
%   \begin{aligned}
%   d &=\frac{FL^3}{192EI}+\frac{\frac{F}{2}\frac{L}{2}}{GA_v}  \\
%     &=\frac{FL^3}{192E\frac{bh^3}{12}}+\frac{\frac{F}{2}\frac{L}{2}}{\frac{E}{2(1+\nu)}\frac{bh}{\kappa}} \\
%    &= \frac{100 N\times 6 m^3}{192 \times 10^8 N/m^2 \times \frac{1}{12} m^4}+ 
%     \frac{\frac{100}{2} N \times \frac{6}{2} m}{\frac{10}{2}\times 10^7 N/m^2\times 1 m^2\times \frac{5}{6}}   \\
%   &=1.35\times 10^{-5} m + 0.36\times 10^{-5} m  \\
%   &=1.71\times 10^{-5} \ m 
%     \end{aligned}
% \end{equation}

% The theoretical solution for $L=6\ m$ was calculated above, while the solutions for other length were calculated by the similar process. 

% In the figures above, only the model with geometry $6m\times 1m \times 1m$ was drawed. In the ESSI models, the geometry $10m\times 1m \times 1m$ and the geometry $20m\times 1m \times 1m$ were also calculated. In three different geometry models, all the element sizes were $1m\times 1m \times 1m$. Therefore, the number of elements used in each model were $6,\ 10\ and\ 20$ respectively.

% The results were listed in Table (\ref{table Results for 8NodeBrick clamped beams of different geometry}).

% \begin{table}[H]
%   \centering
%     \caption{Results for 8NodeBrick clamped beams of different geometry}
%     \label{table Results for 8NodeBrick clamped beams of different geometry}
%     \begin{tabular}{|c|c|c|c|c|c|}
%     \hline
%     Geometry & 8NodeBrick & Theory(bending) & Theory(shear) & Theory(all) & Error   \\  \hline
%     1:6      & 1.100E-05 $m$ & 1.35E-05  $m$     & 2.50E-06  $m$   & 1.60E-05 $m$        & 33.33\% \\ \hline
%     1:10     & 4.500E-05 $m$ & 6.25E-05  $m$     & 5.00E-06  $m$   & 6.75E-05 $m$        & 33.33\% \\ \hline
%     1:20     & 3.400E-04 $m$ & 5.00E-04  $m$     & 1.00E-05  $m$   & 5.10E-04 $m$        & 33.33\% \\
%     \hline
%     \end{tabular}
% \end{table}

% The ESSI model fei files for the table above are \href{https://github.com/yuan-energy/ESSI_Verification/blob/master/8NodeBrick/clamped_beam_different_geometry/clamped_beam_different_geometry.tar.gz?raw=true}{here}





% 
% old table : may be useful.....
% \begin{table}[H]
%   \centering
%   \begin{tabular}{|c|c|c|c|c|}
%     \hline 
%     \multicolumn{5}{|c|}{The edge clamped beam displacement errors}   \\ \hline
%     Element Type  & Force direction  &1:6 & 1:10 & 1:20  \\ \hline 
%     4NodeANDES & in-plane       &    &   & \\ \hline
%     4NodeANDES & out-of-plane        &    &   &  \\ \hline
%     \multicolumn{2}{|c|}{8NodeBrick} &    &   &  \\ \hline
%     \multicolumn{2}{|c|}{27NodeBrick} &   &   &  \\ \hline
%   \end{tabular}
%   % \caption{}
% \end{table}




\newpage
In this section, the beam was cut into smaller elements with element side length 0.5m and 0.25m respectively. And the element side length of the original models is 1.0m. The numerical models were shown in Figure (\ref{fig 8NodeBrick clamped beams with element side length 1.0m}), (\ref{fig 8NodeBrick clamped beams with element side length 0.5m}) and (\ref{fig 8NodeBrick clamped beams with element side length 0.25m}). 

Number of division 1:

\begin{figure}[H]
  \centering
  \includegraphics[width=9cm]{../Figure-files/beam_8brick.pdf}
  \caption{8NodeBrick clamped beams with element side length 1.0m}
  \label{fig 8NodeBrick clamped beams with element side length 1.0m}
\end{figure}

Number of division 2:

\begin{figure}[H]
  \centering
  \includegraphics[width=9cm]{../Figure-files/beam_8brick_more_2.pdf}
  \caption{8NodeBrick clamped beams with element side length 0.5m}
  \label{fig 8NodeBrick clamped beams with element side length 0.5m}
\end{figure}

Number of division 4:

\begin{figure}[H]
  \centering
  \includegraphics[width=9cm]{../Figure-files/beam_8brick_more.pdf}
  \caption{8NodeBrick clamped beams with element side length 0.25m}
  \label{fig 8NodeBrick clamped beams with element side length 0.25m}
\end{figure}


The ESSI results were listed in Table (\ref{table Results for 8NodeBrick clamped beams with more elements}). 
The theoretical solution is 1.60E-5 $m$. 

\begin{table}[H]
  \centering
  \caption{Results for 8NodeBrick clamped beams with more elements}
  \label{table Results for 8NodeBrick clamped beams with more elements}
  \begin{tabular}{|c|c|c|c|c|}
    \hline 
    \multirow{2}{*}{Element Type} 
       & \multicolumn{3}{|c|}{Element side length} \\ \cline{2-4}
       & 1 $m$ & 0.5 $m$ & 0.25 $m$ \\                              \hline
8NodeBrick & 1.10E-05 $m$ & 1.47E-05 $m$ & 1.64E-05 $m$ \\ \hline
Error      & 33.33\%  & 11.09\%  & 0.73\%   \\ \hline
  \end{tabular}
  % \caption{}
\end{table}

The errors were plotted in Figure (\ref{fig error 8NodeBrick clamped beam for different element number}).

\begin{figure}[H]
  % \centering
  \begin{subfigure}{0.5\textwidth}
    \centering
    \includegraphics[width=6cm]{../Figure-files/error8brick_clamped_beam_diff_element.jpeg}
    \caption{Error scale 0\% - 40\%}
  \end{subfigure}
  \begin{subfigure}{0.5\textwidth}
    \centering
    \includegraphics[width=6cm]{../Figure-files/error8brick_clamped_beam_diff_element100.jpeg}
    \caption{Error scale 0\% - 100\%}
  \end{subfigure}
  \captionsetup{justification=centering,margin=3cm}
  \caption{8NodeBrick clamped beam for different element number\\
      Displacement error   versus   Number of division}
  \label{fig error 8NodeBrick clamped beam for different element number}
\end{figure}


The ESSI model fei files for the table above are \href{https://github.com/yuan-energy/ESSI_Verification/blob/master/8NodeBrick/clamped_beam_cut/clamped_beam_cut.tar.gz?raw=true}{here}








\newpage
\subsection{Verification of 8NodeBrick stress in cantilever beams}





Problem description: Length=6m, Width=1m, Height=1m, Force=100N, E=1E8Pa, $\nu=0.0$. Use the shear deformation coefficient $\kappa=1.2$. The force direction was shown in Figure (\ref{fig Problem description for cantilever beams of stress verification}). 

\begin{figure}[H]
  \centering
  \includegraphics[width=7cm]{../Figure-files/cantilever_6.pdf}
  \caption{Problem description for cantilever beams of stress verification}
  \label{fig Problem description for cantilever beams of stress verification}
\end{figure}

The theoretical solution for the stress was calculated below. 

The 8NodeBrick elements were shown in Figure (\ref{fig 8NodeBrick for cantilever beams of stress verification}).
\begin{figure}[H]
  \centering
  \includegraphics[width=9cm]{../Figure-files/beam_8brick_6div_gp.pdf}
  \caption{8NodeBrick for cantilever beams of stress verification}
  \label{fig 8NodeBrick for cantilever beams of stress verification}
\end{figure}

The bending moment at the Gassian Point is 
\begin{equation}
  M=F(L-P_y)=100 N \times (6-0.2113) m = 578.87 N\cdot m
\end{equation}

The bending modulus is 
\begin{equation}
  I= \frac{bh^3}{12}=\frac{1}{12} m^4
\end{equation}

Therefore, the theoretical stress is 
\begin{equation}
  \sigma= \frac{M\cdot z}{I}= \frac{578.87 N\cdot m \times (0.5-0.2113) m }{\frac{1}{12} m^4}= 2005 Pa
\end{equation}



To get a better result, the same geometry beam was also cut into small elements. When more elements were used, the theoretical stress was calculated again with the new coordinates. The calculation process is similar to the process above. 

The numerical models were shown in Figure (\ref{fig 8NodeBrick stress with element side length 1.0m}), (\ref{fig 8NodeBrick stress with element side length 0.5m}) and (\ref{fig 8NodeBrick stress with element side length 0.25m}). 


Number of division 1:

\begin{figure}[H]
  \centering
  \includegraphics[width=9cm]{../Figure-files/beam_8brick.pdf}
  \caption{8NodeBrick stress with element side length 1.0m}
  \label{fig 8NodeBrick stress with element side length 1.0m}
\end{figure}

Number of division 2:

\begin{figure}[H]
  \centering
  \includegraphics[width=9cm]{../Figure-files/beam_8brick_more_2.pdf}
  \caption{8NodeBrick stress with element side length 0.5m}
  \label{fig 8NodeBrick stress with element side length 0.5m}
\end{figure}

Number of division 4:

\begin{figure}[H]
  \centering
  \includegraphics[width=9cm]{../Figure-files/beam_8brick_more.pdf}
  \caption{8NodeBrick stress with element side length 0.25m}
  \label{fig 8NodeBrick stress with element side length 0.25m}
\end{figure}


All the stress results were listed in Table (\ref{table Results for 8NodeBrick stress with more elements}). 


\begin{table}[H]
  \centering
  \caption{Results for 8NodeBrick stress with more elements}
  \label{table Results for 8NodeBrick stress with more elements}
  \begin{tabular}{|c|c|c|c|c|}
    \hline 
    \multirow{2}{*}{Element Type} 
       & \multicolumn{3}{|c|}{Element side length} \\ \cline{2-4}
       & 1 $m$ & 0.5 $m$ & 0.25 $m$ \\                              \hline
8NodeBrick & 1270.17 $Pa$ & 2418.60 $Pa$ & 3085.48 $Pa$ \\ \hline
Theoretical & 2005.26 $Pa$ & 2789.23 $Pa$ & 3191.27 $Pa$ \\ \hline
Error      & 36.66\% & 13.29\% & 3.31\%  \\ \hline
  \end{tabular}
  % \caption{}
\end{table}

The errors were plotted in Figure (\ref{fig 8NodeBrick cantilever beams for stress verification}).
\begin{figure}[H]
  % \centering
  \begin{subfigure}{0.5\textwidth}
    \centering
    \includegraphics[width=6cm]{../Figure-files/error8brick_beam_stress.jpeg}
    \caption{Error scale 0\% - 40\%}
  \end{subfigure}
  \begin{subfigure}{0.5\textwidth}
    \centering
    \includegraphics[width=6cm]{../Figure-files/error8brick_beam_stress100.jpeg}
    \caption{Error scale 0\% - 100\%}
  \end{subfigure}
  \captionsetup{justification=centering,margin=3cm}
  \caption{8NodeBrick cantilever beams for stress verification\\
      Stress error   versus   Number of division}
  \label{fig 8NodeBrick cantilever beams for stress verification}
\end{figure}


The ESSI model fei files for the table above are \href{https://github.com/yuan-energy/ESSI_Verification/blob/master/8NodeBrick/cantilever_stress/cantilever_stress.tar.gz?raw=true}{here}




\newpage
\subsection{Verification of 8NodeBrick square plate with four edges clamped}

Problem description: Length=20m, Width=20m, Height=1m, Force=100N, E=1E8Pa, $\nu=0.3$. 

The four edges are clamped. 

The load is the uniform normal pressure on the whole plate. 


The plate flexural rigidity is 
\begin{equation}
  D=\frac{Eh^3}{12(1-\nu^2)}=\frac{10^8 N/m^2 \times 1^3 m^3 }{12 \times (1-0.3^2) }= 9.1575 \times 10^6 \ N\cdot m
\end{equation}
The theoretical solution is 
\begin{equation}
  d=\alpha_c \frac{q a^4}{D}=0.00406\times \frac{100 N/m^2 \times 20^4 m^4}{9.1575 \times 10^6 \ N\cdot m}=2.2015\times 10^{-3} m
\end{equation}

where $\alpha_c$ is a coefficient, which depends on the ratio of plate length to width. In this problem, the coefficient\footnote{Stephen Timoshenko, Theory of plates and shells (2nd edition). MrGRAW-Hill Inc, page120, 1959.} $\alpha_c$ is 0.00406.


The 8NodeBrick were shown in Figure (\ref{fig 8NodeBrick edges clamped square plate with element side length 10m }) - (\ref{fig 8NodeBrick edges clamped square plate with element side length 0.25m }). 


\begin{figure}[H]
  \centering
  \includegraphics[width=11cm]{../Figure-files/square_plate1.pdf}
  \caption{8NodeBrick edge clamped square plate with element side length 10m }
  \label{fig 8NodeBrick edges clamped square plate with element side length 10m }
\end{figure}

\newpage

\begin{figure}[H]
  \centering
  \includegraphics[width=11cm]{../Figure-files/square_plate2.pdf}
  \caption{8NodeBrick edge clamped square plate with element side length 5m }
  \label{fig 8NodeBrick edges clamped square plate with element side length 5m }
\end{figure}


\begin{figure}[H]
  \centering
  \includegraphics[width=11cm]{../Figure-files/square_plate3.pdf}
  \caption{8NodeBrick edge clamped square plate with element side length 2m }
  \label{fig 8NodeBrick edges clamped square plate with element side length 2m }
\end{figure}

\newpage

\begin{figure}[H]
  \centering
  \includegraphics[width=11cm]{../Figure-files/square_plate4.pdf}
  \caption{8NodeBrick edge clamped square plate with element side length 1m }
  \label{fig 8NodeBrick edges clamped square plate with element side length 1m }
\end{figure}


\begin{figure}[H]
  \centering
  \includegraphics[width=11cm]{../Figure-files/square_plate5.pdf}
  \caption{8NodeBrick edge clamped square plate with element side length 0.5m }
  \label{fig 8NodeBrick edges clamped square plate with element side length 0.5m }
\end{figure}

\newpage

\begin{figure}[H]
  \centering
  \includegraphics[width=11cm]{../Figure-files/square_plate6.pdf}
  \caption{8NodeBrick edge clamped square plate with element side length 0.25m }
  \label{fig 8NodeBrick edges clamped square plate with element side length 0.25m }
\end{figure}



The results were listed in Table (\ref{table Results for 8NodeBrick square plate with four edges clamped}).

\begin{table}[H]
  \centering
  \caption{Results for 8NodeBrick square plate with four edges clamped}
  \label{table Results for 8NodeBrick square plate with four edges clamped}
\begin{tabular}{|c|c|c|c|c|}
\hline
Element type     & 8NodeBrick     & 8NodeBrick     & 8NodeBrick     &  \multirow{3}{*}{\tabincell{c}{Theoretical \\ displacement}} \\ \cline{1-4}
Number of layers & 1layer         & 2layers         & 4layers         &          \\ \cline{1-4}
Element side length & Height:1.00$m$ & Height:0.50$m$ & Height:0.25$m$ &          \\ \hline
10$m$            & 9.75E-05  $m$  & 9.75E-05  $m$  & 9.75E-05 $m$  & 2.20E-03 $m$ \\ \hline
5$m$             & 3.28E-04  $m$  & 3.32E-04  $m$  & 3.32E-04 $m$  & 2.20E-03 $m$ \\ \hline
2$m$             & 1.04E-03  $m$  & 1.10E-03  $m$  & 1.12E-03 $m$  & 2.20E-03 $m$ \\ \hline
1$m$             & 1.56E-03  $m$  & 1.74E-03  $m$  & 1.79E-03 $m$  & 2.20E-03 $m$ \\ \hline
0.5$m$           & 1.80E-03  $m$  & 2.30E-03  $m$  & 2.12E-03 $m$  & 2.20E-03 $m$ \\ \hline
0.25$m$          & 1.87E-03  $m$  & 2.14E-03  $m$  & 2.23E-03 $m$  & 2.20E-03 $m$ \\
\hline
\end{tabular}
\end{table}


The errors were listed in Table (\ref{table Errors for 8NodeBrick square plate with four edges clamped}).

\begin{table}[H]
  \centering
  \caption{Errors for 8NodeBrick square plate with four edges clamped}
  \label{table Errors for 8NodeBrick square plate with four edges clamped}
\begin{tabular}{|c|c|c|c|c|}
\hline
Element type     & 8NodeBrick     & 8NodeBrick     & 8NodeBrick      \\ \hline
Number of layers & 1layer         & 2layers         & 4layers          \\ \hline
Element side length & Height:1.00$m$ & Height:0.50$m$ & Height:0.25$m$  \\ \hline
10$m$            & 95.57\%        & 95.57\%        & 95.57\%        \\ \hline
5$m$             & 85.09\%        & 84.94\%        & 84.91\%        \\ \hline
2$m$             & 52.98\%        & 50.09\%        & 49.25\%        \\ \hline
1$m$             & 28.93\%        & 21.17\%        & 18.72\%        \\ \hline
0.5$m$           & 18.26\%        & 4.58\%         & 3.56\%         \\ \hline
0.25$m$          & 15.05\%        & 2.70\%         & 1.37\%         \\
\hline
\end{tabular}
\end{table}



% \begin{figure}[H]
%   % \centering
%   \begin{subfigure}{0.5\textwidth}
%     \centering
%     \includegraphics[width=6cm]{../Figure-files/error8brick_square_plate_clamped.jpeg}
%     \caption{Error scale 0\% - 80\%}
%   \end{subfigure}
%   \begin{subfigure}{0.5\textwidth}
%     \centering
%     \includegraphics[width=6cm]{../Figure-files/error8brick_square_plate_clamped100.jpeg}
%     \caption{Error scale 0\% - 100\%}
%   \end{subfigure}
%   \captionsetup{justification=centering,margin=3cm}
%   \caption{Two sub}
%   % \caption{}
%   % \label{}
% \end{figure}


The errors were plotted in Figure (\ref{fig 8NodeBrick square plate with edge clamped}).
\begin{figure}[H]
  % \centering
    \centering
    \includegraphics[width=6cm]{../Figure-files/error8brick_square_plate_clamped.jpeg}
  \captionsetup{justification=centering,margin=3cm}
  \caption{8NodeBrick square plate with edge clamped\\
      Displacement error   versus   Number of side division}
  \label{fig 8NodeBrick square plate with edge clamped}
\end{figure}



The ESSI model fei files for the table above are \href{https://github.com/yuan-energy/ESSI_Verification/blob/master/8NodeBrick/square_plate_clamped/square_plate_clamped.tar.gz?raw=true}{here}




% \newpage
% \begin{itemize}
%   \item \textbf{\emph{Square plate with edges clamped: $100m \times 100m \times 1m$}}
% \end{itemize}

% The same verification procedures above were did for the square plate of $100m \times 100m \times 1m$. 

% The 








% \newpage
% \begin{itemize}
%   \item \textbf{\emph{Square plate with edges clamped: different geometry}}
% \end{itemize}

% In the figures above, only the model with geometry $20m\times 20m \times 1m$ was drawed. In the ESSI models, the geometry $6m\times 6m \times 1m$ and the geometry $10m\times 10m \times 1m$ were also calculated. In three different geometry models, all the element sizes were $1m\times 1m \times 1m$.


% The ESSI displacement results were listed below.

% \begin{table}[H]
%   \centering
%   \begin{tabular}{|c|c|c|c|c|c|}
%     \hline 
%     \multirow{2}{*}{Element Type}  & \multirow{2}{*}{Number of layers}   
%        &  \multicolumn{3}{|c|}{Model geometry} \\       \cline{3-5}
%        & & 1:6 & 1:10 & 1:20 \\                              \hline
% 8NodeBrick &  2layers   &2.13E-04 $m$ & 1.54E-03 $m$ & 2.41E-02 $m$  \\ \hline
% 8NodeBrick &  4layers   &2.37E-04 $m$ & 1.71E-03 $m$ & 2.66E-02 $m$  \\ \hline
% \multicolumn{2}{|c|}{Theoretical}    &1.78E-05 $m$ & 1.38E-04 $m$ & 2.20E-03 $m$      \\ \hline
%   \end{tabular}
%   % \caption{}
% \end{table}



% The errors were listed below.

% \begin{table}[H]
%   \centering
%   \begin{tabular}{|c|c|c|c|c|c|}
%     \hline 
%     \multirow{2}{*}{Element Type}  & \multirow{2}{*}{Number of layers}   
%        &  \multicolumn{3}{|c|}{Model geometry} \\       \cline{3-5}
%        & & 1:6 & 1:10 & 1:20 \\                              \hline
% 8NodeBrick &  2layers   &3.58\% & 9.47\% & 11.77\%  \\ \hline
% 8NodeBrick &  4layers   &7.27\% & 0.19\% & 2.66\%   \\ \hline
%   \end{tabular}
%   % \caption{}
% \end{table}












\newpage
\subsection{Verification of 8NodeBrick square plate with four edges simply supported}

Problem description: Length=20m, Width=20m, Height=1m, Force=100N, E=1E8Pa, $\nu=0.3$. 

The four edges are simply supported. 

The load is the uniform normal pressure on the whole plate. 

The plate flexural rigidity is 
\begin{equation}
  D=\frac{Eh^3}{12(1-\nu^2)}=\frac{10^8 N/m^2 \times 1^3 m^3 }{12 \times (1-0.3^2) }= 9.1575 \times 10^6 \ N\cdot m
\end{equation}
The theoretical solution is 
\begin{equation}
  d=\alpha_s \frac{q a^4}{D}=0.00126\times \frac{100 N/m^2 \times 20^4 m^4}{9.1575 \times 10^6 \ N\cdot m}=7.0936\times 10^{-3} m
\end{equation}

where $\alpha_s$ is a coefficient, which depends on the ratio of plate length to width. In this problem, the coefficient\footnote{Stephen Timoshenko, Theory of plates and shells (2nd edition). MrGRAW-Hill Inc, page202, 1959.} $\alpha_s$ is 0.00126.


The 8NodeBrick were shown in Figure (\ref{fig 8NodeBrick edges simply supported square plate with element side length 10m }) - (\ref{fig 8NodeBrick edges simply supported square plate with element side length 0.25m }). 



\begin{figure}[H]
  \centering
  \includegraphics[width=11cm]{../Figure-files/square_plate1.pdf}
  \caption{8NodeBrick edge simply supported square plate with element side length 10m }
  \label{fig 8NodeBrick edges simply supported square plate with element side length 10m }
\end{figure}

\newpage

\begin{figure}[H]
  \centering
  \includegraphics[width=11cm]{../Figure-files/square_plate2.pdf}
  \caption{8NodeBrick edge simply supported square plate with element side length 5m }
  \label{fig 8NodeBrick edges simply supported square plate with element side length 5m }
\end{figure}


\begin{figure}[H]
  \centering
  \includegraphics[width=11cm]{../Figure-files/square_plate3.pdf}
  \caption{8NodeBrick edge simply supported square plate with element side length 2m }
  \label{fig 8NodeBrick edges simply supported square plate with element side length 2m }
\end{figure}

\newpage

\begin{figure}[H]
  \centering
  \includegraphics[width=11cm]{../Figure-files/square_plate4.pdf}
  \caption{8NodeBrick edge simply supported square plate with element side length 1m }
  \label{fig 8NodeBrick edges simply supported square plate with element side length 1m }
\end{figure}


\begin{figure}[H]
  \centering
  \includegraphics[width=11cm]{../Figure-files/square_plate5.pdf}
  \caption{8NodeBrick edge simply supported square plate with element side length 0.5m }
  \label{fig 8NodeBrick edges simply supported square plate with element side length 0.5m }
\end{figure}

\newpage

\begin{figure}[H]
  \centering
  \includegraphics[width=11cm]{../Figure-files/square_plate6.pdf}
  \caption{8NodeBrick edge simply supported square plate with element side length 0.25m }
  \label{fig 8NodeBrick edges simply supported square plate with element side length 0.25m }
\end{figure}


The results were listed in Table (\ref{table Results for 8NodeBrick square plate with four edges simply supported}).

\begin{table}[H]
  \centering
  \caption{Results for 8NodeBrick square plate with four edges simply supported}
  \label{table Results for 8NodeBrick square plate with four edges simply supported}
\begin{tabular}{|c|c|c|c|c|}
\hline
Element type         & 8NodeBrick     & 8NodeBrick     &  \multirow{3}{*}{\tabincell{c}{Theoretical \\ displacement}}    \\ \cline{1-3}
Number of layers          & 2layers         & 4layers         &          \\ \cline{1-3}
Element side length  & Height:0.50$m$ & Height:0.25$m$ &          \\ \hline
10$m$                & 3.75E-004 $m$ & 3.76E-004 $m$ & 7.09E-03 $m$ \\ \hline
5$m$                 & 1.34E-003 $m$ & 1.35E-003 $m$ & 7.09E-03 $m$ \\ \hline
2$m$                 & 4.16E-003 $m$ & 4.27E-003 $m$ & 7.09E-03 $m$ \\ \hline
1$m$                 & 5.98E-003 $m$ & 6.22E-003 $m$ & 7.09E-03 $m$ \\ \hline
0.5$m$               & 6.75E-003 $m$ & 7.04E-003 $m$ & 7.09E-03 $m$ \\ \hline
0.25$m$              & 8.07E-003 $m$ & 7.30E-003 $m$ & 7.09E-03 $m$ \\
\hline
\end{tabular}
\end{table}


The errors were listed in Table (\ref{table Errors for 8NodeBrick square plate with four edges simply supported}).

\begin{table}[H]
  \centering
  \caption{Errors for 8NodeBrick square plate with four edges simply supported}
  \label{table Errors for 8NodeBrick square plate with four edges simply supported}
\begin{tabular}{|c|c|c|c|c|}
\hline
Element type        & 8NodeBrick     & 8NodeBrick      \\ \hline
Number of layers         & 2layers         & 4layers          \\ \hline
Element side length  & Height:0.50$m$ & Height:0.25$m$  \\ \hline
10$m$                & 94.72\% & 94.71\%        \\ \hline
5$m$                 & 81.05\% & 80.91\%        \\ \hline
2$m$                 & 41.31\% & 39.79\%        \\ \hline
1$m$                 & 15.64\% & 12.38\%        \\ \hline
0.5$m$               & 4.88\%  & 0.70\%         \\ \hline
0.25$m$              & 13.74\% & 2.86\%         \\
\hline
\end{tabular}
\end{table}

% \begin{figure}[H]
%   % \centering
%   \begin{subfigure}{0.5\textwidth}
%     \centering
%     \includegraphics[width=6cm]{../Figure-files/error8brick_square_plate_simply_supported.jpeg}
%     \caption{Error scale 0\% - 80\%}
%   \end{subfigure}
%   \begin{subfigure}{0.5\textwidth}
%     \centering
%     \includegraphics[width=6cm]{../Figure-files/error8brick_square_plate_simply_supported100.jpeg}
%     \caption{Error scale 0\% - 100\%}
%   \end{subfigure}
%   \captionsetup{justification=centering,margin=3cm}
%   \caption{Two sub}
%   % \caption{}
%   % \label{}
% \end{figure}

The errors were plotted in Figure (\ref{fig 8NodeBrick square plate with four edge simply supported}).
\begin{figure}[H]
  % \centering
    \centering
    \includegraphics[width=6cm]{../Figure-files/error8brick_square_plate_simply_supported.jpeg}
  \captionsetup{justification=centering,margin=3cm}
  \caption{8NodeBrick square plate with four edges simply supported\\
      Displacement error   versus   Number of side division}
  \label{fig 8NodeBrick square plate with four edge simply supported}
\end{figure}


The ESSI model fei files for the table above are \href{https://github.com/yuan-energy/ESSI_Verification/blob/master/8NodeBrick/square_plate_simply_support/square_plate_simply_support.tar.gz?raw=true}{here}

























\newpage
\subsection{Verification of 8NodeBrick circular plate with all edges clamped}

Problem description: Diameter=20m, Height=1m, Force=100N, E=1E8Pa, $\nu=0.3$. 

The four edges are clamped. 

The load is the uniform normal pressure on the whole plate. 


The plate flexural rigidity is 

\begin{equation}
  D=\frac{Eh^3}{12(1-\nu^2)}=\frac{10^8 N/m^2 \times 1^3 m^3 }{12 \times (1-0.3^2) }= 9.1575 \times 10^6 \ N\cdot m
\end{equation}

The theoretical solution\footnote{Stephen Timoshenko, Theory of plates and shells (2nd edition). MrGRAW-Hill Inc, page55, 1959.} is 

\begin{equation}
  d= \frac{q a^4}{64D}=\frac{100 N/m^2 \times 10^4 m^4}{64 \times 9.1575 \times 10^6 \ N\cdot m}=1.7106\times 10^{-3} m
\end{equation}



The 8NodeBrick were shown in Figure (\ref{fig 8NodeBrick edges clamped circular plate with element side length 10m }) - (\ref{fig 8NodeBrick edges clamped circular plate with element side length 0.25m }). 




\begin{figure}[H]
  \centering
  \includegraphics[width=9cm]{../Figure-files/circular_plate1.pdf}
  \caption{8NodeBrick edge clamped circular plate with element side length 10m }
  \label{fig 8NodeBrick edges clamped circular plate with element side length 10m }
\end{figure}

\newpage

\begin{figure}[H]
  \centering
  \includegraphics[width=9cm]{../Figure-files/circular_plate2.pdf}
  \caption{8NodeBrick edge clamped circular plate with element side length 5m }
  \label{fig 8NodeBrick edges clamped circular plate with element side length 5m }
\end{figure}


\begin{figure}[H]
  \centering
  \includegraphics[width=9cm]{../Figure-files/circular_plate3.pdf}
  \caption{8NodeBrick edge clamped circular plate with element side length 2m }
  \label{fig 8NodeBrick edges clamped circular plate with element side length 2m }
\end{figure}

\newpage

\begin{figure}[H]
  \centering
  \includegraphics[width=9cm]{../Figure-files/circular_plate4.pdf}
  \caption{8NodeBrick edge clamped circular plate with element side length 1m }
  \label{fig 8NodeBrick edges clamped circular plate with element side length 1m }
\end{figure}


\begin{figure}[H]
  \centering
  \includegraphics[width=9cm]{../Figure-files/circular_plate5.pdf}
  \caption{8NodeBrick edge clamped circular plate with element side length 0.5m }
  \label{fig 8NodeBrick edges clamped circular plate with element side length 0.5m }
\end{figure}

\newpage

\begin{figure}[H]
  \centering
  \includegraphics[width=9cm]{../Figure-files/circular_plate6.pdf}
  \caption{8NodeBrick edge clamped circular plate with element side length 0.25m }
  \label{fig 8NodeBrick edges clamped circular plate with element side length 0.25m }
\end{figure}

The results were listed in Table (\ref{table Results for 8NodeBrick circular plate with four edges clamped}).

\begin{table}[H]
  \centering
    \caption{Results for 8NodeBrick circular plate with four edges clamped}
  \label{table Results for 8NodeBrick circular plate with four edges clamped}
\begin{tabular}{|c|c|c|c|c|}
\hline
Element type     & 8NodeBrick     & 8NodeBrick     & 8NodeBrick     &  \multirow{3}{*}{\tabincell{c}{Theoretical \\ displacement}} \\ \cline{1-4}
Number of layers & 1layer         & 2layers         & 4layers         &          \\ \cline{1-4}
Number of diameter divisions & Height:1.00$m$ & Height:0.50$m$ & Height:0.25$m$ &          \\ \hline
4           & 1.97E-04 $m$ & 1.99E-04 $m$ & 2.00E-04 $m$ & 1.71E-03 $m$ \\ \hline
12          & 7.95E-04 $m$ & 8.47E-04 $m$ & 8.62E-04 $m$ & 1.71E-03 $m$ \\ \hline
20          & 1.13E-03 $m$ & 1.25E-03 $m$ & 1.28E-03 $m$ & 1.71E-03 $m$ \\ \hline
40          & 1.36E-03 $m$ & 1.54E-03 $m$ & 1.60E-03 $m$ & 1.71E-03 $m$ \\ \hline
60          & 1.41E-03 $m$ & 1.62E-03 $m$ & 1.68E-03 $m$ & 1.71E-03 $m$ \\ \hline
80          & 1.43E-03 $m$ & 1.64E-03 $m$ & 1.71E-03 $m$ & 1.71E-03 $m$ \\
\hline
\end{tabular}
\end{table}


The errors were listed in Table (\ref{table errors for 8NodeBrick circular plate with four edges clamped}).

\begin{table}[H]
  \centering
      \caption{Errors for 8NodeBrick circular plate with four edges clamped}
  \label{table errors for 8NodeBrick circular plate with four edges clamped}
\begin{tabular}{|c|c|c|c|c|}
\hline
Element type     & 8NodeBrick     & 8NodeBrick     & 8NodeBrick      \\ \hline
Number of layers & 1layer         & 2layers         & 4layers          \\ \hline
Number of diameter divisions & Height:1.00$m$ & Height:0.50$m$ & Height:0.25$m$  \\ \hline
4           & 88.43\% & 88.32\% & 88.30\%       \\ \hline
12          & 53.43\% & 50.35\% & 49.47\%       \\ \hline
20          & 33.79\% & 27.00\% & 24.93\%       \\ \hline
40          & 20.14\% & 9.47\%  & 6.03\%        \\ \hline
60          & 17.11\% & 5.34\%  & 1.51\%        \\ \hline
80          & 16.01\% & 3.80\%  & 0.19\%        \\
\hline
\end{tabular}
\end{table}


% \begin{figure}[H]
%   % \centering
%   \begin{subfigure}{0.5\textwidth}
%     \centering
%     \includegraphics[width=6cm]{../Figure-files/error8brick_circular_plate_clamped.jpeg}
%     \caption{Error scale 0\% - 80\%}
%   \end{subfigure}
%   \begin{subfigure}{0.5\textwidth}
%     \centering
%     \includegraphics[width=6cm]{../Figure-files/error8brick_circular_plate_clamped100.jpeg}
%     \caption{Error scale 0\% - 100\%}
%   \end{subfigure}
%   \captionsetup{justification=centering,margin=3cm}
%   \caption{Two sub}
%   % \caption{}
%   % \label{}
% \end{figure}

The errors were shown in Figure (\ref{fig 8NodeBrick circular plate with edge clamped}).
\begin{figure}[H]
  % \centering
    \centering
    \includegraphics[width=6cm]{../Figure-files/error8brick_circular_plate_clamped.jpeg}
  \captionsetup{justification=centering,margin=3cm}
  \caption{8NodeBrick circular plate with edge clamped\\
      Displacement error   versus   Number of side division}
  \label{fig 8NodeBrick circular plate with edge clamped}
\end{figure}



The ESSI model fei files for the table above are \href{https://github.com/yuan-energy/ESSI_Verification/blob/master/8NodeBrick/circular_plate_clamped/circular_plate_clamped.tar.gz?raw=true}{here}






\newpage
\subsection{Verification of 8NodeBrick circular plate with all edges simply supported}


Problem description: Diameter=20m, Height=1m, Force=100N, E=1E8Pa, $\nu=0.3$. 

The four edges are simply supported. 

The load is the uniform normal pressure on the whole plate. 


The plate flexural rigidity is 

\begin{equation}
  D=\frac{Eh^3}{12(1-\nu^2)}=\frac{10^8 N/m^2 \times 1^3 m^3 }{12 \times (1-0.3^2) }= 9.1575 \times 10^6 \ N\cdot m
\end{equation}

The theoretical solution\footnote{Stephen Timoshenko, Theory of plates and shells (2nd edition). MrGRAW-Hill Inc, page55, 1959.} is 

\begin{equation}
  d= \frac{(5+\nu)  q a^4}{64(1+\nu) D}=\frac{(5+0.3)\times 100 N/m^2 \times 10^4 m^4}{64\times(1+0.3) \times 9.1575 \times 10^6 \ N\cdot m}=6.956\times 10^{-3} m
\end{equation}




The 8NodeBrick were shown in Figure (\ref{fig 8NodeBrick edges simply supported circular plate with element side length 10m }) - (\ref{fig 8NodeBrick edges simply supported circular plate with element side length 0.25m }). 



\begin{figure}[H]
  \centering
  \includegraphics[width=11cm]{../Figure-files/circular_plate1.pdf}
  \caption{8NodeBrick edge simply supported circular plate with element side length 10m }
  \label{fig 8NodeBrick edges simply supported circular plate with element side length 10m }
\end{figure}

\newpage

\begin{figure}[H]
  \centering
  \includegraphics[width=11cm]{../Figure-files/circular_plate2.pdf}
  \caption{8NodeBrick edge simply supported circular plate with element side length 5m }
  \label{fig 8NodeBrick edges simply supported circular plate with element side length 5m }
\end{figure}


\begin{figure}[H]
  \centering
  \includegraphics[width=11cm]{../Figure-files/circular_plate3.pdf}
  \caption{8NodeBrick edge simply supported circular plate with element side length 2m }
  \label{fig 8NodeBrick edges simply supported circular plate with element side length 2m }
\end{figure}

\newpage

\begin{figure}[H]
  \centering
  \includegraphics[width=11cm]{../Figure-files/circular_plate4.pdf}
  \caption{8NodeBrick edge simply supported circular plate with element side length 1m }
  \label{fig 8NodeBrick edges simply supported circular plate with element side length 1m }
\end{figure}


\begin{figure}[H]
  \centering
  \includegraphics[width=11cm]{../Figure-files/circular_plate5.pdf}
  \caption{8NodeBrick edge simply supported circular plate with element side length 0.5m }
  \label{fig 8NodeBrick edges simply supported circular plate with element side length 0.5m }
\end{figure}

\newpage

\begin{figure}[H]
  \centering
  \includegraphics[width=11cm]{../Figure-files/circular_plate6.pdf}
  \caption{8NodeBrick edge simply supported circular plate with element side length 0.25m }
  \label{fig 8NodeBrick edges simply supported circular plate with element side length 0.25m }
\end{figure}





The results were listed in Table (\ref{table Results for 8NodeBrick cicular plate with four edges simply supported}).

\begin{table}[H]
  \centering
  \caption{Results for 8NodeBrick cicular plate with four edges simply supported}
  \label{table Results for 8NodeBrick cicular plate with four edges simply supported}
\begin{tabular}{|c|c|c|c|c|}
\hline
Element type        & 8NodeBrick     & 8NodeBrick     &  \multirow{3}{*}{\tabincell{c}{Theoretical \\ displacement}} \\ \cline{1-3}
Number of layers    & 2layers         & 4layers         &          \\ \cline{1-3}
Number of diameter divisions & Height:0.50$m$ & Height:0.25$m$ &          \\ \hline
4            & 6.35E-04 $m$ & 6.39E-04 $m$ & 6.96E-03 $m$ \\ \hline
12           & 3.46E-03 $m$ & 3.57E-03 $m$ & 6.96E-03 $m$ \\ \hline
20           & 4.96E-03 $m$ & 5.18E-03 $m$ & 6.96E-03 $m$ \\ \hline
40           & 6.05E-03 $m$ & 6.37E-03 $m$ & 6.96E-03 $m$ \\ \hline
60           & 6.30E-03 $m$ & 6.65E-03 $m$ & 6.96E-03 $m$ \\ \hline
80           & 6.39E-03 $m$ & 6.76E-03 $m$ & 6.96E-03 $m$ \\
\hline
\end{tabular}
\end{table}


The errors were listed in Table (\ref{table Errors for 8NodeBrick cicular plate with four edges simply supported}).

\begin{table}[H]
  \centering
  \caption{Errors for 8NodeBrick cicular plate with four edges simply supported}
  \label{table Errors for 8NodeBrick cicular plate with four edges simply supported}
\begin{tabular}{|c|c|c|c|c|}
\hline
Element type     & 8NodeBrick     & 8NodeBrick      \\ \hline
Number of layers      & 2layers         & 4layers          \\ \hline
Number of diameter divisions & Height:0.50$m$ & Height:0.25$m$  \\ \hline
4            & 90.87\% & 90.82\%       \\ \hline
12           & 50.19\% & 48.65\%       \\ \hline
20           & 28.64\% & 25.47\%       \\ \hline
40           & 13.09\% & 8.40\%        \\ \hline
60           & 9.45\%  & 4.36\%        \\ \hline
80           & 8.10\%  & 2.85\%        \\
\hline
\end{tabular}
\end{table}




% \begin{figure}[H]
%   % \centering
%   \begin{subfigure}{0.5\textwidth}
%     \centering
%     \includegraphics[width=6cm]{../Figure-files/error8brick_circular_plate_simply_supported.jpeg}
%     \caption{Error scale 0\% - 80\%}
%   \end{subfigure}
%   \begin{subfigure}{0.5\textwidth}
%     \centering
%     \includegraphics[width=6cm]{../Figure-files/error8brick_circular_plate_simply_supported100.jpeg}
%     \caption{Error scale 0\% - 100\%}
%   \end{subfigure}
%   \captionsetup{justification=centering,margin=3cm}
%   \caption{Two sub}
%   % \caption{}
%   % \label{}
% \end{figure}

The errors were plotted in Figure (\ref{fig 8NodeBrick circular plate with four edge simply supported}).

\begin{figure}[H]
  % \centering
    \centering
    \includegraphics[width=6cm]{../Figure-files/error8brick_circular_plate_simply_supported.jpeg}
  \captionsetup{justification=centering,margin=3cm}
  \caption{8NodeBrick circular plate with edge simply supported\\
      Displacement error   versus   Number of side division}
  \label{fig 8NodeBrick circular plate with four edge simply supported}
\end{figure}


The ESSI model fei files for the table above are \href{https://github.com/yuan-energy/ESSI_Verification/blob/master/8NodeBrick/circular_plate_simply_support/circular_plate_simply_support.tar.gz?raw=true}{here}.


































































\newpage
% \begin{center}
%   \Large\textbf{Verification for 27NodeBrick}
% \end{center}
%\title{Scientific computing in geotechnical engineering}
%\maketitle

\section{Verification of 27NodeBrick elements}
\subsection{Verification of 27NodeBrick cantilever beams}

% The model is \href{https://github.com/yuan-energy/test/archive/master.zip}{here}.

% The model is \href{https://github.com/yuan-energy/test/tree/master/example}{here}.

% The model is \href{https://github.com/yuan-energy/test/blob/master/example_2/example3.zip?raw=true}{here}.



Problem description: Length=6m, Width=1m, Height=1m, Force=100N, E=1E8Pa, $\nu=0.0$. Use the shear deformation coefficient $\kappa=1.2$. The force direction was shown in Figure (\ref{fig Problem description for cantilever 27}). 

\begin{figure}[H]
  \centering
  \includegraphics[width=7cm]{../Figure-files/cantilever_6.pdf}
  \caption{Problem description for cantilever beams}
  \label{fig Problem description for cantilever 27}
\end{figure}


Theoretical displacement (bending and shear deformation):
\begin{equation}
  \begin{aligned}
  d &=\frac{FL^3}{3EI}+\frac{FL}{GA_v} \\
  &= \frac{FL^3}{3E\frac{bh^3}{12}}+\frac{FL}{\frac{E}{2(1+\nu)} \frac{bh}{\kappa}} \\ 
    &= \frac{100 N \times 6^3 m^3}{3\times 10^8 N/m^2 \times \frac{1}{12} m^4}+ 
    \frac{100 N\times 6 m}{\frac{10}{2} \times 10^7 N/m^2\times 1 m^2 \times \frac{5}{6}} \\ 
    &=8.64\times 10^{-4} m + 0.144 \times 10^{-4} m   \\
   & =8.784\times 10^{-4} \ m
   \end{aligned}
\end{equation}



Numerical model:

The 27NodeBrick elements were shown in Figure (\ref{fig 27NodeBrick elements for cantilever beams}).

\begin{figure}[H]
  \centering
  \begin{subfigure}{0.5\textwidth}
    \centering
    \includegraphics[width=9cm]{../Figure-files/beam_27brick_1div.pdf}
    \caption{One 27NodeBrick element}
  \end{subfigure}
  \vskip 8pt
  \begin{subfigure}{0.5\textwidth}
    \centering
    \includegraphics[width=9cm]{../Figure-files/beam_27brick_2div.pdf}
    \caption{Two 27NodeBrick elements}
  \end{subfigure}
  \vskip 8pt
  \begin{subfigure}{0.5\textwidth}
    \centering
    \includegraphics[width=9cm]{../Figure-files/beam_27brick_6div.pdf}
    \caption{Six 27NodeBrick elements}
  \end{subfigure}
  \captionsetup{justification=centering,margin=3cm}
  \caption{27NodeBrick elements for cantilever beams}
  \label{fig 27NodeBrick elements for cantilever beams}
\end{figure}



% 27NodeBrick element:
% \begin{figure}[H]
%   \centering
%   \includegraphics[width=9cm]{../Figure-files/beam_27brick_1div.pdf}
%   % \caption{}
%   % \label{}
% \end{figure}


% \begin{figure}[H]
%   \centering
%   \includegraphics[width=9cm]{../Figure-files/beam_27brick_2div.pdf}
%   % \caption{}
%   % \label{}
% \end{figure}

% \begin{figure}[H]
%   \centering
%   \includegraphics[width=9cm]{../Figure-files/beam_27brick_6div.pdf}
%   % \caption{}
%   % \label{}
% \end{figure}




All the ESSI results were listed in Table (\ref{table 27NodeBrick cantilever beams results for different element number}). 
\begin{table}[H]
  \centering
      \caption{Results for 27NodeBrick cantilever beams of different element numbers}
    \label{table 27NodeBrick cantilever beams results for different element number}
    \begin{tabular}{|c|c|c|c|}
      \hline
      Element number & 1        & 2        & 6         \\  \hline
      27NodeBrick     & 7.07E-04 $m$ & 8.50E-04 $m$ & 8.75E-04 $m$     \\ \hline
      Error           & 19.52\% & 3.19\% & 0.34\%    \\ 
      \hline 
    \end{tabular}
\end{table}



% \begin{figure}[H]
%   \centering
%   \includegraphics[width=9cm]{../Figure-files/error27brick_beam_different_element_number.jpeg}
%   % \caption{}
%   % \label{}
% \end{figure}
The errors were plotted in Figure (\ref{fig error 27NodeBrick cantilever beam for different element number}).

\begin{figure}[H]
  % \centering
  \begin{subfigure}{0.5\textwidth}
    \centering
    \includegraphics[width=6cm]{../Figure-files/error27brick_beam_different_element_number.jpeg}
    \caption{Error scale 0\% - 20\%}
  \end{subfigure}
  \begin{subfigure}{0.5\textwidth}
    \centering
    \includegraphics[width=6cm]{../Figure-files/error27brick_beam_different_element_number100.jpeg}
    \caption{Error scale 0\% - 100\%}
  \end{subfigure}
  \captionsetup{justification=centering,margin=2cm}
  \caption{27NodeBrick cantilever beam for different element number\\
    Displacement error   versus   Number of elements}
  \label{fig error 27NodeBrick cantilever beam for different element number}
\end{figure}


The ESSI model fei files for the table above are \href{https://github.com/yuan-energy/ESSI_Verification/blob/master/27NodeBrick/cantilever_different_element_number/cantilever_different_element_number.tar.gz?raw=true}{here}



% \newpage
% \begin{itemize}
%   \item \textbf{\emph{Cantilever: different geometry}}
% \end{itemize}

% In the figures above, only the model with geometry $6m\times 1m \times 1m$ was drawed. In the ESSI models, the geometry $10m\times 1m \times 1m$ and the geometry $20m\times 1m \times 1m$ were also calculated. In three different geometry models, all the element sizes were $1m\times 1m \times 1m$. Therefore, the number of elements used in each model were $6,\ 10\ and\ 20$ respectively.

% The ESSI results for different geometry were listed in Table (\ref{table Results for 27NodeBrick cantilever beams of different geometry}). 

% \begin{table}[H]
%   \centering
%   \caption{Results for 27NodeBrick cantilever beams of different geometry}
%   \label{table Results for 27NodeBrick cantilever beams of different geometry}
%   \begin{tabular}{|c|c|c|c|c|c|}
%   \hline
%   Geometry & 27NodeBrick & Theoretical(bending) & Theoretical(shear) & Theoretical(all) & Error   \\ \hline
%   1:6      & 8.75E-04 $m$ & 8.64E-04 $m$ & 1.20E-05 $m$ & 8.76E-04 $m$ & 0.06\% \\ \hline
%   1:10     & 4.02E-03 $m$ & 4.00E-03 $m$ & 2.00E-05 $m$ & 4.02E-03 $m$ & 0.02\% \\ \hline
%   1:20     & 3.20E-02 $m$ & 3.20E-02 $m$ & 4.00E-05 $m$ & 3.20E-02 $m$ & 0.01\% \\
%   \hline
%   \end{tabular}
% \end{table}

% The ESSI model fei files for the table above are \href{https://github.com/yuan-energy/ESSI_Verification/blob/master/27NodeBrick/cantilever_different_geometry/cantilever_different_geometry.tar.gz?raw=true}{here}








\newpage
\subsection{Verification of 27NodeBrick cantilever beam for different Poisson's ratio}




Problem description: Length=6m, Width=1m, Height=1m, Force=100N, E=1E8Pa, $\nu=0.0-0.49$.
The force direction was shown in Figure (\ref{fig Problem description for cantilever beams of different Poisson's 27}). 

\begin{figure}[H]
  \centering
  \includegraphics[width=7cm]{../Figure-files/cantilever_6.pdf}
  \caption{Problem description for cantilever beams of different Poisson's ratios}
  \label{fig Problem description for cantilever beams of different Poisson's 27}
\end{figure}

The theoretical solution for $\nu=0.0$ was calculated below, while the solution for other Poisson's ratio were calculated by the similar process. 

Theoretical displacement (bending and shear deformation):
\begin{equation}
  \begin{aligned}
  d &=\frac{FL^3}{3EI}+\frac{FL}{GA_v} \\
  &= \frac{FL^3}{3E\frac{bh^3}{12}}+\frac{FL}{\frac{E}{2(1+\nu)} \frac{bh}{\kappa}} \\ 
    &= \frac{100 N \times 6^3 m^3}{3\times 10^8 N/m^2 \times \frac{1}{12} m^4}+ 
    \frac{100 N\times 6 m}{\frac{10}{2} \times 10^7 N/m^2\times 1 m^2 \times \frac{5}{6}} \\ 
    &=8.64\times 10^{-4} m + 0.144 \times 10^{-4} m   \\
   & =8.784\times 10^{-4} \ m
   \end{aligned}
\end{equation}

The rotation angle at the end:
\begin{equation}
  \theta =\frac{FL^2}{2EI} 
   =\frac{100 N \times 6^2 m^2} {2\times 10^8 N/m^2 \times \frac{1}{12} m^4} 
 =2.16 \times 10^{-4} \ rad = 0.0124 \degree 
\end{equation}

The 27NodeBrick elements for cantilever beams of different Poisson's ratios were shown in Figure (\ref{fig 27NodeBrick elements for cantilever beams of different Poisson's ratios}):
\begin{figure}[H]
  \centering
  \includegraphics[width=9cm]{../Figure-files/beam_27brick_6div.pdf}
  \caption{27NodeBrick elements for cantilever beams of different Poisson's ratios}
  \label{fig 27NodeBrick elements for cantilever beams of different Poisson's ratios}
\end{figure}


All the displacement results were listed in Table (\ref{table Displacement results for 27NodeBrick cantilever beams of different Poisson ratios}). 

\begin{table}[H]
  \centering
    \captionsetup{justification=centering,margin=3cm}
    \caption{\emph{\textbf{Displacement}} results for 27NodeBrick cantilever beams with \textcolor{red}{element side length 1 m}}
  \label{table Displacement results for 27NodeBrick cantilever beams of different Poisson ratios}
  \begin{tabular}{|c|c|c|c|c|c|}
    \hline 
\tabincell{c}{Poisson's \\ ratio}    & \tabincell{c}{27NodeBrick\\displacement} & \tabincell{c}{Theory displacement\\(bending)} & \tabincell{c}{Theory displacement\\(shear)} & \tabincell{c}{Theory\\displacement(all)}   & Error \\ \hline
0.00 & 8.755E-04 $m$ & 8.640E-04 $m$ & 1.440E-05 $m$  & 8.784E-04 $m$  & 0.34\%     \\ \hline
0.05 & 8.757E-04 $m$ & 8.640E-04 $m$ & 1.512E-05 $m$  & 8.791E-04 $m$  & 0.39\%     \\ \hline
0.10 & 8.751E-04 $m$ & 8.640E-04 $m$ & 1.586E-05 $m$  & 8.799E-04 $m$  & 0.54\%     \\ \hline
0.15 & 8.735E-04 $m$ & 8.640E-04 $m$ & 1.659E-05 $m$  & 8.806E-04 $m$  & 0.80\%     \\ \hline
0.20 & 8.708E-04 $m$ & 8.640E-04 $m$ & 1.734E-05 $m$  & 8.813E-04 $m$  & 1.19\%     \\ \hline
0.25 & 8.667E-04 $m$ & 8.640E-04 $m$ & 1.808E-05 $m$  & 8.821E-04 $m$  & 1.74\%     \\ \hline
0.30 & 8.608E-04 $m$ & 8.640E-04 $m$ & 1.884E-05 $m$  & 8.828E-04 $m$  & 2.50\%     \\ \hline
0.35 & 8.520E-04 $m$ & 8.640E-04 $m$ & 1.959E-05 $m$  & 8.836E-04 $m$  & 3.57\%     \\ \hline
0.40 & 8.385E-04 $m$ & 8.640E-04 $m$ & 2.035E-05 $m$  & 8.844E-04 $m$  & 5.18\%     \\ \hline
0.45 & 8.147E-04 $m$ & 8.640E-04 $m$ & 2.111E-05 $m$  & 8.851E-04 $m$  & 7.96\%     \\ \hline
0.49 & 7.711E-04 $m$ & 8.640E-04 $m$ & 2.173E-05 $m$  & 8.857E-04 $m$  & 12.94\%    \\ \hline
  \end{tabular}
  % \caption{}
\end{table}



\begin{table}[H]
  \centering
    \captionsetup{justification=centering,margin=3cm}
    \caption{\emph{\textbf{Displacement}} results for 27NodeBrick cantilever beams with \textcolor{red}{element side length 0.5 m}}
  \label{table Displacement results for 27NodeBrick cantilever beams of different Poisson ratios 2}
  \begin{tabular}{|c|c|c|c|c|c|}
    \hline 
\tabincell{c}{Poisson's \\ ratio}    & \tabincell{c}{27NodeBrick\\displacement} & \tabincell{c}{Theory displacement\\(bending)} & \tabincell{c}{Theory displacement\\(shear)} & \tabincell{c}{Theory\\displacement(all)}   & Error \\ \hline
0.00 & 8.804E-04 $m$ & 8.640E-04 $m$ & 1.440E-05 $m$  & 8.784E-04 $m$  & 0.23\%    \\ \hline
0.05 & 8.808E-04 $m$ & 8.640E-04 $m$ & 1.512E-05 $m$  & 8.791E-04 $m$  & 0.19\%    \\ \hline
0.10 & 8.805E-04 $m$ & 8.640E-04 $m$ & 1.586E-05 $m$  & 8.799E-04 $m$  & 0.08\%    \\ \hline
0.15 & 8.796E-04 $m$ & 8.640E-04 $m$ & 1.659E-05 $m$  & 8.806E-04 $m$  & 0.12\%    \\ \hline
0.20 & 8.778E-04 $m$ & 8.640E-04 $m$ & 1.734E-05 $m$  & 8.813E-04 $m$  & 0.40\%    \\ \hline
0.25 & 8.752E-04 $m$ & 8.640E-04 $m$ & 1.808E-05 $m$  & 8.821E-04 $m$  & 0.78\%    \\ \hline
0.30 & 8.715E-04 $m$ & 8.640E-04 $m$ & 1.884E-05 $m$  & 8.828E-04 $m$  & 1.28\%    \\ \hline
0.35 & 8.663E-04 $m$ & 8.640E-04 $m$ & 1.959E-05 $m$  & 8.836E-04 $m$  & 1.95\%    \\ \hline
0.40 & 8.588E-04 $m$ & 8.640E-04 $m$ & 2.035E-05 $m$  & 8.844E-04 $m$  & 2.89\%    \\ \hline
0.45 & 8.465E-04 $m$ & 8.640E-04 $m$ & 2.111E-05 $m$  & 8.851E-04 $m$  & 4.36\%    \\ \hline
0.49 & 8.248E-04 $m$ & 8.640E-04 $m$ & 2.173E-05 $m$  & 8.857E-04 $m$  & 6.88\%    \\ \hline
  \end{tabular}
  % \caption{}
\end{table}



\begin{table}[H]
  \centering
    \captionsetup{justification=centering,margin=3cm}
    \caption{\emph{\textbf{Displacement}} results for 27NodeBrick cantilever beams with \textcolor{red}{element side length 0.25 m}}
  \label{table Displacement results for 27NodeBrick cantilever beams of different Poisson ratios 3}
  \begin{tabular}{|c|c|c|c|c|c|}
    \hline 
\tabincell{c}{Poisson's \\ ratio}    & \tabincell{c}{27NodeBrick\\displacement} & \tabincell{c}{Theory displacement\\(bending)} & \tabincell{c}{Theory displacement\\(shear)} & \tabincell{c}{Theory\\displacement(all)}   & Error \\ \hline
0.00 & 8.797E-04 $m$ & 8.640E-04 $m$ & 1.440E-05 $m$  & 8.784E-04 $m$  & 0.15\%    \\ \hline
0.05 & 8.801E-04 $m$ & 8.640E-04 $m$ & 1.512E-05 $m$  & 8.791E-04 $m$  & 0.11\%    \\ \hline
0.10 & 8.799E-04 $m$ & 8.640E-04 $m$ & 1.586E-05 $m$  & 8.799E-04 $m$  & 0.01\%    \\ \hline
0.15 & 8.792E-04 $m$ & 8.640E-04 $m$ & 1.659E-05 $m$  & 8.806E-04 $m$  & 0.16\%    \\ \hline
0.20 & 8.778E-04 $m$ & 8.640E-04 $m$ & 1.734E-05 $m$  & 8.813E-04 $m$  & 0.40\%    \\ \hline
0.25 & 8.758E-04 $m$ & 8.640E-04 $m$ & 1.808E-05 $m$  & 8.821E-04 $m$  & 0.71\%    \\ \hline
0.30 & 8.730E-04 $m$ & 8.640E-04 $m$ & 1.884E-05 $m$  & 8.828E-04 $m$  & 1.12\%    \\ \hline
0.35 & 8.692E-04 $m$ & 8.640E-04 $m$ & 1.959E-05 $m$  & 8.836E-04 $m$  & 1.63\%    \\ \hline
0.40 & 8.641E-04 $m$ & 8.640E-04 $m$ & 2.035E-05 $m$  & 8.844E-04 $m$  & 2.29\%    \\ \hline
0.45 & 8.567E-04 $m$ & 8.640E-04 $m$ & 2.111E-05 $m$  & 8.851E-04 $m$  & 3.21\%    \\ \hline
0.49 & 8.452E-04 $m$ & 8.640E-04 $m$ & 2.173E-05 $m$  & 8.857E-04 $m$  & 4.58\%    \\ \hline
  \end{tabular}
  % \caption{}
\end{table}



% \begin{figure}[H]
%   \centering
%   \includegraphics[width=9cm]{../Figure-files/error27brick_beam_different_poisson_ratio_disp.jpeg}
%   % \caption{}
%   % \label{}
% \end{figure}

The errors were plotted in Figure (\ref{fig error 27NodeBrick cantilever beam for different Poisson's ratio}).

\begin{figure}[H]
  % \centering
  \begin{subfigure}{0.5\textwidth}
    \centering
    \includegraphics[width=6cm]{../Figure-files/error27brick_beam_different_poisson_ratio_disp_div.jpeg}
    \caption{Error scale 0\% - 15\%}
  \end{subfigure}
  \begin{subfigure}{0.5\textwidth}
    \centering
    \includegraphics[width=6cm]{../Figure-files/error27brick_beam_different_poisson_ratio_disp_div100.jpeg}
    \caption{Error scale 0\% - 100\%}
  \end{subfigure}
  \captionsetup{justification=centering,margin=3cm}
  \caption{27NodeBrick cantilever beam for different Poisson's ratio\\
     \textbf{\emph{ Displacement error }}  versus   Poisson's ratio}
  \label{fig error 27NodeBrick cantilever beam for different Poisson's ratio}
\end{figure}



The angle results were listed in Table (\ref{table angle results for 27NodeBrick cantilever beams of different Poissons ratios}).
\begin{table}[H]
  \centering
  \captionsetup{justification=centering,margin=3cm}
  \caption{\emph{\textbf{Rotation angle}} results for 27NodeBrick cantilever beams with \textcolor{red}{element side length 1 m}}
  \label{table angle results for 27NodeBrick cantilever beams of different Poissons ratios}
\begin{tabular}{|c|c|c|c|}
\hline
\tabincell{c}{Poisson's \\ ratio} & \tabincell{c}{27NodeBrick \\ angle (unit:\degree)}  & \tabincell{c}{Theory angle\\(unit:\degree)}  & Error   \\ \hline
0.00            & 1.238E-02 & 1.24E-02 & 0.19\% \\ \hline
0.05            & 1.237E-02 & 1.24E-02 & 0.24\% \\ \hline
0.10            & 1.236E-02 & 1.24E-02 & 0.34\% \\ \hline
0.15            & 1.233E-02 & 1.24E-02 & 0.53\% \\ \hline
0.20            & 1.230E-02 & 1.24E-02 & 0.80\% \\ \hline
0.25            & 1.225E-02 & 1.24E-02 & 1.18\% \\ \hline
0.30            & 1.219E-02 & 1.24E-02 & 1.70\% \\ \hline
0.35            & 1.210E-02 & 1.24E-02 & 2.45\% \\ \hline
0.40            & 1.196E-02 & 1.24E-02 & 3.55\% \\ \hline
0.45            & 1.172E-02 & 1.24E-02 & 5.47\% \\ \hline
0.49            & 1.130E-02 & 1.24E-02 & 8.89\% \\
\hline
\end{tabular}
  % \caption{}
\end{table}


\begin{table}[H]
  \centering
  \captionsetup{justification=centering,margin=3cm}
  \caption{\emph{\textbf{Rotation angle}} results for 27NodeBrick cantilever beams with \textcolor{red}{element side length 0.5 m}}
  \label{table angle results for 27NodeBrick cantilever beams of different Poissons ratios 2}
\begin{tabular}{|c|c|c|c|}
\hline
\tabincell{c}{Poisson's \\ ratio} & \tabincell{c}{27NodeBrick \\ angle (unit:\degree)}  & \tabincell{c}{Theory angle\\(unit:\degree)}  & Error   \\ \hline
0.00            & 1.242E-02 & 1.24E-02 & 0.12\% \\ \hline
0.05            & 1.241E-02 & 1.24E-02 & 0.11\% \\ \hline
0.10            & 1.241E-02 & 1.24E-02 & 0.06\% \\ \hline
0.15            & 1.239E-02 & 1.24E-02 & 0.05\% \\ \hline
0.20            & 1.237E-02 & 1.24E-02 & 0.21\% \\ \hline
0.25            & 1.235E-02 & 1.24E-02 & 0.44\% \\ \hline
0.30            & 1.231E-02 & 1.24E-02 & 0.74\% \\ \hline
0.35            & 1.226E-02 & 1.24E-02 & 1.16\% \\ \hline
0.40            & 1.218E-02 & 1.24E-02 & 1.76\% \\ \hline
0.45            & 1.206E-02 & 1.24E-02 & 2.76\% \\ \hline
0.49            & 1.183E-02 & 1.24E-02 & 4.63\% \\
\hline
\end{tabular}
  % \caption{}
\end{table}


\begin{table}[H]
  \centering
  \captionsetup{justification=centering,margin=3cm}
  \caption{\emph{\textbf{Rotation angle}} results for 27NodeBrick cantilever beams with \textcolor{red}{element side length 0.25 m}}
  \label{table angle results for 27NodeBrick cantilever beams of different Poissons ratios 4}
\begin{tabular}{|c|c|c|c|}
\hline
\tabincell{c}{Poisson's \\ ratio} & \tabincell{c}{27NodeBrick \\ angle (unit:\degree)}  & \tabincell{c}{Theory angle\\(unit:\degree)}  & Error   \\ \hline
0.00            & 1.242E-02 & 1.24E-02 & 0.17\% \\ \hline
0.05            & 1.242E-02 & 1.24E-02 & 0.15\% \\ \hline
0.10            & 1.241E-02 & 1.24E-02 & 0.09\% \\ \hline
0.15            & 1.240E-02 & 1.24E-02 & 0.02\% \\ \hline
0.20            & 1.238E-02 & 1.24E-02 & 0.17\% \\ \hline
0.25            & 1.235E-02 & 1.24E-02 & 0.38\% \\ \hline
0.30            & 1.232E-02 & 1.24E-02 & 0.64\% \\ \hline
0.35            & 1.228E-02 & 1.24E-02 & 0.98\% \\ \hline
0.40            & 1.222E-02 & 1.24E-02 & 1.42\% \\ \hline
0.45            & 1.214E-02 & 1.24E-02 & 2.06\% \\ \hline
0.49            & 1.202E-02 & 1.24E-02 & 3.08\% \\
\hline
\end{tabular}
  % \caption{}
\end{table}

The errors were plotted in Figure (\ref{table angle error 27NodeBrick cantilever beam for different Poisson ratio}).

\begin{figure}[H]
  % \centering
  \begin{subfigure}{0.5\textwidth}
    \centering
    \includegraphics[width=6cm]{../Figure-files/error27brick_beam_different_poisson_ratio_angle_div.jpeg}
    \caption{Error scale 0\% - 10\%}
  \end{subfigure}
  \begin{subfigure}{0.5\textwidth}
    \centering
    \includegraphics[width=6cm]{../Figure-files/error27brick_beam_different_poisson_ratio_angle_div100.jpeg}
    \caption{Error scale 0\% - 100\%}
  \end{subfigure}
  \captionsetup{justification=centering,margin=3cm}
  \caption{27NodeBrick cantilever beam for different Poisson's ratio\\
      \textbf{\emph{Rotation angle error} }  versus   Poisson's ratio}
  \label{table angle error 27NodeBrick cantilever beam for different Poisson ratio}
\end{figure}




The ESSI model fei files for the table above are \href{https://github.com/yuan-energy/ESSI_Verification/blob/master/27NodeBrick/cantilever_different_Poisson/cantilever_different_Poisson.tar.gz?raw=true}{here}



\newpage
Then, different values of elastic modulus were also tried. The errors were plotted below.
\begin{figure}[H]
  \centering
  \includegraphics[width=13cm]{../Figure-files/3D_error.jpg}
  \caption{The influence of Poisson's ratio and elastic modulus on the errors} 
  \label{fig The influence of Poisson's ratio and elastic modulus on the errors}
\end{figure}


According to Fig.(\ref{fig The influence of Poisson's ratio and elastic modulus on the errors})), the different values of elastic modulus will not influence the error. 

However, the different Poisson's ratio will influence the error. The error will increase with the Poisson's ratio increase. 

% The reason is that we often have coefficients $\frac{E}{1-2\nu}$ in the stiffness matrix. When the Poisson's ratio approaches 0.5, the coefficients $\frac{E}{1-2\nu}$ will become too large. In this situation, this kind of stiffness matrix is called \emph{\textbf{ill-conditioned}} matrix, which need special algorithms to deal with, like regularization\footnote{Neumaier, Arnold. "Solving ill-conditioned and singular linear systems: A tutorial on regularization." SIAM review 40.3 (1998): 636-666.} and iteration.




\newpage
\subsection{Test of irregular shaped 27NodeBrick cantilever beams}

Cantilever model was used as an example. 
Three different shapes were tested. 


In the first test, the upper two nodes of each element were moved one half element size along the $y-axis$, while the lower two nodes were kept at the same location.  The element shape was shown in Figure (\ref{fig irregular shape 1 27NodeBrick cantilever beams }).

\begin{figure}[H]
  \centering
  \includegraphics[width=9cm]{../Figure-files/beam_brick27_shape1.pdf}
  \caption{27NodeBrick cantilever beams for irregular \textbf{\emph{Shape 1}} }
  \label{fig irregular shape 1 27NodeBrick cantilever beams }
\end{figure}


In the second test, the upper two nodes of each element were moved 90\% element size along the $y-axis$, while the lower two nodes were moved 90\% element size in the other direction along the $y-axis$. The element shape was shown in Figure (\ref{fig irregular shape 2 27NodeBrick cantilever beams }).

\begin{figure}[H]
  \centering
  \includegraphics[width=9cm]{../Figure-files/beam_brick27_shape2.pdf}
  \caption{27NodeBrick cantilever beams for irregular \textbf{\emph{Shape 2}} }
  \label{fig irregular shape 2 27NodeBrick cantilever beams }
\end{figure}



In the third test, the upper two nodes of each element were moved one half element size with different directions along the $y-axis$, while the lower two nodes were kept at the same location. The element shape was shown in Figure (\ref{fig irregular shape 3 27NodeBrick cantilever beams }).

\begin{figure}[H]
  \centering
  \includegraphics[width=9cm]{../Figure-files/beam_brick27_shape3.pdf}
  \caption{27NodeBrick cantilever beams for irregular \textbf{\emph{Shape 3}} }
  \label{fig irregular shape 3 27NodeBrick cantilever beams }
\end{figure}

The boundary conditions were shown in Figure (\ref{fig 27NodeBrick cantilever beam boundary conditions shape 1}), (\ref{fig 27NodeBrick cantilever beam boundary conditions shape 2}) and (\ref{fig 27NodeBrick cantilever beam boundary conditions shape 3}) .

\begin{figure}[H]
  \centering
    \begin{subfigure}{0.5\textwidth}
      \centering
      \includegraphics[width=9cm]{../Figure-files/beam_brick27_shape1_vertical.pdf}
      \caption{Veritical force}
    \end{subfigure}
    \begin{subfigure}{0.5\textwidth}
      \centering
      \includegraphics[width=9cm]{../Figure-files/beam_brick27_shape1_horizontal.pdf}
      \caption{Horizontal force}
    \end{subfigure}
  \caption{27NodeBrick cantilever beam boundary conditions for irregular \textbf{\emph{Shape 1}} }
  \label{fig 27NodeBrick cantilever beam boundary conditions shape 1}
\end{figure}


\begin{figure}[H]
  \centering
    \begin{subfigure}{0.5\textwidth}
      \centering
      \includegraphics[width=9cm]{../Figure-files/beam_brick27_shape2_vertical.pdf}
      \caption{Veritical force}
    \end{subfigure}
    \begin{subfigure}{0.5\textwidth}
      \centering
      \includegraphics[width=9cm]{../Figure-files/beam_brick27_shape2_horizontal.pdf}
      \caption{Horizontal force}
    \end{subfigure}
  \caption{27NodeBrick cantilever beam boundary conditions for irregular \textbf{\emph{Shape 2}} }
  \label{fig 27NodeBrick cantilever beam boundary conditions shape 2}
\end{figure}


\begin{figure}[H]
  \centering
    \begin{subfigure}{0.5\textwidth}
      \centering
      \includegraphics[width=9cm]{../Figure-files/beam_brick27_shape3_vertical.pdf}
      \caption{Veritical force}
    \end{subfigure}
    \begin{subfigure}{0.5\textwidth}
      \centering
      \includegraphics[width=9cm]{../Figure-files/beam_brick27_shape3_horizontal.pdf}
      \caption{Horizontal force}
    \end{subfigure}
  \caption{27NodeBrick cantilever beam boundary conditions for irregular \textbf{\emph{Shape 3}} }
  \label{fig 27NodeBrick cantilever beam boundary conditions shape 3}
\end{figure}


The ESSI results were listed in Table (\ref{table Results for 27NodeBrick cantilever beams of irregular shapes}). 
\begin{table}[H]
  \centering
  \caption{Results for 27NodeBrick cantilever beams of irregular shapes}
  \label{table Results for 27NodeBrick cantilever beams of irregular shapes}
  \begin{tabular}{|c|c|c|c|c|c|}
    \hline 
    \multicolumn{6}{|c|}{Displacements for irregular shaped element}   \\ \hline
    Element Type   & Force direction & Normal shape & Shape 1 & Shape 2 & Shape 3  \\ \hline 
    27NodeBrick     & Vertical ($z$)     & 8.755E-04 $m$  & 8.819E-04 $m$ & 8.709E-04 $m$ & 8.837E-04 $m$   \\ \hline
    27NodeBrick     & Transverse ($y$)   & 8.755E-04 $m$  & 8.831E-04 $m$ & 8.462E-04 $m$ & 8.824E-04 $m$   \\ \hline
    Theoretical    &      -              & 8.784E-04 $m$  & 8.784E-04 $m$ & 8.784E-04 $m$ & 8.784E-04 $m$ \\ \hline
  \end{tabular}
  % \caption{}
\end{table}

The errors were listed in Table (\ref{table Errors for irregular shaped 27NodeBrick compared to theoretical solution}) and (\ref{talbe Errors for irregular shaped 27NodeBrick compared to normal shape}).


\begin{table}[H]
  \centering
  \caption{Errors for irregular shaped 27NodeBrick compared to theoretical solution}
  \label{table Errors for irregular shaped 27NodeBrick compared to theoretical solution}
  \begin{tabular}{|c|c|c|c|c|c|}
    \hline 
    \multicolumn{6}{|c|}{Errors for irregular shaped element, compared to theoretical solutions}   \\ \hline
    Element Type   & Force direction & Normal shape & Shape 1 & Shape 2 & Shape 3  \\ \hline 
    27NodeBrick     & Vertical ($z$)     & 0.34\% & 0.40\% & 0.85\% & 0.60\%  \\ \hline
    27NodeBrick     & Transverse ($y$)   & 0.34\% & 0.54\% & 3.67\% & 0.46\%  \\ \hline
  \end{tabular}
  % \caption{}
\end{table}

\begin{table}[H]
  \centering
    \caption{Errors for irregular shaped 27NodeBrick compared to normal shape}
  \label{talbe Errors for irregular shaped 27NodeBrick compared to normal shape}
  \begin{tabular}{|c|c|c|c|c|c|}
    \hline 
    \multicolumn{6}{|c|}{Errors for irregular shaped element, compared to normal shape}   \\ \hline
    Element Type   & Force direction & Normal shape & Shape 1 & Shape 2 & Shape 3  \\ \hline 
    27NodeBrick     & Vertical ($z$)    & 0.00\% & 0.74\% & 0.52\% & 0.94\%       \\ \hline
    27NodeBrick     & Transverse ($y$)  & 0.00\% & 0.87\% & 3.34\% & 0.79\%       \\ \hline
  \end{tabular}
  % \caption{}
\end{table}

The ESSI model fei files for the table above are \href{https://github.com/yuan-energy/ESSI_Verification/blob/master/27NodeBrick/cantilever_irregular_element/cantilever_irregular_element.tar.gz?raw=true}{here}



% The errors were listed below, compared to the theoretical solution.
% \begin{table}[H]
%   \centering
%   \begin{tabular}{|c|c|c|c|c|}
%     \hline 
%     \multicolumn{5}{|c|}{Test for brick shape displacement errors}   \\ \hline
%     Element Type  & Normal shape & Shape 1 & Shape 2 & Shape 3  \\ \hline 
%     27NodeBrick     &     &    &   & \\ \hline
%     27NodeBrick    &     &    &   &  \\ \hline
%   \end{tabular}
%   % \caption{}
% \end{table}

\newpage
Then, the beam was divided into small elements. 

Problem description: Length=12m, Width=2m, Height=2m, Force=400N/m, E=1E8Pa, $\nu=0.0$. Use the shear deformation coefficient $\kappa=1.2$. The force direction was shown in Figure (\ref{fig Problem description for cantilever beams under uniform pressure 27}). 

\begin{figure}[H]
  \centering
  \includegraphics[width=7cm]{../Figure-files/cantilever_12_uniform_load.pdf}
  \caption{Problem description for cantilever beams under uniform pressure  }
  \label{fig Problem description for cantilever beams under uniform pressure 27}
\end{figure}


Theoretical displacement (bending and shear deformation):
\begin{equation}
  \begin{aligned}
  d &=\frac{qL^4}{8EI} + \frac{q \frac{L^2}{2}}{GA_v} \\ 
    &=\frac{qL^4}{8E\frac{bh^3}{12} }+\frac{q \frac{L^2}{2}}{\frac{E}{2(1+\nu)}\frac{bh}{\kappa}} \\
    &= \frac{400 N/m \times 12^4 m^4}{8\times 10^8 N/m^2 \times \frac{2^4}{12} m^4} 
       + \frac{400 N/m \times \frac{12^2}{2} m^2} {\frac{10^8}{2} N/m^2 \times 2m\times 2m\times \frac{5}{6}} \\ 
    &=7.776\times 10^{-3} m  +1.728\times 10^{-4} m \\
    &=7.9488\times 10^{-3} m
   \end{aligned}
\end{equation}

The ESSI displacement results were listed in Table (\ref{table Results for 27NodeBrick cantilever beams of irregular shapes with more elements}).
\begin{table}[H]
  \centering
  \caption{Results for 27NodeBrick cantilever beams of irregular shapes with more elements}
  \label{table Results for 27NodeBrick cantilever beams of irregular shapes with more elements}
\begin{tabular}{|c|c|c|c|c|c|}
\hline
\multirow{2}{*}{Element Type} & \multirow{2}{*}{Shape}  & \multirow{2}{*}{Force direction}  & \multicolumn{3}{|c|}{Number of division} \\  \cline{4-6}
                        &        &                  &  1 &  2 &  4  \\ \hline
27NodeBrick              & shape1 & Vertical ($z$)   & 7.913E-03 $m$ & 7.946E-03 $m$ & 7.948E-03  $m$ \\ \hline
27NodeBrick              & shape1 & Transverse ($y$) & 7.903E-03 $m$ & 7.946E-03 $m$ & 7.948E-03  $m$ \\ \hline
27NodeBrick              & shape2 & Vertical ($z$)   & 7.741E-03 $m$ & 7.930E-03 $m$ & 7.947E-03  $m$ \\ \hline
27NodeBrick              & shape2 & Transverse ($y$) & 7.371E-03 $m$ & 7.894E-03 $m$ & 7.944E-03  $m$ \\ \hline
27NodeBrick              & shape3 & Vertical ($z$)   & 1.982E-03 $m$ & 7.946E-03 $m$ & 7.948E-03  $m$ \\ \hline
27NodeBrick              & shape3 & Transverse ($y$) & 1.979E-03 $m$ & 7.947E-03 $m$ & 7.948E-03  $m$ \\ \hline
 \multicolumn{3}{|c|}{Theoretical solution}      & 7.9488E-03 $m$  & 7.9488E-03 $m$  & 7.9488E-03  $m$ \\
\hline
\end{tabular}
\end{table}

The error were listed in Table (\ref{table Errors for 27NodeBrick cantilever beams of irregular shapes with more elements}). 

\begin{table}[H]
  \centering
  \caption{Errors for 27NodeBrick cantilever beams of irregular shapes with more elements}
  \label{table Errors for 27NodeBrick cantilever beams of irregular shapes with more elements}
\begin{tabular}{|c|c|c|c|c|c|}
\hline
\multirow{2}{*}{Element Type} & \multirow{2}{*}{Shape}  & \multirow{2}{*}{Force direction}  & \multicolumn{3}{|c|}{Number of division} \\  \cline{4-6}
                        &        &                  &  1 &  2 &  4  \\ \hline
27NodeBrick   & shape1      & Vertical ($z$)   & 0.45\%  & 0.04\% & 0.01\%    \\ \hline
27NodeBrick   & shape1      & Transverse ($y$) & 0.32\%  & 0.03\% & 0.01\%    \\ \hline
27NodeBrick   & shape2      & Vertical ($z$)   & 2.61\%  & 0.23\% & 0.03\%    \\ \hline
27NodeBrick   & shape2      & Transverse ($y$) & 7.27\%  & 0.69\% & 0.06\%    \\ \hline
27NodeBrick   & shape3      & Vertical ($z$)   & 75.06\% & 0.04\% & 0.01\%    \\ \hline
27NodeBrick   & shape3      & Transverse ($y$) & 75.11\% & 0.03\% & 0.01\%    \\
\hline
\end{tabular}
\end{table}

% \begin{figure}[H]
%   \centering
%   \includegraphics[width=9cm]{../Figure-files/error27brick_beam_irregular_shape1.jpeg}
%   % \caption{}
%   % \label{}
% \end{figure}

The errors were shown in Figure (\ref{fig shape 1 27NodeBrick cantilever beam for irregular more elements}), (\ref{fig shape 2 27NodeBrick cantilever beam for irregular more elements}) and (\ref{fig shape 3 27NodeBrick cantilever beam for irregular more elements}). 

\begin{figure}[H]
  % \centering
  \begin{subfigure}{0.5\textwidth}
    \centering
    \includegraphics[width=6cm]{../Figure-files/error27brick_beam_irregular_shape1.jpeg}
    \caption{Error scale 0\% - 0.4\%}
  \end{subfigure}
  \begin{subfigure}{0.5\textwidth}
    \centering
    \includegraphics[width=6cm]{../Figure-files/error27brick_beam_irregular_shape1100.jpeg}
    \caption{Error scale 0\% - 100\%}
  \end{subfigure}
  \captionsetup{justification=centering,margin=3cm}
  \caption{27NodeBrick cantilever beam for irregular \textbf{\emph{Shape 1}}\\
      Displacement error   versus   Number of division}
  \label{fig shape 1 27NodeBrick cantilever beam for irregular more elements}
\end{figure}


% \begin{figure}[H]
%   \centering
%   \includegraphics[width=9cm]{../Figure-files/error27brick_beam_irregular_shape2.jpeg}
%   % \caption{}
%   % \label{}
% \end{figure}



\begin{figure}[H]
  % \centering
  \begin{subfigure}{0.5\textwidth}
    \centering
    \includegraphics[width=6cm]{../Figure-files/error27brick_beam_irregular_shape2.jpeg}
    \caption{Error scale 0\% - 8\%}
  \end{subfigure}
  \begin{subfigure}{0.5\textwidth}
    \centering
    \includegraphics[width=6cm]{../Figure-files/error27brick_beam_irregular_shape2100.jpeg}
    \caption{Error scale 0\% - 100\%}
  \end{subfigure}
  \captionsetup{justification=centering,margin=3cm}
  \caption{27NodeBrick cantilever beam for irregular \textbf{\emph{Shape 2}}\\
      Displacement error   versus   Number of division}
  \label{fig shape 2 27NodeBrick cantilever beam for irregular more elements}
\end{figure}


% \begin{figure}[H]
%   \centering
%   \includegraphics[width=9cm]{../Figure-files/error27brick_beam_irregular_shape3.jpeg}
%   % \caption{}
%   % \label{}
% \end{figure}


\begin{figure}[H]
  % \centering
  \begin{subfigure}{0.5\textwidth}
    \centering
    \includegraphics[width=6cm]{../Figure-files/error27brick_beam_irregular_shape3.jpeg}
    \caption{Error scale 0\% - 80\%}
  \end{subfigure}
  \begin{subfigure}{0.5\textwidth}
    \centering
    \includegraphics[width=6cm]{../Figure-files/error27brick_beam_irregular_shape3100.jpeg}
    \caption{Error scale 0\% - 100\%}
  \end{subfigure}
  \captionsetup{justification=centering,margin=3cm}
  \caption{27NodeBrick cantilever beam for irregular \textbf{\emph{Shape 3}}\\
      Displacement error   versus   Number of division}
  \label{fig shape 3 27NodeBrick cantilever beam for irregular more elements}
\end{figure}



The ESSI model fei files for the table above are \href{https://github.com/yuan-energy/ESSI_Verification/blob/master/27NodeBrick/cantilever_irregular_element_cut/cantilever_irregular_element_cut.tar.gz?raw=true}{here}












% \newpage
% \subsection{Verification of 27NodeBrick edge clamped beams }

% Problem description: Length=6m, Width=1m, Height=1m, Force=100N, E=1E8Pa, $\nu=0.0$. Use the shear deformation coefficient $\kappa=1.2$.  The force direction was shown in Figure (\ref{fig Problem description for clamped beams 27}). 

% \begin{figure}[H]
%   \centering
%   \includegraphics[width=7cm]{../Figure-files/clamped_beam.pdf}
%   \caption{Problem description for clamped beams}
%   \label{fig Problem description for clamped beams 27}
% \end{figure}

% % \subsection{Verification of edge clamped beams - one line elements}



% The elment types and element sizes were same to the cantilever model. Only the boundary conditions and external force locations were changed. 

% The 27NodeBrick elements were shown in Figure (\ref{fig 27NodeBrick elements for clamped beams}). 
% \begin{figure}[H]
%   \centering
%   \includegraphics[width=9cm]{../Figure-files/beam_27brick.pdf}
%   \caption{27NodeBrick elements for clamped beams}
%   \label{fig 27NodeBrick elements for clamped beams}
% \end{figure}

% Theoretical displacement (bending and shear deformation):
% \begin{equation}
%   \begin{aligned}
%   d &=\frac{FL^3}{192EI}+\frac{\frac{F}{2}\frac{L}{2}}{GA_v}  \\
%     &=\frac{FL^3}{192E\frac{bh^3}{12}}+\frac{\frac{F}{2}\frac{L}{2}}{\frac{E}{2(1+\nu)}\frac{bh}{\kappa}} \\
%    &= \frac{100 N\times 6 m^3}{192 \times 10^8 N/m^2 \times \frac{1}{12} m^4}+ 
%     \frac{\frac{100}{2} N \times \frac{6}{2} m}{\frac{10}{2}\times 10^7 N/m^2\times 1 m^2\times \frac{5}{6}}   \\
%   &=1.35\times 10^{-5} m + 0.36\times 10^{-5} m  \\
%   &=1.71\times 10^{-5} \ m 
%     \end{aligned}
% \end{equation}

% The theoretical solution for $L=6\ m$ was calculated above, while the solutions for other length were calculated by the similar process. 

% In the figures above, only the model with geometry $6m\times 1m \times 1m$ was drawed. In the ESSI models, the geometry $10m\times 1m \times 1m$ and the geometry $20m\times 1m \times 1m$ were also calculated. In three different geometry models, all the element sizes were $1m\times 1m \times 1m$. Therefore, the number of elements used in each model were $6,\ 10\ and\ 20$ respectively.


% The results were listed in Table (\ref{table Results for 27NodeBrick clamped beams of different geometry}).

% \begin{table}[H]
%   \centering
%     \caption{Results for 27NodeBrick clamped beams of different geometry}
%     \label{table Results for 27NodeBrick clamped beams of different geometry}
%     \begin{tabular}{|c|c|c|c|c|c|}
%     \hline
%     Geometry & 27NodeBrick & Theory(bending) & Theory(shear) & Theory(all) & Error   \\  \hline
%     1:6      & 1.636E-05 $m$ & 1.35E-05  $m$     & 2.50E-06  $m$   & 1.60E-05 $m$        & 0.83\% \\ \hline
%     1:10     & 6.727E-05 $m$ & 6.25E-05  $m$     & 5.00E-06  $m$   & 6.75E-05 $m$        & 0.34\% \\ \hline
%     1:20     & 5.095E-04 $m$ & 5.00E-04  $m$     & 1.00E-05  $m$   & 5.10E-04 $m$        & 0.09\% \\
%     \hline
%     \end{tabular}
% \end{table}

% The ESSI model fei files for the table above are \href{https://github.com/yuan-energy/ESSI_Verification/blob/master/27NodeBrick/clamped_beam_different_geometry/clamped_beam_different_geometry.tar.gz?raw=true}{here}





% 
% old table : may be useful.....
% \begin{table}[H]
%   \centering
%   \begin{tabular}{|c|c|c|c|c|}
%     \hline 
%     \multicolumn{5}{|c|}{The edge clamped beam displacement errors}   \\ \hline
%     Element Type  & Force direction  &1:6 & 1:10 & 1:20  \\ \hline 
%     4NodeANDES & in-plane       &    &   & \\ \hline
%     4NodeANDES & out-of-plane        &    &   &  \\ \hline
%     \multicolumn{2}{|c|}{27NodeBrick} &    &   &  \\ \hline
%     \multicolumn{2}{|c|}{27NodeBrick} &   &   &  \\ \hline
%   \end{tabular}
%   % \caption{}
% \end{table}




\newpage
In this section, the beam was cut into smaller elements with element side length 0.5m and 0.25m respectively. And the element side length of the original models is 1.0m. The numerical models were shown in Figure (\ref{fig 27NodeBrick clamped beams with element side length 1.0m}), (\ref{fig 27NodeBrick clamped beams with element side length 0.5m}) and (\ref{fig 27NodeBrick clamped beams with element side length 0.25m}). 

Number of division 1:

\begin{figure}[H]
  \centering
  \includegraphics[width=9cm]{../Figure-files/beam_27brick.pdf}
  \caption{27NodeBrick clamped beams with element side length 1.0m}
  \label{fig 27NodeBrick clamped beams with element side length 1.0m}
\end{figure}

Number of division 2:

\begin{figure}[H]
  \centering
  \includegraphics[width=9cm]{../Figure-files/beam_8brick_more_2.pdf}
  \caption{27NodeBrick clamped beams with element side length 0.5m}
  \label{fig 27NodeBrick clamped beams with element side length 0.5m}
\end{figure}

Number of division 4:

\begin{figure}[H]
  \centering
  \includegraphics[width=9cm]{../Figure-files/beam_8brick_more.pdf}
  \caption{27NodeBrick clamped beams with element side length 0.25m}
  \label{fig 27NodeBrick clamped beams with element side length 0.25m}
\end{figure}


The ESSI results were listed in Table (\ref{table Results for 27NodeBrick clamped beams with more elements}). 
The theoretical solution is 1.60E-5 $m$. 

\begin{table}[H]
  \centering
  \caption{Results for 27NodeBrick clamped beams with more elements}
  \label{table Results for 27NodeBrick clamped beams with more elements}
  \begin{tabular}{|c|c|c|c|c|}
    \hline 
    \multirow{2}{*}{Element Type} 
       & \multicolumn{3}{|c|}{Element side length} \\ \cline{2-4}
       & 1 $m$ & 0.5 $m$ & 0.25 $m$ \\                              \hline
27NodeBrick & 1.64E-05 $m$  & 1.70E-05 $m$ & 1.71E-05 $m$ \\ \hline
Error       & 0.83\%   & 3.25\%   & 3.70\%     \\ \hline
  \end{tabular}
  % \caption{}
\end{table}






% \begin{figure}[H]
%   \centering
%   \includegraphics[width=02cm]{../Figure-files/error27brick_clamped_beam_diff_element.jpeg}
%   % \caption{}
%   % \label{}
% \end{figure}

The errors were plotted in Figure (\ref{fig error 27NodeBrick clamped beam for different element number}).


\begin{figure}[H]
  % \centering
  \begin{subfigure}{0.5\textwidth}
    \centering
    \includegraphics[width=6cm]{../Figure-files/error27brick_clamped_beam_diff_element.jpeg}
    \caption{Error scale 0\% - 4\%}
  \end{subfigure}
  \begin{subfigure}{0.5\textwidth}
    \centering
    \includegraphics[width=6cm]{../Figure-files/error27brick_clamped_beam_diff_element100.jpeg}
    \caption{Error scale 0\% - 100\%}
  \end{subfigure}
  \captionsetup{justification=centering,margin=3cm}
  \caption{27NodeBrick clamped beam for different element number\\
      Displacement error   versus   Number of division}
  \label{fig error 27NodeBrick clamped beam for different element number}
\end{figure}


The ESSI model fei files for the table above are \href{https://github.com/yuan-energy/ESSI_Verification/blob/master/27NodeBrick/clamped_beam_cut/clamped_beam_cut.tar.gz?raw=true}{here}

\newpage
\subsection{Verification of 27NodeBrick stress in cantilever beams}





Problem description: Length=6m, Width=1m, Height=1m, Force=100N, E=1E8Pa, $\nu=0.0$. Use the shear deformation coefficient $\kappa=1.2$. The force direction was shown in Figure (\ref{fig Problem description for cantilever beams of stress verification 27}). 

\begin{figure}[H]
  \centering
  \includegraphics[width=7cm]{../Figure-files/cantilever_6.pdf}
  \caption{Problem description for cantilever beams of stress verification}
  \label{fig Problem description for cantilever beams of stress verification 27}
\end{figure}

The theoretical solution for the stress was calculated below. 





The 27NodeBrick elements were shown in Figure (\ref{fig 27NodeBrick for cantilever beams of stress verification}).

\begin{figure}[H]
  \centering
  \includegraphics[width=9cm]{../Figure-files/beam_27brick_6div_gp.pdf}
  \caption{27NodeBrick for cantilever beams of stress verification}
  \label{fig 27NodeBrick for cantilever beams of stress verification}
\end{figure}

The bending moment at the Gassian Point is 
\begin{equation}
  M=F(L-P_y)=100 N \times (6-0.1127) m = 588.73 N\cdot m
\end{equation}

The bending modulus is 
\begin{equation}
  I= \frac{bh^3}{12}=\frac{1}{12} m^4
\end{equation}

Therefore, the theoretical stress is 
\begin{equation}
  \sigma= \frac{M\cdot z}{I}= \frac{588.73 N\cdot m \times (0.5-0.1127) m }{\frac{1}{12} m^4}= 2736 Pa
\end{equation}



To get a better result, the same geometry beam was also cut into small elements. When more elements were used, the theoretical stress was calculated again with the new coordinates. The calculation process is similar to the process above. 


The numerical models were shown in Figure (\ref{fig 27NodeBrick stress with element side length 1.0m}), (\ref{fig 27NodeBrick stress with element side length 0.5m}) and (\ref{fig 27NodeBrick stress with element side length 0.25m}). 


Number of division 1:

\begin{figure}[H]
  \centering
  \includegraphics[width=9cm]{../Figure-files/beam_8brick.pdf}
  \caption{27NodeBrick stress with element side length 1.0m}
  \label{fig 27NodeBrick stress with element side length 1.0m}
\end{figure}

Number of division 2:

\begin{figure}[H]
  \centering
  \includegraphics[width=9cm]{../Figure-files/beam_8brick_more_2.pdf}
  \caption{27NodeBrick stress with element side length 0.5m}
  \label{fig 27NodeBrick stress with element side length 0.5m}
\end{figure}

Number of division 4:

\begin{figure}[H]
  \centering
  \includegraphics[width=9cm]{../Figure-files/beam_8brick_more.pdf}
  \caption{27NodeBrick stress with element side length 0.25m}
  \label{fig 27NodeBrick stress with element side length 0.25m}
\end{figure}


All the stress results were listed in Table (\ref{table Results for 27NodeBrick stress with more elements}). 


\begin{table}[H]
  \centering
  \caption{Results for 27NodeBrick stress with more elements}
  \label{table Results for 27NodeBrick stress with more elements}
  \begin{tabular}{|c|c|c|c|c|}
    \hline 
    \multirow{2}{*}{Element Type} 
       & \multicolumn{3}{|c|}{Element side length} \\ \cline{2-4}
       & 1 $m$ & 0.5 $m$ & 0.25 $m$ \\                              \hline
27NodeBrick & 2719.81 $Pa$ & 3198.19 $Pa$ & 3464.76 $Pa$ \\ \hline
Theoretical & 2736.17 $Pa$ & 3164.27 $Pa$ & 3381.18 $Pa$ \\ \hline
Error       & 0.60\%  & 1.07\%  & 2.47\%        \\ \hline
  \end{tabular}
\end{table}

% \begin{figure}[H]
%   \centering
%   \includegraphics[width=02cm]{../Figure-files/error27brick_beam_stress.jpeg}
%   % \caption{}
%   % \label{}
% \end{figure}


\begin{figure}[H]
  % \centering
  \begin{subfigure}{0.5\textwidth}
    \centering
    \includegraphics[width=6cm]{../Figure-files/error27brick_beam_stress.jpeg}
    \caption{Error scale 0\% - 2.5\%}
  \end{subfigure}
  \begin{subfigure}{0.5\textwidth}
    \centering
    \includegraphics[width=6cm]{../Figure-files/error27brick_beam_stress100.jpeg}
    \caption{Error scale 0\% - 100\%}
  \end{subfigure}
  \captionsetup{justification=centering,margin=3cm}
  \caption{27NodeBrick cantilever beams for stress verification\\
      Stress error   versus   Number of division}
  % \caption{}
  % \label{}
\end{figure}



The ESSI model fei files for the table above are \href{https://github.com/yuan-energy/ESSI_Verification/blob/master/27NodeBrick/cantilever_stress/cantilever_stress.tar.gz?raw=true}{here}




\newpage
\subsection{Verification of 27NodeBrick square plate with four edges clamped}

Problem description: Length=20m, Width=20m, Height=1m, Force=100N, E=1E8Pa, $\nu=0.3$. 

The four edges are clamped. 

The load is the uniform normal pressure on the whole plate. 


The plate flexural rigidity is 
\begin{equation}
  D=\frac{Eh^3}{12(1-\nu^2)}=\frac{10^8 N/m^2 \times 1^3 m^3 }{12 \times (1-0.3^2) }= 9.1575 \times 10^6 \ N\cdot m
\end{equation}
The theoretical solution is 
\begin{equation}
  d=\alpha_c \frac{q a^4}{D}=0.00406\times \frac{100 N/m^2 \times 20^4 m^4}{9.1575 \times 10^6 \ N\cdot m}=2.2015\times 10^{-3} m
\end{equation}

where $\alpha_c$ is a coefficient, which depends on the ratio of plate length to width. In this problem, the coefficient\footnote{Stephen Timoshenko, Theory of plates and shells (2nd edition). MrGRAW-Hill Inc, page120, 1959.} $\alpha_c$ is 0.00406.

The 27NodeBrick were shown in Figure (\ref{fig 27NodeBrick edges clamped square plate with element side length 10m }) - (\ref{fig 27NodeBrick edges clamped square plate with element side length 0.25m }). 


\begin{figure}[H]
  \centering
  \includegraphics[width=11cm]{../Figure-files/square_plate1.pdf}
  \caption{27NodeBrick edge clamped square plate with element side length 10m }
  \label{fig 27NodeBrick edges clamped square plate with element side length 10m }
\end{figure}

\newpage

\begin{figure}[H]
  \centering
  \includegraphics[width=11cm]{../Figure-files/square_plate2.pdf}
  \caption{27NodeBrick edge clamped square plate with element side length 5m }
  \label{fig 27NodeBrick edges clamped square plate with element side length 5m }
\end{figure}


\begin{figure}[H]
  \centering
  \includegraphics[width=11cm]{../Figure-files/square_plate3.pdf}
  \caption{27NodeBrick edge clamped square plate with element side length 2m }
  \label{fig 27NodeBrick edges clamped square plate with element side length 2m }
\end{figure}

\newpage

\begin{figure}[H]
  \centering
  \includegraphics[width=11cm]{../Figure-files/square_plate4.pdf}
  \caption{27NodeBrick edge clamped square plate with element side length 1m }
  \label{fig 27NodeBrick edges clamped square plate with element side length 1m }
\end{figure}


\begin{figure}[H]
  \centering
  \includegraphics[width=11cm]{../Figure-files/square_plate5.pdf}
  \caption{27NodeBrick edge clamped square plate with element side length 0.5m }
  \label{fig 27NodeBrick edges clamped square plate with element side length 0.5m }
\end{figure}

\newpage

\begin{figure}[H]
  \centering
  \includegraphics[width=11cm]{../Figure-files/square_plate6.pdf}
  \caption{27NodeBrick edge clamped square plate with element side length 0.25m }
  \label{fig 27NodeBrick edges clamped square plate with element side length 0.25m }
\end{figure}



The results were listed in Table (\ref{table Results for 27NodeBrick square plate with four edges clamped}).

\begin{table}[H]
  \centering
  \caption{Results for 27NodeBrick square plate with four edges clamped}
  \label{table Results for 27NodeBrick square plate with four edges clamped}
\begin{tabular}{|c|c|c|c|c|}
\hline
Element type     & 27NodeBrick     & 27NodeBrick     & 27NodeBrick     &  \multirow{3}{*}{\tabincell{c}{Theoretical \\ displacement}} \\ \cline{1-4}
Number of layers & 1layer         & 2layers         & 4layers         &          \\ \cline{1-4}
Element side length & Height:1.00$m$ & Height:0.50$m$ & Height:0.25$m$ &          \\ \hline
10$m$            & 4.82E-004 $m$ & 4.82E-004 $m$ & 4.82E-004 $m$  & 2.20E-03 $m$ \\ \hline
5$m$             & 1.97E-003 $m$ & 1.98E-003 $m$ & 1.98E-003 $m$  & 2.20E-03 $m$ \\ \hline
2$m$             & 2.25E-003 $m$ & 2.26E-003 $m$ & 2.26E-003 $m$  & 2.20E-03 $m$ \\ \hline
1$m$             & 2.28E-003 $m$ & 2.29E-003 $m$ & 2.29E-003 $m$  & 2.20E-03 $m$ \\ \hline
0.5$m$           & 2.29E-003 $m$ & 2.30E-003 $m$ & 2.30E-003 $m$  & 2.20E-03 $m$ \\ \hline
0.25$m$          & 2.29E-003 $m$ & 2.30E-003 $m$ &  -\footnotemark          & 2.20E-03 $m$ \\
\hline
\end{tabular}
\end{table}

\footnotetext{This model run out of memory on machine cml01 (memory: 23.5GB). This model has 233,289 nodes with 3 dofs, which may require 40GB memory.}

The errors were listed in Table (\ref{table Errors for 27NodeBrick square plate with four edges clamped}).

\begin{table}[H]
  \centering
  \caption{Errors for 27NodeBrick square plate with four edges clamped}
  \label{table Errors for 27NodeBrick square plate with four edges clamped}
\begin{tabular}{|c|c|c|c|c|}
\hline
Element type     & 27NodeBrick     & 27NodeBrick     & 27NodeBrick      \\ \hline
Number of layers & 1layer         & 2layers         & 4layers          \\ \hline
Element side length & Height:1.00$m$ & Height:0.50$m$ & Height:0.25$m$  \\ \hline
10$m$            & 78.11\% & 78.10\% & 78.10\%       \\ \hline
5$m$             & 10.67\% & 10.19\% & 10.16\%       \\ \hline
2$m$             & 2.23\%  & 2.79\%  & 2.83\%        \\ \hline
1$m$             & 3.56\%  & 4.16\%  & 4.22\%        \\ \hline
0.5$m$           & 3.96\%  & 4.58\%  & 4.65\%        \\ \hline
0.25$m$          & 4.08\%  & 4.70\%  &    -         \\
\hline
\end{tabular}
\end{table}

The errors were plotted in Figure (\ref{fig 27NodeBrick square plate with edge clamped}).

\begin{figure}[H]
  \centering
  \includegraphics[width=7cm]{../Figure-files/error27brick_square_plate_clamped100.jpeg}
  \captionsetup{justification=centering,margin=3cm}
  \caption{27NodeBrick square plate with edge clamped\\
      Displacement error   versus   Number of side division}
  \label{fig 27NodeBrick square plate with edge clamped}
\end{figure}

The ESSI model fei files for the table above are \href{https://github.com/yuan-energy/ESSI_Verification/blob/master/27NodeBrick/square_plate_clamped/square_plate_clamped.tar.gz?raw=true}{here}



\newpage
\subsection{Verification of 27NodeBrick square plate with four edges simply supported}

Problem description: Length=20m, Width=20m, Height=1m, Force=100N, E=1E8Pa, $\nu=0.3$. 

The four edges are simply supported. 

The load is the uniform normal pressure on the whole plate. 

The plate flexural rigidity is 
\begin{equation}
  D=\frac{Eh^3}{12(1-\nu^2)}=\frac{10^8 N/m^2 \times 1^3 m^3 }{12 \times (1-0.3^2) }= 9.1575 \times 10^6 \ N\cdot m
\end{equation}
The theoretical solution is 
\begin{equation}
  d=\alpha_s \frac{q a^4}{D}=0.00126\times \frac{100 N/m^2 \times 20^4 m^4}{9.1575 \times 10^6 \ N\cdot m}=7.0936\times 10^{-3} m
\end{equation}

where $\alpha_s$ is a coefficient, which depends on the ratio of plate length to width. In this problem, the coefficient\footnote{Stephen Timoshenko, Theory of plates and shells (2nd edition). MrGRAW-Hill Inc, page202, 1959.} $\alpha_s$ is 0.00126.

The 27NodeBrick were shown in Figure (\ref{fig 27NodeBrick edges simply supported square plate with element side length 10m }) - (\ref{fig 27NodeBrick edges simply supported square plate with element side length 0.25m }). 



\begin{figure}[H]
  \centering
  \includegraphics[width=11cm]{../Figure-files/square_plate1.pdf}
  \caption{27NodeBrick edge simply supported square plate with element side length 10m }
  \label{fig 27NodeBrick edges simply supported square plate with element side length 10m }
\end{figure}

\newpage

\begin{figure}[H]
  \centering
  \includegraphics[width=11cm]{../Figure-files/square_plate2.pdf}
  \caption{27NodeBrick edge simply supported square plate with element side length 5m }
  \label{fig 27NodeBrick edges simply supported square plate with element side length 5m }
\end{figure}


\begin{figure}[H]
  \centering
  \includegraphics[width=11cm]{../Figure-files/square_plate3.pdf}
  \caption{27NodeBrick edge simply supported square plate with element side length 2m }
  \label{fig 27NodeBrick edges simply supported square plate with element side length 2m }
\end{figure}

\newpage

\begin{figure}[H]
  \centering
  \includegraphics[width=11cm]{../Figure-files/square_plate4.pdf}
  \caption{27NodeBrick edge simply supported square plate with element side length 1m }
  \label{fig 27NodeBrick edges simply supported square plate with element side length 1m }
\end{figure}


\begin{figure}[H]
  \centering
  \includegraphics[width=11cm]{../Figure-files/square_plate5.pdf}
  \caption{27NodeBrick edge simply supported square plate with element side length 0.5m }
  \label{fig 27NodeBrick edges simply supported square plate with element side length 0.5m }
\end{figure}

\newpage

\begin{figure}[H]
  \centering
  \includegraphics[width=11cm]{../Figure-files/square_plate6.pdf}
  \caption{27NodeBrick edge simply supported square plate with element side length 0.25m }
  \label{fig 27NodeBrick edges simply supported square plate with element side length 0.25m }
\end{figure}


The results were listed in Table (\ref{table Results for 27NodeBrick square plate with four edges simply supported}).

\begin{table}[H]
  \centering
  \caption{Results for 27NodeBrick square plate with four edges simply supported}
  \label{table Results for 27NodeBrick square plate with four edges simply supported}
\begin{tabular}{|c|c|c|c|c|}
\hline
Element type         & 27NodeBrick     & 27NodeBrick     &  \multirow{3}{*}{\tabincell{c}{Theoretical \\ displacement}}    \\ \cline{1-3}
Number of layers          & 2layers         & 4layers         &          \\ \cline{1-3}
Element side length  & Height:0.50$m$ & Height:0.25$m$ &          \\ \hline
10$m$                & 6.54E-003 $m$ & 6.54E-003 $m$ & 7.09E-03 $m$ \\ \hline
5$m$                 & 7.24E-003 $m$ & 7.24E-003 $m$ & 7.09E-03 $m$ \\ \hline
2$m$                 & 7.44E-003 $m$ & 7.44E-003 $m$ & 7.09E-03 $m$ \\ \hline
1$m$                 & 7.49E-003 $m$ & 7.49E-003 $m$ & 7.09E-03 $m$ \\ \hline
0.5$m$               & 7.50E-003 $m$ & 7.50E-003 $m$ & 7.09E-03 $m$ \\ \hline
0.25$m$              & 7.51E-003 $m$ &    -\footnotemark & 7.09E-03 $m$ \\
\hline
\end{tabular}
\end{table}

\footnotetext{This model run out of memory on machine cml01 (memory: 23.5GB). This model has 233,289 nodes with 3 dofs, which may require 40GB memory.}

The errors were listed in Table (\ref{table Errors for 27NodeBrick square plate with four edges simply supported}).

\begin{table}[H]
  \centering
  \caption{Errors for 27NodeBrick square plate with four edges simply supported}
  \label{table Errors for 27NodeBrick square plate with four edges simply supported}
\begin{tabular}{|c|c|c|c|c|}
\hline
Element type        & 27NodeBrick     & 27NodeBrick      \\ \hline
Number of layers         & 2layers         & 4layers          \\ \hline
Element side length  & Height:0.50$m$ & Height:0.25$m$  \\ \hline
10$m$                & 7.87\% & 7.85\%        \\ \hline
5$m$                 & 2.07\% & 2.10\%        \\ \hline
2$m$                 & 4.85\% & 4.89\%        \\ \hline
1$m$                 & 5.54\% & 5.58\%        \\ \hline
0.5$m$               & 5.74\% & 5.79\%        \\ \hline
0.25$m$              & 5.80\% &    -           \\
\hline
\end{tabular}
\end{table}

% \begin{figure}[H]
%   \centering
%   \includegraphics[width=9cm]{../Figure-files/error27brick_square_plate_simply_supported.jpeg}
%   \captionsetup{justification=centering,margin=3cm}
%   \caption{27NodeBrick square plate with edge simply supported\\
%       Displacement error   versus   Number of side division}
%   % \label{}
% \end{figure}

The errors were plotted in Figure (\ref{fig 27NodeBrick square plate with four edge simply supported}).

\begin{figure}[H]
  % \centering
  \begin{subfigure}{0.5\textwidth}
    \centering
    \includegraphics[width=6cm]{../Figure-files/error27brick_square_plate_simply_supported.jpeg}
    \caption{Error scale 0\% - 8\%}
  \end{subfigure}
  \begin{subfigure}{0.5\textwidth}
    \centering
    \includegraphics[width=6cm]{../Figure-files/error27brick_square_plate_simply_supported100.jpeg}
    \caption{Error scale 0\% - 100\%}
  \end{subfigure}
  \captionsetup{justification=centering,margin=3cm}
  \caption{27NodeBrick square plate with edge simply supported\\
      Displacement error   versus   Number of side division}
  \label{fig 27NodeBrick square plate with four edge simply supported}
\end{figure}



% \begin{figure}[H]
%   \centering
%   \includegraphics[width=9cm]{../Figure-files/err_disp27brick_square_plate_simply_support2.pdf}
%   % \caption{}
%   % \label{}
% \end{figure}


The ESSI model fei files for the table above are \href{https://github.com/yuan-energy/ESSI_Verification/blob/master/27NodeBrick/square_plate_simply_support/square_plate_simply_support.tar.gz?raw=true}{here}

























\newpage
\subsection{Verification of 27NodeBrick circular plate with all edges clamped}

Problem description: Diameter=20m, Height=1m, Force=100N, E=1E8Pa, $\nu=0.3$. 

The four edges are clamped. 

The load is the uniform normal pressure on the whole plate. 


The plate flexural rigidity is 

\begin{equation}
  D=\frac{Eh^3}{12(1-\nu^2)}=\frac{10^8 N/m^2 \times 1^3 m^3 }{12 \times (1-0.3^2) }= 9.1575 \times 10^6 \ N\cdot m
\end{equation}

The theoretical solution\footnote{Stephen Timoshenko, Theory of plates and shells (2nd edition). MrGRAW-Hill Inc, page55, 1959.} is 

\begin{equation}
  d= \frac{q a^4}{64D}=\frac{100 N/m^2 \times 10^4 m^4}{64 \times 9.1575 \times 10^6 \ N\cdot m}=1.7106\times 10^{-3} m
\end{equation}

The 27NodeBrick were shown in Figure (\ref{fig 27NodeBrick edges clamped circular plate with element side length 10m }) - (\ref{fig 27NodeBrick edges clamped circular plate with element side length 0.25m }). 




\begin{figure}[H]
  \centering
  \includegraphics[width=9cm]{../Figure-files/circular_plate1.pdf}
  \caption{27NodeBrick edge clamped circular plate with element side length 10m }
  \label{fig 27NodeBrick edges clamped circular plate with element side length 10m }
\end{figure}

\newpage

\begin{figure}[H]
  \centering
  \includegraphics[width=9cm]{../Figure-files/circular_plate2.pdf}
  \caption{27NodeBrick edge clamped circular plate with element side length 5m }
  \label{fig 27NodeBrick edges clamped circular plate with element side length 5m }
\end{figure}


\begin{figure}[H]
  \centering
  \includegraphics[width=9cm]{../Figure-files/circular_plate3.pdf}
  \caption{27NodeBrick edge clamped circular plate with element side length 2m }
  \label{fig 27NodeBrick edges clamped circular plate with element side length 2m }
\end{figure}

\newpage

\begin{figure}[H]
  \centering
  \includegraphics[width=9cm]{../Figure-files/circular_plate4.pdf}
  \caption{27NodeBrick edge clamped circular plate with element side length 1m }
  \label{fig 27NodeBrick edges clamped circular plate with element side length 1m }
\end{figure}


\begin{figure}[H]
  \centering
  \includegraphics[width=9cm]{../Figure-files/circular_plate5.pdf}
  \caption{27NodeBrick edge clamped circular plate with element side length 0.5m }
  \label{fig 27NodeBrick edges clamped circular plate with element side length 0.5m }
\end{figure}

\newpage

\begin{figure}[H]
  \centering
  \includegraphics[width=9cm]{../Figure-files/circular_plate6.pdf}
  \caption{27NodeBrick edge clamped circular plate with element side length 0.25m }
  \label{fig 27NodeBrick edges clamped circular plate with element side length 0.25m }
\end{figure}

The results were listed in Table (\ref{table Results for 27NodeBrick circular plate with four edges clamped}).

\begin{table}[H]
  \centering
    \caption{Results for 27NodeBrick circular plate with four edges clamped}
  \label{table Results for 27NodeBrick circular plate with four edges clamped}
\begin{tabular}{|c|c|c|c|c|}
\hline
Element type     & 27NodeBrick     & 27NodeBrick     & 27NodeBrick     &  \multirow{3}{*}{\tabincell{c}{Theoretical \\ displacement}} \\ \cline{1-4}
Number of layers & 1layer         & 2layers         & 4layers         &          \\ \cline{1-4}
Number of diameter divisions & Height:1.00$m$ & Height:0.50$m$ & Height:0.25$m$ &          \\ \hline
4           & 2.777E-03 $m$ & 2.788E-03 $m$ & 2.789E-03 $m$ & 1.706E-03 $m$ \\ \hline
12          & 2.772E-03 $m$ & 2.786E-03 $m$ & 2.787E-03 $m$ & 1.706E-03 $m$ \\ \hline
20          & 2.545E-03 $m$ & 2.556E-03 $m$ & 2.558E-03 $m$ & 1.706E-03 $m$ \\ \hline
40          & 1.758E-03 $m$ & 1.768E-03 $m$ & 1.769E-03 $m$ & 1.706E-03 $m$ \\ \hline
60          & 1.762E-03 $m$ & 1.772E-03 $m$ & 1.773E-03 $m$ & 1.706E-03 $m$ \\ \hline
80          & 1.763E-03 $m$ & 1.773E-03 $m$ & 1.774E-03 $m$ & 1.706E-03 $m$ \\
\hline
\end{tabular}
\end{table}


The errors were listed in Table (\ref{table errors for 27NodeBrick circular plate with four edges clamped}).

\begin{table}[H]
  \centering
      \caption{Errors for 27NodeBrick circular plate with four edges clamped}
  \label{table errors for 27NodeBrick circular plate with four edges clamped}
\begin{tabular}{|c|c|c|c|c|}
\hline
Element type     & 27NodeBrick     & 27NodeBrick     & 27NodeBrick      \\ \hline
Number of layers & 1layer         & 2layers         & 4layers          \\ \hline
Number of diameter divisions & Height:1.00$m$ & Height:0.50$m$ & Height:0.25$m$  \\ \hline
4           & 62.75\% & 63.42\% & 63.47\%       \\ \hline
12          & 62.46\% & 63.27\% & 63.34\%       \\ \hline
20          & 49.14\% & 49.82\% & 49.91\%       \\ \hline
40          & 3.03\%  & 3.62\%  & 3.68\%        \\ \hline
60          & 3.25\%  & 3.83\%  & 3.91\%        \\ \hline
80          & 3.32\%  & 3.91\%  & 3.99\%        \\
\hline
\end{tabular}
\end{table}

The errors were shown in Figure (\ref{fig 27NodeBrick circular plate with edge clamped}).
\begin{figure}[H]
  \centering
  \includegraphics[width=7cm]{../Figure-files/error27brick_circular_plate_clamped100.jpeg}
  \captionsetup{justification=centering,margin=3cm}
  \caption{27NodeBrick circular plate with edge clamped\\
      Displacement error   versus   Number of side division}
  \label{fig 27NodeBrick circular plate with edge clamped}
\end{figure}



The ESSI model fei files for the table above are \href{https://github.com/yuan-energy/ESSI_Verification/blob/master/27NodeBrick/circular_plate_clamped/circular_plate_clamped.tar.gz?raw=true}{here}






\newpage
\subsection{Verification of 27NodeBrick circular plate with all edges simply supported}


Problem description: Diameter=20m, Height=1m, Force=100N, E=1E8Pa, $\nu=0.3$. 

The four edges are simply supported. 

The load is the uniform normal pressure on the whole plate. 


The plate flexural rigidity is 

\begin{equation}
  D=\frac{Eh^3}{12(1-\nu^2)}=\frac{10^8 N/m^2 \times 1^3 m^3 }{12 \times (1-0.3^2) }= 9.1575 \times 10^6 \ N\cdot m
\end{equation}

The theoretical solution\footnote{Stephen Timoshenko, Theory of plates and shells (2nd edition). MrGRAW-Hill Inc, page55, 1959.} is 

\begin{equation}
  d= \frac{(5+\nu)  q a^4}{64(1+\nu) D}=\frac{(5+0.3)\times 100 N/m^2 \times 10^4 m^4}{64\times(1+0.3) \times 9.1575 \times 10^6 \ N\cdot m}=6.956\times 10^{-3} m
\end{equation}



The 27NodeBrick were shown in Figure (\ref{fig 27NodeBrick edges simply supported circular plate with element side length 10m }) - (\ref{fig 27NodeBrick edges simply supported circular plate with element side length 0.25m }). 



\begin{figure}[H]
  \centering
  \includegraphics[width=11cm]{../Figure-files/circular_plate1.pdf}
  \caption{27NodeBrick edge simply supported circular plate with element side length 10m }
  \label{fig 27NodeBrick edges simply supported circular plate with element side length 10m }
\end{figure}

\newpage

\begin{figure}[H]
  \centering
  \includegraphics[width=11cm]{../Figure-files/circular_plate2.pdf}
  \caption{27NodeBrick edge simply supported circular plate with element side length 5m }
  \label{fig 27NodeBrick edges simply supported circular plate with element side length 5m }
\end{figure}


\begin{figure}[H]
  \centering
  \includegraphics[width=11cm]{../Figure-files/circular_plate3.pdf}
  \caption{27NodeBrick edge simply supported circular plate with element side length 2m }
  \label{fig 27NodeBrick edges simply supported circular plate with element side length 2m }
\end{figure}

\newpage

\begin{figure}[H]
  \centering
  \includegraphics[width=11cm]{../Figure-files/circular_plate4.pdf}
  \caption{27NodeBrick edge simply supported circular plate with element side length 1m }
  \label{fig 27NodeBrick edges simply supported circular plate with element side length 1m }
\end{figure}


\begin{figure}[H]
  \centering
  \includegraphics[width=11cm]{../Figure-files/circular_plate5.pdf}
  \caption{27NodeBrick edge simply supported circular plate with element side length 0.5m }
  \label{fig 27NodeBrick edges simply supported circular plate with element side length 0.5m }
\end{figure}

\newpage

\begin{figure}[H]
  \centering
  \includegraphics[width=11cm]{../Figure-files/circular_plate6.pdf}
  \caption{27NodeBrick edge simply supported circular plate with element side length 0.25m }
  \label{fig 27NodeBrick edges simply supported circular plate with element side length 0.25m }
\end{figure}





The results were listed in Table (\ref{table Results for 27NodeBrick cicular plate with four edges simply supported}).

\begin{table}[H]
  \centering
  \caption{Results for 27NodeBrick cicular plate with four edges simply supported}
  \label{table Results for 27NodeBrick cicular plate with four edges simply supported}
\begin{tabular}{|c|c|c|c|c|}
\hline
Element type        & 27NodeBrick     & 27NodeBrick     &  \multirow{3}{*}{\tabincell{c}{Theoretical \\ displacement}} \\ \cline{1-3}
Number of layers    & 2layers         & 4layers         &          \\ \cline{1-3}
Number of diameter divisions & Height:0.50$m$ & Height:0.25$m$ &          \\ \hline
4            & 7.259E-03 $m$ & 7.261E-03 $m$ & 6.956E-03 $m$ \\ \hline
12           & 7.083E-03 $m$ & 7.084E-03 $m$ & 6.956E-03 $m$ \\ \hline
20           & 7.064E-03 $m$ & 7.065E-03 $m$ & 6.956E-03 $m$ \\ \hline
40           & 7.018E-03 $m$ & 7.019E-03 $m$ & 6.956E-03 $m$ \\ \hline
60           & 7.029E-03 $m$ & 7.030E-03 $m$ & 6.956E-03 $m$ \\ \hline
80           & 7.032E-03 $m$ & 7.034E-03 $m$ & 6.956E-03 $m$ \\
\hline
\end{tabular}
\end{table}


The errors were listed in Table (\ref{table Errors for 27NodeBrick cicular plate with four edges simply supported}).

\begin{table}[H]
  \centering
  \caption{Errors for 27NodeBrick cicular plate with four edges simply supported}
  \label{table Errors for 27NodeBrick cicular plate with four edges simply supported}
  \begin{tabular}{|c|c|c|c|c|}
  \hline
  Element type     & 27NodeBrick     & 27NodeBrick      \\ \hline
  Number of layers      & 2layers         & 4layers          \\ \hline
  Number of diameter divisions & Height:0.50$m$ & Height:0.25$m$  \\ \hline
  4            & 4.36\% & 4.38\%      \\ \hline
  12           & 1.82\% & 1.83\%      \\ \hline
  20           & 1.56\% & 1.57\%      \\ \hline
  40           & 0.88\% & 0.90\%      \\ \hline
  60           & 1.04\% & 1.06\%      \\ \hline
  80           & 1.09\% & 1.11\%      \\
  \hline
  \end{tabular}
\end{table}

% \begin{figure}[H]
%   \centering
%   \includegraphics[width=9cm]{../Figure-files/error27brick_circular_plate_simply_supported.jpeg}
%   % \caption{}
%   % \label{}
% \end{figure}

The errors were plotted in Figure (\ref{fig 27NodeBrick circular plate with four edge simply supported}).
\begin{figure}[H]
  % \centering
  \begin{subfigure}{0.5\textwidth}
    \centering
    \includegraphics[width=6cm]{../Figure-files/error27brick_circular_plate_simply_supported.jpeg}
    \caption{Error scale 0\% - 5\%}
  \end{subfigure}
  \begin{subfigure}{0.5\textwidth}
    \centering
    \includegraphics[width=6cm]{../Figure-files/error27brick_circular_plate_simply_supported100.jpeg}
    \caption{Error scale 0\% - 100\%}
  \end{subfigure}
  \captionsetup{justification=centering,margin=3cm}
  \caption{27NodeBrick circular plate with edge simply supported\\
      Displacement error   versus   Number of side division}
  \label{fig 27NodeBrick circular plate with four edge simply supported}
\end{figure}






The ESSI model fei files for the table above are \href{https://github.com/yuan-energy/ESSI_Verification/blob/master/27NodeBrick/circular_plate_simply_support/circular_plate_simply_support.tar.gz?raw=true}{here}







% 27NodeBrick ends
% 27NodeBrick ends
% 27NodeBrick ends
% 27NodeBrick ends
% 27NodeBrick ends
% 27NodeBrick ends
% 27NodeBrick ends
% 27NodeBrick ends

% 4NodeANDES starts
% 4NodeANDES starts
% 4NodeANDES starts
% 4NodeANDES starts
% 4NodeANDES starts
% 4NodeANDES starts


















\newpage
\begin{center}
  \Large\textbf{Verification for 4NodeANDES}
\end{center}


\vskip 24pt

\section{Verification of 4NodeANDES elements}
\subsection{Verification of 4NodeANDES cantilever beams}





Problem description: Length=6m, Width=1m, Height=1m, Force=100N, E=1E8Pa, $\nu=0.0$. Use the shear deformation coefficient $\kappa=1.2$. The force direction was shown in Figure (\ref{fig Problem description for cantilever 4}). 

\begin{figure}[H]
  \centering
  \includegraphics[width=7cm]{../Figure-files/cantilever_6.pdf}
  \caption{Problem description for cantilever beams}
  \label{fig Problem description for cantilever 4}
\end{figure}


Theoretical displacement (bending and shear deformation):
\begin{equation}
  \begin{aligned}
  d &=\frac{FL^3}{3EI}+\frac{FL}{GA_v} \\
  &= \frac{FL^3}{3E\frac{bh^3}{12}}+\frac{FL}{\frac{E}{2(1+\nu)} \frac{bh}{\kappa}} \\ 
    &= \frac{100 N \times 6^3 m^3}{3\times 10^8 N/m^2 \times \frac{1}{12} m^4}+ 
    \frac{100 N\times 6 m}{\frac{10}{2} \times 10^7 N/m^2\times 1 m^2 \times \frac{5}{6}} \\ 
    &=8.64\times 10^{-4} m + 0.144 \times 10^{-4} m   \\
   & =8.784\times 10^{-4} \ m
   \end{aligned}
\end{equation}



4NodeANDES element model:

\vskip 12pt

\begin{itemize}
  \item \textbf{\emph{Force direction: perpendicular to plane (bending)}}
\end{itemize}
When the force direction is perpendicular to the plane, only the bending deformation is calculated in 4NodeANDES elements. 


The 4NodeANDES elements were shown in Figure (\ref{fig 4NodeANDES elements for cantilever beams under force perpendicular to plane}).

\begin{figure}[H]
  \centering
  \begin{subfigure}{0.5\textwidth}
    \centering
    \includegraphics[width=9cm]{../Figure-files/beam_ANDES_xy_bending_1div.pdf}
    \caption{One 4NodeANDES element}
  \end{subfigure}
  \vskip 8pt
  \begin{subfigure}{0.5\textwidth}
    \centering
    \includegraphics[width=9cm]{../Figure-files/beam_ANDES_xy_bending_2div.pdf}
    \caption{Two 4NodeANDES elements}
  \end{subfigure}
  \vskip 8pt
  \begin{subfigure}{0.5\textwidth}
    \centering
    \includegraphics[width=9cm]{../Figure-files/beam_ANDES_xy_bending_6div.pdf}
    \caption{Six 4NodeANDES elements}
  \end{subfigure}
  \captionsetup{justification=centering,margin=3cm}
  \caption{4NodeANDES elements for cantilever beams under force perpendicular to plane}
  \label{fig 4NodeANDES elements for cantilever beams under force perpendicular to plane}
\end{figure}


% One element:

% \begin{figure}[H]
%   \centering
%   \includegraphics[width=9cm]{../Figure-files/beam_ANDES_xy_bending_1div.pdf}
%   % \caption{}
%   % \label{}
% \end{figure}

% Two elements: 

% \begin{figure}[H]
%   \centering
%   \includegraphics[width=9cm]{../Figure-files/beam_ANDES_xy_bending_2div.pdf}
%   % \caption{}
%   % \label{}
% \end{figure}


% Six elements: 

% \begin{figure}[H]
%   \centering
%   \includegraphics[width=9cm]{../Figure-files/beam_ANDES_xy_bending_6div.pdf}
%   % \caption{}
%   % \label{}
% \end{figure}

\begin{itemize}
  \item \textbf{\emph{Force direction: inplane force }}
\end{itemize}
When the force direction is inplane, both the bending and shear deformation are calculated in 4NodeANDES elements. 


The 4NodeANDES elements under inplane force were shown in Figure (\ref{fig 4NodeANDES elements for cantilever beams under inplane force}).

\begin{figure}[H]
  \centering
  \begin{subfigure}{0.5\textwidth}
    \centering
    \includegraphics[width=9cm]{../Figure-files/beam_ANDES_yz_inPlane_1div.pdf}
    \caption{One 4NodeANDES element}
  \end{subfigure}
  \vskip 8pt
  \begin{subfigure}{0.5\textwidth}
    \centering
    \includegraphics[width=9cm]{../Figure-files/beam_ANDES_yz_inPlane_2div.pdf}
    \caption{Two 4NodeANDES elements}
  \end{subfigure}
  \vskip 8pt
  \begin{subfigure}{0.5\textwidth}
    \centering
    \includegraphics[width=9cm]{../Figure-files/beam_ANDES_yz_inPlane_6div.pdf}
    \caption{Six 4NodeANDES elements}
  \end{subfigure}
  \captionsetup{justification=centering,margin=3cm}
  \caption{4NodeANDES elements for cantilever beams under inplane force}
  \label{fig 4NodeANDES elements for cantilever beams under inplane force}
\end{figure}


% One element:

% \begin{figure}[H]
%   \centering
%   \includegraphics[width=9cm]{../Figure-files/beam_ANDES_yz_inPlane_1div.pdf}
%   % \caption{}
%   % \label{}
% \end{figure}

% Two elements: 
% \begin{figure}[H]
%   \centering
%   \includegraphics[width=9cm]{../Figure-files/beam_ANDES_yz_inPlane_2div.pdf}
%   % \caption{}
%   % \label{}
% \end{figure}

% Six elements: 

% \begin{figure}[H]
%   \centering
%   \includegraphics[width=9cm]{../Figure-files/beam_ANDES_yz_inPlane_6div.pdf}
%   % \caption{}
%   % \label{}
% \end{figure}



The ESSI results for the force \textbf{\emph{perpendicular to plane (bending)}} were listed in Table (\ref{table 4NodeANDES cantilever beams results for different element number}). 
The theoretical solution is 8.784E-04 $m$.
\begin{table}[H]
  \centering
    \captionsetup{justification=centering,margin=2cm}
      \caption{Results for 4NodeANDES cantilever beams under the force perpendicular to plane (bending)}
    \label{table 4NodeANDES cantilever beams results for different element number}
    \begin{tabular}{|c|c|c|c|}
      \hline
      Element number & 1        & 2        & 6         \\  \hline
      4NodeANDES     & 6.56E-04 $m$ & 8.27E-04 $m$ & 8.86E-04 $m$     \\ \hline
      Error          & 25.34\% & 5.87\% & 0.83\%    \\ 
      \hline 
    \end{tabular}
\end{table}
The ESSI results for the \textbf{\emph{inplane force}} were listed in Table (\ref{table 4NodeANDES cantilever beams results for different element number 2}). 

The theoretical solution is 8.784E-04 $m$.


\begin{table}[H]
  \centering
      \captionsetup{justification=centering,margin=2cm}
      \caption{Results for 4NodeANDES cantilever beams under the inplane force}
    \label{table 4NodeANDES cantilever beams results for different element number 2}
    \begin{tabular}{|c|c|c|c|}
      \hline
      Element number & 1        & 2        & 6         \\  \hline
      4NodeANDES     &6.70E-04 $m$& 8.27E-04 $m$& 8.64E-04 $m$     \\ \hline
      Error          &  23.77\% & 5.89\% & 1.65\%          \\ 
      \hline 
    \end{tabular}
\end{table}

% \begin{figure}[H]
%   \centering
%   \includegraphics[width=9cm]{../Figure-files/error4andes_beam_different_element_number.jpeg}
%   % \caption{}
%   % \label{}
% \end{figure}

The errors were plotted in Figure (\ref{fig error 4NodeANDES cantilever beam for different element number}).

\begin{figure}[H]
  % \centering
  \begin{subfigure}{0.5\textwidth}
    \centering
    \includegraphics[width=6cm]{../Figure-files/error4andes_beam_different_element_number.jpeg}
    \caption{Error scale 0\% - 30\%}
  \end{subfigure}
  \begin{subfigure}{0.5\textwidth}
    \centering
    \includegraphics[width=6cm]{../Figure-files/error4andes_beam_different_element_number100.jpeg}
    \caption{Error scale 0\% - 100\%}
  \end{subfigure}
  \captionsetup{justification=centering,margin=2cm}
  \caption{4NodeANDES cantilever beam for different element number\\
    Displacement error   versus   Number of elements}
  \label{fig error 4NodeANDES cantilever beam for different element number}
\end{figure}


The ESSI model fei files for the table above are \href{https://github.com/yuan-energy/ESSI_Verification/blob/master/4NodeANDES/cantilever_different_element_number/cantilever_different_element_number.tar.gz?raw=true}{here}


% \newpage
% \begin{itemize}
%   \item \textbf{\emph{Beam: different geometry}}
% \end{itemize}

% In the figures above, only the model with geometry $6m\times 1m \times 1m$ was drawed. In the ESSI models, the geometry $10m\times 1m \times 1m$ and the geometry $20m\times 1m \times 1m$ were also calculated. In three different geometry models, all the element sizes were $1m\times 1m \times 1m$. Therefore, the number of elements used in each model were $6,\ 10\ and\ 20$ respectively.

% The ESSI results for the force \textbf{\emph{perpendicular to plane (bending)}} were listed in Table (\ref{table Results for 4NodeANDES cantilever beams of different geometry}).

% \begin{table}[H]
%   \centering
%   \caption{Results for 4NodeANDES cantilever for the force perpendicular to plane (bending)}
%   \label{table Results for 4NodeANDES cantilever beams of different geometry}
%   \begin{tabular}{|c|c|c|c|c|c|}
%   \hline
%   Geometry & 4NodeANDES & Theoretical(bending) & Theoretical(shear) & Theoretical(all) & Error   \\ \hline
%   1:6      & 8.64E-04 $m$ & 8.64E-04 $m$ & 1.20E-05 $m$ & 8.76E-04 $m$ & 1.38\% \\ \hline
%   1:10     & 4.00E-03 $m$ & 4.00E-03 $m$ & 2.00E-05 $m$ & 4.02E-03 $m$ & 0.50\% \\ \hline
%   1:20     & 3.20E-02 $m$ & 3.20E-02 $m$ & 4.00E-05 $m$ & 3.20E-02 $m$ & 0.13\% \\
%   \hline
%   \end{tabular}
% \end{table}


% The ESSI results for the \textbf{\emph{inplane force}} were listed in Table (\ref{table Results for 4NodeANDES cantilever beams of different geometry 2}).



% \begin{table}[H]
%   \centering
%   \caption{Results for 4NodeANDES cantilever beams for the inplane force}
%   \label{table Results for 4NodeANDES cantilever beams of different geometry 2}
%   \begin{tabular}{|c|c|c|c|c|c|}
%   \hline
%   Geometry & 4NodeANDES & Theoretical(bending) & Theoretical(shear) & Theoretical(all) & Error   \\ \hline
%   1:6      & 8.86E-04 $m$ & 8.64E-04 $m$ & 1.20E-05 $m$ & 8.76E-04 $m$ & 1.11\% \\ \hline
%   1:10     & 4.04E-03 $m$ & 4.00E-03 $m$ & 2.00E-05 $m$ & 4.02E-03 $m$ & 0.42\% \\ \hline
%   1:20     & 3.21E-02 $m$ & 3.20E-02 $m$ & 4.00E-05 $m$ & 3.20E-02 $m$ & 0.04\% \\
%   \hline
%   \end{tabular}
% \end{table}


% The ESSI model fei files for the table above are \href{https://github.com/yuan-energy/ESSI_Verification/blob/master/4NodeANDES/cantilever_different_geometry/cantilever_different_geometry.tar.gz?raw=true}{here}








\newpage
\subsection{Verification of 4NodeANDES cantilever beam for different Poisson's ratio}




Problem description: Length=6m, Width=1m, Height=1m, Force=100N, E=1E8Pa, $\nu=0.0-0.49$. The force direction was shown in Figure (\ref{fig Problem description for cantilever beams of different Poisson's 4}). 

\begin{figure}[H]
  \centering
  \includegraphics[width=7cm]{../Figure-files/cantilever_6.pdf}
  \caption{Problem description for cantilever beams of different Poisson's ratios}
  \label{fig Problem description for cantilever beams of different Poisson's 4}
\end{figure}


The theoretical solution for $\nu=0.0$ was calculated below, while the solution for other Poisson's ratio were calculated by the similar process.

Theoretical displacement (bending and shear deformation):
\begin{equation}
  \begin{aligned}
  d &=\frac{FL^3}{3EI}+\frac{FL}{GA_v} \\
  &= \frac{FL^3}{3E\frac{bh^3}{12}}+\frac{FL}{\frac{E}{2(1+\nu)} \frac{bh}{\kappa}} \\ 
    &= \frac{100 N \times 6^3 m^3}{3\times 10^8 N/m^2 \times \frac{1}{12} m^4}+ 
    \frac{100 N\times 6 m}{\frac{10}{2} \times 10^7 N/m^2\times 1 m^2 \times \frac{5}{6}} \\ 
    &=8.64\times 10^{-4} m + 0.144 \times 10^{-4} m   \\
   & =8.784\times 10^{-4} \ m
   \end{aligned}
\end{equation}

The rotation angle at the end:
\begin{equation}
  \theta =\frac{FL^2}{2EI} 
   =\frac{100 N \times 6^2 m^2} {2\times 10^8 N/m^2 \times \frac{1}{12} m^4} 
 =2.16 \times 10^{-4} \ rad = 0.0124 \degree 
\end{equation}

The 4NodeANDES elements for cantilever beams of different Poisson's ratios were shown in Figure (\ref{fig 4NodeANDES elements for cantilever beams of different Poisson's ratios}) and (\ref{fig 4NodeANDES elements for cantilever beams of different Poisson's ratios 2}):

\begin{figure}[H]
  \centering
  \includegraphics[width=9cm]{../Figure-files/beam_ANDES_xy_bending_6div.pdf}
  \captionsetup{justification=centering,margin=3cm}
  \caption{4NodeANDES elements for different Poisson's ratios under the force perpendicular to plane (bending)}
  \label{fig 4NodeANDES elements for cantilever beams of different Poisson's ratios}
\end{figure}


\begin{figure}[H]
  \centering
  \includegraphics[width=9cm]{../Figure-files/beam_ANDES_yz_inPlane_6div.pdf}
  \captionsetup{justification=centering,margin=3cm}
  \caption{4NodeANDES elements for different Poisson's ratios under the inplane force}
  \label{fig 4NodeANDES elements for cantilever beams of different Poisson's ratios 2}
\end{figure}


The ESSI results for the force \textbf{\emph{perpendicular to plane (bending)}} were listed in Table (\ref{table Displacement results for 4NodeANDES cantilever beams of different Poissons ratios}) - (\ref{table Displacement results for 4NodeANDES cantilever beams of different Poissons ratios 4}).



\begin{table}[H]
  \centering
    \captionsetup{justification=centering,margin=3cm}
  \caption{\emph{\textbf{ Displacement error }} results for 4NodeANDES with \textcolor{red}{element side length 1 m} under the force perpendicular to plane (bending)}
  \begin{tabular}{|c|c|c|c|c|c|}
    \hline 
\tabincell{c}{Poisson's \\ ratio}    & \tabincell{c}{4NodeANDES\\displacement} & \tabincell{c}{Theory displacement\\(bending)} & \tabincell{c}{Theory displacement\\(shear)} & \tabincell{c}{Theory\\displacement(all)}   & Error \\ \hline
0.00 & 8.639E-04 $m$ & 8.640E-04 $m$ & 1.440E-05  $m$  & 8.784E-04 $m$  & 1.38\%    \\ \hline
0.05 & 8.635E-04 $m$ & 8.640E-04 $m$ & 1.512E-05  $m$  & 8.791E-04 $m$  & 1.49\%    \\ \hline
0.10 & 8.622E-04 $m$ & 8.640E-04 $m$ & 1.586E-05  $m$  & 8.799E-04 $m$  & 1.71\%    \\ \hline
0.15 & 8.599E-04 $m$ & 8.640E-04 $m$ & 1.659E-05  $m$  & 8.806E-04 $m$  & 2.04\%    \\ \hline
0.20 & 8.566E-04 $m$ & 8.640E-04 $m$ & 1.734E-05  $m$  & 8.813E-04 $m$  & 2.48\%    \\ \hline
0.25 & 8.522E-04 $m$ & 8.640E-04 $m$ & 1.808E-05  $m$  & 8.821E-04 $m$  & 3.05\%    \\ \hline
0.30 & 8.466E-04 $m$ & 8.640E-04 $m$ & 1.884E-05  $m$  & 8.828E-04 $m$  & 3.75\%    \\ \hline
0.35 & 8.398E-04 $m$ & 8.640E-04 $m$ & 1.959E-05  $m$  & 8.836E-04 $m$  & 4.59\%    \\ \hline
0.40 & 8.315E-04 $m$ & 8.640E-04 $m$ & 2.035E-05  $m$  & 8.844E-04 $m$  & 5.60\%    \\ \hline
0.45 & 8.216E-04 $m$ & 8.640E-04 $m$ & 2.111E-05  $m$  & 8.851E-04 $m$  & 6.78\%    \\ \hline
0.49 & 8.124E-04 $m$ & 8.640E-04 $m$ & 2.173E-05  $m$  & 8.857E-04 $m$  & 7.88\%    \\ \hline
  \end{tabular}
  \label{table Displacement results for 4NodeANDES cantilever beams of different Poissons ratios}
\end{table}


\begin{table}[H]
  \centering
    \captionsetup{justification=centering,margin=3cm}
  \caption{\emph{\textbf{ Displacement error }} results for 4NodeANDES with \textcolor{red}{element side length 0.5 m} under the force perpendicular to plane (bending)}
  \begin{tabular}{|c|c|c|c|c|c|}
    \hline 
\tabincell{c}{Poisson's \\ ratio}    & \tabincell{c}{4NodeANDES\\displacement} & \tabincell{c}{Theory displacement\\(bending)} & \tabincell{c}{Theory displacement\\(shear)} & \tabincell{c}{Theory\\displacement(all)}   & Error \\ \hline
0.00 & 8.724E-04 $m$ & 8.640E-04 $m$ & 1.440E-05  $m$  & 8.784E-04 $m$  & 0.68\%    \\ \hline
0.05 & 8.724E-04 $m$ & 8.640E-04 $m$ & 1.512E-05  $m$  & 8.791E-04 $m$  & 0.76\%    \\ \hline
0.10 & 8.717E-04 $m$ & 8.640E-04 $m$ & 1.586E-05  $m$  & 8.799E-04 $m$  & 0.93\%    \\ \hline
0.15 & 8.703E-04 $m$ & 8.640E-04 $m$ & 1.659E-05  $m$  & 8.806E-04 $m$  & 1.17\%    \\ \hline
0.20 & 8.682E-04 $m$ & 8.640E-04 $m$ & 1.734E-05  $m$  & 8.813E-04 $m$  & 1.49\%    \\ \hline
0.25 & 8.652E-04 $m$ & 8.640E-04 $m$ & 1.808E-05  $m$  & 8.821E-04 $m$  & 1.91\%    \\ \hline
0.30 & 8.615E-04 $m$ & 8.640E-04 $m$ & 1.884E-05  $m$  & 8.828E-04 $m$  & 2.42\%    \\ \hline
0.35 & 8.569E-04 $m$ & 8.640E-04 $m$ & 1.959E-05  $m$  & 8.836E-04 $m$  & 3.02\%    \\ \hline
0.40 & 8.514E-04 $m$ & 8.640E-04 $m$ & 2.035E-05  $m$  & 8.844E-04 $m$  & 3.73\%    \\ \hline
0.45 & 8.449E-04 $m$ & 8.640E-04 $m$ & 2.111E-05  $m$  & 8.851E-04 $m$  & 4.54\%    \\ \hline
0.49 & 8.388E-04 $m$ & 8.640E-04 $m$ & 2.173E-05  $m$  & 8.857E-04 $m$  & 5.30\%    \\ \hline
  \end{tabular}
  \label{table Displacement results for 4NodeANDES cantilever beams of different Poissons ratios 22}
\end{table}




\begin{table}[H]
  \centering
    \captionsetup{justification=centering,margin=3cm}
  \caption{\emph{\textbf{ Displacement error }} results for 4NodeANDES with \textcolor{red}{element side length 0.25 m} under the force perpendicular to plane (bending)}
  \begin{tabular}{|c|c|c|c|c|c|}
    \hline 
\tabincell{c}{Poisson's \\ ratio}    & \tabincell{c}{4NodeANDES\\displacement} & \tabincell{c}{Theory displacement\\(bending)} & \tabincell{c}{Theory displacement\\(shear)} & \tabincell{c}{Theory\\displacement(all)}   & Error \\ \hline
0.00 & 8.640E-04 $m$ & 8.640E-04 $m$ & 1.440E-05  $m$  & 8.784E-04 $m$  & 1.64\%   \\ \hline
0.05 & 8.637E-04 $m$ & 8.640E-04 $m$ & 1.512E-05  $m$  & 8.791E-04 $m$  & 1.75\%   \\ \hline
0.10 & 8.627E-04 $m$ & 8.640E-04 $m$ & 1.586E-05  $m$  & 8.799E-04 $m$  & 1.95\%   \\ \hline
0.15 & 8.611E-04 $m$ & 8.640E-04 $m$ & 1.659E-05  $m$  & 8.806E-04 $m$  & 2.21\%   \\ \hline
0.20 & 8.588E-04 $m$ & 8.640E-04 $m$ & 1.734E-05  $m$  & 8.813E-04 $m$  & 2.56\%   \\ \hline
0.25 & 8.559E-04 $m$ & 8.640E-04 $m$ & 1.808E-05  $m$  & 8.821E-04 $m$  & 2.97\%   \\ \hline
0.30 & 8.523E-04 $m$ & 8.640E-04 $m$ & 1.884E-05  $m$  & 8.828E-04 $m$  & 3.46\%   \\ \hline
0.35 & 8.480E-04 $m$ & 8.640E-04 $m$ & 1.959E-05  $m$  & 8.836E-04 $m$  & 4.03\%   \\ \hline
0.40 & 8.429E-04 $m$ & 8.640E-04 $m$ & 2.035E-05  $m$  & 8.844E-04 $m$  & 4.69\%   \\ \hline
0.45 & 8.370E-04 $m$ & 8.640E-04 $m$ & 2.111E-05  $m$  & 8.851E-04 $m$  & 5.44\%   \\ \hline
0.49 & 8.316E-04 $m$ & 8.640E-04 $m$ & 2.173E-05  $m$  & 8.857E-04 $m$  & 6.11\%   \\ \hline
  \end{tabular}
  \label{table Displacement results for 4NodeANDES cantilever beams of different Poissons ratios 4}
\end{table}




The errors were plotted in Figure (\ref{table Displacement error 4NodeANDES cantilever beam for different Poisson ratio}).


\begin{figure}[H]
  % \centering
  \begin{subfigure}{0.5\textwidth}
    \centering
    \includegraphics[width=6cm]{../Figure-files/error4andes_beam_dif_poisson_disp_bend_div.jpeg}
    \caption{Error scale 0\% - 15\%}
  \end{subfigure}
  \begin{subfigure}{0.5\textwidth}
    \centering
    \includegraphics[width=6cm]{../Figure-files/error4andes_beam_dif_poisson_disp_bend_div100.jpeg}
    \caption{Error scale 0\% - 100\%}
  \end{subfigure}
  \captionsetup{justification=centering,margin=2cm}
  \caption{4NodeANDES cantilever beam for force perpendicular to the plane(bending)\\
      \emph{\textbf{Displacement error}}   versus   Poisson's ratio}
  \label{table Displacement error 4NodeANDES cantilever beam for different Poisson ratio}
\end{figure}






The ESSI results for the \textbf{\emph{inplane force}} were listed in Table (\ref{table Displacement results for 4NodeANDES cantilever beams of different Poissons ratios 2}) - (\ref{table Displacement results for 4NodeANDES cantilever beams of different Poissons ratios 2 div4}).

\begin{table}[H]
  \centering
      \captionsetup{justification=centering,margin=3cm}
      \caption{\emph{\textbf{ Displacement error }} results for 4NodeANDES with \textcolor{red}{element side length 1 m}  under the inplane force}
  \begin{tabular}{|c|c|c|c|c|c|}
    \hline 
\tabincell{c}{Poisson's \\ ratio}    & \tabincell{c}{4NodeANDES\\displacement} & \tabincell{c}{Theory displacement\\(bending)} & \tabincell{c}{Theory displacement\\(shear)} & \tabincell{c}{Theory\\displacement(all)}   & Error \\ \hline
0.00 & 8.790E-04 $m$ & 8.640E-04 $m$ & 1.440E-05  $m$  & 8.784E-04 $m$  & 0.07\%    \\ \hline
0.05 & 8.799E-04 $m$ & 8.640E-04 $m$ & 1.512E-05  $m$  & 8.791E-04 $m$  & 0.09\%    \\ \hline
0.10 & 8.809E-04 $m$ & 8.640E-04 $m$ & 1.586E-05  $m$  & 8.799E-04 $m$  & 0.12\%    \\ \hline
0.15 & 8.821E-04 $m$ & 8.640E-04 $m$ & 1.659E-05  $m$  & 8.806E-04 $m$  & 0.17\%    \\ \hline
0.20 & 8.835E-04 $m$ & 8.640E-04 $m$ & 1.734E-05  $m$  & 8.813E-04 $m$  & 0.25\%    \\ \hline
0.25 & 8.853E-04 $m$ & 8.640E-04 $m$ & 1.808E-05  $m$  & 8.821E-04 $m$  & 0.37\%    \\ \hline
0.30 & 8.878E-04 $m$ & 8.640E-04 $m$ & 1.884E-05  $m$  & 8.828E-04 $m$  & 0.56\%    \\ \hline
0.35 & 8.913E-04 $m$ & 8.640E-04 $m$ & 1.959E-05  $m$  & 8.836E-04 $m$  & 0.87\%    \\ \hline
0.40 & 8.971E-04 $m$ & 8.640E-04 $m$ & 2.035E-05  $m$  & 8.844E-04 $m$  & 1.44\%    \\ \hline
0.45 & 9.107E-04 $m$ & 8.640E-04 $m$ & 2.111E-05  $m$  & 8.851E-04 $m$  & 2.89\%    \\ \hline
0.49 & 9.901E-04 $m$ & 8.640E-04 $m$ & 2.173E-05  $m$  & 8.857E-04 $m$  & 11.79\%   \\ \hline
  \end{tabular}
  \label{table Displacement results for 4NodeANDES cantilever beams of different Poissons ratios 2}
\end{table}

\begin{table}[H]
  \centering
      \captionsetup{justification=centering,margin=3cm}
      \caption{\emph{\textbf{ Displacement error }} results for 4NodeANDES with \textcolor{red}{element side length 0.5 m}  under the inplane force}
  \begin{tabular}{|c|c|c|c|c|c|}
    \hline 
\tabincell{c}{Poisson's \\ ratio}    & \tabincell{c}{4NodeANDES\\displacement} & \tabincell{c}{Theory displacement\\(bending)} & \tabincell{c}{Theory displacement\\(shear)} & \tabincell{c}{Theory\\displacement(all)}   & Error \\ \hline
0.00 & 8.784E-04 $m$ & 8.640E-04 $m$ & 1.440E-05  $m$  & 8.784E-04 $m$  & 0.00\%   \\ \hline
0.05 & 8.788E-04 $m$ & 8.640E-04 $m$ & 1.512E-05  $m$  & 8.791E-04 $m$  & 0.04\%   \\ \hline
0.10 & 8.787E-04 $m$ & 8.640E-04 $m$ & 1.586E-05  $m$  & 8.799E-04 $m$  & 0.13\%   \\ \hline
0.15 & 8.782E-04 $m$ & 8.640E-04 $m$ & 1.659E-05  $m$  & 8.806E-04 $m$  & 0.27\%   \\ \hline
0.20 & 8.772E-04 $m$ & 8.640E-04 $m$ & 1.734E-05  $m$  & 8.813E-04 $m$  & 0.47\%   \\ \hline
0.25 & 8.759E-04 $m$ & 8.640E-04 $m$ & 1.808E-05  $m$  & 8.821E-04 $m$  & 0.70\%   \\ \hline
0.30 & 8.742E-04 $m$ & 8.640E-04 $m$ & 1.884E-05  $m$  & 8.828E-04 $m$  & 0.98\%   \\ \hline
0.35 & 8.722E-04 $m$ & 8.640E-04 $m$ & 1.959E-05  $m$  & 8.836E-04 $m$  & 1.29\%   \\ \hline
0.40 & 8.699E-04 $m$ & 8.640E-04 $m$ & 2.035E-05  $m$  & 8.844E-04 $m$  & 1.63\%   \\ \hline
0.45 & 8.679E-04 $m$ & 8.640E-04 $m$ & 2.111E-05  $m$  & 8.851E-04 $m$  & 1.94\%   \\ \hline
0.49 & 8.709E-04 $m$ & 8.640E-04 $m$ & 2.173E-05  $m$  & 8.857E-04 $m$  & 1.67\%   \\ \hline
  \end{tabular}
  \label{table Displacement results for 4NodeANDES cantilever beams of different Poissons ratios 2 div2}
\end{table}

\begin{table}[H]
  \centering
      \captionsetup{justification=centering,margin=3cm}
      \caption{\emph{\textbf{ Displacement error }} results for 4NodeANDES with \textcolor{red}{element side length 0.25 m} under the inplane force}
  \begin{tabular}{|c|c|c|c|c|c|}
    \hline 
\tabincell{c}{Poisson's \\ ratio}    & \tabincell{c}{4NodeANDES\\displacement} & \tabincell{c}{Theory displacement\\(bending)} & \tabincell{c}{Theory displacement\\(shear)} & \tabincell{c}{Theory\\displacement(all)}   & Error \\ \hline
0.00 & 8.782E-04 $m$ & 8.640E-04 $m$ &1.440E-05  $m$  & 8.784E-04 $m$  & 0.02\%   \\ \hline
0.05 & 8.786E-04 $m$ & 8.640E-04 $m$ &1.512E-05  $m$  & 8.791E-04 $m$  & 0.06\%   \\ \hline
0.10 & 8.788E-04 $m$ & 8.640E-04 $m$ &1.586E-05  $m$  & 8.799E-04 $m$  & 0.12\%   \\ \hline
0.15 & 8.786E-04 $m$ & 8.640E-04 $m$ &1.659E-05  $m$  & 8.806E-04 $m$  & 0.23\%   \\ \hline
0.20 & 8.781E-04 $m$ & 8.640E-04 $m$ &1.734E-05  $m$  & 8.813E-04 $m$  & 0.37\%   \\ \hline
0.25 & 8.774E-04 $m$ & 8.640E-04 $m$ &1.808E-05  $m$  & 8.821E-04 $m$  & 0.53\%   \\ \hline
0.30 & 8.763E-04 $m$ & 8.640E-04 $m$ &1.884E-05  $m$  & 8.828E-04 $m$  & 0.74\%   \\ \hline
0.35 & 8.750E-04 $m$ & 8.640E-04 $m$ &1.959E-05  $m$  & 8.836E-04 $m$  & 0.97\%   \\ \hline
0.40 & 8.734E-04 $m$ & 8.640E-04 $m$ &2.035E-05  $m$  & 8.844E-04 $m$  & 1.24\%   \\ \hline
0.45 & 8.717E-04 $m$ & 8.640E-04 $m$ &2.111E-05  $m$  & 8.851E-04 $m$  & 1.52\%   \\ \hline
0.49 & 8.706E-04 $m$ & 8.640E-04 $m$ &2.173E-05  $m$  & 8.857E-04 $m$  & 1.71\%   \\ \hline
  \end{tabular}
  \label{table Displacement results for 4NodeANDES cantilever beams of different Poissons ratios 2 div4}
\end{table}



The errors were plotted in Figure (\ref{table Displacement error 4NodeANDES cantilever beam for different Poisson ratio}).


% \begin{figure}[H]
%   % \centering
%   \begin{subfigure}{0.5\textwidth}
%     \centering
%     \includegraphics[width=6cm]{../Figure-files/error4andes_beam_dif_poisson_disp_bend_div.jpeg}
%     \caption{Error scale 0\% - 15\%}
%   \end{subfigure}
%   \begin{subfigure}{0.5\textwidth}
%     \centering
%     \includegraphics[width=6cm]{../Figure-files/error4andes_beam_dif_poisson_disp_bend_div100.jpeg}
%     \caption{Error scale 0\% - 100\%}
%   \end{subfigure}
%   \captionsetup{justification=centering,margin=2cm}
%   \caption{4NodeANDES cantilever beam for force perpendicular to the plane(bending)\\
%       \emph{\textbf{Displacement error}}   versus   Poisson's ratio}
%   \label{table Displacement error 4NodeANDES cantilever beam for different Poisson ratio}
% \end{figure}


\begin{figure}[H]
  % \centering
  \begin{subfigure}{0.5\textwidth}
    \centering
    \includegraphics[width=6cm]{../Figure-files/error4andes_beam_dif_poisson_disp_inplane_div.jpeg}
    \caption{Error scale 0\% - 10\%}
  \end{subfigure}
  \begin{subfigure}{0.5\textwidth}
    \centering
    \includegraphics[width=6cm]{../Figure-files/error4andes_beam_dif_poisson_disp_inplane_div100.jpeg}
    \caption{Error scale 0\% - 100\%}
  \end{subfigure}
  \captionsetup{justification=centering,margin=2cm}
  \caption{4NodeANDES cantilever beam for inplane force\\
      \emph{\textbf{Displacement error}}   versus   Poisson's ratio}
  \label{table Displacement error 4NodeANDES cantilever beam for different Poisson ratio 2}
\end{figure}



The angle results for the force \textbf{\emph{perpendicular to plane (bending)}} were listed in Table (\ref{table angle results for 4NodeANDES cantilever beams of different Poissons ratios}). 

\begin{table}[H]
  \centering
    \captionsetup{justification=centering,margin=3cm}
    \caption{\emph{\textbf{Rotation angle}} results for \textcolor{red}{element side length 1 m} under the force perpendicular to plane (bending)}
  \label{table angle results for 4NodeANDES cantilever beams of different Poissons ratios}
\begin{tabular}{|c|c|c|c|}
\hline
\tabincell{c}{Poisson's \\ ratio} & \tabincell{c}{4NodeANDES \\ angle (unit:\degree)}  & \tabincell{c}{Theory angle\\(unit:\degree)}  & Error   \\ \hline
0.00            & 1.238E-02 & 1.240E-02 & 0.19\% \\ \hline
0.05            & 1.237E-02 & 1.240E-02 & 0.23\% \\ \hline
0.10            & 1.236E-02 & 1.240E-02 & 0.34\% \\ \hline
0.15            & 1.234E-02 & 1.240E-02 & 0.52\% \\ \hline
0.20            & 1.230E-02 & 1.240E-02 & 0.78\% \\ \hline
0.25            & 1.226E-02 & 1.240E-02 & 1.12\% \\ \hline
0.30            & 1.221E-02 & 1.240E-02 & 1.54\% \\ \hline
0.35            & 1.214E-02 & 1.240E-02 & 2.07\% \\ \hline
0.40            & 1.206E-02 & 1.240E-02 & 2.70\% \\ \hline
0.45            & 1.197E-02 & 1.240E-02 & 3.46\% \\ \hline
0.49            & 1.188E-02 & 1.240E-02 & 4.16\% \\
\hline
\end{tabular}
  % \caption{}
\end{table}



\begin{table}[H]
  \centering
    \captionsetup{justification=centering,margin=3cm}
    \caption{\emph{\textbf{Rotation angle}} results for \textcolor{red}{element side length 0.5 m} the force perpendicular to plane (bending)}
  \label{table angle results for 4NodeANDES cantilever beams of different Poissons ratios div2}
\begin{tabular}{|c|c|c|c|}
\hline
\tabincell{c}{Poisson's \\ ratio} & \tabincell{c}{4NodeANDES \\ angle (unit:\degree)}  & \tabincell{c}{Theory angle\\(unit:\degree)}  & Error   \\ \hline
0.00            & 1.239E-02 & 1.240E-02 & 0.10\% \\ \hline
0.05            & 1.238E-02 & 1.240E-02 & 0.13\% \\ \hline
0.10            & 1.237E-02 & 1.240E-02 & 0.22\% \\ \hline
0.15            & 1.236E-02 & 1.240E-02 & 0.36\% \\ \hline
0.20            & 1.233E-02 & 1.240E-02 & 0.55\% \\ \hline
0.25            & 1.230E-02 & 1.240E-02 & 0.81\% \\ \hline
0.30            & 1.226E-02 & 1.240E-02 & 1.13\% \\ \hline
0.35            & 1.221E-02 & 1.240E-02 & 1.52\% \\ \hline
0.40            & 1.216E-02 & 1.240E-02 & 1.97\% \\ \hline
0.45            & 1.209E-02 & 1.240E-02 & 2.51\% \\ \hline
0.49            & 1.203E-02 & 1.240E-02 & 3.00\% \\
\hline
\end{tabular}
  % \caption{}
\end{table}



\begin{table}[H]
  \centering
    \captionsetup{justification=centering,margin=3cm}
    \caption{\emph{\textbf{Rotation angle}} results for \textcolor{red}{element side length 0.25 m} under the force perpendicular to plane (bending)}
  \label{table angle results for 4NodeANDES cantilever beams of different Poissons ratios div4}
\begin{tabular}{|c|c|c|c|}
\hline
\tabincell{c}{Poisson's \\ ratio} & \tabincell{c}{4NodeANDES \\ angle (unit:\degree)}  & \tabincell{c}{Theory angle\\(unit:\degree)}  & Error   \\ \hline
0.00            & 1.238E-02 & 1.240E-02 & 0.19\% \\ \hline
0.05            & 1.237E-02 & 1.240E-02 & 0.21\% \\ \hline
0.10            & 1.237E-02 & 1.240E-02 & 0.28\% \\ \hline
0.15            & 1.235E-02 & 1.240E-02 & 0.39\% \\ \hline
0.20            & 1.233E-02 & 1.240E-02 & 0.56\% \\ \hline
0.25            & 1.230E-02 & 1.240E-02 & 0.78\% \\ \hline
0.30            & 1.227E-02 & 1.240E-02 & 1.05\% \\ \hline
0.35            & 1.223E-02 & 1.240E-02 & 1.38\% \\ \hline
0.40            & 1.218E-02 & 1.240E-02 & 1.77\% \\ \hline
0.45            & 1.212E-02 & 1.240E-02 & 2.23\% \\ \hline
0.49            & 1.207E-02 & 1.240E-02 & 2.64\% \\
\hline
\end{tabular}
  % \caption{}
\end{table}








The errors were plotted in Figure (\ref{table angle error 4NodeANDES cantilever beam for different Poisson ratio}).

\begin{figure}[H]
  % \centering
  \begin{subfigure}{0.5\textwidth}
    \centering
    \includegraphics[width=6cm]{../Figure-files/error4andes_beam_dif_poisson_angle_bend_div.jpeg}
    \caption{Error scale 0\% - 5\%}
  \end{subfigure}
  \begin{subfigure}{0.5\textwidth}
    \centering
    \includegraphics[width=6cm]{../Figure-files/error4andes_beam_dif_poisson_angle_bend_div100.jpeg}
    \caption{Error scale 0\% - 100\%}
  \end{subfigure}
  \captionsetup{justification=centering,margin=2cm}
  \caption{4NodeANDES cantilever beam for force perpendicular to the plane(bending)\\
      \emph{\textbf{Rotation angle error}}   versus   Poisson's ratio}
  \label{table angle error 4NodeANDES cantilever beam for different Poisson ratio}
\end{figure}











The ESSI results for the \textbf{\emph{inplane force}} were listed in Table (\ref{table angle results for 4NodeANDES cantilever beams of different Poissons ratios 2} - (\ref{table angle results for 4NodeANDES cantilever beams of different Poissons ratios 2 div4}). 

\begin{table}[H]
  \centering
          \captionsetup{justification=centering,margin=3cm}
      \caption{\emph{\textbf{Rotation angle}} results for \textcolor{red}{element side length 1 m} under the inplane force}
  \label{table angle results for 4NodeANDES cantilever beams of different Poissons ratios 2}
\begin{tabular}{|c|c|c|c|}
\hline
\tabincell{c}{Poisson's \\ ratio} & \tabincell{c}{4NodeANDES \\ angle (unit:\degree)}  & \tabincell{c}{Theory angle\\(unit:\degree)}  & Error   \\ \hline
0.00            & 1.254E-02 & 1.240E-02 & 1.14\%  \\ \hline
0.05            & 1.255E-02 & 1.240E-02 & 1.19\%  \\ \hline
0.10            & 1.256E-02 & 1.240E-02 & 1.26\%  \\ \hline
0.15            & 1.257E-02 & 1.240E-02 & 1.35\%  \\ \hline
0.20            & 1.258E-02 & 1.240E-02 & 1.47\%  \\ \hline
0.25            & 1.260E-02 & 1.240E-02 & 1.64\%  \\ \hline
0.30            & 1.263E-02 & 1.240E-02 & 1.89\%  \\ \hline
0.35            & 1.269E-02 & 1.240E-02 & 2.30\%  \\ \hline
0.40            & 1.278E-02 & 1.240E-02 & 3.08\%  \\ \hline
0.45            & 1.305E-02 & 1.240E-02 & 5.28\%  \\ \hline
0.49            & 1.506E-02 & 1.240E-02 & 21.43\% \\
\hline
\end{tabular}
  % \caption{}
\end{table}

\begin{table}[H]
  \centering
          \captionsetup{justification=centering,margin=3cm}
      \caption{\emph{\textbf{Rotation angle}} results for \textcolor{red}{element side length 0.5 m} under the inplane force}
  \label{table angle results for 4NodeANDES cantilever beams of different Poissons ratios 2 div2}
\begin{tabular}{|c|c|c|c|}
\hline
\tabincell{c}{Poisson's \\ ratio} & \tabincell{c}{4NodeANDES \\ angle (unit:\degree)}  & \tabincell{c}{Theory angle\\(unit:\degree)}  & Error   \\ \hline
0.00            & 1.271E-02 & 1.240E-02 & 2.51\% \\ \hline
0.05            & 1.272E-02 & 1.240E-02 & 2.56\% \\ \hline
0.10            & 1.272E-02 & 1.240E-02 & 2.58\% \\ \hline
0.15            & 1.272E-02 & 1.240E-02 & 2.60\% \\ \hline
0.20            & 1.273E-02 & 1.240E-02 & 2.63\% \\ \hline
0.25            & 1.273E-02 & 1.240E-02 & 2.67\% \\ \hline
0.30            & 1.274E-02 & 1.240E-02 & 2.77\% \\ \hline
0.35            & 1.277E-02 & 1.240E-02 & 2.98\% \\ \hline
0.40            & 1.283E-02 & 1.240E-02 & 3.47\% \\ \hline
0.45            & 1.299E-02 & 1.240E-02 & 4.79\% \\ \hline
0.49            & 1.361E-02 & 1.240E-02 & 9.78\% \\
\hline
\end{tabular}
  % \caption{}
\end{table}

\begin{table}[H]
  \centering
      \captionsetup{justification=centering,margin=3cm}
      \caption{\emph{\textbf{Rotation angle}} results for \textcolor{red}{element side length 0.25 m} under the inplane force}
  \label{table angle results for 4NodeANDES cantilever beams of different Poissons ratios 2 div4}
\begin{tabular}{|c|c|c|c|}
\hline
\tabincell{c}{Poisson's \\ ratio} & \tabincell{c}{4NodeANDES \\ angle (unit:\degree)}  & \tabincell{c}{Theory angle\\(unit:\degree)}  & Error   \\ \hline
0.00            & 1.268E-02 & 1.240E-02 & 2.24\% \\ \hline
0.05            & 1.268E-02 & 1.240E-02 & 2.27\% \\ \hline
0.10            & 1.268E-02 & 1.240E-02 & 2.30\% \\ \hline
0.15            & 1.269E-02 & 1.240E-02 & 2.31\% \\ \hline
0.20            & 1.269E-02 & 1.240E-02 & 2.33\% \\ \hline
0.25            & 1.269E-02 & 1.240E-02 & 2.35\% \\ \hline
0.30            & 1.270E-02 & 1.240E-02 & 2.41\% \\ \hline
0.35            & 1.271E-02 & 1.240E-02 & 2.53\% \\ \hline
0.40            & 1.275E-02 & 1.240E-02 & 2.83\% \\ \hline
0.45            & 1.284E-02 & 1.240E-02 & 3.58\% \\ \hline
0.49            & 1.312E-02 & 1.240E-02 & 5.77\% \\
\hline
\end{tabular}
  % \caption{}
\end{table}


% \begin{figure}[H]
%   \centering
%   \includegraphics[width=9cm]{../Figure-files/error4andes_beam_different_poisson_ratio_angle.jpeg}
%   % \caption{}
%   % \label{}
% \end{figure}


The errors were plotted in Figure (\ref{table angle error 4NodeANDES cantilever beam for different Poisson ratio}).

\begin{figure}[H]
  % \centering
  \begin{subfigure}{0.5\textwidth}
    \centering
    \includegraphics[width=6cm]{../Figure-files/error4andes_beam_dif_poisson_angle_inplane_div.jpeg}
    \caption{Error scale 0\% - 25\%}
  \end{subfigure}
  \begin{subfigure}{0.5\textwidth}
    \centering
    \includegraphics[width=6cm]{../Figure-files/error4andes_beam_dif_poisson_angle_inplane_div100.jpeg}
    \caption{Error scale 0\% - 100\%}
  \end{subfigure}
  \captionsetup{justification=centering,margin=2cm}
  \caption{4NodeANDES cantilever beam for inplane force\\
      \emph{\textbf{Rotation angle error}}   versus   Poisson's ratio}
  \label{table angle error 4NodeANDES cantilever beam for different Poisson ratio 2}
\end{figure}



The ESSI model fei files for the table above are \href{https://github.com/yuan-energy/ESSI_Verification/blob/master/4NodeANDES/cantilever_different_Poisson/cantilever_different_Poisson.tar.gz?raw=true}{here}






\newpage
\subsection{Test of irregular shaped 4NodeANDES cantilever beams}

Cantilever model was used as an example. 
Three different shapes were tested. 


In the \emph{\textbf{first}} test, the upper two nodes of each element were moved one half element size along the $y-axis$, while the lower two nodes were kept at the same location. The element shape was shown in Figure (\ref{fig irregular shape 1 4NodeANDES cantilever beams }).


\begin{figure}[H]
  \centering
    \begin{subfigure}{0.5\textwidth}
      \centering
      \includegraphics[width=9cm]{../Figure-files/beam_ANDES_xy_bending_pureshape1.pdf}
      \caption{Horizontal plane}
    \end{subfigure}
    \begin{subfigure}{0.5\textwidth}
      \centering
      \includegraphics[width=9cm]{../Figure-files/beam_ANDES_yz_inPlane_pureshape1.pdf}
      \caption{Veritical  plane}
    \end{subfigure}
  \caption{4NodeANDES cantilever beam for irregular \textbf{\emph{Shape 1}} }
  \label{fig irregular shape 1 4NodeANDES cantilever beams }
\end{figure}





In the \emph{\textbf{second}} test, the upper nodes of each element were moved 50\% element size along the $y-axis$, while the lower nodes were moved 50\% element size in the other direction along the $y-axis$. The element shape was shown in Figure (\ref{fig irregular shape 2 4NodeANDES cantilever beams }).


\begin{figure}[H]
  \centering
    \begin{subfigure}{0.5\textwidth}
      \centering
      \includegraphics[width=9cm]{../Figure-files/beam_ANDES_xy_bending_pureshape2.pdf}
      \caption{Horizontal plane}
    \end{subfigure}
    \begin{subfigure}{0.5\textwidth}
      \centering
      \includegraphics[width=9cm]{../Figure-files/beam_ANDES_yz_inPlane_pureshape2.pdf}
      \caption{Veritical  plane}
    \end{subfigure}
  \caption{4NodeANDES cantilever beam for irregular \textbf{\emph{Shape 2}} }
  \label{fig irregular shape 2 4NodeANDES cantilever beams }
\end{figure}





In the \emph{\textbf{third}} test, the upper two nodes of each element were moved 90\% element size with different directions along the $y-axis$, while the lower nodes  were moved 90\% element size in the other direction along the $y-axis$. The element shape was shown in Figure (\ref{fig irregular shape 3 4NodeANDES cantilever beams }).


\begin{figure}[H]
  \centering
    \begin{subfigure}{0.5\textwidth}
      \centering
      \includegraphics[width=9cm]{../Figure-files/beam_ANDES_xy_bending_pureshape3.pdf}
      \caption{Horizontal plane}
    \end{subfigure}
    \begin{subfigure}{0.5\textwidth}
      \centering
      \includegraphics[width=9cm]{../Figure-files/beam_ANDES_yz_inPlane_pureshape3.pdf}
      \caption{Veritical  plane}
    \end{subfigure}
  \caption{4NodeANDES cantilever beam for irregular \textbf{\emph{Shape 3}} }
  \label{fig irregular shape 3 4NodeANDES cantilever beams }
\end{figure}






The boundary conditions were shown in Figure (\ref{fig irregular shape 1 4NodeANDES cantilever beams bc}), (\ref{fig irregular shape 2 4NodeANDES cantilever beams bc}) and (\ref{fig irregular shape 3 4NodeANDES cantilever beams bc}).



\begin{figure}[H]
  \centering
    \begin{subfigure}{0.5\textwidth}
      \centering
      \includegraphics[width=9cm]{../Figure-files/beam_ANDES_xy_bending_shape1.pdf}
      \caption{Horizontal plane}
    \end{subfigure}
    \begin{subfigure}{0.5\textwidth}
      \centering
      \includegraphics[width=9cm]{../Figure-files/beam_ANDES_yz_inPlane_shape1.pdf}
      \caption{Veritical  plane}
    \end{subfigure}
      \captionsetup{justification=centering,margin=3cm}
  \caption{4NodeANDES cantilever beam boundary conditions for irregular \textbf{\emph{Shape 1}} }
  \label{fig irregular shape 1 4NodeANDES cantilever beams bc}
\end{figure}



\begin{figure}[H]
  \centering
    \begin{subfigure}{0.5\textwidth}
      \centering
      \includegraphics[width=9cm]{../Figure-files/beam_ANDES_xy_bending_shape2.pdf}
      \caption{Horizontal plane}
    \end{subfigure}
    \begin{subfigure}{0.5\textwidth}
      \centering
      \includegraphics[width=9cm]{../Figure-files/beam_ANDES_yz_inPlane_shape2.pdf}
      \caption{Veritical  plane}
    \end{subfigure}
      \captionsetup{justification=centering,margin=3cm}
  \caption{4NodeANDES cantilever beam boundary conditions for irregular \textbf{\emph{Shape 2}} }
  \label{fig irregular shape 2 4NodeANDES cantilever beams bc}
\end{figure}





\begin{figure}[H]
  \centering
    \begin{subfigure}{0.5\textwidth}
      \centering
      \includegraphics[width=9cm]{../Figure-files/beam_ANDES_xy_bending_shape3.pdf}
      \caption{Horizontal plane}
    \end{subfigure}
    \begin{subfigure}{0.5\textwidth}
      \centering
      \includegraphics[width=9cm]{../Figure-files/beam_ANDES_yz_inPlane_shape3.pdf}
      \caption{Veritical  plane}
    \end{subfigure}
      \captionsetup{justification=centering,margin=3cm}
  \caption{4NodeANDES cantilever beam boundary conditions for irregular \textbf{\emph{Shape 3}} }
  \label{fig irregular shape 3 4NodeANDES cantilever beams bc}
\end{figure}













The ESSI results were listed in Table (\ref{table Results for 4NodeANDES cantilever beams of irregular shapes}). 
\begin{table}[H]
  \centering
  \caption{Results for 4NodeANDES cantilever beams of irregular shapes}
  \label{table Results for 4NodeANDES cantilever beams of irregular shapes}
  \begin{tabular}{|c|c|c|c|c|c|}
    \hline 
    \multicolumn{6}{|c|}{Displacements for irregular shaped element}   \\ \hline
    Element Type   & Force direction & Normal shape & Shape 1 & Shape 2 & Shape 3  \\ \hline 
    4NodeANDES     & \tabincell{c}{perpendicular to\\plane (bending)}    & 8.639E-04 $m$ & 8.602E-04 $m$ & 8.534E-04 $m$ & 7.851E-04 $m$   \\ \hline
    4NodeANDES     & inplane force   & 8.857E-04 $m$ & 7.036E-04 $m$ & 4.263E-04 $m$ & 1.909E-04 $m$   \\ \hline
    Theoretical    &      -             & 8.784E-04 $m$  & 8.784E-04 $m$ & 8.784E-04 $m$ & 8.784E-04 $m$ \\ \hline
  \end{tabular}
  % \caption{}
\end{table}

The errors were listed in Table (\ref{table Errors for irregular shaped 4NodeANDES compared to theoretical solution}) and (\ref{talbe Errors for irregular shaped 4NodeANDES compared to normal shape}).


\begin{table}[H]
  \centering
  \caption{Errors for irregular shaped 4NodeANDES compared to theoretical solution}
  \label{table Errors for irregular shaped 4NodeANDES compared to theoretical solution}
  \begin{tabular}{|c|c|c|c|c|c|}
    \hline 
    \multicolumn{6}{|c|}{Errors for irregular shaped element, compared to theoretical solutions}   \\ \hline
    Element Type   & Force direction & Normal shape & Shape 1 & Shape 2 & Shape 3  \\ \hline 
    4NodeANDES     & \tabincell{c}{perpendicular to\\plane (bending)}    & 1.65\% & 2.07\% & 2.85\% & 10.63\%  \\ \hline
    4NodeANDES     & inplane force    & 0.83\% & 19.90\% & 51.47\% & 78.27\%  \\ \hline
  \end{tabular}
  % \caption{}
\end{table}

\begin{table}[H]
  \centering
    \caption{Errors for irregular shaped 4NodeANDES compared to normal shape}
  \label{talbe Errors for irregular shaped 4NodeANDES compared to normal shape}
  \begin{tabular}{|c|c|c|c|c|c|}
    \hline 
    \multicolumn{6}{|c|}{Errors for irregular shaped element, compared to normal shape}   \\ \hline
    Element Type   & Force direction & Normal shape & Shape 1 & Shape 2 & Shape 3  \\ \hline 
    4NodeANDES     & \tabincell{c}{perpendicular to\\plane (bending)}    & 0.00\% & 0.42\%  & 1.22\%  & 9.12\%        \\ \hline
    4NodeANDES     & inplane force                                      & 0.00\% & 20.56\% & 51.87\% & 78.45\%       \\ \hline
  \end{tabular}
  % \caption{}
\end{table}

The ESSI model fei files for the table above are \href{https://github.com/yuan-energy/ESSI_Verification/blob/master/4NodeANDES/cantilever_irregular_element/cantilever_irregular_element.tar.gz?raw=true}{here}



% The errors were listed below, compared to the theoretical solution.
% \begin{table}[H]
%   \centering
%   \begin{tabular}{|c|c|c|c|c|}
%     \hline 
%     \multicolumn{5}{|c|}{Test for brick shape displacement errors}   \\ \hline
%     Element Type  & Normal shape & Shape 1 & Shape 2 & Shape 3  \\ \hline 
%     4NodeANDES     &     &    &   & \\ \hline
%     4NodeANDES    &     &    &   &  \\ \hline
%   \end{tabular}
%   % \caption{}
% \end{table}

\newpage
Then, the beam was divided into small elements. 


Problem description: Length=6m, Width=1m, Height=1m, Force=100N, E=1E8Pa, $\nu=0.0$. Use the shear deformation coefficient $\kappa=1.2$. The force direction was shown in Figure (\ref{fig Problem description for cantilever beams under uniform pressure 4}).

\begin{figure}[H]
  \centering
  \includegraphics[width=7cm]{../Figure-files/cantilever_6_uniform_load.pdf}
  \caption{Problem description for cantilever beams under uniform pressure  }
  \label{fig Problem description for cantilever beams under uniform pressure 4}
\end{figure}


Theoretical displacement (bending and shear deformation):
\begin{equation}
  \begin{aligned}
  d &=\frac{qL^4}{8EI} + \frac{q \frac{L^2}{2}}{GA_v} \\ 
    &=\frac{qL^4}{8E\frac{bh^3}{12} }+\frac{q \frac{L^2}{2}}{\frac{E}{2(1+\nu)}\frac{bh}{\kappa}} \\
    &= \frac{400 N/m \times 12^4 m^4}{8\times 10^8 N/m^2 \times \frac{2^4}{12} m^4} 
       + \frac{400 N/m \times \frac{12^2}{2} m^2} {\frac{10^8}{2} N/m^2 \times 2m\times 2m\times \frac{5}{6}} \\ 
    &=7.776\times 10^{-3} m  +1.728\times 10^{-4}  m \\
    &=7.9488\times 10^{-3} m
   \end{aligned}
\end{equation}





The ESSI displacement results were listed in Table (\ref{table Results for 4NodeANDES cantilever beams of irregular shapes with more elements}).
\begin{table}[H]
  \centering
  \caption{Results for 4NodeANDES cantilever beams of irregular shapes with more elements}
  \label{table Results for 4NodeANDES cantilever beams of irregular shapes with more elements}
\begin{tabular}{|c|c|c|c|c|c|}
% \midrule
\hline
\multirow{2}{*}{Element Type} & \multirow{2}{*}{Shape}  & \multirow{2}{*}{Force direction}  & \multicolumn{3}{|c|}{Number of division} \\  \cline{4-6}
                        &        &                  &  1 &  2 &  4  \\ \hline
4NodeANDES              & shape1 & \tabincell{c}{perpendicular to\\plane (bending)} & 7.750E-03 $m$ & 7.768E-03  $m$& 7.774E-03  $m$  \\      \hline
4NodeANDES              & shape1 & \tabincell{c}{inplane  \\   force     }           & 6.822E-03 $m$ & 7.569E-03  $m$& 7.832E-03  $m$  \\ \hline
4NodeANDES              & shape2 & \tabincell{c}{perpendicular to\\plane (bending)}  & 7.656E-03 $m$ & 7.734E-03  $m$& 7.765E-03  $m$  \\       \hline
4NodeANDES              & shape2 & \tabincell{c}{inplane  \\   force     }          & 3.875E-03 $m$ & 5.855E-03  $m$& 7.074E-03  $m$  \\       \hline
4NodeANDES              & shape3 & \tabincell{c}{perpendicular to\\plane (bending)} & 6.637E-03 $m$ & 7.139E-03  $m$& 7.521E-03  $m$  \\        \hline
4NodeANDES              & shape3 & \tabincell{c}{inplane  \\   force     }          & 1.555E-03 $m$ & 2.424E-03  $m$& 3.896E-03  $m$  \\        \hline
 \multicolumn{3}{|c|}{Theoretical solution}      & 7.9488E-03 $m$  & 7.9488E-03 $m$  & 7.9488E-03  $m$ \\
\hline
\end{tabular}
\end{table}

The error were listed in Table (\ref{table Errors for 4NodeANDES cantilever beams of irregular shapes with more elements}). 

\begin{table}[H]
  \centering
  \caption{Errors for 4NodeANDES cantilever beams of irregular shapes with more elements}
  \label{table Errors for 4NodeANDES cantilever beams of irregular shapes with more elements}
\begin{tabular}{|c|c|c|c|c|c|}
\hline
\multirow{2}{*}{Element Type} & \multirow{2}{*}{Shape}  & \multirow{2}{*}{Force direction}  & \multicolumn{3}{|c|}{Number of division} \\  \cline{4-6}
                        &        &                  &  1 &  2 &  4  \\ \hline
4NodeANDES   & shape1      & \tabincell{c}{perpendicular to\\plane (bending)} & 2.51\%  & 2.28\%  & 2.20\%     \\ \hline
4NodeANDES   & shape1      & \tabincell{c}{inplane  \\   force     }          & 14.18\% & 4.78\%  & 1.48\%     \\ \hline
4NodeANDES   & shape2      & \tabincell{c}{perpendicular to\\plane (bending)} & 3.68\%  & 2.71\%  & 2.31\%     \\ \hline
4NodeANDES   & shape2      & \tabincell{c}{inplane  \\   force     }          & 51.25\% & 26.34\% & 11.00\%    \\ \hline
4NodeANDES   & shape3      & \tabincell{c}{perpendicular to\\plane (bending)} & 16.51\% & 10.19\% & 5.38\%     \\ \hline
4NodeANDES   & shape3      & \tabincell{c}{inplane  \\   force     }          & 80.44\% & 69.51\% & 50.98\%    \\
\hline
\end{tabular}
\end{table}

% \begin{figure}[H]
%   \centering
%   \includegraphics[width=9cm]{../Figure-files/error4andes_beam_irregular_shape1.jpeg}
%   % \caption{}
%   % \label{}
% \end{figure}

The errors were shown in Figure (\ref{fig shape 1 4NodeANDES cantilever beam for irregular more elements}), (\ref{fig shape 2 4NodeANDES cantilever beam for irregular more elements}) and (\ref{fig shape 3 4NodeANDES cantilever beam for irregular more elements}). 

\begin{figure}[H]
  % \centering
  \begin{subfigure}{0.5\textwidth}
    \centering
    \includegraphics[width=6cm]{../Figure-files/error4andes_beam_irregular_shape1.jpeg}
    \caption{Error scale 0\% - 15\%}
  \end{subfigure}
  \begin{subfigure}{0.5\textwidth}
    \centering
    \includegraphics[width=6cm]{../Figure-files/error4andes_beam_irregular_shape1100.jpeg}
    \caption{Error scale 0\% - 100\%}
  \end{subfigure}
  \captionsetup{justification=centering,margin=2cm}
  \caption{4NodeANDES cantilever beam for irregular \emph{\textbf{Shape 1}}\\
      Displacement error   versus   Number of division}
  \label{fig shape 1 4NodeANDES cantilever beam for irregular more elements}
\end{figure}


% \begin{figure}[H]
%   \centering
%   \includegraphics[width=9cm]{../Figure-files/error4andes_beam_irregular_shape2.jpeg}
%   % \caption{}
%   % \label{}
% \end{figure}

\begin{figure}[H]
  % \centering
  \begin{subfigure}{0.5\textwidth}
    \centering
    \includegraphics[width=6cm]{../Figure-files/error4andes_beam_irregular_shape2.jpeg}
    \caption{Error scale 0\% - 60\%}
  \end{subfigure}
  \begin{subfigure}{0.5\textwidth}
    \centering
    \includegraphics[width=6cm]{../Figure-files/error4andes_beam_irregular_shape2100.jpeg}
    \caption{Error scale 0\% - 100\%}
  \end{subfigure}
  \captionsetup{justification=centering,margin=2cm}
  \caption{4NodeANDES cantilever beam for irregular \emph{\textbf{Shape 2}}\\
      Displacement error   versus   Number of division}
  \label{fig shape 2 4NodeANDES cantilever beam for irregular more elements}
\end{figure}

% \begin{figure}[H]
%   \centering
%   \includegraphics[width=9cm]{../Figure-files/error4andes_beam_irregular_shape3.jpeg}
%   % \caption{}
%   % \label{}
% \end{figure}

\begin{figure}[H]
  % \centering
  \begin{subfigure}{0.5\textwidth}
    \centering
    \includegraphics[width=6cm]{../Figure-files/error4andes_beam_irregular_shape3.jpeg}
    \caption{Error scale 0\% - 80\%}
  \end{subfigure}
  \begin{subfigure}{0.5\textwidth}
    \centering
    \includegraphics[width=6cm]{../Figure-files/error4andes_beam_irregular_shape3100.jpeg}
    \caption{Error scale 0\% - 100\%}
  \end{subfigure}
  \captionsetup{justification=centering,margin=2cm}
  \caption{4NodeANDES cantilever beam for irregular \emph{\textbf{Shape 3}}\\
      Displacement error   versus   Number of division}
  \label{fig shape 3 4NodeANDES cantilever beam for irregular more elements}
\end{figure}


The ESSI model fei files for the table above are \href{https://github.com/yuan-energy/ESSI_Verification/blob/master/4NodeANDES/cantilever_irregular_element_cut/cantilever_irregular_element_cut.tar.gz?raw=true}{here}












% \newpage
% \subsection{Verification of 4NodeANDES edge clamped beams }

% Problem description: Length=6m, Width=1m, Height=1m, Force=100N, E=1E8Pa, $\nu=0.0$. Use the shear deformation coefficient $\kappa=1.2$. The force direction was shown in Figure (\ref{fig Problem description for clamped beams 4}). 

% \begin{figure}[H]
%   \centering
%   \includegraphics[width=7cm]{../Figure-files/clamped_beam.pdf}
%   \caption{Problem description for clamped beams}
%   \label{fig Problem description for clamped beams 4}
% \end{figure}

% % \subsection{Verification of edge clamped beams - one line elements}



% The elment types and element sizes were same to the cantilever model. Only the boundary conditions and external force locations were changed. 

% The 4NodeANDES elements were shown in Figure (\ref{fig 4NodeANDES elements for clamped beams}) and (\ref{fig 4NodeANDES elements for clamped beams 2}).

% \begin{figure}[H]
%   \centering
%   \includegraphics[width=9cm]{../Figure-files/beam_ANDES_xy_bending.pdf}
%     \captionsetup{justification=centering,margin=4cm}
%   \caption{4NodeANDES elements for clamped beams horizontal plane}
%   \label{fig 4NodeANDES elements for clamped beams}
% \end{figure}


% \begin{figure}[H]
%   \centering
%   \includegraphics[width=9cm]{../Figure-files/beam_ANDES_yz_inPlane.pdf}
%     \captionsetup{justification=centering,margin=4cm}
%   \caption{4NodeANDES elements for clamped beams vertical plane}
%   \label{fig 4NodeANDES elements for clamped beams 2}
% \end{figure}

% Theoretical displacement (bending and shear deformation):
% \begin{equation}
%   \begin{aligned}
%   d &=\frac{FL^3}{192EI}+\frac{\frac{F}{2}\frac{L}{2}}{GA_v}  \\
%     &=\frac{FL^3}{192E\frac{bh^3}{12}}+\frac{\frac{F}{2}\frac{L}{2}}{\frac{E}{2(1+\nu)}\frac{bh}{\kappa}} \\
%    &= \frac{100 N\times 6 m^3}{192 \times 10^8 N/m^2 \times \frac{1}{12} m^4}+ 
%     \frac{\frac{100}{2} N \times \frac{6}{2} m}{\frac{10}{2}\times 10^7 N/m^2\times 1 m^2\times \frac{5}{6}}   \\
%   &=1.35\times 10^{-5} m + 0.36\times 10^{-5} m  \\
%   &=1.71\times 10^{-5} \ m 
%     \end{aligned}
% \end{equation}

% The theoretical solution for $L=6\ m$ was calculated above, while the solutions for other length were calculated by the similar process. 

% In the figures above, only the model with geometry $6m\times 1m \times 1m$ was drawed. In the ESSI models, the geometry $10m\times 1m \times 1m$ and the geometry $20m\times 1m \times 1m$ were also calculated. In three different geometry models, all the element sizes were $1m\times 1m \times 1m$. Therefore, the number of elements used in each model were $6,\ 10\ and\ 20$ respectively.

% The ESSI results for the force \textbf{\emph{perpendicular to plane (bending)}} were listed in Table (\ref{table Results for 4NodeANDES clamped beams of different geometry}). 


% \begin{table}[H]
%   \centering
%   \captionsetup{justification=centering,margin=3cm}
%       \caption{Results for 4NodeANDES clamped beams under the force perpendicular to plane (bending)}
%     \label{table Results for 4NodeANDES clamped beams of different geometry}
%     \begin{tabular}{|c|c|c|c|c|c|}
%     \hline
%     Geometry & 4NodeANDES & Theory(bending) & Theory(shear) & Theory(all) & Error   \\  \hline
%     1:6      & 1.347E-05 $m$ & 1.35E-05  $m$     & 2.50E-06  $m$   & 1.60E-05 $m$ & 18.36\% \\ \hline
%     1:10     & 6.245E-05 $m$ & 6.25E-05  $m$     & 5.00E-06  $m$   & 6.75E-05 $m$ & 7.48\%  \\ \hline
%     1:20     & 4.999E-04 $m$ & 5.00E-04  $m$     & 1.00E-05  $m$   & 5.10E-04 $m$ & 1.98\%  \\
%     \hline
%     \end{tabular}
% \end{table}


% The ESSI results for the \textbf{\emph{inplane force}} were listed in Table (\ref{table Results for 4NodeANDES clamped beams of different geometry 2}). 


% \begin{table}[H]
%   \centering
%   \captionsetup{justification=centering,margin=3cm}
%       \caption{Results for 4NodeANDES clamped beams under the inplane force}
%     \label{table Results for 4NodeANDES clamped beams of different geometry 2}
%     \begin{tabular}{|c|c|c|c|c|c|}
%     \hline
%     Geometry & 4NodeANDES & Theory(bending) & Theory(shear) & Theory(all) & Error   \\  \hline
%     1:6      & 1.622E-05 $m$ & 1.35E-05  $m$     & 2.50E-06  $m$   & 1.60E-05 $m$        & 1.70\% \\ \hline
%     1:10     & 6.796E-05 $m$ & 6.25E-05  $m$     & 5.00E-06  $m$   & 6.75E-05 $m$        & 0.68\% \\ \hline
%     1:20     & 5.123E-04 $m$ & 5.00E-04  $m$     & 1.00E-05  $m$   & 5.10E-04 $m$        & 0.45\% \\
%     \hline
%     \end{tabular}
% \end{table}



% The ESSI model fei files for the table above are \href{https://github.com/yuan-energy/ESSI_Verification/blob/master/4NodeANDES/clamped_beam_different_geometry/clamped_beam_different_geometry.tar.gz?raw=true}{here}









% 
% old table : may be useful.....
% \begin{table}[H]
%   \centering
%   \begin{tabular}{|c|c|c|c|c|}
%     \hline 
%     \multicolumn{5}{|c|}{The edge clamped beam displacement errors}   \\ \hline
%     Element Type  & Force direction  &1:6 & 1:10 & 1:20  \\ \hline 
%     4NodeANDES & in-plane       &    &   & \\ \hline
%     4NodeANDES & out-of-plane        &    &   &  \\ \hline
%     \multicolumn{2}{|c|}{4NodeANDES} &    &   &  \\ \hline
%     \multicolumn{2}{|c|}{4NodeANDES} &   &   &  \\ \hline
%   \end{tabular}
%   % \caption{}
% \end{table}




\newpage
In this section, the beam was cut into smaller elements with element side length 0.5m and 0.25m respectively. And the element side length of the original models is 1.0m. The numerical models were shown in Figure (\ref{fig 4NodeANDES clamped beams with element side length 1.0m}), (\ref{fig 4NodeANDES clamped beams with element side length 0.5m}) and (\ref{fig 4NodeANDES clamped beams with element side length 0.25m}). 

Number of division 1:

\begin{figure}[H]
  \centering
    \begin{subfigure}{0.5\textwidth}
      \centering
      \includegraphics[width=9cm]{../Figure-files/beam_ANDES_xy_bending.pdf}
      \caption{Horizontal plane}
    \end{subfigure}
    \begin{subfigure}{0.5\textwidth}
      \centering
      \includegraphics[width=9cm]{../Figure-files/beam_ANDES_yz_inPlane.pdf}
      \caption{Veritical  plane}
    \end{subfigure}
  \caption{4NodeANDES clamped beam with element side length 1.0m }
  \label{fig 4NodeANDES clamped beams with element side length 1.0m}
\end{figure}


Number of division 2:

\begin{figure}[H]
  \centering
    \begin{subfigure}{0.5\textwidth}
      \centering
      \includegraphics[width=9cm]{../Figure-files/beam_ANDES_xy_more_2.pdf}
      \caption{Horizontal plane}
    \end{subfigure}
    \begin{subfigure}{0.5\textwidth}
      \centering
      \includegraphics[width=9cm]{../Figure-files/beam_ANDES_yz_more_2.pdf}
      \caption{Veritical  plane}
    \end{subfigure}
  \caption{4NodeANDES clamped beam with element side length 0.5m }
  \label{fig 4NodeANDES clamped beams with element side length 0.5m}
\end{figure}






Number of division 4:

\begin{figure}[H]
  \centering
    \begin{subfigure}{0.5\textwidth}
      \centering
      \includegraphics[width=9cm]{../Figure-files/beam_ANDES_xy_more.pdf}
      \caption{Horizontal plane}
    \end{subfigure}
    \begin{subfigure}{0.5\textwidth}
      \centering
      \includegraphics[width=9cm]{../Figure-files/beam_ANDES_yz_more.pdf}
      \caption{Veritical  plane}
    \end{subfigure}
  \caption{4NodeANDES clamped beam with element side length 0.25m }
  \label{fig 4NodeANDES clamped beams with element side length 0.25m}
\end{figure}





The ESSI results for the force \textbf{\emph{perpendicular to plane (bending)}} were listed in Table (\ref{table Results for 4NodeANDES clamped beams with more elements}).  
The theoretical solution is 1.60E-5 $m$. 

\begin{table}[H]
  \centering
    \caption{Results for 4NodeANDES clamped beams under the force perpendicular to plane (bending)}
  \label{table Results for 4NodeANDES clamped beams with more elements}
  \begin{tabular}{|c|c|c|c|c|}
    \hline 
    \multirow{2}{*}{Element Type} 
       & \multicolumn{3}{|c|}{Element side length} \\ \cline{2-4}
       & 1 $m$ & 0.5 $m$ & 0.25 $m$ \\                              \hline
4NodeANDES & 1.347E-05 $m$ & 1.35E-05  $m$& 1.35E-05 $m$  \\ \hline
Error      & 18.36\%   & 18.24\%  & 18.18\%       \\ \hline
  \end{tabular}
  % \caption{}
\end{table}

The ESSI results for the \textbf{\emph{inplane force}} were listed in Table (\ref{table Results for 4NodeANDES clamped beams with more elements 2}). 
The theoretical solution is 1.60E-5 $m$. 

\begin{table}[H]
  \centering
      \caption{Results for 4NodeANDES clamped beams under the inplane force}
  \label{table Results for 4NodeANDES clamped beams with more elements 2}
  \begin{tabular}{|c|c|c|c|c|}
    \hline 
    \multirow{2}{*}{Element Type} 
       & \multicolumn{3}{|c|}{Element side length} \\ \cline{2-4}
       & 1 $m$ & 0.5 $m$ & 0.25 $m$ \\                              \hline
4NodeANDES  & 1.62E-05 $m$ & 1.65E-05 $m$ & 1.69E-05 $m$ \\ \hline
Error       & 1.70\%   & 0.12\%   & 2.12\%       \\         \hline
  \end{tabular}
  % \caption{}
\end{table}

% \begin{figure}[H]
%   \centering
%   \includegraphics[width=9cm]{../Figure-files/error4andes_clamped_beam_diff_element.jpeg}
%   % \caption{}
%   % \label{}
% \end{figure}


The errors were plotted in Figure (\ref{fig error 4NodeANDES clamped beam for different element number}).

\begin{figure}[H]
  % \centering
  \begin{subfigure}{0.5\textwidth}
    \centering
    \includegraphics[width=6cm]{../Figure-files/error4andes_clamped_beam_diff_element.jpeg}
    \caption{Error scale 0\% - 20\%}
  \end{subfigure}
  \begin{subfigure}{0.5\textwidth}
    \centering
    \includegraphics[width=6cm]{../Figure-files/error4andes_clamped_beam_diff_element100.jpeg}
    \caption{Error scale 0\% - 100\%}
  \end{subfigure}
  \captionsetup{justification=centering,margin=2cm}
  \caption{4NodeANDES clamped beam for different element number\\
      Displacement error   versus   Number of division}
  \label{fig error 4NodeANDES clamped beam for different element number}
\end{figure}


The ESSI model fei files for the table above are \href{https://github.com/yuan-energy/ESSI_Verification/blob/master/4NodeANDES/clamped_beam_cut/clamped_beam_cut.tar.gz?raw=true}{here}







\newpage
\subsection{Verification of 4NodeANDES square plate with four edges clamped}

Problem description: Length=20m, Width=20m, Height=1m, Force=100N, E=1E8Pa, $\nu=0.3$. 

The four edges are clamped. 

The load is the uniform normal pressure on the whole plate. 


The plate flexural rigidity is 
\begin{equation}
  D=\frac{Eh^3}{12(1-\nu^2)}=\frac{10^8 N/m^2 \times 1^3 m^3 }{12 \times (1-0.3^2) }= 9.1575 \times 10^6 \ N\cdot m
\end{equation}
The theoretical solution is 
\begin{equation}
  d=\alpha_c \frac{q a^4}{D}=0.00406\times \frac{100 N/m^2 \times 20^4 m^4}{9.1575 \times 10^6 \ N\cdot m}=2.2015\times 10^{-3} m
\end{equation}

where $\alpha_c$ is a coefficient, which depends on the ratio of plate length to width. In this problem, the coefficient\footnote{Stephen Timoshenko, Theory of plates and shells (2nd edition). MrGRAW-Hill Inc, page120, 1959.} $\alpha_c$ is 0.00406.



The 4NodeANDES were shown in Figure (\ref{fig 4NodeANDES edges clamped square plate with element side length 10m }) - (\ref{fig 4NodeANDES edges clamped square plate with element side length 0.25m }). 


\begin{figure}[H]
  \centering
  \includegraphics[width=11cm]{../Figure-files/square_plate1.pdf}
  \caption{4NodeANDES edge clamped square plate with element side length 10m }
  \label{fig 4NodeANDES edges clamped square plate with element side length 10m }
\end{figure}

\newpage

\begin{figure}[H]
  \centering
  \includegraphics[width=11cm]{../Figure-files/square_plate2.pdf}
  \caption{4NodeANDES edge clamped square plate with element side length 5m }
  \label{fig 4NodeANDES edges clamped square plate with element side length 5m }
\end{figure}


\begin{figure}[H]
  \centering
  \includegraphics[width=11cm]{../Figure-files/square_plate3.pdf}
  \caption{4NodeANDES edge clamped square plate with element side length 2m }
  \label{fig 4NodeANDES edges clamped square plate with element side length 2m }
\end{figure}

\newpage

\begin{figure}[H]
  \centering
  \includegraphics[width=11cm]{../Figure-files/square_plate4.pdf}
  \caption{4NodeANDES edge clamped square plate with element side length 1m }
  \label{fig 4NodeANDES edges clamped square plate with element side length 1m }
\end{figure}


\begin{figure}[H]
  \centering
  \includegraphics[width=11cm]{../Figure-files/square_plate5.pdf}
  \caption{4NodeANDES edge clamped square plate with element side length 0.5m }
  \label{fig 4NodeANDES edges clamped square plate with element side length 0.5m }
\end{figure}

\newpage

\begin{figure}[H]
  \centering
  \includegraphics[width=11cm]{../Figure-files/square_plate6.pdf}
  \caption{4NodeANDES edge clamped square plate with element side length 0.25m }
  \label{fig 4NodeANDES edges clamped square plate with element side length 0.25m }
\end{figure}



The results were listed in Table (\ref{table Results for 4NodeANDES square plate with four edges clamped}).

\begin{table}[H]
  \centering
  \caption{Results for 4NodeANDES square plate with four edges clamped}
  \label{table Results for 4NodeANDES square plate with four edges clamped}
\begin{tabular}{|c|c|c|}
\hline
Element type     & 4NodeANDES        &  \multirow{2}{*}{\tabincell{c}{Theoretical \\ displacement}} \\ \cline{1-2}
Element side length & Height:1.00$m$ &        \\ \hline
10$m$            & 2.33E-003 $m$ & 2.20E-03 $m$ \\ \hline
5$m$             & 2.75E-003 $m$ & 2.20E-03 $m$ \\ \hline
2$m$             & 2.58E-003 $m$ & 2.20E-03 $m$ \\ \hline
1$m$             & 2.54E-003 $m$ & 2.20E-03 $m$ \\ \hline
0.5$m$           & 2.53E-003 $m$ & 2.20E-03 $m$ \\ \hline
0.25$m$          & 2.53E-003 $m$ & 2.20E-03 $m$ \\
\hline
\end{tabular}
\end{table}



The errors were listed in Table (\ref{table Errors for 4NodeANDES square plate with four edges clamped}).

\begin{table}[H]
  \centering
    \caption{Errors for 4NodeANDES square plate with four edges clamped}
  \label{table Errors for 4NodeANDES square plate with four edges clamped}
\begin{tabular}{|c|c|}
\hline
Element type     & 4NodeANDES          \\ \hline
Element side length & Height:1.00$m$   \\ \hline
10$m$            & 5.65\%         \\ \hline
5$m$             & 24.98\%        \\ \hline
2$m$             & 16.97\%        \\ \hline
1$m$             & 15.28\%        \\ \hline
0.5$m$           & 14.84\%        \\ \hline
0.25$m$          & 14.73\%       \\
\hline
\end{tabular}
\end{table}

% \begin{figure}[H]
%   \centering
%   \includegraphics[width=9cm]{../Figure-files/error4andes_square_plate_clamped.jpeg}
%   % \caption{}
%   % \label{}
% \end{figure}
The errors were plotted in Figure (\ref{fig 4NodeANDES square plate with edge clamped}).

\begin{figure}[H]
  % \centering
  \begin{subfigure}{0.5\textwidth}
    \centering
    \includegraphics[width=6cm]{../Figure-files/error4andes_square_plate_clamped.jpeg}
    \caption{Error scale 0\% - 25\%}
  \end{subfigure}
  \begin{subfigure}{0.5\textwidth}
    \centering
    \includegraphics[width=6cm]{../Figure-files/error4andes_square_plate_clamped100.jpeg}
    \caption{Error scale 0\% - 100\%}
  \end{subfigure}
  \captionsetup{justification=centering,margin=2cm}
  \caption{4NodeANDES square plate with edge clamped\\
      Displacement error   versus   Number of side division}
  \label{fig 4NodeANDES square plate with edge clamped}
\end{figure}



The ESSI model fei files for the table above are \href{https://github.com/yuan-energy/ESSI_Verification/blob/master/4NodeANDES/square_plate_clamped/square_plate_clamped.tar.gz?raw=true}{here}
















% \newpage
% \begin{itemize}
%   \item \textbf{\emph{Square plate with edges clamped: different geometry}}
% \end{itemize}

% In the figures above, only the model with geometry $20m\times 20m \times 1m$ was drawed. In the ESSI models, the geometry $6m\times 6m \times 1m$ and the geometry $10m\times 10m \times 1m$ were also calculated. In three different geometry models, all the element sizes were $1m\times 1m $ with thickness $1m$.


% The ESSI displacement results were listed below.

% \begin{table}[H]
%   \centering
%   \begin{tabular}{|c|c|c|c|c|}
%     \hline 
%     \multirow{2}{*}{Element Type} 
%        & \multicolumn{3}{|c|}{Model geometry} \\ \cline{2-4}
%        & 1:6 & 1:10 & 1:20 \\                              \hline
% 4NodeANDES & 2.05E-05 $m$ & 1.58E-04 $m$ & 2.53E-03 $m$  \\ \hline
% Theoretical& 1.78E-05 $m$ & 1.38E-04 $m$ & 2.20E-03 $m$      \\ \hline
% Error      & 15.11\% & 14.84\% & 14.73\%       \\ \hline
%   \end{tabular}
%   % \caption{}
% \end{table}








\newpage
\subsection{Verification of 4NodeANDES square plate with four edges simply supported}

Problem description: Length=20m, Width=20m, Height=1m, Force=100N, E=1E8Pa, $\nu=0.3$. 

The four edges are simply supported. 

The load is the uniform normal pressure on the whole plate. 

The plate flexural rigidity is 
\begin{equation}
  D=\frac{Eh^3}{12(1-\nu^2)}=\frac{10^8 N/m^2 \times 1^3 m^3 }{12 \times (1-0.3^2) }= 9.1575 \times 10^6 \ N\cdot m
\end{equation}
The theoretical solution is 
\begin{equation}
  d=\alpha_s \frac{q a^4}{D}=0.00126\times \frac{100 N/m^2 \times 20^4 m^4}{9.1575 \times 10^6 \ N\cdot m}=7.0936\times 10^{-3} m
\end{equation}

where $\alpha_s$ is a coefficient, which depends on the ratio of plate length to width. In this problem, the coefficient\footnote{Stephen Timoshenko, Theory of plates and shells (2nd edition). MrGRAW-Hill Inc, page202, 1959.} $\alpha_s$ is 0.00126.

The 4NodeANDES were shown in Figure (\ref{fig 4NodeANDES edges simply supported square plate with element side length 10m }) - (\ref{fig 4NodeANDES edges simply supported square plate with element side length 0.25m }). 



\begin{figure}[H]
  \centering
  \includegraphics[width=11cm]{../Figure-files/square_plate1.pdf}
  \caption{4NodeANDES edge simply supported square plate with element side length 10m }
  \label{fig 4NodeANDES edges simply supported square plate with element side length 10m }
\end{figure}

\newpage

\begin{figure}[H]
  \centering
  \includegraphics[width=11cm]{../Figure-files/square_plate2.pdf}
  \caption{4NodeANDES edge simply supported square plate with element side length 5m }
  \label{fig 4NodeANDES edges simply supported square plate with element side length 5m }
\end{figure}


\begin{figure}[H]
  \centering
  \includegraphics[width=11cm]{../Figure-files/square_plate3.pdf}
  \caption{4NodeANDES edge simply supported square plate with element side length 2m }
  \label{fig 4NodeANDES edges simply supported square plate with element side length 2m }
\end{figure}

\newpage

\begin{figure}[H]
  \centering
  \includegraphics[width=11cm]{../Figure-files/square_plate4.pdf}
  \caption{4NodeANDES edge simply supported square plate with element side length 1m }
  \label{fig 4NodeANDES edges simply supported square plate with element side length 1m }
\end{figure}


\begin{figure}[H]
  \centering
  \includegraphics[width=11cm]{../Figure-files/square_plate5.pdf}
  \caption{4NodeANDES edge simply supported square plate with element side length 0.5m }
  \label{fig 4NodeANDES edges simply supported square plate with element side length 0.5m }
\end{figure}

\newpage

\begin{figure}[H]
  \centering
  \includegraphics[width=11cm]{../Figure-files/square_plate6.pdf}
  \caption{4NodeANDES edge simply supported square plate with element side length 0.25m }
  \label{fig 4NodeANDES edges simply supported square plate with element side length 0.25m }
\end{figure}


The results were listed in Table (\ref{table Results for 4NodeANDES square plate with four edges simply supported}).

\begin{table}[H]
  \centering
  \caption{Results for 4NodeANDES square plate with four edges simply supported}
  \label{table Results for 4NodeANDES square plate with four edges simply supported}
\begin{tabular}{|c|c|c|}
\hline
Element type     & 4NodeANDES        &  \multirow{2}{*}{\tabincell{c}{Theoretical \\ displacement}} \\ \cline{1-2}
Element side length & Height:1.00$m$ &        \\ \hline
10$m$            & 1.14E-002 $m$ & 7.09E-03 $m$ \\ \hline
5$m$             & 1.03E-002 $m$ & 7.09E-03 $m$ \\ \hline
2$m$             & 9.78E-003 $m$ & 7.09E-03 $m$ \\ \hline
1$m$             & 9.70E-003 $m$ & 7.09E-03 $m$ \\ \hline
0.5$m$           & 9.68E-003 $m$ & 7.09E-03 $m$ \\ \hline
0.25$m$          & 9.67E-003 $m$ & 7.09E-03 $m$ \\
\hline
\end{tabular}
\end{table}


The errors were listed in Table (\ref{table Errors for 4NodeANDES square plate with four edges simply supported}).

\begin{table}[H]
  \centering
  \caption{Errors for 4NodeANDES square plate with four edges simply supported}
  \label{table Errors for 4NodeANDES square plate with four edges simply supported}
\begin{tabular}{|c|c|}
\hline
Element type     & 4NodeANDES          \\ \hline
Element side length & Height:1.00$m$   \\ \hline
10$m$            & 60.34\%        \\ \hline
5$m$             & 45.14\%        \\ \hline
2$m$             & 37.83\%        \\ \hline
1$m$             & 36.69\%        \\ \hline
0.5$m$           & 36.40\%        \\ \hline
0.25$m$          & 36.32\%       \\
\hline
\end{tabular}
\end{table}


% \begin{figure}[H]
%   \centering
%   \includegraphics[width=9cm]{../Figure-files/error4andes_square_plate_simply_supported.jpeg}
%   % \caption{}
%   % \label{}
% \end{figure}

The errors were plotted in Figure (\ref{fig 4NodeANDES square plate with four edge simply supported}).
\begin{figure}[H]
  % \centering
  \begin{subfigure}{0.5\textwidth}
    \centering
    \includegraphics[width=6cm]{../Figure-files/error4andes_square_plate_simply_supported.jpeg}
    \caption{Error scale 0\% - 70\%}
  \end{subfigure}
  \begin{subfigure}{0.5\textwidth}
    \centering
    \includegraphics[width=6cm]{../Figure-files/error4andes_square_plate_simply_supported100.jpeg}
    \caption{Error scale 0\% - 100\%}
  \end{subfigure}
  \captionsetup{justification=centering,margin=2cm}
  \caption{4NodeANDES square plate with edge simply supported\\
      Displacement error   versus   Number of side division}
  \label{fig 4NodeANDES square plate with four edge simply supported}
\end{figure}


The ESSI model fei files for the table above are \href{https://github.com/yuan-energy/ESSI_Verification/blob/master/4NodeANDES/square_plate_simply_support/square_plate_simply_support.tar.gz?raw=true}{here}
















% \newpage
% \begin{itemize}
%   \item \textbf{\emph{Square plate with edges simply supported: different geometry}}
% \end{itemize}

% In the figures above, only the model with geometry $20m\times 20m \times 1m$ was drawed. In the ESSI models, the geometry $6m\times 6m \times 1m$ and the geometry $10m\times 10m \times 1m$ were also calculated. In three different geometry models, all the element sizes were $1m\times 1m $ with thickness $1m$.


% The ESSI displacement results were listed below.

% \begin{table}[H]
%   \centering
%   \begin{tabular}{|c|c|c|c|c|}
%     \hline 
%     \multirow{2}{*}{Element Type} 
%        & \multicolumn{3}{|c|}{Model geometry} \\ \cline{2-4}
%        & 1:6 & 1:10 & 1:20 \\                              \hline
% 4NodeANDES & 7.85E-05 $m$ & 6.05E-04 $m$ & 9.67E-03 $m$  \\ \hline
% Theoretical& 5.75E-05 $m$ & 4.43E-04 $m$ & 7.09E-03 $m$      \\ \hline
% Error      & 36.57\% & 36.40\% & 36.32\%       \\ \hline
%   \end{tabular}
%   % \caption{}
% \end{table}





















\newpage
\subsection{Verification of 4NodeANDES circular plate with all edges clamped}

Problem description: Diameter=20m, Height=1m, Force=100N, E=1E8Pa, $\nu=0.3$. 

The four edges are clamped. 

The load is the uniform normal pressure on the whole plate. 


The plate flexural rigidity is 

\begin{equation}
  D=\frac{Eh^3}{12(1-\nu^2)}=\frac{10^8 N/m^2 \times 1^3 m^3 }{12 \times (1-0.3^2) }= 9.1575 \times 10^6 \ N\cdot m
\end{equation}

The theoretical solution\footnote{Stephen Timoshenko, Theory of plates and shells (2nd edition). MrGRAW-Hill Inc, page55, 1959.} is 

\begin{equation}
  d= \frac{q a^4}{64D}=\frac{100 N/m^2 \times 10^4 m^4}{64 \times 9.1575 \times 10^6 \ N\cdot m}=1.7106\times 10^{-3} m
\end{equation}

The 4NodeANDES were shown in Figure (\ref{fig 4NodeANDES edges clamped circular plate with element side length 10m }) - (\ref{fig 4NodeANDES edges clamped circular plate with element side length 0.25m }). 




\begin{figure}[H]
  \centering
  \includegraphics[width=9cm]{../Figure-files/circular_plate1.pdf}
  \caption{4NodeANDES edge clamped circular plate with element side length 10m }
  \label{fig 4NodeANDES edges clamped circular plate with element side length 10m }
\end{figure}

\newpage

\begin{figure}[H]
  \centering
  \includegraphics[width=9cm]{../Figure-files/circular_plate2.pdf}
  \caption{4NodeANDES edge clamped circular plate with element side length 5m }
  \label{fig 4NodeANDES edges clamped circular plate with element side length 5m }
\end{figure}


\begin{figure}[H]
  \centering
  \includegraphics[width=9cm]{../Figure-files/circular_plate3.pdf}
  \caption{4NodeANDES edge clamped circular plate with element side length 2m }
  \label{fig 4NodeANDES edges clamped circular plate with element side length 2m }
\end{figure}

\newpage

\begin{figure}[H]
  \centering
  \includegraphics[width=9cm]{../Figure-files/circular_plate4.pdf}
  \caption{4NodeANDES edge clamped circular plate with element side length 1m }
  \label{fig 4NodeANDES edges clamped circular plate with element side length 1m }
\end{figure}


\begin{figure}[H]
  \centering
  \includegraphics[width=9cm]{../Figure-files/circular_plate5.pdf}
  \caption{4NodeANDES edge clamped circular plate with element side length 0.5m }
  \label{fig 4NodeANDES edges clamped circular plate with element side length 0.5m }
\end{figure}

\newpage

\begin{figure}[H]
  \centering
  \includegraphics[width=9cm]{../Figure-files/circular_plate6.pdf}
  \caption{4NodeANDES edge clamped circular plate with element side length 0.25m }
  \label{fig 4NodeANDES edges clamped circular plate with element side length 0.25m }
\end{figure}

The results were listed in Table (\ref{table Results for 4NodeANDES circular plate with four edges clamped}).

\begin{table}[H]
  \centering
    \caption{Results for 4NodeANDES circular plate with four edges clamped}
  \label{table Results for 4NodeANDES circular plate with four edges clamped}
\begin{tabular}{|c|c|c|}
\hline
Element type     & 4NodeANDES        &  \multirow{2}{*}{\tabincell{c}{Theoretical \\ displacement}} \\ \cline{1-2}
Element side length & Height:1.00$m$ &        \\ \hline
10$m$            & 1.69E-003 $m$ & 1.706E-03 $m$ \\ \hline
5$m$             & 1.97E-003 $m$ & 1.706E-03 $m$ \\ \hline
2$m$             & 1.97E-003 $m$ & 1.706E-03 $m$ \\ \hline
1$m$             & 1.96E-003 $m$ & 1.706E-03 $m$ \\ \hline
0.5$m$           & 1.96E-003 $m$ & 1.706E-03 $m$ \\ \hline
0.25$m$          & 1.96E-003 $m$ & 1.706E-03 $m$ \\
\hline
\end{tabular}
\end{table}


The errors were listed in Table (\ref{table errors for 4NodeANDES circular plate with four edges clamped}).

\begin{table}[H]
  \centering
      \caption{Errors for 4NodeANDES circular plate with four edges clamped}
  \label{table errors for 4NodeANDES circular plate with four edges clamped}
\begin{tabular}{|c|c|}
\hline
Element type     & 4NodeANDES          \\ \hline
Element side length & Height:1.00$m$   \\ \hline
10$m$            & 0.71\%         \\ \hline
5$m$             & 15.43\%        \\ \hline
2$m$             & 15.31\%        \\ \hline
1$m$             & 15.16\%        \\ \hline
0.5$m$           & 15.13\%        \\ \hline
0.25$m$          & 15.12\%       \\
\hline
\end{tabular}
\end{table}

% \begin{figure}[H]
%   \centering
%   \includegraphics[width=9cm]{../Figure-files/error4andes_circular_plate_clamped.jpeg}
%   % \caption{}
%   % \label{}
% \end{figure}


The errors were shown in Figure (\ref{fig 4NodeANDES circular plate with edge clamped}).
\begin{figure}[H]
  % \centering
  \begin{subfigure}{0.5\textwidth}
    \centering
    \includegraphics[width=6cm]{../Figure-files/error4andes_circular_plate_clamped.jpeg}
    \caption{Error scale 0\% - 20\%}
  \end{subfigure}
  \begin{subfigure}{0.5\textwidth}
    \centering
    \includegraphics[width=6cm]{../Figure-files/error4andes_circular_plate_clamped100.jpeg}
    \caption{Error scale 0\% - 100\%}
  \end{subfigure}
  \captionsetup{justification=centering,margin=3cm}
  \caption{4NodeANDES circular plate with edge clamped\\
      Displacement error   versus   Number of side division}
  \label{fig 4NodeANDES circular plate with edge clamped}
\end{figure}



The ESSI model fei files for the table above are \href{https://github.com/yuan-energy/ESSI_Verification/blob/master/4NodeANDES/circular_plate_clamped/circular_plate_clamped.tar.gz?raw=true}{here}














% \newpage
% \begin{itemize}
%   \item \textbf{\emph{Circular plate with edges clamped: different geometry}}
% \end{itemize}

% In the figures above, only the model with diameter $20m$ thickness $1m$ was drawed. In the ESSI models, the diameter $6m$ thickness $1m$  and diameter $10m$ thickness $1m$  were also calculated. In three different geometry models, all the element sizes were close to standard element, like $1m\times 1m$ with thickness $1m$.


% The ESSI displacement results were listed below.

% \begin{table}[H]
%   \centering
%   \begin{tabular}{|c|c|c|c|c|}
%     \hline 
%     \multirow{2}{*}{Element Type} 
%        & \multicolumn{3}{|c|}{Model geometry} \\ \cline{2-4}
%        & 1:6 & 1:10 & 1:20 \\                              \hline
% 4NodeANDES & 2.55E-04 $m$ & 1.97E-03$m$  & 3.14E-02 $m$  \\ \hline
% Theoretical& 2.21E-04 $m$ & 1.71E-03$m$  & 2.73E-02 $m$      \\ \hline
% Error      & 15.43\% & 15.31\% & 15.16\%       \\ \hline
%   \end{tabular}
%   % \caption{}
% \end{table}



















\newpage
\subsection{Verification of 4NodeANDES circular plate with all edges simply supported}


Problem description: Diameter=20m, Height=1m, Force=100N, E=1E8Pa, $\nu=0.3$. 

The four edges are simply supported. 

The load is the uniform normal pressure on the whole plate. 


The plate flexural rigidity is 

\begin{equation}
  D=\frac{Eh^3}{12(1-\nu^2)}=\frac{10^8 N/m^2 \times 1^3 m^3 }{12 \times (1-0.3^2) }= 9.1575 \times 10^6 \ N\cdot m
\end{equation}

The theoretical solution\footnote{Stephen Timoshenko, Theory of plates and shells (2nd edition). MrGRAW-Hill Inc, page55, 1959.} is 

\begin{equation}
  d= \frac{(5+\nu)  q a^4}{64(1+\nu) D}=\frac{(5+0.3)\times 100 N/m^2 \times 10^4 m^4}{64\times(1+0.3) \times 9.1575 \times 10^6 \ N\cdot m}=6.956\times 10^{-3} m
\end{equation}


The 4NodeANDES were shown in Figure (\ref{fig 4NodeANDES edges simply supported circular plate with element side length 10m }) - (\ref{fig 4NodeANDES edges simply supported circular plate with element side length 0.25m }). 



\begin{figure}[H]
  \centering
  \includegraphics[width=11cm]{../Figure-files/circular_plate1.pdf}
  \caption{4NodeANDES edge simply supported circular plate with element side length 10m }
  \label{fig 4NodeANDES edges simply supported circular plate with element side length 10m }
\end{figure}

\newpage

\begin{figure}[H]
  \centering
  \includegraphics[width=11cm]{../Figure-files/circular_plate2.pdf}
  \caption{4NodeANDES edge simply supported circular plate with element side length 5m }
  \label{fig 4NodeANDES edges simply supported circular plate with element side length 5m }
\end{figure}


\begin{figure}[H]
  \centering
  \includegraphics[width=11cm]{../Figure-files/circular_plate3.pdf}
  \caption{4NodeANDES edge simply supported circular plate with element side length 2m }
  \label{fig 4NodeANDES edges simply supported circular plate with element side length 2m }
\end{figure}

\newpage

\begin{figure}[H]
  \centering
  \includegraphics[width=11cm]{../Figure-files/circular_plate4.pdf}
  \caption{4NodeANDES edge simply supported circular plate with element side length 1m }
  \label{fig 4NodeANDES edges simply supported circular plate with element side length 1m }
\end{figure}


\begin{figure}[H]
  \centering
  \includegraphics[width=11cm]{../Figure-files/circular_plate5.pdf}
  \caption{4NodeANDES edge simply supported circular plate with element side length 0.5m }
  \label{fig 4NodeANDES edges simply supported circular plate with element side length 0.5m }
\end{figure}

\newpage

\begin{figure}[H]
  \centering
  \includegraphics[width=11cm]{../Figure-files/circular_plate6.pdf}
  \caption{4NodeANDES edge simply supported circular plate with element side length 0.25m }
  \label{fig 4NodeANDES edges simply supported circular plate with element side length 0.25m }
\end{figure}





The results were listed in Table (\ref{table Results for 4NodeANDES cicular plate with four edges simply supported}).

\begin{table}[H]
  \centering
  \caption{Results for 4NodeANDES cicular plate with four edges simply supported}
  \label{table Results for 4NodeANDES cicular plate with four edges simply supported}
\begin{tabular}{|c|c|c|}
\hline
Element type     & 4NodeANDES        &  \multirow{2}{*}{\tabincell{c}{Theoretical \\ displacement}} \\ \cline{1-2}
Element side length & Height:1.00$m$ &        \\ \hline
10$m$            & 7.50E-003 $m$ & 6.956E-03 $m$ \\ \hline
5$m$             & 7.29E-003 $m$ & 6.956E-03 $m$ \\ \hline
2$m$             & 7.25E-003 $m$ & 6.956E-03 $m$ \\ \hline
1$m$             & 7.23E-003 $m$ & 6.956E-03 $m$ \\ \hline
0.5$m$           & 7.22E-003 $m$ & 6.956E-03 $m$ \\ \hline
0.25$m$          & 7.22E-003 $m$ & 6.956E-03 $m$ \\
\hline
\end{tabular}
\end{table}


The errors were listed in Table (\ref{table Errors for 4NodeANDES cicular plate with four edges simply supported}).

\begin{table}[H]
  \centering
  \caption{Errors for 4NodeANDES cicular plate with four edges simply supported}
  \label{table Errors for 4NodeANDES cicular plate with four edges simply supported}
\begin{tabular}{|c|c|}
\hline
Element type     & 4NodeANDES          \\ \hline
Element side length & Height:1.00$m$   \\ \hline
10$m$            & 7.75\%        \\ \hline
5$m$             & 4.73\%        \\ \hline
2$m$             & 4.15\%        \\ \hline
1$m$             & 3.89\%        \\ \hline
0.5$m$           & 3.84\%        \\ \hline
0.25$m$          & 3.82\%       \\
\hline
\end{tabular}
\end{table}

% \begin{figure}[H]
%   \centering
%   \includegraphics[width=9cm]{../Figure-files/error4andes_circular_plate_simply_supported.jpeg}
%   % \caption{}
%   % \label{}
% \end{figure}
The errors were plotted in Figure (\ref{fig 4NodeANDES circular plate with four edge simply supported}).
\begin{figure}[H]
  % \centering
  \begin{subfigure}{0.5\textwidth}
    \centering
    \includegraphics[width=6cm]{../Figure-files/error4andes_circular_plate_simply_supported.jpeg}
    \caption{Error scale 0\% - 8\%}
  \end{subfigure}
  \begin{subfigure}{0.5\textwidth}
    \centering
    \includegraphics[width=6cm]{../Figure-files/error4andes_circular_plate_simply_supported100.jpeg}
    \caption{Error scale 0\% - 100\%}
  \end{subfigure}
  \captionsetup{justification=centering,margin=3cm}
  \caption{4NodeANDES circular plate with edge simply supported\\
      Displacement error   versus   Number of side division}
  \label{fig 4NodeANDES circular plate with four edge simply supported}
\end{figure}


The ESSI model fei files for the table above are \href{https://github.com/yuan-energy/ESSI_Verification/blob/master/4NodeANDES/circular_plate_simply_support/circular_plate_simply_support.tar.gz?raw=true}{here}













% \newpage
% \begin{itemize}
%   \item \textbf{\emph{Circular plate with edges simply supported: different geometry}}
% \end{itemize}

% In the figures above, only the model with diameter $20m$ thickness $1m$ was drawed. In the ESSI models, the diameter $6m$ thickness $1m$  and diameter $10m$ thickness $1m$  were also calculated. In three different geometry models, all the element sizes were close to standard element, like $1m\times 1m$ with thickness $1m$.


% The ESSI displacement results were listed below.

% \begin{table}[H]
%   \centering
%   \begin{tabular}{|c|c|c|c|c|}
%     \hline 
%     \multirow{2}{*}{Element Type} 
%        & \multicolumn{3}{|c|}{Model geometry} \\ \cline{2-4}
%        & 1:6 & 1:10 & 1:20 \\                              \hline
% 4NodeANDES & 9.44E-04 & 7.25E-03 & 1.16E-01 $m$  \\ \hline
% Theoretical& 9.02E-04 & 6.96E-03 & 1.11E-01 $m$      \\ \hline
% Error      & 4.73\% & 4.15\% & 3.89\%       \\ \hline
%   \end{tabular}
%   % \caption{}
% \end{table}























\section{G/Gmax plot by Drucker-Prager Armstrong Frederick}

\subsection{Plot stress-strain} ~



\begin{figure}[H]
  \caption{The stress-strain diagram for different $\gamma_c$}
  \centering
    \includegraphics[width=17cm]{../Figure-files/G_stress_strain_plot.png}
    \label{fig:stress_strain}
\end{figure}


When $\gamma_c$ are very small, the materials are in the elastic range. 

\newpage
\subsection{Plot G/Gmax} ~

\begin{figure}[H]
  \caption{The G/Gmax diagram with multiple loops}
  \centering
    \includegraphics[width=17cm]{../Figure-files/G_Gmax_plot.png}
    \label{fig:G_Gmax}
\end{figure}


The $x-axis$ labels are the unitless $\gamma_c$, \textbf{not} in percent.

So the max shear strain is 1 percent here.

\begin{figure}[H]
  \caption{The G/Gmax diagram in one loading}
  \centering
    \includegraphics[width=17cm]{../Figure-files/G_Gmax_plot_one_loading.png}
    \label{fig:G_Gmax_1loading}
\end{figure}


The G/Gmax curve above is plotted by a different method. Only one loading is used to plot G/Gmax. The different plot functions are attached at the end. The curves between different plotting methods are almost the same. 


\newpage
\subsection{Plot damping ratio} ~

\begin{figure}[H]
  \caption{The damping ratio}
  \centering
    \includegraphics[width=17cm]{../Figure-files/G_damping_ratio.png}
    \label{fig:damping}
\end{figure}


\newpage

The parameters I am using
\begin{lstlisting}[language=C++, frame=single] 
  // Yuan is testing different Armstrong Frederick Drucker Prager (AFDP) parameters:
  // For Real ESSI, Units are processed by other functions.
  // Here, all units should be the SI units.
  // initial_von_mises_radius is actually the variable 'k' in the function of yield suface 
  double initial_von_mises_radius  =0.0008;
  double kinematic_hardening_AF_ha =12;   
  double kinematic_hardening_AF_cr =1000; 
  double isotropic_hardening_rate  =0.0;
  double elastic_modulus           =4e6;   //4MPa   
  double poisson_ratio             =0.25;
  double density                   =0.0;
  double initial_confining_stress  =3e5;   //300kPa
\end{lstlisting}


In the Appendix (\ref{sec_Appendix}), you would see the whole cpp code I am using to define the materials.

\newpage
\subsection{Appendix (Code)} \label{sec_Appendix} ~


\subsubsection{Framework: A cpp main function to test the materials directly} ~



\begin{lstlisting}[language=C++, frame=single]  

#include "ClassicElastoplasticMaterial.h"
#include "ConstitutiveModels/DruckerPragerArmstrongFrederick.h"
#include <iostream>
#include <fstream>
#include <vector>
#include <math.h>  

using namespace std;

void print_this_step(const int &printStep, NDMaterialLT* test_material, ofstream & out_strain,ofstream & out_stress){
  // Terminal: print the input strain:
  cout<<"-----------------------------step "<<printStep<<" start------------------------------"<<endl;
  auto input_strain = test_material -> getStrainTensor();
  cout<<"The current strain: "<<endl;
  cout<<input_strain<<endl;

  // Terminal: Output the calculated stress
  auto stress_result = test_material -> getStressTensor();
  cout<<"The output stress: "<<endl;
  cout<<stress_result<<endl;
  cout<<"--------------------------------step end-------------------------------"<<endl;

  // output to file as well
  out_strain<<input_strain(0,2)<<endl;
  out_stress<<stress_result(0,2)<<endl;
}

int main(){
  // The corresponding constructor for reference:
    //First constructor, creates a material at its "ground state" from its parameters.
    // DruckerPragerArmstrongFrederick(int tag_in,
    //  double k0_in,
    //  double ha_alpha,
    //  double cr_alpha,
    //  double H_k,
    //  double E,
    //  double nu,
    //  double rho_,
    //  double p0) ;

  // Define and input the material properties. 
  int    material_tag              =1;

  // Yuan is testing different Armstrong Frederick Drucker Prager (AFDP) parameters:
  // For Real ESSI, units are processed by other functions.
  // Here, all units should be the SI units.
  // initial_von_mises_radius is actually the variable 'k' in the function of yield suface 
  double initial_von_mises_radius  =0.0008;
  double kinematic_hardening_AF_ha =12;   
  double kinematic_hardening_AF_cr =1000; 
  double isotropic_hardening_rate  =0.0;
  double elastic_modulus           =4e6;   //4MPa   
  double poisson_ratio             =0.25;
  double density                   =0.0;
  double initial_confining_stress  =3e5;   //300kPa

  // Set the integration rule and tolerance to the material:
  int method = (int) NDMaterialLT_Constitutive_Integration_Method::Euler_One_Step;
  double f_relative_tol=0.001;
  double stress_relative_tol=0.001;
  double n_max_iterations=10;

    NDMaterialLT::set_constitutive_integration_method(
      method,
      f_relative_tol,
      stress_relative_tol,
      n_max_iterations);

    
  // generate the uniformly distributed values on the logarithmic axis. 
  double strain_start=0.0001/100.0;
  double strain_end  =1.0/100.0;
  double Lstart=log10(strain_start);
  double Lend=log10(strain_end);
    int num_gammac=51;
  std::vector<double> gammac(num_gammac,0.0);

  for (int i = 0; i < num_gammac; ++i)
  {
    gammac[i] = pow(10.0, Lstart+(Lend-Lstart)*i/(num_gammac-1) );
  }
  
  double steplength_loading=strain_start/10.0;

    // the Gammac for the step
  double theGammac=0.0;
  for (int i_gamma = 0; i_gamma < num_gammac; ++i_gamma)
  {
    // gammac is the maximum shear strain for this cyclic loading
    // theGammac=steplength_gammac*(i_gamma+1);
    theGammac=gammac[i_gamma];

    
    // Declare the new materials
      NDMaterialLT* DPrackerAF_material = new DruckerPragerArmstrongFrederick( material_tag,
        initial_von_mises_radius,
        kinematic_hardening_AF_ha,
        kinematic_hardening_AF_cr,
        isotropic_hardening_rate,
        elastic_modulus,
        poisson_ratio,
        density,
        initial_confining_stress) ;

      // Output the parameters:
      cout<<"-----------------------------Start Testing----------------------------"<<endl;
    cout<<"The input Drucker Prager Armstrong Frederick material parameters:"<<endl;
    cout<<"initial_von_mises_radius  : "<<initial_von_mises_radius<<endl;
    cout<<"kinematic_hardening_AF_ha : "<<kinematic_hardening_AF_ha<<endl;
    cout<<"kinematic_hardening_AF_cr : "<<kinematic_hardening_AF_cr<<endl;
    cout<<"isotropic_hardening_rate  : "<<isotropic_hardening_rate<<endl;
    cout<<"elastic_modulus           : "<<elastic_modulus<<endl;
    cout<<"poisson_ratio             : "<<poisson_ratio<<endl;
    cout<<"density                   : "<<density<<endl;
    cout<<"initial_confining_stress  : "<<initial_confining_stress<<endl;
    cout<<"-----------------------------------------------------------------------"<<endl;

    
    // Print out the initial state of the materials:
    cout<<"The initial state:"<<endl;
    auto initial_strain = DPrackerAF_material -> getStrainTensor();
    cout<<"The initial strain state: "<<endl;
    cout<<initial_strain<<endl;

    auto initial_stress = DPrackerAF_material -> getStressTensor();
    cout<<"The initial stress state: "<<endl;
    cout<<initial_stress<<endl;
    cout<<"-----------------------------------------------------------------------"<<endl;
    

    // create the output filename here:
    ofstream outfile_strain;
    ofstream outfile_stress;
    string outfilename_strain= "shearStrain"+ to_string(i_gamma+1) +".txt" ;
    string outfilename_stress= "shearStress"+ to_string(i_gamma+1) +".txt" ;
    
    outfile_strain.open(outfilename_strain);
    outfile_stress.open(outfilename_stress);

    // write the initial state to the file 
    outfile_strain<<"0.0"<<endl;
    outfile_stress<<"0.0"<<endl;


    cout<<"-----------------------------Start loading-----------------------------"<<endl;

    int mystep=0;
    DTensor2 myinputStrain( 3, 3, 0.0 );
    
    short loop=0;
    for (int loop = 0; loop < 2; ++loop){
      // Input positive shear strain increment each step
      myinputStrain(0,2)=steplength_loading;
      myinputStrain(2,0)=myinputStrain(0,2);
      for (int i = 0; i*steplength_loading < theGammac ; ++i)
      {
        DPrackerAF_material->setTrialStrainIncr(myinputStrain);
        DPrackerAF_material->commitState();
        print_this_step(++mystep,DPrackerAF_material,outfile_strain,outfile_stress);
      }

      // Input nagative shear strain increment each step
      myinputStrain(0,2)=-steplength_loading;
      myinputStrain(2,0)=myinputStrain(0,2);
      for (int i = 0; i*steplength_loading < theGammac*2; ++i)
      {
        DPrackerAF_material->setTrialStrainIncr(myinputStrain);
        DPrackerAF_material->commitState();
        print_this_step(++mystep,DPrackerAF_material,outfile_strain,outfile_stress);
      }

      // Input positive shear strain increment each step
      myinputStrain(0,2)=steplength_loading;
      myinputStrain(2,0)=myinputStrain(0,2);
      for (int i = 0; i*steplength_loading < theGammac; ++i)
      {
        DPrackerAF_material->setTrialStrainIncr(myinputStrain);
        DPrackerAF_material->commitState();
        print_this_step(++mystep,DPrackerAF_material,outfile_strain,outfile_stress);
      }
    } 

    // garbage collection before exit this loop.
    outfile_strain.close();
    outfile_stress.close();
    delete DPrackerAF_material;
  }

  return 0;

}

// previous data:
// calculate 9 values only. The plotted curve are not smooth but this will be fast.
  // double Seed_strain[]={0.0001, 0.0003, 0.001, 0.003, 0.01, 0.03, 0.1, 0.3, 1};
    // int num_gammac=sizeof(Seed_strain)/sizeof(double);
  // vector<double> gammac (Seed_strain, Seed_strain + num_gammac);
  // for(auto& g: gammac) g=g/100;

\end{lstlisting}




\newpage
\subsubsection{The function to plot the results with 51 loops} ~

I did not use LBNL's function to plot the results, since I am not using the HDF5 output. I write my own function plot the results. 



\begin{lstlisting}[language=Python, frame=single]  
#!/usr/bin/python

import numpy as np
import matplotlib.pyplot as plt
pi=np.pi

# Open and read files 
# fig1 for stress-strain results:
fig1=plt.figure()
G_Gmax=[1.0]
gamma=[0.0]
xi=[0.0]
for i in range(1,10,1):

  filename_strain ="shearStrain"+str(i)+".txt"
  filename_stress ="shearStress"+str(i)+".txt"
  with open(filename_strain) as fStrain:
    shearStrain = fStrain.read()

  with open(filename_stress) as fStress:
    shearStress = fStress.read()

  # split the string to strains.
  strain_spl=shearStrain.split('\n')
  stress_spl=shearStress.split('\n')
  # remove the last blankspace.
  strain_spl.pop()
  stress_spl.pop()
  # convert string to int
  strain=map(float, strain_spl)
  stress=map(float, stress_spl)
  

  strain_max=max(strain)
  stress_max=max(stress)
  G_sec=stress_max/strain_max

  stress_2nd=stress[1]
  strain_2nd=strain[1]
  G_max=stress_2nd/strain_2nd

  thisG_Gmax=G_sec/G_max
  G_Gmax.append(thisG_Gmax)
  gamma.append(strain_max)

  # ------------------------------------
    # add the subplot to the stress-strain results
  thisFig = fig1.add_subplot(3,3,i)
  thisFig.plot(strain,stress)
  plt.xlabel(r'$\gamma$')
  plt.ylabel(r'$\tau$')
  plt.grid(True)
  plt.autoscale(enable=True,axis='x',tight=True) #
  # plt.xlim( (-0.1, 0.1) )
  # plt.ylim((-1e6,1e6))


  # ------------------------------------
  # calculate the damping ratio
  # find the number of steps in the one loop
  num_zero_strain=0
  strain_len=len(strain)
  index=strain_len
  while True:
    # the last is zero.
    index=index-1
    # find the second last and third last zeros.
    if abs(strain[index-1]) <1e-15:
      num_zero_strain=num_zero_strain+1
    if num_zero_strain==2:
      break;
  num_steps_perloop=strain_len-index+1
  strain_last_loop=strain[-num_steps_perloop:]
  stress_last_loop=stress[-num_steps_perloop:]

  stress_bottom_last_loop= min(stress_last_loop)
  stress_top_bottom = [i-stress_bottom_last_loop for i in stress_last_loop]

  
  A_loop=0.0
  for i in range(num_steps_perloop-1):
    A_loop=A_loop+(stress_top_bottom[i]+stress_top_bottom[i+1])/2*(strain_last_loop[i+1]-strain_last_loop[i])
  
  damping_ratio=A_loop/strain_max/strain_max/2.0/pi/G_sec
  xi.append(damping_ratio)
  print "-------------------------"
  print "G_sec:",G_sec
  print "strain_max",strain_max
  print "A_loop",A_loop
  print "G_Gmax", thisG_Gmax
  print "damping_ratio", damping_ratio
  print "-------------------------"

# For fig1 strain-stress plot
# reduce spacing between subplots to minimize the overlaps.
# plt.tight_layout()

# ------------------------------------
# plot the G_Gmax results
fig2=plt.figure()
plt.semilogx(gamma,G_Gmax,label = "DPAF")
# plot Seed results for comparison
Seed_strain_percent = [0.0001, 0.0003, 0.001, 0.003, 0.01, 0.03, 0.1, 0.3, 1.0];
Seed_G = [1, 0.99, 0.96, 0.9, 0.76, 0.57, 0.3, 0.15, 0.06];
Seed_strain=[x/100.0 for x in Seed_strain_percent];
plt.semilogx(Seed_strain, Seed_G, label="Seed")
plt.legend()
# plt.xlim((0,0.011))
plt.autoscale(enable=True,axis='x',tight=True) #
plt.ylim((0,1.1))
plt.xlabel('$\gamma_c$',fontsize=25)
plt.ylabel('G/Gmax',fontsize=25)
plt.grid(True)
# ------------------------------------
# plot the damping ratio results 
fig3=plt.figure()
plt.semilogx(gamma,xi,label = "DPAF")
sameStrain=Seed_strain
Seed_damping=[0.0, 0.0, 0.01, 0.02, 0.033, 0.05, 0.1, 0.155, 0.21]
plt.semilogx(sameStrain,Seed_damping,label="Seed")
plt.legend()

# plt.ylim((0,0.4))

plt.xlabel(r'$\gamma_c$',fontsize=25)
plt.ylabel(r'$\xi$',fontsize=25)
plt.grid(True)
# ------------------------------------
# disp all the results on screen
plt.show()
\end{lstlisting}



\newpage
\subsubsection{The function to plot the G/Gmax with one loading only} ~


\begin{lstlisting}[language=python, frame=single]  
#!/usr/bin/python

import numpy as np
import matplotlib.pyplot as plt

# Open and read files 
# open one file only 
# 51 is the final loading file only.
i=51

filename_strain ="shearStrain"+str(i)+".txt"
filename_stress ="shearStress"+str(i)+".txt"
with open(filename_strain) as fStrain:
  shearStrain = fStrain.read()

with open(filename_stress) as fStress:
  shearStress = fStress.read()

# split the string to strains.
strain_spl=shearStrain.split('\n')
stress_spl=shearStress.split('\n')
# remove the last blankspace.
strain_spl.pop()
stress_spl.pop()
# convert string to int
strain=map(float, strain_spl)
stress=map(float, stress_spl)

index=0
while True:
  index=index+1
  # find the second zero.
  if strain[index] > strain[index+1]:
    break;

strain_1load=strain[:index]
stress_1load=stress[:index]


stress_2nd=stress[1]
strain_2nd=strain[1]
G_max=stress_2nd/strain_2nd

G_Gmax=[1.0]
gamma=[0.0]

for i in range(index-1):
  G_sec=stress_1load[i+1]/strain_1load[i+1]
  thisG_Gmax=G_sec/G_max
  G_Gmax.append(thisG_Gmax)
  gamma.append(strain_1load[i]) 



# ------------------------------------
# plot the G_Gmax results
fig2=plt.figure()
plt.semilogx(gamma,G_Gmax,label = "DPAF")
# plot Seed results for comparison
Seed_strain_percent = [0.0001, 0.0003, 0.001, 0.003, 0.01, 0.03, 0.1, 0.3, 1.0];
Seed_G = [1, 0.99, 0.96, 0.9, 0.76, 0.57, 0.3, 0.15, 0.06];
Seed_strain=[x/100.0 for x in Seed_strain_percent];
plt.semilogx(Seed_strain, Seed_G, label="Seed")
plt.legend()
# plt.xlim((0,0.011))
plt.autoscale(enable=True,axis='x',tight=True) #
plt.ylim((0,1.1))
plt.xlabel('$\gamma_c$',fontsize=25)
plt.ylabel('G/Gmax',fontsize=25)
plt.grid(True)
# ------------------------------------
plt.show()
\end{lstlisting}















% \paragraph{References:} ~

\nocite{*}
\bibliographystyle{plain}
\bibliography{reference}




%-------------------------------------------------------------------------------------------------------------%
%-------------------------------------------------------------------------------------------------------------%

\end{document}


        10        20        30        40        50        60        70        80
12345678901234567890123456789012345678901234567890123456789012345678901234567890
